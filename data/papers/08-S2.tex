\documentclass[a4, 11pt]{report}


\pagestyle{myheadings}
\markboth{}{Paper II, 2008
\ \ \ \ \ 
\today 
}               

\RequirePackage{amssymb}
\RequirePackage{amsmath}
\RequirePackage{graphicx}
\RequirePackage{color}
\RequirePackage[flushleft]{paralist}[2013/06/09]



\RequirePackage{geometry}
\geometry{%
  a4paper,
  lmargin=2cm,
  rmargin=2.5cm,
  tmargin=3.5cm,
  bmargin=2.5cm,
  footskip=12pt,
  headheight=24pt}


\newcommand{\comment}[1]{{\bf Comment} {\it #1}}
%\renewcommand{\comment}[1]{}

\newcommand{\bluecomment}[1]{{\color{blue}#1}}
%\renewcommand{\comment}[1]{}
\newcommand{\redcomment}[1]{{\color{red}#1}}



\usepackage{epsfig}
\usepackage{pstricks-add}
\usepackage{tgheros} %% changes sans-serif font to TeX Gyre Heros (tex-gyre)
\renewcommand{\familydefault}{\sfdefault} %% changes font to sans-serif
%\usepackage{sfmath}  %%%% this makes equation sans-serif
%\input RexFigs


\setlength{\parskip}{10pt}
\setlength{\parindent}{0pt}

\newlength{\qspace}
\setlength{\qspace}{20pt}


\newcounter{qnumber}
\setcounter{qnumber}{0}

\newenvironment{question}%
 {\vspace{\qspace}
  \begin{enumerate}[\bfseries 1\quad][10]%
    \setcounter{enumi}{\value{qnumber}}%
    \item%
 }
{
  \end{enumerate}
  \filbreak
  \stepcounter{qnumber}
 }


\newenvironment{questionparts}[1][1]%
 {
  \begin{enumerate}[\bfseries (i)]%
    \setcounter{enumii}{#1}
    \addtocounter{enumii}{-1}
    \setlength{\itemsep}{5mm}
    \setlength{\parskip}{8pt}
 }
 {
  \end{enumerate}
 }



\DeclareMathOperator{\cosec}{cosec}
\DeclareMathOperator{\Var}{Var}

\def\d{{\mathrm d}}
\def\e{{\mathrm e}}
\def\g{{\mathrm g}}
\def\h{{\mathrm h}}
\def\f{{\mathrm f}}
\def\p{{\mathrm p}}
\def\s{{\mathrm s}}
\def\t{{\mathrm t}}


\def\A{{\mathrm A}}
\def\B{{\mathrm B}}
\def\E{{\mathrm E}}
\def\F{{\mathrm F}}
\def\G{{\mathrm G}}
\def\H{{\mathrm H}}
\def\P{{\mathrm P}}


\def\bb{\mathbf b}
\def \bc{\mathbf c}
\def\bx {\mathbf x}
\def\bn {\mathbf n}

\newcommand{\low}{^{\vphantom{()}}}
%%%%% to lower suffices: $X\low_1$ etc


\newcommand{\subone}{ {\vphantom{\dot A}1}}
\newcommand{\subtwo}{ {\vphantom{\dot A}2}}




\def\le{\leqslant}
\def\ge{\geqslant}


\def\var{{\rm Var}\,}

\newcommand{\ds}{\displaystyle}
\newcommand{\ts}{\textstyle}
\def\half{{\textstyle \frac12}}
\def\l{\left(}
\def\r{\right)}



\begin{document}
\setcounter{page}{2}

 
\section*{Section A: \ \ \ Pure Mathematics}

%%%%%%%%%%Q1
\begin{question}
A sequence of points $(x_1,y_1)$, $(x_2,y_2)$, $\ldots$ in the
cartesian plane is generated by first choosing $(x_1,y_1)$ then
applying the rule, for $n=1$, $2$, $\ldots$,
\[
(x_{n+1}, y_{n+1}) = (x_n^2-y_n^2 +a, \; 2x_ny_n+b+2)\,,
\]
where $a$ and $b$ are given real constants.

\begin{questionparts}
\item  In the case $a=1$ and  $b=-1$, find the values
of $(x_1,y_1)$ for which the sequence is constant.
\item  Given that $(x_1,y_1) = (-1,1)$, find the  values
of $a$ and $b$ for which the sequence has  period
2.
\end{questionparts}
\end{question}

%%%%%%%%%%Q2
\begin{question}
Let $a_n$ be the coefficient of $x^n$ in the series expansion, 
in ascending powers of $x$, of
\[\displaystyle
\frac{1+x}{(1-x)^2(1+x^2)}
\,,
\]
where $\vert x \vert <1\,$.
Show, using partial fractions,
 that either $a_n =n+1$ or $a_n = n+2$ according to the value of $n$.

Hence find a decimal approximation, to nine significant figures, 
for the fraction $ \displaystyle \frac{11\,000}{8181}$.
\newline
[You are not required to justify the accuracy of your approximation.]



\end{question}

%%%%%%%%% Q3
\begin{question}
\begin{questionparts}
\item  
Find the coordinates of the turning points of the curve
$y=27x^3-27x^2+4$. Sketch the curve  and deduce that
$x^2(1-x)\le 4/27$ for all $x\ge0\,$.

Given that each of the  numbers $a$, $b$ and $c$ lies between 
$0$ and $1$, prove by contradiction that at least one of the 
numbers $bc(1-a)$, $ca(1-b)$ and $ab(1-c)$ is less than
or equal to $4/27$.

\item  Given that each of the  numbers $p$ and $q$ lies between 
$0$ and $1$,
prove that at least one of the numbers
 $p(1-q)$ and $q(1-p)$ is less than or equal to $1/4$. 

\end{questionparts}
\end{question}

%%%%%% Q4 
\begin{question}
A curve is given by
\[x^2+y^2 +2axy = 1,\] where $a$ is a constant satisfying  $0<a<1$.
Show that the gradient of the curve
at the point~$P$ with coordinates $(x,y)$  is 
\[\displaystyle - \frac {x+ay}{ax+y}\,,\]
provided $ax+y \ne0$.
Show that $\theta$,  the acute angle between $OP$ and the normal to the 
curve at $P$, satisfies
\[
\tan\theta = a\vert y^2-x^2\vert\;.
\]

Show further that, 
if $\ \displaystyle \frac{\d \theta}{\d x}=0$ at $P$, then: 
\begin{questionparts}
\item
$a(x^2+y^2)+2xy=0\,$; 
\item
$(1+a)(x^2+y^2+2xy)=1\,$;
\item
$\displaystyle \tan\theta = \frac a{\sqrt{1-a^2}}\,$. 
\end{questionparts}
\end{question}

%%%%%%%%% Q5
\begin{question}
Evaluate the integrals 
\[\displaystyle
\ \int_0^{\frac{1}{2}\pi} \frac{\sin 2x}{1+\sin^2x} \d x \ 
\text{ \ 
and \
}
\displaystyle \ \int_0^{\frac{1}{2}\pi} \frac{\sin x}{1+\sin^2x} \d x
\;.
\]
 
Show, using the binomial expansion, that  $(1+\sqrt2\,)^5<99$.
Show also that $\sqrt 2 > 1.4$. Deduce that
$
2^{\sqrt2} > 1+ \sqrt2\,
$.
Use this result to  determine which of the above integrals is greater.

	\end{question}
	
%%%%%%%%% Q6
\begin{question}
A curve has the equation $y=\f(x)$, where
\[
\f(x) = \cos \Big( 2x+ \frac \pi 3\Big) + \sin \Big ( \frac{3x}2 - \frac
\pi 4\Big).
\] 


\begin{questionparts}

\item Find the period of $\f(x)$.
 
\item Determine all values of $x$ in the interval $-\pi\le x \le \pi$ 
for which $\f(x)=0$. 
Find a value of~$x$ in this interval at which 
 the curve touches the $x$-axis 
without crossing it.

\item Find the  value or values
 of $x$ in the interval $0\le x \le 2\pi$ for which
$\f(x)=2\,$.

\end{questionparts}
\end{question}
	
%%%%%%%%% Q7
\begin{question}
\begin{questionparts}  
\item By writing $y=u{(1+x^2)\vphantom{\dot A}}^{\frac12}$, 
where $u$ is a function of $x$,
find the solution of the equation
\[
\frac 1 y \frac{\d y} {\d x} = xy + \frac x {1+x^2}
\]
for which $y=1$ when $x=0$.

\item Find the solution of the equation
\[
\frac 1 y \frac{\d y} {\d x} = x^2y + \frac {x^2 } {1+x^3}
\]
for which $y=1$ when $x=0$.

\item Give, without proof, a conjecture for 
 the solution of the equation
\[
\frac 1 y \frac{\d y} {\d x} = x^{n-1}y + \frac {x^{n-1} } {1+x^n}
\]
for which $y=1$ when $x=0$, where $n$ is an integer greater than 1.
\end{questionparts}
\end{question}
		
%%%%%%%%% Q8
\begin{question}
The points $A$ and $B$ have position vectors $\bf a$ and $\bf b$, 
respectively, relative to the  origin $O$. The points $A$, $B$ and $O$ are
not collinear. The point $P$ 
lies on 
$AB$ between $A$ and $B$ such that 
\[
AP : PB = (1-\lambda):\lambda\,.
\]
Write down 
the position vector of $P$  in terms of $\bf a$,
  $\bf b$ and $\lambda$.
Given that $OP$ bisects $\angle AOB$, determine $\lambda$
in terms of $a$ and $b$, where $a=\vert \bf a\vert$ and $b=\vert
\bb\vert$.

The point $Q$ also lies on $AB$ between $A$ and $B$,
and is such that $AP=BQ$. 
Prove that $$OQ^2-OP^2=(b-a)^2\,.$$
\end{question}	
		

		
	
\newpage
\section*{Section B: \ \ \ Mechanics}


	
%%%%%%%%%% Q9
\begin{question}
In this question, use $g=10\,$m\,s$^{-2}$.

In cricket, a fast bowler
projects a ball at $40\,$m\,s$^{-1}$  from a point $h\,$m above the ground,
which is horizontal, and at an angle $\alpha$ above the 
horizontal.
The trajectory is such that the ball will
strike the  stumps at ground level  a horizontal distance
of  $20\,$m  from the
point of projection.

\begin{questionparts}
\item 
Determine, in terms of $h$, the two possible values of $\tan\alpha$.

Explain which of these two values is the more appropriate one, and
deduce
that the ball hits the stumps after approximately half a second.
\item State the range of values of $h$ for which the bowler
projects the ball below the horizontal.

\item In the case $h=2.5$, give an approximate value in degrees,
  correct to two significant figures, for
  $\alpha$. You need not justify the accuracy of your 
approximation. 
\end{questionparts}
[You may use the small-angle approximations $\cos\theta \approx 1$ and 
$\sin\theta\approx \theta$.]
	\end{question}
	
%%%%%%%%%% Q10 
\begin{question}	
The lengths of the sides of a rectangular billiards table $ABCD$ are given by
 $AB = DC = a$ and \mbox{$AD=BC = 2b$}.
There are small pockets at the midpoints $M$ and $N$
of the sides $AD$ and $BC$, respectively.
The sides of the table may be taken as smooth vertical walls.

A small ball is projected along the table from the corner $A$. 
It strikes the side 
$BC$ at $X$, then the side $DC$ at $Y$
 and then goes directly into the pocket at $M$.
The angles $BAX$, $CXY$ and $DY\!M$ are $\alpha$, $\beta$ and $\gamma$
respectively.
On each stage of its path, the ball moves with constant speed in a straight
line,
the speeds being $u$, $v$ and $w$ respectively. 
The coefficient of restitution between the ball and the sides is $e$,
where $e>0$.

\begin{questionparts}
\item Show that $\tan\alpha \tan \beta = e$ and find $\gamma$ in
  terms
of $\alpha$.
\item Show that $\displaystyle \tan\alpha = \frac {(1+2e)b} {(1+e)a}$
and deduce that the shot is  possible whatever the value of~$e$.
 
\item Find an expression in terms of $e$ for the fraction of the
kinetic energy of the ball that is lost during the motion.
\end{questionparts}
\end{question}

%%%%%%%%%% Q11

\begin{question}
A wedge of mass $km$ has the shape (in cross-section) of a right-angled
triangle. It stands on a smooth horizontal surface with one face
vertical. The inclined face makes an angle $\theta$ with the 
horizontal surface.
A particle $P$, of mass $m$, is placed on
the inclined face and released from rest. 
The horizontal face of the wedge
is smooth, but the inclined face is rough
and the coefficient of friction between $P$ and 
this face is $\mu$.

\begin{questionparts}
\item When
$P$ is released, it slides down the inclined
plane at an acceleration $a$ relative to the wedge. Show that
the acceleration of the wedge is 
\[
\frac {a \cos\theta}{k+1}\,.
\]

 To a
stationary observer, $P$ appears to descend along a straight line
inclined at an angle~$45^\circ$ to the horizontal. Show that
\[
\tan\theta = \frac k {k+1}\,.
\]


In the case $k=3$, find an expression for $a$ in terms of $g$ and $\mu$.
 

\item What happens when $P$ is released if $\tan\theta \le \mu$?
\end{questionparts}
\end{question}
	

	
	\newpage
\section*{Section C: \ \ \ Probability and Statistics}


%%%%%%%%%% Q12
\begin{question}
In the High Court of  Farnia, the outcome of each case
is determined by three judges: the ass, the beaver and the centaur.
Each judge decides its verdict independently. Being simple
creatures,
they make their decisions entirely at random. Past verdicts show that 
the ass gives a guilty verdict with probability $p$, the beaver
gives a guilty verdict with probability $p/3$ and the  
centaur gives a guilty verdict with probability $p^2$. 

Let $X$ be the number of guilty verdicts given by the three judges in
a case. Given
that $\E(X)= 4/3$, find the value of $p$.

The probability that a defendant brought to trial is guilty is 
$t$. The King
pronounces that the defendant is guilty if at least two of the judges
give a guilty verdict; otherwise, he pronounces the defendant not
guilty.
Find the value of $t$ such that the probability that the King
pronounces
correctly is $1/2$.
\end{question}

%%%%%%%%%% Q13
\begin{question}
Bag $P$ and bag $Q$ each  contain $n$ counters, where $n\ge2$. The
counters are identical
in shape and size, but coloured either black or white. First, $k$ 
counters ($0\le k\le n$) are drawn at random from bag $P$ 
and placed in bag $Q$. Then, $k$ counters are drawn at random from 
bag $Q$ and placed in bag $P$.

\begin{questionparts}
\item If initially
 $n-1$  counters  
in bag $P$ are white and one is black, and 
all $n$ counters in bag~$Q$ are white,  find the probability 
in terms of $n$ and $k$ that the black counter ends up in bag $P$.

Find the value or values of $k$ for which this probability is maximised.

\item If initially
 $n-1$  counters  
in bag~$P$ are white and one is black, and
 $n-1$ counters in bag~$Q$ are white and one is black, 
find the probability 
in terms of $n$ and $k$ that the black counters end up in the same bag.

Find the value or values of $k$ for which this probability is maximised.

\end{questionparts}
\end{question}

\end{document}
