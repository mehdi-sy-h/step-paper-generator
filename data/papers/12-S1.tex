\documentclass[a4, 11pt]{report}


\pagestyle{myheadings}
\markboth{}{Paper I, 2012
\ \ \ \ \ 
\today 
}               

\RequirePackage{amssymb}
\RequirePackage{amsmath}
\RequirePackage{graphicx}
\RequirePackage{color}
\RequirePackage[flushleft]{paralist}[2013/06/09]



\RequirePackage{geometry}
\geometry{%
  a4paper,
  lmargin=2cm,
  rmargin=2.5cm,
  tmargin=3.5cm,
  bmargin=2.5cm,
  footskip=12pt,
  headheight=24pt}


\newcommand{\comment}[1]{{\bf Comment} {\it #1}}
%\renewcommand{\comment}[1]{}

\newcommand{\bluecomment}[1]{{\color{blue}#1}}
%\renewcommand{\comment}[1]{}
\newcommand{\redcomment}[1]{{\color{red}#1}}



\usepackage{epsfig}
\usepackage{pstricks-add}
\usepackage{tgheros} %% changes sans-serif font to TeX Gyre Heros (tex-gyre)
\renewcommand{\familydefault}{\sfdefault} %% changes font to sans-serif
%\usepackage{sfmath}  %%%% this makes equation sans-serif
%\input RexFigs


\setlength{\parskip}{10pt}
\setlength{\parindent}{0pt}

\newlength{\qspace}
\setlength{\qspace}{20pt}


\newcounter{qnumber}
\setcounter{qnumber}{0}

\newenvironment{question}%
 {\vspace{\qspace}
  \begin{enumerate}[\bfseries 1\quad][10]%
    \setcounter{enumi}{\value{qnumber}}%
    \item%
 }
{
  \end{enumerate}
  \filbreak
  \stepcounter{qnumber}
 }


\newenvironment{questionparts}[1][1]%
 {
  \begin{enumerate}[\bfseries (i)]%
    \setcounter{enumii}{#1}
    \addtocounter{enumii}{-1}
    \setlength{\itemsep}{5mm}
    \setlength{\parskip}{8pt}
 }
 {
  \end{enumerate}
 }



\DeclareMathOperator{\cosec}{cosec}
\DeclareMathOperator{\Var}{Var}

\def\d{{\mathrm d}}
\def\e{{\mathrm e}}
\def\g{{\mathrm g}}
\def\h{{\mathrm h}}
\def\f{{\mathrm f}}
\def\p{{\mathrm p}}
\def\s{{\mathrm s}}
\def\t{{\mathrm t}}


\def\A{{\mathrm A}}
\def\B{{\mathrm B}}
\def\E{{\mathrm E}}
\def\F{{\mathrm F}}
\def\G{{\mathrm G}}
\def\H{{\mathrm H}}
\def\P{{\mathrm P}}


\def\bb{\mathbf b}
\def \bc{\mathbf c}
\def\bx {\mathbf x}
\def\bn {\mathbf n}

\newcommand{\low}{^{\vphantom{()}}}
%%%%% to lower suffices: $X\low_1$ etc


\newcommand{\subone}{ {\vphantom{\dot A}1}}
\newcommand{\subtwo}{ {\vphantom{\dot A}2}}




\def\le{\leqslant}
\def\ge{\geqslant}
\def\arcosh{{\rm arcosh}\,}


\def\var{{\rm Var}\,}

\newcommand{\ds}{\displaystyle}
\newcommand{\ts}{\textstyle}
\def\half{{\textstyle \frac12}}
\def\l{\left(}
\def\r{\right)}



\begin{document}
\setcounter{page}{2}

 
\section*{Section A: \ \ \ Pure Mathematics}

%%%%%%%%%%Q1
\begin{question}
The line $L$ has equation $y=c-mx$, with $m>0$ and $c>0$. 
It passes through the  point 
$R(a,b)$ and cuts the axes at the points $P(p,0)$ and $Q(0,q)$,
where $a$, $b$, $p$ and $q$ are all positive. Find $p$ and $q$
in terms of $a$, $b$ and $m$.

As $L$ varies with $R$ remaining fixed, show that
the minimum value of the sum of the distances 
of $P$ and $Q$ from the origin is $(a^{\frac12} + b^{\frac12})^2$,
and find in  a similar form 
the minimum distance between $P$ and $Q$. (You may assume
that any stationary values   of these distances are minima.) 
\end{question}

%%%%%%%%%%Q2
\begin{question}
\begin{questionparts}
\item
Sketch the curve 
$
y= x^4-6x^2+9 
$
giving the coordinates of the stationary points.

Let $n$ be the number of distinct real values of $x$ for which 
\[
x^4-6x^2 +b=0.
\] 
State the values of $b$, if any, for which
\ (a) $n=0\,$;
\ (b) $n=1\,$;
\ (c) $n=2\,$;
\ (d) $n=3\,$;
\ (e) $n=4\,$.

\item
For which values of $a$ does the curve $y= x^4-6x^2 +ax +b$
have a point at which 
both $\dfrac{\d y}{\d x}=0$ and $\dfrac{\d^2y}{\d x^2}=0\,$?

 For these values of $a$,
find the number of distinct real values of $x$ for which 
$\vphantom{\dfrac{A}{B}}$
\[
x^4-6x^2 +ax +b=0\,,
\]
 in the different cases that arise according to the
value of $b$. 

\item Sketch the curve $y= x^4-6x^2 +ax$ in the case $a>8\,$.

\end{questionparts}
\end{question}

%%%%%%%%% Q3
\begin{question}
\begin{questionparts}
\item
Sketch the curve $y=\sin x$ for $0\le x \le \tfrac12 \pi$
and add to your diagram the 
tangent to the curve at the origin and the chord joining the 
origin to the point $(b, \sin b)$, where $0<b<\frac12\pi$.

By considering areas, show that 
\[
1-\tfrac12 b^2 <\cos b < 1-\tfrac 12 b \sin b\,.
\]

\item By considering the curve $y=a^x$, where $a>1$, show
that 
\[
\frac{2(a-1)}{a+1} < \ln a < -1 + \sqrt{2a-1\,}\,.
\]
[{\bf Hint}: You may wish to write $a^x$ as $\e^{x\ln a}$.]
\end{questionparts} 
\end{question}

%%%%%% Q4 
\begin{question}
The curve $C$ has equation $xy = \frac12$.
The tangents to $C$ at the distinct
points 
$
P\big(p, 
\frac1 
{
\rule[0pt]{0pt}{2.7mm}
\mbox{\fontsize{10pt}{13}\selectfont$2p$}
}
\big)
$
and
$
Q\big(q, 
\frac1 
{
\rule[0pt]{0pt}{2.7mm}
\mbox{\fontsize{10pt}{13}\selectfont$2q$}
}
\big),
$
where $p$ and $q$ are positive,
intersect at $T$ and the normals to $C$
 at these points intersect at~$N$. Show that 
 $T$ is the point 
\[
\left( \frac{2pq}{p+q}\,,\, \frac 1 {p+q}\right)\!.
\] 

In the case $pq=\frac12$, find the coordinates of $N$. Show (in this case)
that
$T$ and $N$ lie on the line $y=x$ and are such that the 
product of their distances from the origin is constant.
\end{question}

%%%%%%%%% Q5
\begin{question}
 Show that 
\[
\int_0^{\frac14\pi} \sin (2x) \ln(\cos x)\, \d x = \frac14(\ln 2 -1)\,,
\]
and that 
\[
\int_0^{\frac14\pi} \cos (2x) \ln(\cos x)\, \d x = \frac18(\pi -\ln 4-2)\,.
\]

Hence evaluate
\[
\int_{\frac14\pi}^{\frac12\pi}
 \big ( \cos(2x) + \sin (2x)\big) \, \ln \big( \cos x + \sin x\big)\, \d x\,.
\]
	\end{question}
	
%%%%%%%%% Q6
\begin{question}
A thin circular path with diameter $AB$ is laid on horizontal ground.
A vertical flagpole is erected with its base at a point $D$
on the diameter $AB$. The angles of 
elevation of the top of the flagpole from $A$ and $B$ are $\alpha$ and
$\beta$ respectively (both are acute). 
The point $C$ lies on the circular path  
with $DC$ perpendicular to $AB$ and the angle of elevation of the top of the
flagpole from $C$ is $\phi$. Show that $\cot\alpha\cot \beta  = \cot^2\phi$.


Show that, for any $p$ and $q$,
\[
\cos p \cos q \sin^2\tfrac12(p+q) 
- \sin p\sin q \cos^2 \tfrac12 (p+q) = 
\tfrac12 \cos(p+q) -\tfrac12 \cos(p+q)\cos(p-q)
.\]
Deduce that, if $p$ and $q$ are positive and  $ p+q  \le \tfrac12 \pi$,
then
\[
 \cot p\cot q\,
\ge \cot^2 \tfrac12(p+q) \,
\]
and hence show 
that $\phi \le \tfrac12(\alpha+\beta)$ 
when $ \alpha +\beta \le \tfrac12 \pi\,$.
\end{question}
	
%%%%%%%%% Q7
\begin{question}
A sequence of numbers $t_0$, $t_1$, $t_2$, $\ldots\,$ satisfies
\[
\ \ \ \ \ \ \ \ \ \ \ \ \ \ \ \ \ \ \ \ \ \ \ \  
t_{n+2} = p t_{n+1}+qt_{n}  \ \ \ \ \ \ \ \ \ \ (n\ge0),
\]
where $p$ and $q$ are real. Throughout this question, $x$, $y$ and $z$ are non-zero real numbers.
\begin{questionparts}
\item Show that, if $t_n=x$ for all values of $n$,
then $p+q=1$ and $x$ can be any (non-zero) real number.
\item 
Show that, if $t_{2n} = x$ and $t_{2n+1}=y$ for all values of $n$,
then $q\pm p=1$. Deduce that either $x=y$ or $x=-y$, unless
$p$ and $q$ take certain values that you should identify.
\item 
Show that, if $t_{3n} = x$,  $t_{3n+1}=y$ and $t_{3n+2}=z$
for all values of $n$,
 then
\[ 
p^3+q^3 +3pq-1=0\,.
\]
 Deduce that either $p+q=1$ or 
$(p-q)^2 +(p+1)^2+(q+1)^2=0$. Hence show that either 
$x=y=z$ or $x+y+z=0$.
\end{questionparts}
\end{question}
		
%%%%%%%%% Q8
\begin{question}
\begin{questionparts}
\item
Show that substituting  $y=xv$, where $v$ is a function of $x$,  in the 
 differential equation
\[ 
\hphantom{(x\ne0)\hspace*{2cm}}
xy \frac{\d y}{\d x} +y^2- 2x^2 =0
{\hspace*{2cm}(x\ne0)}
\]
leads to  the differential equation
\[
xv\frac{\d v}{\d x} +2v^2 -2=0\,.
\]
Hence show that the general solution can be written in the form 
\[
x^2(y^2 -x^2) = C 
\,,
\] where $C$ is a constant. 


\item
Find the general solution of the differential equation
\[
\hphantom{(x\ne0)\hspace*{2cm}}
 y \frac{\d y}{\d x} +6x +5y=0\, 
{\hspace*{2cm}(x\ne0)}
.
\]

\end{questionparts}
\end{question}	
		

		
	
\newpage
\section*{Section B: \ \ \ Mechanics}


	
%%%%%%%%%% Q9
\begin{question}
A tall shot-putter
projects a small shot 
from a point $2.5\,$m above the ground, which is horizontal. 
The speed of projection is
$10\,$m\,s$^{- 1}$  and the  angle of projection is $\theta$
 above the horizontal. 
Taking the acceleration due to gravity to be $10\,$m\,s$^{-2}$, 
show that the time, 
in seconds, that elapses before the shot hits the ground is 
\[
\frac1{\sqrt2}\left ( \sqrt{1-c}+ \sqrt{2-c}\right),
\]
where $c = \cos2\theta$.

Find an expression for the range in terms of $c$ and show that
it is greatest when $c= \frac15\,$.

 Show that the extra distance attained 
by projecting the shot at this angle rather than at an angle of $45^\circ$
 is $5(\sqrt6 -\sqrt2 -1)\,$m.
	\end{question}
	
%%%%%%%%%% Q10 
\begin{question}	
I        stand  at the top of a vertical well.
The   
depth of the well, from the top to the surface of the water, is
$D$. I  drop  a stone from 	
the top of the well and 
measure  the time that
 elapses between the release of the stone and the moment when 
I   hear the splash  of the stone entering the water.

In order to gauge the depth of the well,  
I  climb  a distance $\delta$  down into the well 
and drop a stone from my new position. The
time until I hear the splash is $t$ less than the previous time.
Show that
\[
t = \sqrt{\frac{2D}g} -
 \sqrt{\frac{2(D-\delta)}g} + \frac \delta u\,,
\]
where $u$ is the (constant) speed of sound.
Hence show that
\[
D = \tfrac12 gT^2\,,
\]
where $T= \dfrac12 \beta + \dfrac \delta{\beta g}$ 
and $\beta = t - \dfrac \delta u\,$.
	

Taking
$u=300\,$m\,s$^{-1}$ 
and $g=10\,$m\,s$^{-2}$,
show that if $t= \frac 15\,$s and  $\delta=10\,$m, 
 the well is approximately $185\,$m deep.
\end{question}

%%%%%%%%%% Q11

\begin{question}
The diagram shows two particles, $A$ of mass $5m$ and $B$ of mass $3m$, 
connected by a light  inextensible string
which passes over two smooth, 
light, fixed pulleys, $Q$ and  $R$, 
and under a smooth pulley $P$ which has mass $M$ and is free to move 
vertically.

Particles $A$ and $B$ lie on fixed rough planes
inclined to the horizontal at angles of $\arctan \frac 7{24}$ and
$\arctan\frac43$ respectively.
The segments $AQ$ and $RB$ of the string are 
parallel to their respective planes, and segments $QP$
 and 
$PR$ are vertical. 
The coefficient of friction between each particle and its plane is $\mu$.

\begin{center}
\psset{xunit=1.0cm,yunit=1.0cm,algebraic=true,dimen=middle,dotstyle=o,dotsize=3pt 0,linewidth=0.3pt,arrowsize=3pt 2,arrowinset=0.25}
\begin{pspicture*}(-1.07,-0.16)(9.6,4.76)
\pscircle[fillcolor=black,fillstyle=solid,opacity=0.15](5,4){0.25}
\pscircle[fillcolor=black,fillstyle=solid,opacity=0.15](6.5,4){0.25}
\pspolygon[linewidth=0pt,fillcolor=black,fillstyle=solid,opacity=0.3](-1,0)(5,0)(5,4)
\pspolygon[linewidth=0pt,fillcolor=black,fillstyle=solid,opacity=0.3](9.5,0)(6.5,0)(6.5,4)

\psline(1.63,2.12)(4.85,4.2)
\psline(8.29,2.04)(6.712,4.13)
\psline(5.25,4)(5.25,1.5)
\psline(6.25,3.99)(6.25,1.5)
\rput[tl](4.85,4.68){$Q$}
\rput[tl](6.38,4.64){$R$}
\rput[tl](8.29,2.72){$B$}
\rput[tl](5.6,0.85){$P$}
\rput[tl](1.54,2.85){$A$}
\begin{scriptsize}
\psdots[dotsize=28.5pt,dotstyle=*](5.75,1.5)
\psdots[dotsize=17pt 0,dotstyle=*](1.63,2.12)
\psdots[dotsize=15pt 0,dotstyle=*](8.29,2.04)
\end{scriptsize}
\end{pspicture*}
\end{center}

\begin{questionparts}
\item Given that the system is in equilibrium, 
with both $A$ and $B$ on the point of moving up their    planes,
determine the value of $\mu$  and show that $M = 6m$.

\item In the case when $M = 9m$, determine the 
initial accelerations of $A$, $B$ and $P$ in terms of $g$.


\end{questionparts}
\end{question}
	

	
	\newpage
\section*{Section C: \ \ \ Probability and Statistics}


%%%%%%%%%% Q12
\begin{question}
Fire extinguishers may become faulty at any time
after  manufacture and 
are tested 
annually on the
 anniversary of  manufacture.

The time $T$ years after manufacture
 until a fire extinguisher
becomes faulty
 is modelled by the continuous
probability density function 
\[
\f(t) =
\begin{cases}
\dfrac{2t}{(1+t^2)^2}& \text{for $t\ge0$}\,,\\[4mm]
 \ \ \ \ 0& \text{otherwise}.
\end{cases}
\]
A faulty fire extinguisher will fail an annual test
with probability $p$, in which case it is destroyed immediately. A non-faulty
fire extinguisher will always pass the test. All of the annual tests
are independent.

Show that the probability 
that a randomly chosen fire extinguisher will be destroyed exactly
three years after its manufacture is $p(5p^2-13p +9)/10$. 

Find the probability that a randomly chosen
 fire extinguisher
that was destroyed exactly three years after its manufacture
was faulty 18 months after its manufacture.
\end{question}

%%%%%%%%%% Q13
\begin{question}
I choose at random an integer in the 
range 10000 to 99999, all choices being
equally likely. Given that my choice
does not contain the digits 0, 6, 7, 8 or 9,
show that the expected number of different digits
in my choice is 3.3616.
\end{question}

\end{document}
