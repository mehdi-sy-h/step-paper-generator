\documentclass[a4, 11pt]{report}


\pagestyle{myheadings}
\markboth{}{Paper II, 2013
\ \ \ \ \ 
\today 
}               

\RequirePackage{amssymb}
\RequirePackage{amsmath}
\RequirePackage{graphicx}
\RequirePackage{color}
\RequirePackage[flushleft]{paralist}[2013/06/09]



\RequirePackage{geometry}
\geometry{%
  a4paper,
  lmargin=2cm,
  rmargin=2.5cm,
  tmargin=3.5cm,
  bmargin=2.5cm,
  footskip=12pt,
  headheight=24pt}


\newcommand{\comment}[1]{{\bf Comment} {\it #1}}
%\renewcommand{\comment}[1]{}

\newcommand{\bluecomment}[1]{{\color{blue}#1}}
%\renewcommand{\comment}[1]{}
\newcommand{\redcomment}[1]{{\color{red}#1}}



\usepackage{epsfig}
\usepackage{pstricks-add}
\usepackage{tgheros} %% changes sans-serif font to TeX Gyre Heros (tex-gyre)
\renewcommand{\familydefault}{\sfdefault} %% changes font to sans-serif
%\usepackage{sfmath}  %%%% this makes equation sans-serif
%\input RexFigs


\setlength{\parskip}{10pt}
\setlength{\parindent}{0pt}

\newlength{\qspace}
\setlength{\qspace}{20pt}


\newcounter{qnumber}
\setcounter{qnumber}{0}

\newenvironment{question}%
 {\vspace{\qspace}
  \begin{enumerate}[\bfseries 1\quad][10]%
    \setcounter{enumi}{\value{qnumber}}%
    \item%
 }
{
  \end{enumerate}
  \filbreak
  \stepcounter{qnumber}
 }


\newenvironment{questionparts}[1][1]%
 {
  \begin{enumerate}[\bfseries (i)]%
    \setcounter{enumii}{#1}
    \addtocounter{enumii}{-1}
    \setlength{\itemsep}{5mm}
    \setlength{\parskip}{8pt}
 }
 {
  \end{enumerate}
 }



\DeclareMathOperator{\cosec}{cosec}
\DeclareMathOperator{\Var}{Var}

\def\d{{\mathrm d}}
\def\e{{\mathrm e}}
\def\g{{\mathrm g}}
\def\h{{\mathrm h}}
\def\f{{\mathrm f}}
\def\p{{\mathrm p}}
\def\q{{\mathrm q}}
\def\s{{\mathrm s}}
\def\t{{\mathrm t}}


\def\A{{\mathrm A}}
\def\B{{\mathrm B}}
\def\E{{\mathrm E}}
\def\F{{\mathrm F}}
\def\G{{\mathrm G}}
\def\H{{\mathrm H}}
\def\P{{\mathrm P}}


\def\bb{\mathbf b}
\def \bc{\mathbf c}
\def\bx {\mathbf x}
\def\bn {\mathbf n}

\newcommand{\low}{^{\vphantom{()}}}
%%%%% to lower suffices: $X\low_1$ etc


\newcommand{\subone}{ {\vphantom{\dot A}1}}
\newcommand{\subtwo}{ {\vphantom{\dot A}2}}




\def\le{\leqslant}
\def\ge{\geqslant}
\def\arcosh{{\rm arcosh}\,}


\def\var{{\rm Var}\,}

\newcommand{\ds}{\displaystyle}
\newcommand{\ts}{\textstyle}
\def\half{{\textstyle \frac12}}
\def\l{\left(}
\def\r{\right)}



\begin{document}
\setcounter{page}{2}

 
\section*{Section A: \ \ \ Pure Mathematics}

%%%%%%%%%%Q1
\begin{question}
\begin{questionparts}
\item Find the value of $m$ for which  the line 
$y = mx$ touches  the curve $y = \ln x\,$.  

If instead the line intersects the curve
when $x = a$ and $x = b$, where $a < b$, show  that $a^b = b^a$. 
Show by means of a sketch  that $a < \e < b$.

\item  
The line $y=mx+c$, where $c>0$,
 intersects the curve $y=\ln x$ when $x=p$ and $x=q$,
where $p<q$.
Show by means of a sketch, or otherwise,  that $p^q>q^p\,$.

\item
Show by means of a sketch that the
 straight line through the points
$(p, \ln p)$ and $(q, \ln q)$, where $\e\le p<q\,$,  intersects  the $y$-axis
at a positive value of $y$.  Which is greater, $\pi^\e$ or $\e^\pi$?

\item
Show, using a sketch or otherwise, that if $0<p<q$ and 
 $\dfrac{\ln q - \ln p}{q-p} = \e^{-1}$, then $q^p>p^q$.  
 
\end{questionparts}
\end{question}
\vspace{-0.8cm}
%%%%%%%%%%Q2
\begin{question}
For $n\ge 0$, let 
\[
I_n = \int_0^1 x^n(1-x)^n\d x\,.
\]

\begin{questionparts}
\item
For $n\ge 1$, show by means of a substitution that 
\[
 \int_0^1 x^{n-1}(1-x)^n\d x = \int_0^1 x^n(1-x)^{n-1}\d x\,
\]
and deduce that 
\[
2
 \int_0^1 x^{n-1}(1-x)^n\d x = I_{n-1}\,.
\]

Show also, for $n\ge1$, that
\[
I_n = \frac n {n+1} \int_0^1 x^{n-1} (1-x)^{n+1} \d x
\]
and hence that $I_n = \dfrac{n}{2(2n+1)} I_{n-1}\,.$


\item When $n$ is a
 positive integer, show that
\[
I_n = \frac{(n!)^2}{(2n+1)!}\,.
\]

\item
Use the substitution $x= \sin^2 \theta$ to show
that $I_{\frac12}= \frac \pi 8$, and evaluate $I_{\frac32}$.
\end{questionparts}
\end{question}

%%%%%%%%% Q3
\begin{question}
\begin{questionparts}
\item 
Given that the cubic equation
$x^3+3ax^2 + 3bx +c=0$ has three distinct real roots and $c<0$, 
show with the help of sketches that
either exactly one of the roots is positive 
or all three of the roots are positive.
 
\item 
Given that the     
equation $x^3 +3ax^2+3bx+c=0$ has three distinct real positive roots 
show that
\begin{equation*}
a^2>b>0, \ \ \ \ a<0, \ \ \ \ c<0\,.
\tag{$*$}
\end{equation*}

\noindent[{\bf Hint}: Consider the turning points.]  
\item Given that the 
equation $x^3 +3ax^2+3bx+c=0$ has three distinct real roots and that
\begin{equation*}
 ab<0, \ \ \ \ c>0\,,
\end{equation*}
determine, with the help of sketches,  the signs of the roots.
  
\item 
Show by means of an explicit example 
(giving values for $a$, $b$ and $c$)
that it is possible 
for  the conditions ($*$) to be satisfied even though
the corresponding cubic equation 
has only one real root.               

\end{questionparts}
\end{question}

%%%%%% Q4 
\begin{question}
The line passing through the point $(a,0)$ with gradient $b$ intersects 
the circle of unit radius centred at the origin at $P$ and $Q$, and $M$ is the
midpoint of
 the chord $PQ$.
Find the coordinates of $M$ in terms of $a$ and $b$.

\begin{questionparts}
 
\item Suppose $b$ is fixed and positive. As  
$a$ varies,  $M$  traces out a curve (the {\em locus} of $M$).
Show that 
$x=- by$
on this curve.
Given that  $a$ varies with $-1\le a \le 1$, 
 show that the locus is a line 
segment of length~$2b/(1+b^2)^\frac12$.
Give
a sketch 
showing the locus 
and the unit circle. 


\item Find the locus of $M$ in the following cases, giving in
each case its cartesian equation, describing it geometrically and sketching
it in relation to the unit circle:
\begin{itemize}

\item [({\bf a})] $a$ is fixed with $0<a<1$, and $b$ varies 
with  $-\infty <b< \infty$;

\item [({\bf b})] $ab=1$, and $b$  varies  with  
 $0<b\le1$.
\end{itemize}
\end{questionparts}
\end{question}

%%%%%%%%% Q5
\begin{question}
\begin{questionparts}
\item A function $\f(x)$ satisfies $\f(x) = \f(1-x)$ for all $x$.
Show, by differentiating with respect to $x$, that $\f'(\frac12) =0\,$.
If, in addition,  $\f(x) = \f(\frac1x)$ for all (non-zero) $x$, show that 
$\f'(-1)=0$ and that  $\f'(2)=0$.

\item  The function $\f$ is defined, for $x\ne0$ and $x\ne1$, by
\[
\f(x) = \frac {(x^2-x+1)^3}{(x^2-x)^2} \,. 
\]
Show that $\f(x)= \f(\frac 1 x)$ and $\f(x) = \f(1-x)$. 

Given that it has exactly three stationary points, 
sketch the 
curve  $y=\f(x)$. 

\item
Hence, or otherwise,
 find all the roots of the equation $\f(x) = \dfrac {27} 4\,$
and state the ranges of values of $x$ for which $\f(x) > \dfrac{27} 4\,$. 

 Find also all  the roots of the  equation $\f(x) = \dfrac{343}{36}\,$
and state the ranges of values of $x$ for which 
$\f(x) >   \dfrac{343}{36}$.

 \end{questionparts}
	\end{question}
	
%%%%%%%%% Q6
\begin{question}
In this question, the following theorem may be used.\newline 
{\sl Let $u_1$, $u_2$, $\ldots$ be a sequence of (real) numbers. 
If the sequence is bounded above (that is, $u_n\le b$ for all $n$, where $b$
is some  fixed number) and increasing 
(that is, $u_n\ge u_{n-1}$ for all $n$), then
the sequence tends to a limit (that is, converges).}

The sequence  $u_1$, $u_2$, $\ldots$ is defined by $u_1=1$ and 
\[
u_{n+1} = 1+\frac 1{u_n} \ \ \ \ \ \ \ \ \ \ (n\ge1)\,.
\tag{$*$}
\]


\begin{questionparts}
\item Show that, for $n\ge3$, 
\[
u_{n+2}-u_n = \frac{u_{n} - u_{n-2}}{(1+u_n)(1+u_{n-2})}
.
\]
\item Prove, by induction or otherwise, that $1\le u_n \le 2$ for all $n$.
\item 
Show that the sequence $u_1$, $u_3$, $u_5$, $\ldots$
tends to a limit, 
and that the sequence $u_2$, $u_4$, $u_6$, $\ldots$
tends to a limit. Find these limits and  deduce that the sequence 
 $u_1$, $u_2$, $u_3$, $\ldots\,$ tends to a limit.

Would this conclusion change if the sequence were 
defined by $(*)$ and $u_1=3$? 
\end{questionparts}
\end{question}
	
%%%%%%%%% Q7
\begin{question}
\begin{questionparts}
\item
Write down a solution of the equation
\[
x^2-2y^2 =1\,,
\tag{$*$}
\]
for which $x$ and $y$ are non-negative integers.

Show that, if  $x=p$, $y=q$ is a solution of ($*$),
then so also is $x=3p+4q$, $y=2p+3q$. Hence find two   
solutions of $(*)$ for which $x$ is a positive odd integer and $y$ is a 
positive even integer.
\item Show that, if $x$ is an odd integer and $y$ is an 
even integer, $(*)$ can be written in the form
\[
n^2 = \tfrac12 m(m+1)\,,
\]
where $m$ and $n$ are integers.


\item
The positive integers $a$, $b$ and $c$ satisfy
\[
b^3=c^4-a^2\,,
\]
where $b$ is a prime number. Express $a$ and $c^2$ in terms of $b$ in the 
two cases that arise.

Find a solution of $a^2+b^3=c^4$, where $a$, $b$ and $c$ are positive 
integers but  $b$ is not prime.
\end{questionparts}
\end{question}
\vspace{-0.9cm}	
%%%%%%%%% Q8
\begin{question}
The function $\f$ satisfies  $\f(x)>0$ for $x\ge0$ and  is strictly
decreasing (which means that 
$\f(b)<\f(a)$
for $b>a$).     
   
\begin{questionparts}
\item

For $t\ge0$, let  $A_0(t)$ be the area of the largest rectangle 
with sides parallel to the coordinate axes that can fit in the region
bounded by the curve $y=\f(x)$, 
the $y$-axis 
and the line $y=\f(t)$. Show that $A_0(t)$ 
can be written in the form
\[
A_0(t) =x_0\left( \f(x_0) -\f(t)\right),
\]
where $x_0$ satisfies $x_0 \f'(x_0) +\f(x_0) = \f(t)\,$. 

\item The function g is defined,
for $t>  0$,  by
\[
\g(t) =\frac 1t \int_0^t \f(x) \d x\,.
\]
Show that $t \g'(t) = \f(t) -\g(t)\,$.

Making use of a sketch show that, for $t>0$, 
\[
 \int_0^t \left( \f(x) - \f(t)\right) \d x > 
A_0(t)
\]
and deduce that $-t^2 \g'(t)> A_0(t)$.

\item In the case $\f(x)= \dfrac 1 {1+x}\,$, use the above to establish the inequality
\[
\ln \sqrt{1+t} > 1 - \frac 1 {\sqrt{1+t}}
\,,
\]
for $t>0$.
 
\end{questionparts}
\end{question}	
		

		
	
\newpage
\section*{Section B: \ \ \ Mechanics}


	
%%%%%%%%%% Q9
\begin{question}
The diagram shows three identical discs in equilibrium in 
a vertical plane. Two discs rest, not in contact with each other,
 on a horizontal surface
and the third disc rests on the other two. The angle at the upper
vertex of the triangle joining the centres of the discs is $2\theta$.

\begin{center}
\psset{xunit=0.7cm,yunit=0.7cm,algebraic=true,dimen=middle,dotstyle=o,dotsize=3pt 0,linewidth=0.3pt,arrowsize=3pt 2,arrowinset=0.25}
\begin{pspicture*}(-7,-0.42)(7,6.86)
\psline(-7,0)(7,0)
\pscircle(-3,2){1.4}
\pscircle(3,2){1.4}
\pscircle(0,4.64){1.4}
\psline(0,4.64)(-3,2)
\psline(0,4.64)(3,2)
\psline(0,4.64)(0,0.9)
\parametricplot{-1.5707963267948966}{-0.722030440522891}{1*cos(t)+0|1*sin(t)+4.64}
\rput[tl](0.16,4.25){$\theta$}
\end{pspicture*}
\end{center}

\noindent
The weight of each disc is $W$.
The coefficient of friction between a disc and the horizontal surface
is $\mu$ and the coefficient of friction between the discs is also $\mu$.

\begin{questionparts} 
\item Show that the normal reaction between the horizontal surface and 
a disc in contact with the surface is $\frac32 W\,$. 

\item Find the normal reaction between 
two discs in contact and show that the magnitude of the                          frictional force between two discs in contact is 
$\dfrac{W\sin\theta}{2(1+\cos\theta)}\,$.

\item Show that if
$\mu <2- \surd3\,$ there is no value of $\theta$ for which 
equilibrium is possible.
\end{questionparts}
	\end{question}
	
%%%%%%%%%% Q10 
\begin{question}	
A particle is projected 
at an angle of elevation $\alpha$ (where $\alpha>0$) from a point
$A$ on horizontal ground.
At a general
point in its trajectory the angle of elevation of the particle
from $A$ is $\theta$ and 
its direction of motion is at  an angle $\phi$ above the horizontal
(with $\phi\ge0$ for the first half of the trajectory and $\phi\le0$
for the second half).

Let $B$ denote the point on the trajectory at which $\theta = \frac12 \alpha$
and let $C$ denote the point on the trajectory at which
 $\phi = -\frac12\alpha$.

\begin{questionparts}
\item Show that, at a general point on the trajectory,  
$2\tan\theta = \tan \alpha + \tan\phi\,$.
\item Show that, if $B$ and $C$ are the same point, then 
$ \alpha =  60^\circ\,$.
 \item Given that 
 $\alpha < 60^\circ\,$,
determine whether the particle reaches the  point $B$ first or the
point  $C$ first.
\end{questionparts}
\end{question}

%%%%%%%%%% Q11

\begin{question}
Three identical particles lie, not touching one another, in a straight line
on a smooth horizontal surface. One particle is projected with speed
$u$ directly towards the other two which are at rest. The coefficient of 
restitution in all collisions is $e$, where $0<e<1\,$.

\begin{questionparts}
\item Show that, after the second collision, the speeds of the particles are
$\frac12u(1-e)$, $\frac14u (1-e^2)$ and $\frac14u(1+e)^2$. 
Deduce that there will be a third  collision         
whatever the value of~$e$.
\item Show that there will be a fourth collision if and only if 
$e$ is less than a particular value which you should determine. 
\end{questionparts} 
\end{question}
	

	
	\newpage
\section*{Section C: \ \ \ Probability and Statistics}


%%%%%%%%%% Q12
\begin{question}
The random variable $U$ has a Poisson distribution with parameter
$\lambda$. The random variables $X$ and $Y$ are defined as follows.
\begin{align*}
X&=
 \begin{cases} 
U & \text{ if $U$ is 1, 3, 5, 7, $\ldots\,$} \\
0 & \text{ otherwise}
\end{cases}
\\
Y&=
 \begin{cases} 
U & \text{ if $U$ is 2, 4, 6, 8, $\ldots\,$ } \\
0 & \text{ otherwise} 
\end{cases}
\end{align*}


\begin{questionparts}
\item Find $\E(X)$ and $\E(Y)$ in terms of $\lambda$, $\alpha$ and 
$\beta$, where
\[
\alpha = 1+\frac{\lambda^2}{2!}+\frac{\lambda^4}{4!} +\cdots\,
\text{ \ \ and \ \ }
\beta = \frac{\lambda}{1!} + \frac{\lambda^3}{3!} + \frac{\lambda^5}{5!}
+\cdots\,.
\]
\item
Show that 
\[
\var(X) = \frac{\lambda\alpha+\lambda^2\beta}{\alpha+\beta}
 - \frac{\lambda^2\alpha^2}{(\alpha+\beta)^2}
\]
and obtain the corresponding expression for $\var(Y)$. Are there any 
non-zero values of~$\lambda$ for which 
$ \var(X) + \var(Y) = \var(X+Y)\,$? 
\end{questionparts}
\end{question}

%%%%%%%%%% Q13
\begin{question}
A biased coin has probability $p$ of showing a head
and probability $q$ of showing a tail, where $p\ne0$, $q\ne0$
and $p\ne q$. When the coin is tossed repeatedly, runs occur. 
A {\em straight run} of length $n$ is a sequence of $n$ consecutive
heads or $n$ consecutive tails. An {\em alternating run} of length 
$n$ is a sequence of length $n$ alternating between heads and tails. An 
alternating run can start with either a head or a tail.

Let $S$ be the length of the longest straight run beginning with the
first toss and let $A$ be the  length of the longest 
alternating  run beginning with the
first toss.

\begin{questionparts}
\item Explain why  $\P(A=1)=p^2+q^2$ and find $\P(S=1)$. Show that
 $\P(S=1)<\P(A=1)$.
\item Show that
$\P(S=2)= \P(A=2)$
and determine the relationship between
$\P(S=3)$ and $ \P(A=3)$.

 \item Show that, for $n>1$, $\P(S=2n)>\P(A=2n)$ and determine
the corresponding relationship between $\P(S=2n+1)$ and $\P(A=2n+1)$.
[You are advised {\em not} to  use $p+q=1$ in this part.] 
\end{questionparts}
\end{question}

\end{document}
