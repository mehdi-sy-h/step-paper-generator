\documentclass[a4, 11pt]{report}


\pagestyle{myheadings}
\markboth{}{Paper III, 2008
\ \ \ \ \ 
\today 
}               

\RequirePackage{amssymb}
\RequirePackage{amsmath}
\RequirePackage{graphicx}
\RequirePackage{color}
\RequirePackage[flushleft]{paralist}[2013/06/09]



\RequirePackage{geometry}
\geometry{%
  a4paper,
  lmargin=2cm,
  rmargin=2.5cm,
  tmargin=3.5cm,
  bmargin=2.5cm,
  footskip=12pt,
  headheight=24pt}


\newcommand{\comment}[1]{{\bf Comment} {\it #1}}
%\renewcommand{\comment}[1]{}

\newcommand{\bluecomment}[1]{{\color{blue}#1}}
%\renewcommand{\comment}[1]{}
\newcommand{\redcomment}[1]{{\color{red}#1}}



\usepackage{epsfig}
\usepackage{pstricks-add}
\usepackage{tgheros} %% changes sans-serif font to TeX Gyre Heros (tex-gyre)
\renewcommand{\familydefault}{\sfdefault} %% changes font to sans-serif
%\usepackage{sfmath}  %%%% this makes equation sans-serif
%\input RexFigs


\setlength{\parskip}{10pt}
\setlength{\parindent}{0pt}

\newlength{\qspace}
\setlength{\qspace}{20pt}


\newcounter{qnumber}
\setcounter{qnumber}{0}

\newenvironment{question}%
 {\vspace{\qspace}
  \begin{enumerate}[\bfseries 1\quad][10]%
    \setcounter{enumi}{\value{qnumber}}%
    \item%
 }
{
  \end{enumerate}
  \filbreak
  \stepcounter{qnumber}
 }


\newenvironment{questionparts}[1][1]%
 {
  \begin{enumerate}[\bfseries (i)]%
    \setcounter{enumii}{#1}
    \addtocounter{enumii}{-1}
    \setlength{\itemsep}{5mm}
    \setlength{\parskip}{8pt}
 }
 {
  \end{enumerate}
 }



\DeclareMathOperator{\cosec}{cosec}
\DeclareMathOperator{\Var}{Var}

\def\d{{\mathrm d}}
\def\e{{\mathrm e}}
\def\g{{\mathrm g}}
\def\h{{\mathrm h}}
\def\f{{\mathrm f}}
\def\p{{\mathrm p}}
\def\s{{\mathrm s}}
\def\t{{\mathrm t}}


\def\A{{\mathrm A}}
\def\B{{\mathrm B}}
\def\E{{\mathrm E}}
\def\F{{\mathrm F}}
\def\G{{\mathrm G}}
\def\H{{\mathrm H}}
\def\P{{\mathrm P}}


\def\bb{\mathbf b}
\def \bc{\mathbf c}
\def\bx {\mathbf x}
\def\bn {\mathbf n}

\newcommand{\low}{^{\vphantom{()}}}
%%%%% to lower suffices: $X\low_1$ etc


\newcommand{\subone}{ {\vphantom{\dot A}1}}
\newcommand{\subtwo}{ {\vphantom{\dot A}2}}




\def\le{\leqslant}
\def\ge{\geqslant}
\def\arcosh{{\rm arcosh}\,}


\def\var{{\rm Var}\,}

\newcommand{\ds}{\displaystyle}
\newcommand{\ts}{\textstyle}
\def\half{{\textstyle \frac12}}
\def\l{\left(}
\def\r{\right)}



\begin{document}
\setcounter{page}{2}

 
\section*{Section A: \ \ \ Pure Mathematics}

%%%%%%%%%%Q1
\begin{question}
Find all values of $a$, $b$, $x$ and $y$ that satisfy the
simultaneous
equations
\begin{alignat*}{3}
a&+b &  &=1 &\\
ax&+by & &= \tfrac13& \\
ax^2&+by^2& &=\tfrac15& \\
ax^3 &+by^3& &=\tfrac17\,.&
\end{alignat*}


\noindent{\bf [} {\bf Hint}: you may wish to start by multiplying the second
equation by $x+y$. {\bf ]}
\end{question}

%%%%%%%%%%Q2
\begin{question}
Let $S_k(n) \equiv \sum\limits_{r=0}^n r^k\,$, where $k$ is a
positive
integer, so that
\[
S_1(n) \equiv  \tfrac12 n(n+1)
\text{ \ \ \ \ and \ \ \ \ }
S_2(n) \equiv \tfrac16 n(n+1)(2n+1)\,.
\]

\begin{questionparts}
\item
By considering $\sum\limits_{r=0}^n \left[ (r+1)^k-r^k\right]\, $,
show that 
\[
kS_{k-1}(n)=(n+1)^k -(n+1) -  
\binom{k}{2} S_{k-2}(n) 
- \binom {k}{3} S_{k-3}(n) - \cdots
- \binom{k}{k-1} S_{1}(n)
\;.
\tag{$*$}
\] 
Obtain simplified expressions for $S_3(n)$ and~$S_4(n)$.

\item
Explain, using $(*)$, why $S_k(n)$ is a polynomial of degree $k+1$ in
$n$.   Show that in this polynomial the constant term is zero and 
the sum of the coefficients is 1.
\end{questionparts}
\end{question}

%%%%%%%%% Q3
\begin{question}
The point $P(a\cos\theta\,,\, b\sin\theta)$, where $a>b>0$, lies
on the ellipse 
\[\dfrac {x^2}{a^2} + \dfrac {y^2}{b^2}=1\,.\]
The point $S(-ea\,,\,0)$, where $b^2=a^2(1-e^2)\,$, is a focus of
the ellipse. The point $N$ is the foot of the perpendicular from 
the origin, $O$, to the tangent to the ellipse at $P$. The lines
$SP$ and $ON$ intersect at $T$. Show that the $y$-coordinate of 
$T$ is 
\[\dfrac{b\sin\theta}{1+e\cos\theta}\,.\]

Show that $T$ lies on the circle with centre $S$ and radius $a$.



\end{question}

%%%%%% Q4 
\begin{question}
\begin{questionparts}
\item Show, with the aid of  a sketch, that $y>  \tanh (y/2)$ for $y>0$ 
and deduce
that 
\begin{equation}
\arcosh x > \dfrac{x-1}{\sqrt{x^2-1}} 
\text{  \ \ for \ \ } x>1.
\tag{$*$}
\end{equation}

\item By integrating $(*)$, show that $\arcosh x > 2
  \dfrac{{x-1}}{\sqrt{x^2-1}} $ for $x>1$.

\item Show that 
 $\arcosh x >3
  \dfrac{\sqrt{x^2-1}}{{x+2}} 
$ for $x>1$.

\end{questionparts}

\noindent[{\bf Note:} $\arcosh x $ is another notation for
  $\cosh^{-1}x$.\ ]

\end{question}

%%%%%%%%% Q5
\begin{question}
The functions ${\rm T}_n(x)$, for $n=0$, 1, 2, $\ldots\,$, satisfy
the recurrence relation
\[
{\rm T}_{n+1}(x) -2x {\rm T}_n(x) + {\rm T}_{n-1}(x) =0\,
\ \ \ \ \ \ \ (n\ge1).
\tag{$*$}
\]

Show by induction that 
\[
\left({\rm T}_n(x)\right)^2 - {\rm T}_{n-1}(x) {\rm T}_{n+1}(x) = \f(x)\,,
\]
where $\f(x) = \left({\rm T}_1(x)\right)^2 - {\rm T}_0(x){\rm T}_2(x)\,$.

In the case $\f(x)\equiv 0$, determine (with proof) an expression
for ${\rm T}_n(x)$ in terms of ${\rm T}_0(x)$ (assumed to be non-zero)
and  ${\rm r}(x)$, where 
${\rm r}(x) =   {\rm T}_1(x)/ {\rm T}_0(x)$. 
Find the two possible expressions for  ${\rm r}(x)$
 in terms of $x$.

%Conjecture (without proof) the general form of the solution of $(*)$.
	\end{question}
	
%%%%%%%%% Q6
\begin{question}
In this question, $p$ denotes $\dfrac{\d y}{\d x}\,$.
\begin{questionparts}
\item Given that 
\[
y=p^2 +2 xp\,,
\]
show by differentiating with respect to $x$ that 
\[
\frac{\d x}{\d p} = -2 -  \frac {2x} p .
\]
Hence show that $x = -\frac23p +Ap^{-2}\,,$ where $A$ is an arbitrary
 constant.

Find $y$ in terms of $x$ if $p=-3$ when $x=2$.

\item Given instead that 
\[ y=2xp +p \ln p\,,\]
and that $p=1$ when $x=-\frac14$, show that 
$x=-\frac12 \ln p - \frac14\,$ and find $y$ in terms of $x$.

\end{questionparts}
\end{question}
	
%%%%%%%%% Q7
\begin{question}
The  points $A$, $B$ and  $C$ 
 in the Argand diagram are the vertices of an equilateral triangle
 described
anticlockwise. 
Show that the complex numbers
$a$, $b$ and $c$ representing $A$, $B$ and $C$
satisfy  \[2c= (a+b) +\mathrm{i}\sqrt3(b-a).\]
Find a similar relation in the case that
$A$, $B$ and  $C$ 
  are the vertices of an equilateral triangle
 described
clockwise. 

\begin{questionparts}
\item The quadrilateral $DEFG$ lies in the Argand diagram. Show that
points $P$, $Q$, $R$ and $S$ can be chosen so that  
$PDE$, $QEF$, $RFG$ and $SGD$ are equilateral triangles and $PQRS$ is 
a parallelogram.

\item The triangle $LMN$ lies in the Argand diagram.
  Show that the centroids $U$, $V$ and $W$ of the
  equilateral
triangles drawn externally on the sides of  $LMN$ are the vertices
of an equilateral triangle.

\noindent
[{\bf Note:} The {\em centroid} of a triangle with vertices 
represented by the complex numbers $x$,~$y$ and~$z$ is the point
represented by $\frac13(x+y+z)\,$.]
\end{questionparts}
\end{question}
		
%%%%%%%%% Q8
\begin{question}
\begin{questionparts}
\item
The coefficients in the series
\[
S= \tfrac13 x + \tfrac 16 x^2 + \tfrac1{12} x^3 + \cdots +  a_rx^r +
\cdots
\]
satisfy a recurrence relation of the form $a_{r+1} + p a_r =0$. Write
down the value of $p$.

By considering $(1+px)S$, find an expression for the sum to infinity 
of $S$  (assuming that it exists). Find also
an expression for the sum of the first $n+1$ terms of $S$.

\item
The coefficients in the series 
\[
T=2 + 8x +18x^2+37 x^3 +\cdots + a_rx^r + \cdots
\]
satisfy a recurrence relation of the form $a_{r+2}+pa_{r+1} +qa_r=0$.
Find an expression for the sum to infinity of $T$ (assuming that it
exists). By expressing $T$ in partial fractions, or
otherwise,
find 
an expression for the sum of the first $n+1$ terms of $T$.
\end{questionparts}
\end{question}	
		

		
	
\newpage
\section*{Section B: \ \ \ Mechanics}


	
%%%%%%%%%% Q9
\begin{question}
A particle of mass $m$  is initially at rest 
on a rough horizontal surface. The particle experiences a force 
$mg\sin \pi t$, where $t$ is time, acting  in 
a fixed horizontal direction. 
The coefficient of  friction between the particle and the surface is $\mu$. 
Given that the particle starts to 
move first at $t=T_0$,  state the relation between
$T_0$ and $\mu$.

\begin{questionparts}
\item For $\mu = \mu_0$, 
the particle comes to rest for the first time at $t=1$.
Sketch the acceleration-time graph for $0\le t \le 1$. Show that
\[
1+\left(1-\mu_0^2\right)^{\frac12}
 -\mu_0\pi +\mu_0 \arcsin \mu_0 =0\,. 
\]

\item For $\mu=\mu_0$
sketch the acceleration-time graph for $0\le t\le 3$.
Describe the motion of the particle in this case and in the case $\mu=0$.
\end{questionparts}

\noindent[{\bf Note:} $\arcsin x$ is another notation for
  $\sin^{-1}x$.\ ]
	\end{question}
	
%%%%%%%%%% Q10 
\begin{question}	
A long string consists of $n$ short
light strings joined together, each of natural length
$\ell$ and modulus of elasticity $\lambda$. 
It hangs vertically at rest, suspended from one end.
Each of the short strings has a particle of mass $m$ attached to its lower
end.  The short strings are numbered $1$ to $n$, the $n$th short
string  being at the top. By considering the tension in the $r$th
short string, determine the length
of the long string. Find also the elastic energy stored in the long string.


A uniform heavy rope of mass $M$ and natural length $L_0$
has modulus of elasticity $\lambda$.
The rope hangs vertically at rest, suspended from one end. 
Show that the length, $L$,  of the rope  is given by
\[
L=L_0\biggl(1+ \frac{Mg}{2\lambda}\biggr),
\]
and find an expression in  terms of $L$, $L_0$ and $\lambda$ for the 
elastic energy stored in the rope.
\end{question}

%%%%%%%%%% Q11

\begin{question}
A circular wheel of radius $r$ has moment of inertia $I$ about
its
axle, which is fixed in a horizontal position. A light string is
wrapped around the circumference of the wheel and a particle of 
mass $m$ hangs from the free end. The system is released from rest
and the particle descends.
The string does not slip on the wheel.

As the particle descends, the wheel turns through
$n_1$ revolutions, and the string then detaches from the wheel. At this
moment, the angular speed of the wheel is $\omega_0$.  The wheel then
turns through a further
 $n_2$ revolutions, in time $T$, before coming to rest.
The couple on the wheel due to resistance is constant.

Show that
\[ \frac12 \omega_0 T = 2 \pi n_2\]
and
\[
I =\dfrac {mgrn_1T^2 -4\pi mr^2n_2^2}{4\pi n_2(n_1+n_2)}\;.
\]
\end{question}
	

	
	\newpage
\section*{Section C: \ \ \ Probability and Statistics}


%%%%%%%%%% Q12
\begin{question}
Let $X$ be a random variable with a  Laplace distribution, so that  
its probability 
density function is given by
\[
\f(x) = \frac12 \e^{-\vert x \vert }\;, 
\text{ \ \ \ $-\infty < x < \infty $\ }.
\tag{$*$}
\]
Sketch $\f(x)$. Show that its moment generating function 
${\rm M}_X(\theta)$
  is given by 
${\rm M}_X(\theta)
= (1-\theta^2)^{-1}$ and hence find the variance of $X$.

A frog
is jumping up and down, attempting to land on the same 
spot each time. In fact, in each of $n$ successive jumps
he always lands on a fixed
straight line but when he lands
 from  the $i$th jump ($i=1\,,2\,,\ldots\,,n$) 
his displacement from the point from which he jumped is $X_i\,$cm, 
where $X_i$ has the distribution $(*)$. His displacement from his starting
point after
$n$~jumps is $Y\,$cm (so that 
$Y=\sum\limits_{i=1}^n X_i$).
Each jump
is independent of the others.

Obtain the moment generating function  for $Y/ \sqrt {2n}$
 and, by considering
its logarithm, show that this moment generating function tends to 
$\exp(\frac12\theta^2)$ as $n\to\infty$.

Given that 
$\exp(\frac12\theta^2)$ is the moment generating function
 of the standard Normal
random variable, estimate the least number of jumps such that there
is a $5\%$ chance that the frog lands 25~cm or more from his starting
point.
\end{question}

%%%%%%%%%% Q13
\begin{question}
A box contains $n$ 
pieces of string, each of which has two
ends. I select two string ends at random and tie them together. This
creates
either a ring (if the two ends are from the same string) or a longer
piece of string. I repeat the process of tying together string ends
chosen
at random until there are none left.

Find the expected number of rings created at the first step and
hence obtain an expression for the expected number of rings created
by the end of the process. Find also an expression for the variance
of the number of rings created.

Given that $\ln 20 \approx 3$ and that
$1+ \frac12 + \cdots + \frac 1n \approx \ln n$ for large $n$,
 determine approximately the expected
number of rings created in the case $n=40\,000$. 
\end{question}

\end{document}
