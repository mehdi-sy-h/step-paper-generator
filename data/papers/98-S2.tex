\documentclass[a4, 11pt]{report}


\pagestyle{myheadings}
\markboth{}{Paper II, 1998
\ \ \ \ \ 
\today 
}               

\RequirePackage{amssymb}
\RequirePackage{amsmath}
\RequirePackage{graphicx}
\RequirePackage{color}
\RequirePackage[flushleft]{paralist}[2013/06/09]



\RequirePackage{geometry}
\geometry{%
  a4paper,
  lmargin=2cm,
  rmargin=2.5cm,
  tmargin=3.5cm,
  bmargin=2.5cm,
  footskip=12pt,
  headheight=24pt}


\newcommand{\comment}[1]{{\bf Comment} {\it #1}}
%\renewcommand{\comment}[1]{}

\newcommand{\bluecomment}[1]{{\color{blue}#1}}
%\renewcommand{\comment}[1]{}
\newcommand{\redcomment}[1]{{\color{red}#1}}



\usepackage{epsfig}
\usepackage{pstricks-add}
\usepackage{tgheros} %% changes sans-serif font to TeX Gyre Heros (tex-gyre)
\renewcommand{\familydefault}{\sfdefault} %% changes font to sans-serif
%\usepackage{sfmath}  %%%% this makes equation sans-serif
%\input RexFigs


\setlength{\parskip}{10pt}
\setlength{\parindent}{0pt}

\newlength{\qspace}
\setlength{\qspace}{20pt}


\newcounter{qnumber}
\setcounter{qnumber}{0}

\newenvironment{question}%
 {\vspace{\qspace}
  \begin{enumerate}[\bfseries 1\quad][10]%
    \setcounter{enumi}{\value{qnumber}}%
    \item%
 }
{
  \end{enumerate}
  \filbreak
  \stepcounter{qnumber}
 }


\newenvironment{questionparts}[1][1]%
 {
  \begin{enumerate}[\bfseries (i)]%
    \setcounter{enumii}{#1}
    \addtocounter{enumii}{-1}
    \setlength{\itemsep}{5mm}
    \setlength{\parskip}{8pt}
 }
 {
  \end{enumerate}
 }



\DeclareMathOperator{\cosec}{cosec}
\DeclareMathOperator{\Var}{Var}

\def\d{{\rm d}}
\def\e{{\rm e}}
\def\g{{\rm g}}
\def\h{{\rm h}}
\def\f{{\rm f}}
\def\p{{\rm p}}
\def\s{{\rm s}}
\def\t{{\rm t}}


\def\A{{\rm A}}
\def\B{{\rm B}}
\def\E{{\rm E}}
\def\F{{\rm F}}
\def\G{{\rm G}}
\def\H{{\rm H}}
\def\P{{\rm P}}


\def\bb{\mathbf b}
\def \bc{\mathbf c}
\def\bx {\mathbf x}
\def\bn {\mathbf n}

\newcommand{\low}{^{\vphantom{()}}}
%%%%% to lower suffices: $X\low_1$ etc


\newcommand{\subone}{ {\vphantom{\dot A}1}}
\newcommand{\subtwo}{ {\vphantom{\dot A}2}}




\def\le{\leqslant}
\def\ge{\geqslant}


\def\var{{\rm Var}\,}

\newcommand{\ds}{\displaystyle}
\newcommand{\ts}{\textstyle}




\begin{document}
\setcounter{page}{2}

 
\section*{Section A: \ \ \ Pure Mathematics}

%%%%%%%%%%Q1
\begin{question}
Show that, if $n$ is an integer such that
$$(n-3)^3+n^3=(n+3)^3,\eqno{(*)}$$
then $n$ is even and $n^2$ is a factor of $54$. Deduce that
there is no integer $n$ which satisfies the equation $(*)$.

Show that, if $n$ is an integer such that

$$(n-6)^3+n^3=(n+6)^3,\eqno{(**)}$$
then $n$ is even. Deduce that there
is no integer $n$ which satisfies the equation $(**)$.
\end{question}

%%%%%%%%%%Q2
\begin{question}
Use the first four terms of the
binomial expansion of  $(1-1/50)^{1/2}$, writing $1/50 = 2/100$ to simplify 
the calculation,  to derive the approximation
$\sqrt 2 \approx 1.414214$.

Calculate similarly an approximation to
the cube root of 2 to six decimal places by considering
$(1+N/125)^a$, where $a$ and $N$ are suitable numbers.

\noindent
[You need not justify the accuracy of your approximations.]
\end{question}

%%%%%%%%% Q3
\begin{question}
Show that the sum $S_N$ of the first $N$ terms of the series
$${1\over1\cdot2\cdot3}+{3\over2\cdot3\cdot4}+{5\over3\cdot4\cdot5}+\cdots
+{2n-1\over n(n+1)(n+2)}+\cdots$$
is
$${1\over2}\left({3\over2}+{1\over N+1}-{5\over N+2}\right).$$
What is the limit of $S_N$ as $N\to\infty$?

The numbers $a_n$ are such that 
$${a_n\over a_{n-1}}={(n-1)(2n-1)\over (n+2)(2n-3)}.$$
Find an expression for $a_n/a_1$  and hence, or otherwise,
evaluate $\sum\limits_{n=1}^\infty a_n$ when $\displaystyle a_1=\frac{2}{9}\;$.
\end{question}

%%%%%% Q4 
\begin{question}
The integral $I_n$ is defined by
$$I_n=\int_0^\pi(\pi/2-x)\sin(nx+x/2)\,{\rm cosec}\,(x/2)\,\d x,$$
where $n$ is a positive integer.
Evaluate $I_n-I_{n-1}$,
and hence evaluate  $I_n$ leaving your
answer in the form of a sum.
	\end{question}

%%%%%%%%% Q5
\begin{question}
Define the modulus of a complex number $z$ and give the geometric
interpretation
of $\vert\,z_1-z_2\,\vert$ for two complex numbers $z_1$ and $z_2$. On the 
basis of this interpretation establish the inequality
$$\vert\,z_1+z_2\,\vert\le \vert\,z_1\,\vert+\vert\,z_2\,\vert.$$

Use this result to prove, by induction, the corresponding inequality for
$\vert\,z_1+\cdots+z_n\,\vert$.

The complex numbers $a_1,\,a_2,\,\ldots,\,a_n$ 
satisfy $|a_i|\le 3$ ($i=1, 2,  \ldots ,  n$). Prove that the equation
$$a_1z+a_2z^2\cdots +a_nz^n=1$$
has no solution $z$ with $\vert\,z\,\vert\le 1/4$.
	\end{question}
	
%%%%%%%%% Q6
\begin{question}
Two curves are given parametrically by
\[
x_{1}=(\theta+\sin\theta),\qquad y_{1}=(1+\cos\theta),\tag{1}
\]and
\[
x_{2}=(\theta-\sin\theta),\qquad y_{1}=-(1+\cos\theta),\tag{2}
\]
Find the gradients of the tangents to the curves at the points where 
$\theta= \pi/2$ and $\theta=3\pi/2$. 

Sketch, using the same axes, the
curves
for $0\le\theta \le 2\pi$.

Find the equation of the normal to the
curve (1) at the point with parameter $\theta$. Show that this normal is
a tangent to the curve (2).
\end{question}
	
%%%%%%%%% Q7
\begin{question}
\begin{eqnarray*}
{\rm f}(x)&=& \tan x-x,\\
{\rm g}(x)&=& 2-2\cos x-x\sin x,\\
{\rm h}(x)&=& 2x+x\cos 2x-\tfrac{3}{2}\sin 2x,\\
{\rm F}(x)&=& {x(\cos x)^{1/3}\over\sin x}.
\end{eqnarray*}
\vspace{1mm}
\begin{questionparts}
\item By considering $\f(0)$ and $\f'(x)$, show that $\f(x)>0$
for $0<x<\tfrac{1}{2}\pi$.

\item Show similarly that $\g(x)>0$ for $0<x<\tfrac{1}{2}\pi$.

\item Show that $\h(x)>0$ for $0<x<\tfrac{1}{4}\pi$, and hence that
\[x(\sin^2x+3\cos^2x)-3\sin x\cos x>0\] for $0<x<\tfrac{1}{4}\pi$.

\item By considering $\displaystyle {{\rm F}'(x)\over {\rm F}(x)}$,
 show that ${\rm F}'(x)<0$ for $0<x<\tfrac{1}{4}\pi$.
\end{questionparts}
\end{question}
		
%%%%%%%%% Q8
\begin{question}	
Points $\mathbf{A},\mathbf{B},\mathbf{C}$ in three dimensions have coordinate vectors
$\mathbf{a},\mathbf{b},\mathbf{c}$, respectively. Show that the lines joining the vertices of the
triangle $ABC$ to the mid-points of the opposite sides meet at a point $R$.

$P$ is a point which is {\bf not} in the plane $ABC$.
Lines are drawn through the mid-points of $BC$, $CA$ and $AB$ parallel to
$PA$, $PB$ and $PC$ respectively. Write down the vector equations of the
lines and show by inspection that these lines
meet at a common point $Q$.

Prove further that the line $PQ$ meets the plane $ABC$ at $R$.
\end{question}	
		

		
	
\newpage
\section*{Section B: \ \ \ Mechanics}


	
%%%%%%%%%% Q9
\begin{question}
A light smoothly jointed planar framework in the form of a regular hexagon
$ABCDEF$ is
suspended smoothly from $A$ and a weight 1kg is suspended from $C$. 
The framework is kept rigid by three light
rods $BD$, $BE$ and $BF$.
What is the 
direction and magnitude of the supporting force which must be exerted on the
framework at $A$?

Indicate on a labelled
diagram which rods are in thrust (compression) and which are in tension.

Find the magnitude of the force in $BE$.
	\end{question}
	
%%%%%%%%%% Q10
\begin{question}	
A wedge of mass $M$ rests on a smooth horizontal surface. The face of the
wedge is a smooth plane inclined at an angle $\alpha$ to the horizontal.
A particle of mass $m$ slides down the face of the wedge, starting from rest.
At a later time $t$, the speed $V$ of the wedge, the speed $v$ of the particle
and the angle $\beta$ of the velocity of the particle below the horizontal
are as shown in the diagram.

\begin{center}
\psset{xunit=0.55cm,yunit=0.55cm,algebraic=true,dotstyle=o,dotsize=3pt 0,linewidth=0.5pt,arrowsize=3pt 2,arrowinset=0.25} \begin{pspicture*}(-3.96,-2.9)(9.6,6.78) \psline(0,0)(0,6) \psline(8,0)(0,6) \psline(8,0)(0,0) \psline{->}(1.13,2.31)(-1.98,2.31) \rput[tl](6.8,0.52){$\alpha$} \rput[tl](3.05,2.02){$v$} \psline(3.31,1.38)(4.47,-2.55) \rput[tl](4.08,-0.14){$\beta$} \rput[tl](-2.83,2.54){$V$} \psline{->}(2.41,4.53)(3.03,2.24) \begin{scriptsize} \psdots[dotsize=10pt 0,dotstyle=*](2.41,4.53) \end{scriptsize} \end{pspicture*}
\par\end{center}

\noindent Let $y$ be the vertical distance
descended by the particle. Derive the following results, stating in \textbf{(ii)}
and \textbf{(iii)} the mechanical principles you use:

\begin{questionparts}
\item $V\sin\alpha=v\sin(\beta-\alpha)$;

\item $\tan\beta=(1+m/M)\tan\alpha$;

\item $2gy=v^2(M+m\cos^2\beta)/M$.
\end{questionparts}

Write down a differential equation for $y$ and hence show that
$$y={gMt^2\sin^2\beta \over 2\,(M+m\cos^2\beta)}.$$
\end{question}

%%%%%%%%%% Q11

\begin{question}
A fielder, who is perfectly placed to catch
a ball struck by the batsman in a game of cricket, watches the
ball in flight. 
Assuming that the ball is struck at the fielder's eye level and is caught
just in front of her eye,
show that $\frac{ {\rm d}}{{\rm d t}} (\tan\theta ) $ is constant,
where $\theta$ is  the 
angle between the horizontal and the fielder's line of sight.

In order to catch the next  ball, which is also struck towards her but
at a different velocity, the fielder runs at constant
speed $v$ towards the batsman. Assuming that the ground is horizontal,
show that the fielder should choose $v$ so that
 $\frac{ {\rm d}}{{\rm d t}} (\tan\theta ) $
remains constant.
\end{question}
	

	
	\newpage
\section*{Section C: \ \ \ Probability and Statistics}


%%%%%%%%%% Q12
\begin{question}
The diagnostic test AL has a probability 0.9 of giving a positive result when
applied to a person suffering from the rare disease mathematitis. It also
has a probability 1/11 of giving a false positive result when applied to a 
non-sufferer. It is known that only $1\%$ of the population suffer from the 
disease. Given that the test AL is positive when applied to Frankie, who is
chosen at random from the population, what is
the probability that Frankie is a sufferer?

In an attempt to identify sufferers more accurately, a second diagnostic test
STEP is given to those for whom the test AL gave a positive result. The 
probablility of STEP giving a positive result on a sufferer is
0.9, and the probability that it gives a false positive 
result on a non-sufferer is $p$.
Half of those for whom AL was positive and on whom STEP then also gives
a positive result are sufferers. Find $p$.

\end{question}

%%%%%%%%%% Q13
\begin{question}
A random variable $X$ has the probability density function
\[
\mathrm{f}(x)=\begin{cases}
\lambda\mathrm{e}^{-\lambda x} & x\geqslant0,\\
0 & x<0.
\end{cases}
\]
Show that
$${\rm  P}(X>s+t\,\vert X>t) = {\rm P}(X>s).$$

The time it takes an assistant to serve a customer in a certain shop
is a random variable with the above distribution
and the times for different customers are independent. If, when I enter the
shop, the only two assistants are
serving one customer each, what is the probability that these customers
are both still being served at time $t$ after I arrive?

One of the assistants finishes serving his customer
and immediately starts serving me.
 What is the probability that I am still
being served when the other  customer has finished being served?
\end{question}

%%%%%%%%%% Q14
\begin{question}
The staff of Catastrophe College are paid a salary of $A$ pounds per year.
With a Teaching Assessment Exercise impending it is decided to try to lower
the student failure rate  by offering each lecturer an alternative
salary of $B/(1+X)$ pounds, where $X$ is the number of his or her 
students who fail
the end of year examination. Dr Doom has $N$ students,
each with independent probability $p$ of failure. Show that
she should accept the new salary scheme if
$$A(N+1)p<B(1-(1-p)^{N+1}).$$

Under what circumstances could $X$, for Dr Doom,
be modelled by a Poisson random variable?
What would Dr Doom's expected salary be under this model?
\end{question}
	
\end{document}
