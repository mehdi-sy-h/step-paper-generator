\documentclass[a4, 11pt]{report}


\pagestyle{myheadings}
\markboth{}{Paper II, 2012
\ \ \ \ \ 
\today 
}               

\RequirePackage{amssymb}
\RequirePackage{amsmath}
\RequirePackage{graphicx}
\RequirePackage{color}
\RequirePackage[flushleft]{paralist}[2013/06/09]



\RequirePackage{geometry}
\geometry{%
  a4paper,
  lmargin=2cm,
  rmargin=2.5cm,
  tmargin=3.5cm,
  bmargin=2.5cm,
  footskip=12pt,
  headheight=24pt}


\newcommand{\comment}[1]{{\bf Comment} {\it #1}}
%\renewcommand{\comment}[1]{}

\newcommand{\bluecomment}[1]{{\color{blue}#1}}
%\renewcommand{\comment}[1]{}
\newcommand{\redcomment}[1]{{\color{red}#1}}



\usepackage{epsfig}
\usepackage{pstricks-add}
\usepackage{tgheros} %% changes sans-serif font to TeX Gyre Heros (tex-gyre)
\renewcommand{\familydefault}{\sfdefault} %% changes font to sans-serif
%\usepackage{sfmath}  %%%% this makes equation sans-serif
%\input RexFigs


\setlength{\parskip}{10pt}
\setlength{\parindent}{0pt}

\newlength{\qspace}
\setlength{\qspace}{20pt}


\newcounter{qnumber}
\setcounter{qnumber}{0}

\newenvironment{question}%
 {\vspace{\qspace}
  \begin{enumerate}[\bfseries 1\quad][10]%
    \setcounter{enumi}{\value{qnumber}}%
    \item%
 }
{
  \end{enumerate}
  \filbreak
  \stepcounter{qnumber}
 }


\newenvironment{questionparts}[1][1]%
 {
  \begin{enumerate}[\bfseries (i)]%
    \setcounter{enumii}{#1}
    \addtocounter{enumii}{-1}
    \setlength{\itemsep}{5mm}
    \setlength{\parskip}{8pt}
 }
 {
  \end{enumerate}
 }



\DeclareMathOperator{\cosec}{cosec}
\DeclareMathOperator{\Var}{Var}

\def\d{{\mathrm d}}
\def\e{{\mathrm e}}
\def\g{{\mathrm g}}
\def\h{{\mathrm h}}
\def\f{{\mathrm f}}
\def\p{{\mathrm p}}
\def\q{{\mathrm q}}
\def\s{{\mathrm s}}
\def\t{{\mathrm t}}


\def\A{{\mathrm A}}
\def\B{{\mathrm B}}
\def\E{{\mathrm E}}
\def\F{{\mathrm F}}
\def\G{{\mathrm G}}
\def\H{{\mathrm H}}
\def\P{{\mathrm P}}


\def\bb{\mathbf b}
\def \bc{\mathbf c}
\def\bx {\mathbf x}
\def\bn {\mathbf n}

\newcommand{\low}{^{\vphantom{()}}}
%%%%% to lower suffices: $X\low_1$ etc


\newcommand{\subone}{ {\vphantom{\dot A}1}}
\newcommand{\subtwo}{ {\vphantom{\dot A}2}}




\def\le{\leqslant}
\def\ge{\geqslant}
\def\arcosh{{\rm arcosh}\,}


\def\var{{\rm Var}\,}

\newcommand{\ds}{\displaystyle}
\newcommand{\ts}{\textstyle}
\def\half{{\textstyle \frac12}}
\def\l{\left(}
\def\r{\right)}



\begin{document}
\setcounter{page}{2}

 
\section*{Section A: \ \ \ Pure Mathematics}

%%%%%%%%%%Q1
\begin{question}
Write down the general term in the expansion in powers of 
$x$ of 
$(1-x^6)^{-2}\,$.
\begin{questionparts}
\item
Find      the coefficient of $x^{24}$ in 
the expansion in powers of $x$ of 
\[
(1-x^6)^{-2} (1-x^3)^{-1}\,.\]



Obtain also, and simplify, formulae for the 
coefficient of $x^n$ in the different
cases that arise.

\item Show that the coefficient of $x^{24}$ 
in the expansion in powers of $x$ of
\[ 
(1-x^6)^{-2} (1-x^3)^{-1} (1-x)^{-1}\,\] is $55$,
and find the coefficients of 
$x^{25}$ and $x^{66}$.




\end{questionparts}
\end{question}

%%%%%%%%%%Q2
\begin{question}
If $\p(x)$ and $\q(x)$ are polynomials of degree $m$ and $n$,
respectively, what is the degree of $\p(\q(x))$?

\begin{questionparts}
\item
The polynomial $\p(x)$ satisfies
\[
\p(\p(\p(x)))- 3 \p(x)= -2x\,
\]
for all $x$.
Explain carefully why $\p(x)$ must be of degree 1, and 
find all polynomials that satisfy this equation.

\item 
Find all polynomials that satisfy
\[
2\p(\p(x)) +3 [\p(x)]^2 -4\p(x) =x^4
\]   
for all $x$. 
\end{questionparts}
\end{question}

%%%%%%%%% Q3
\begin{question}
Show that, for any function f (for which the integrals exist),
\[
\int_0^\infty \f\big(x+\sqrt{1+x^2}\big) \,\d x = \frac12 \int_1^\infty
\left(1+\frac 1 {t^2}\right) \f(t)\, \d t \,.
\] 
Hence 
evaluate
\[
\int_0^\infty \frac1 {2x^2 +1 + 2 x\sqrt{x^2+1} \ } \, \, \d x
\,,
\]
and, using the substitution $x=\tan\theta$,
\[
\int_0^{\frac12\pi} \frac{1}{(1+\sin\theta)^3}\,\d \theta
\,.
\] 
\end{question}

%%%%%% Q4 
\begin{question}
In this question, you may assume that the infinite series 
\[
\ln(1+x) = x-\frac{x^2}2 + \frac{x^3}{3} -\frac {x^4}4 +\cdots
+ (-1)^{n+1} \frac {x^n}{n} + \cdots
\]
is valid for $\vert x \vert <1$.

\begin{questionparts}
\item
Let $n$ be an integer greater than 1. Show that, for any positive
integer $k$,
\[
\frac1{(k+1)n^{k+1}} 
<
\frac1{kn^{k}}\,.
\]
Hence show that $\displaystyle \ln\! \left(1+\frac1n\right) <\frac1n\,$. Deduce
that
\[
\left(1+\frac1n\right)^{\!n}<\e\,.
\]
\item
Show, using  an expansion in powers of $\dfrac1y\,$,  
that 
$ \displaystyle
 \ln \! \left(\frac{2y+1}{2y-1}\right) 
> \frac 1y
%= \sum _{r=0}^\infty \frac 1{(2r+1)(2y)^{2r}}\,.
$ 
for $y>\frac12$. 

Deduce that, for any positive integer $n$, 
\[
\e < \left(1+\frac1n\right)^{\! n+\frac12}\,.
\] 
\item Use parts (i) and (ii) to show that as $n\to\infty$ 
\[
 \left(1+\frac1n\right)^{\!n}  
\to \e\,.
\]
\end{questionparts}
\end{question}

%%%%%%%%% Q5
\begin{question}
\begin{questionparts}
\item Sketch the curve $y=\f(x)$, where
\[
\text{\hphantom{($x\e a\pm1$)}\hspace{2cm}}
\f(x) = \frac 1 {(x-a)^2 -1} 
\hspace{2cm}(x\ne a\pm1),
\]
and $a$ is a constant.

\item  The function $\g(x)$ is defined by
\[
\text{\hphantom{($x\e a\pm1, \ x\ne b\pm1$)}\hspace{1mm}}
\g(x) = \frac 1 {\big( (x-a)^2-1 \big) \big( (x-b)^2 -1\big)}
\hspace{1cm}(x\ne a\pm1, \ x\ne b\pm1),
\]
where $a$ and $b$ are constants, and $b>a$.
Sketch the curves $y=\g(x)$
in the two cases $b>a+2$ and $b=a+2$, finding the  
 values of $x$ at the stationary points.  

\end{questionparts}
	\end{question}
	
%%%%%%%%% Q6
\begin{question}
A cyclic quadrilateral $ABCD$ has sides $AB$, $BC$, $CD$ and $DA$
of lengths $a$, $b$, $c$ and $d$, respectively. The area of the 
quadrilateral is $Q$, and   angle $DAB$ is $\theta$.

Find an expression for $\cos\theta$ in terms of $a$, $b$, $c$ and $d$,
and an expression 
for $\sin\theta$ in terms of  $a$, $b$, $c$, $d$ and $Q$.
Hence show  that
\[
16Q^2 = 4(ad+bc)^2 - (a^2+d^2-b^2-c^2)^2
\,,
\]
and deduce that 
\[
Q^2 = (s-a)(s-b)(s-c)(s-d)\,,
\]
where  $s= \frac12(a+b+c+d)$.

Deduce a formula for the area of a triangle with sides of length
$a$, $b$ and $c$.
\end{question}
	
%%%%%%%%% Q7
\begin{question}
Three distinct points, $X_1$, $X_2$ and $X_3$,
with position vectors ${\bf x}_1$, ${\bf x}_2$ and ${\bf x}_3$ 
respectively, lie on a   circle of radius 1 with 
its centre at the origin $O$.
The point  $G$ 
has position vector $\frac13({\bf x}_1+{\bf x}_2+{\bf x}_3)$.
        The line through $X_1$ and $G$ meets the circle again at 
the point $Y_1$ and the points
$Y_2$ and $Y_3$ are defined correspondingly.

Given that $\overrightarrow{GY_1} =-\lambda_1 \overrightarrow{GX_1}$,
where $\lambda_1$  is a positive scalar, show that
\[
\overrightarrow{OY_1}= \tfrac13 \big(
(1-2\lambda_1){\bf x}_1 +(1+\lambda_1)({\bf x}_2+{\bf x}_3)\big)
\]
and hence that
\[
\lambda_1 = \frac
{3-\alpha-\beta-\gamma} {3+\alpha -2\beta-2\gamma}
\,,\]
where $\alpha = {\bf x}_2 \,.\, {\bf x}_3$,
$\beta = {\bf x}_3\,.\, {\bf x}_1$ and
$\gamma = {\bf x}_1\,.\, {\bf x}_2$.

Deduce that $\dfrac {GX_1}{GY_1} + \dfrac {GX_2}{GY_2} +
\dfrac {GX_3}{GY_3} =3
\,$.
\end{question}
		
%%%%%%%%% Q8
\begin{question}
The positive numbers $\alpha$, $\beta$ and $q$ satisfy
$\beta-\alpha >q$. Show that
\[
\frac{\alpha^2+\beta^2 -q^2}{\alpha\beta}-2>0\,.
\]

The sequence $u_0$, $u_1$, $\ldots$ is defined by $u_0=\alpha$,
$u_1=\beta$ and 
\[
\ \ \ \ \ \ \ \ \ \ \ \ \ \ \ 
u_{n+1} = \frac {u_{n}^2 -q^2}{u_{n-1}}
\ \ \ \ \ \ \ \ \ \ \ (n\ge1),
\]
where $\alpha$, $\beta$ and $q$ are given positive numbers (and $\alpha$ and $\beta$
are such that
no term in the sequence is zero). 
Prove that $u_n(u_n+u_{n+2}) = u_{n+1}(u_{n-1}+u_{n+1})\,$.
Prove also that                  
\[
u_{n+1} -pu_n + u_{n-1}=0
\]
for some number $p$ which you should express in terms of $\alpha$, $\beta$ and $q$. 

Hence, or otherwise, show that if $\beta> \alpha+q$, the sequence is strictly increasing
(that is, $u_{n+1}-u_n > 0$ for all $n$). 
Comment on the case $\beta =\alpha +q$.
\end{question}	
		

		
	
\newpage
\section*{Section B: \ \ \ Mechanics}


	
%%%%%%%%%% Q9
\begin{question}
A tennis ball
is projected from a height of 
$2h$ above horizontal ground with speed $u$ and  at an angle of $\alpha$
below the horizontal. It travels in a plane 
perpendicular to a vertical net of height $h$ which 
is a horizontal distance of $a$ from the point of 
projection. Given that the ball passes over the net,
show that 
\[
\frac 1{u^2}< 
 \frac 
{2(h-a\tan\alpha)}
{ga^2\sec^2\alpha}\,.
\]  

The ball lands before it has travelled a horizontal distance
of $b$ from the point of projection. Show that
\[ 
\sqrt{u^2\sin^2\alpha +4gh \ } < \frac{bg}{u\cos\alpha} + u \sin\alpha\,.
\]

Hence show that 
\[
\tan\alpha < \frac{h(b^2-2a^2)}{ab(b-a)}\,.
\]
	\end{question}
	
%%%%%%%%%% Q10 
\begin{question}	
A hollow circular cylinder of internal radius $r$
is held fixed with its axis 
horizontal. A uniform rod of length $2a$ (where $a<r$)
 rests in equilibrium
inside the cylinder inclined at an angle of 
$\theta$ to the horizontal, where $\theta\ne0$.
The vertical plane containing the rod is perpendicular to the 
axis of the cylinder. The coefficient of friction
between the cylinder and each end of the rod is $\mu$, where $\mu>0$. 

Show that, if the rod is on the point of slipping, then
the normal reactions $R_1$ and $R_2$ of the lower and higher
ends of the
rod, respectively, on the cylinder are related by
\[
\mu(R_1+R_2) = (R_1-R_2)\tan\phi
\]
where $\phi$ is the angle between the rod and the radius to
an  end of the rod.

Show further that
\[
\tan\theta = \frac {\mu r^2}{r^2 -a^2(1+\mu^2)}\,.
\]

Deduce that $\lambda <\phi $, where $\tan\lambda =\mu$.
\end{question}

%%%%%%%%%% Q11

\begin{question}
A small block of mass $km$
is initially at rest on a smooth horizontal surface.
Particles $P_1$, $P_2$, $P_3$, $\ldots$ are fired, in order, along the 
surface from a fixed point
towards the
 block. 
The mass of the $i$th particle is $im$ ($i$ = 1, 2, $\ldots$)
and the speed
at which it is fired is $u/i\,$.
 Each particle that collides with the block is
embedded in it. 
Show that, if the $n$th particle collides with the block, the 
 speed of the block after the collision is
\[
\frac{2nu}{2k +n(n+1)}\,.
\]

In the case $2k = N(N+1)$, where $N$ is a positive integer,
determine the number of collisions that occur.
Show that the total kinetic energy lost in all the collisions
is
\[
\tfrac12 mu^2\bigg( \sum_{n=2}^{N+1} \frac 1 n \bigg)\,.
\]
\end{question}
	

	
	\newpage
\section*{Section C: \ \ \ Probability and Statistics}


%%%%%%%%%% Q12
\begin{question}
A modern villa has complicated lighting controls. In order 
for the light in the swimming pool to be on, a particular switch in the 
hallway must be on and a particular switch in the kitchen must be on.
There are four identical switches in the hallway and
four identical switches in the kitchen. Guests
cannot tell whether the switches are on or off, or what they 
control.

Each Monday morning a guest arrives, and 
the switches in the hallway are either all  on or all off.
The  probability that they are all on is $p$
and the  probability that they are all off is $1-p$.
The switches in the kitchen are each on or off, independently,
 with probability
$\frac12$. 

\begin{questionparts}
\item
On the first Monday, a guest 
presses one switch in the hallway at random and one 
switch in the kitchen at random. Find the probability that
the swimming pool light is on at the end of this process.
Show that the probability that the guest has pressed the 
 swimming pool light switch
in the hallway, given that the light is on at the end of the process,
is~$\displaystyle \frac{1-p}{1+2p}$.

\item On each of seven Mondays,  guests go through the above process
independently of each other, and each time the swimming pool light
is found to be on at the end of the process.
 Given that the most likely number of days on which
 the swimming pool light switch in the hallway was pressed is 3, 
show that $\frac14 <p< \frac{5}{14}$.
\end{questionparts}
\end{question}

%%%%%%%%%% Q13
\begin{question}
In this question, you may assume that 
$\displaystyle \int_0^\infty \!\!\! 
\e^{-x^2/2} \d x = \sqrt{\tfrac12 \pi}\,$.

The number of supermarkets situated in any given region
can be modelled by a Poisson random variable, where the 
mean is $k$ times the area of the given region.
Find the probability that
there are no supermarkets  within a circle of radius $y$.

The random variable $Y$ denotes the distance
between a randomly chosen point in the region and
the nearest  supermarket. Write down
$\P(Y<y)$ and hence show that the probability density function
of $Y$ is  $\displaystyle 2\pi y k \e^{-\pi k y^2}$ for $y\ge0$.

Find $\E(Y)$ and show that $\var(Y) = \dfrac{4-\pi}{4\pi k}$.  
\end{question}

\end{document}
