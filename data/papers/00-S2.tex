\documentclass[a4, 11pt]{report}


\pagestyle{myheadings}
\markboth{}{Paper II, 2000
\ \ \ \ \ 
\today 
}               

\RequirePackage{amssymb}
\RequirePackage{amsmath}
\RequirePackage{graphicx}
\RequirePackage{color}
\RequirePackage[flushleft]{paralist}[2013/06/09]



\RequirePackage{geometry}
\geometry{%
  a4paper,
  lmargin=2cm,
  rmargin=2.5cm,
  tmargin=3.5cm,
  bmargin=2.5cm,
  footskip=12pt,
  headheight=24pt}


\newcommand{\comment}[1]{{\bf Comment} {\it #1}}
%\renewcommand{\comment}[1]{}

\newcommand{\bluecomment}[1]{{\color{blue}#1}}
%\renewcommand{\comment}[1]{}
\newcommand{\redcomment}[1]{{\color{red}#1}}



\usepackage{epsfig}
\usepackage{pstricks-add}
\usepackage{tgheros} %% changes sans-serif font to TeX Gyre Heros (tex-gyre)
\renewcommand{\familydefault}{\sfdefault} %% changes font to sans-serif
%\usepackage{sfmath}  %%%% this makes equation sans-serif
%\input RexFigs


\setlength{\parskip}{10pt}
\setlength{\parindent}{0pt}

\newlength{\qspace}
\setlength{\qspace}{20pt}


\newcounter{qnumber}
\setcounter{qnumber}{0}

\newenvironment{question}%
 {\vspace{\qspace}
  \begin{enumerate}[\bfseries 1\quad][10]%
    \setcounter{enumi}{\value{qnumber}}%
    \item%
 }
{
  \end{enumerate}
  \filbreak
  \stepcounter{qnumber}
 }


\newenvironment{questionparts}[1][1]%
 {
  \begin{enumerate}[\bfseries (i)]%
    \setcounter{enumii}{#1}
    \addtocounter{enumii}{-1}
    \setlength{\itemsep}{5mm}
    \setlength{\parskip}{8pt}
 }
 {
  \end{enumerate}
 }



\DeclareMathOperator{\cosec}{cosec}
\DeclareMathOperator{\Var}{Var}

\def\d{{\mathrm d}}
\def\e{{\mathrm e}}
\def\g{{\mathrm g}}
\def\h{{\mathrm h}}
\def\f{{\mathrm f}}
\def\p{{\mathrm p}}
\def\s{{\mathrm s}}
\def\t{{\mathrm t}}


\def\A{{\mathrm A}}
\def\B{{\mathrm B}}
\def\E{{\mathrm E}}
\def\F{{\mathrm F}}
\def\G{{\mathrm G}}
\def\H{{\mathrm H}}
\def\P{{\mathrm P}}


\def\bb{\mathbf b}
\def \bc{\mathbf c}
\def\bx {\mathbf x}
\def\bn {\mathbf n}

\newcommand{\low}{^{\vphantom{()}}}
%%%%% to lower suffices: $X\low_1$ etc


\newcommand{\subone}{ {\vphantom{\dot A}1}}
\newcommand{\subtwo}{ {\vphantom{\dot A}2}}




\def\le{\leqslant}
\def\ge{\geqslant}


\def\var{{\rm Var}\,}

\newcommand{\ds}{\displaystyle}
\newcommand{\ts}{\textstyle}
\def\half{{\textstyle \frac12}}




\begin{document}
\setcounter{page}{2}

 
\section*{Section A: \ \ \ Pure Mathematics}

%%%%%%%%%%Q1
\begin{question}
 A number of the form $1/N$, where $N$ is  an integer greater
than 1, is called a {\it unit fraction}.

Noting that 
\[
 \frac1 2 =\frac13 + \frac16\\\ 
\mbox{ \ and \ }
\frac13 
= \frac14
+ \frac1{12},
\]
guess a general result of the form
$$
\frac1N
=\frac1a +\frac1b
\eqno(*)
$$
and hence
 prove that any unit fraction can be expressed as the sum of 
two distinct unit fractions.

By writing $(*)$ in the form 
\[
 (a-N)(b-N)=N^2
\]
and by considering the factors of $N^2$, show that if $N$ is
prime, then there is only one way of expressing $1/N$ as the
sum of two distinct unit fractions.



Prove similarly that any fraction of the form $2/N$, where $N$ is
prime number greater than 2, 
can be expressed uniquely as the sum of two distinct unit
fractions. 
\end{question}

%%%%%%%%%%Q2
\begin{question}
 Prove that if ${(x-a)^{2}}$ is a factor of the polynomial
$\p(x)$, then $\p'(a)=0$.
Prove a corresponding result if $(x-a)^4$ is a factor of $\p(x).$

Given that the polynomial 
$$
x^6+4x^5-5x^4-40x^3-40x^2+32x+k
$$
has a factor of the form ${(x-a)}^4$, find $k$.
\end{question}

%%%%%%%%% Q3
\begin{question}
The lengths of the sides $BC$, $CA$, $AB$ of the triangle
$ABC$  are denoted by $a$, $b$, $c$, respectively. Given that 
$$ 
b = 8+{\epsilon}_1, \,
c=3+{\epsilon}_2,\,
A=\tfrac{1}{3}\pi + {\epsilon}_3,
$$
where ${\epsilon}_1$, ${\epsilon}_2$, and $ {\epsilon}_3$ are small, show that
$a \approx 7 + {\eta}$, where 
${\eta}= {\left(13 \, {{\epsilon}_1}-2\,{\epsilon}_2
+ 24{\sqrt 3} \;{{\epsilon}_3}\right)}/14$.

Given now that 
$$
{\vert {\epsilon}_1} \vert 
\le 2 \times 10^{-3}, \ \ \
{\vert {\epsilon}_2} \vert \le 4\cdot 9\times 10^{-2}, \ \ \  
{\vert {\epsilon}_3} \vert \le  \sqrt3 \times 10^{-3},
$$
find the range of possible values of ${\eta}$.
\end{question}

%%%%%% Q4 
\begin{question}
 Prove that
\[
(\cos\theta +\mathrm{i}\sin\theta)
(\cos\phi +\mathrm{i}\sin\phi)
=
\cos(\theta+\phi) +\mathrm{i}\sin(\theta+\phi)
\]
and that, for every positive integer $n$,
$$
{(\cos {\theta} + \mathrm{i}\sin {\theta})}^n
= \cos{n{\theta}} + \mathrm{i}\sin{n{\theta}}.
$$


By considering 
$(5-\mathrm{i})^2(1+\mathrm{i})$,
or otherwise,
prove that
\[
\arctan\left(\frac{7}{17}\right)+2\arctan\left(\frac{1}{5}\right)=\frac{\pi}{4}\,.
\]
Prove also that 
\[
3\arctan\left(\frac{1}{4}\right)+\arctan\left(\frac{1}{20}\right)+\arctan\left(\frac{1}{1985}\right)=\frac{\pi}{4}\,.
\]
[Note that $\arctan\theta$ is another notation for $\tan^{-1}\theta$.]

\end{question}

%%%%%%%%% Q5
\begin{question}
It is required to approximate a given function $\f(x)$,
over the interval $0 \le x \le 1$,
by the linear function $\lambda x$, where $\lambda$ is chosen
to minimise
\[
\int_0^1 \big(\f(x)-\lambda x \big)^{\!2} \,\d x .
\] 
Show that
\[
\lambda   =  3  \int_0^1 x\f(x)\,\d x.
\]
The residual error, $R$, of this approximation process is
such that 
 \[
R^2 =  \int_0^1 \big(\f(x)-\lambda x \big)^{\!2}\,\d x.
\]
Show that
\[
R^2 =  \int_0^1 \big(\f(x)\big)^{\!2}\,\d x -\tfrac{1}{3}
\lambda ^2.
\]
Given now that 
$\f(x)= \sin (\pi x/n)$, 
show that \textbf{(i)} for large 
$n$, $\lambda \approx \pi/n$ and \textbf{(ii)}
$\lim_{n \to \infty}R = 0.$

Explain why, prior to any calculation, these results are to be
expected. 

[You may assume that, when $\theta$ is small,
$\sin \theta
\approx \theta-\frac{1}{6}\theta^3$ and 
$\cos \theta \approx 1 - \frac{1}{2}\theta^2.$]

	\end{question}
	
%%%%%%%%% Q6
\begin{question}
Show that
\[
\sin\theta = \frac {2t}{1+t^2}, \ \ \
\cos\theta = \frac{1-t^2}{1+t^2}, \ \ \
\frac{1+\cos\theta}{\sin\theta}  = \tan (\tfrac{1}{2}\pi-\tfrac{1}{2}\theta),
\]
where $t =\tan\frac{1}{2}\theta$.


Use the substitution $t =\tan\frac{1}{2}\theta$ to show that, for $0<\alpha<\frac{1}{2}\pi$,
\[
\int_0^{\frac{1}{2}\pi}
{1 \over {1 + \cos\alpha \sin \theta}} \,\d\theta  
=\frac{\alpha}{\sin\alpha}\,,
\]
and deduce a similar result for
\[
\int_0^{\frac{1}{2}\pi}
{1 \over {1 + \sin\alpha \cos \theta}} \,\d\theta  \,.
\]

%$$ 
\end{question}
	
%%%%%%%%% Q7
\begin{question}
The  line $l$ has vector equation ${\bf r} = \lambda {\bf s}$,
where 
\[
{\bf s} = (\cos\theta+\sqrt3\,) \; {\bf i} +(\surd2\;\sin\theta)\;{\bf j}
+(\cos\theta-\sqrt3\,)\;{\bf k} 
\]
and $\lambda$ is a scalar parameter. Find an expression for the 
angle between $l$ and the line 
\mbox{${\bf r} = \mu(a\, {\bf i} + b\,{\bf j} +c\,  {\bf k})$}.
Show that there is a line $m$ through the origin
such that, whatever the value of 
 $\theta$, the acute angle between $l$ and $m$ is $\pi/6$.


A plane has  equation  $x-z=4\sqrt3$. The line $l$ meets this plane 
at $P$. Show that, as $\theta$ varies, $P$ describes a circle,
with its centre on $m$. Find the radius of this circle.
\end{question}
		
%%%%%%%%% Q8
\begin{question}	




\begin{questionparts}
\item
Let $y$ be the solution of the differential equation
\[
\frac{\d y}{\d x} +  4x\e^{-x^2} {(y+3)}^{\frac12} = 0 \qquad (x \ge 0),
\]
that satisfies the condition $y=6$
when $x=0$.
Find $y$ in terms of $x$ and show that  
$y\to1$ 
as $x \to \infty$.

\item
Let $y$ be any solution of the differential equation
\[
\frac{\d y}{\d x}  -x\e^{6 x^2} (y+3)^{1-k} = 0 \qquad (x \ge 0).
\]
%that satisfies the condition $y=6$
%when $x=0$.
Find a value of  $k$ such that,
as $x \to \infty$,
$\e^{-3x^2}y$ 
tends to a finite non-zero  limit, which you should determine.
\end{questionparts}

\noindent
[The approximations, valid for small $\theta$, $\sin\theta \approx \theta$
and $\cos\theta \approx 1-{\textstyle\frac12}\,\theta^2$ may be assumed.]
\end{question}	
		

		
	
\newpage
\section*{Section B: \ \ \ Mechanics}


	
%%%%%%%%%% Q9
\begin{question}
In an aerobatics display, Jane  and Karen jump 
from a great height and 
go through a period of free fall before opening their parachutes.
While in free fall at speed  $v$, Jane experiences  air resistance
$kv$ per unit mass but  Karen, who spread-eagles, 
experiences air resistance \mbox{$kv + (2k^2/g)v^2$} per unit mass.
Show that Jane's speed can never reach $g/k$. Obtain the corresponding
result for Karen.

Jane opens her parachute when  her speed is 
$g/(3{k})$. 
Show that she has then been in free fall
for time 
$k^{-1}\ln (3/2)$. 
 
Karen also opens her parachute  when  her speed is 
$g/(3{k})$. Find the time she has then been in free fall.
	\end{question}
	
%%%%%%%%%% Q10
\begin{question}	
A long light 
inextensible string passes over a fixed smooth light pulley.
 A particle of mass 4~kg is attached to one end $A$ of this string
and the 
other end is attached to a second smooth light pulley.
A long light inextensible string $BC$ passes over the second pulley
and has a particle 
of mass 2 kg attached at $B$ and a particle of
mass of 1 kg  attached at $C$.  The
system is held in equilibrium in a vertical plane. 
The  string $BC$ is then released from rest.
Find the accelerations of the two moving particles.

After $T$ seconds, the end $A$ is released so that all three particles
are now moving in a vertical plane.
Find the accelerations of $A$, $B$ and $C$ in this second phase of
the motion.
Find also, in terms of $g$ and $T$, the speed of $A$ when $B$ has
moved through a total distance of $0.6gT^{2}$~metres. 
\end{question}

%%%%%%%%%% Q11

\begin{question}
The string $AP$ has a natural length of $1\!\cdot5\!$ metres and modulus of
elasticity equal to $5g$ newtons. The end $A$
is attached to the ceiling of a room of height $2\!\cdot\!5$ metres and 
a particle of mass \mbox{$0\!\cdot\!5$ kg}
is attached to the end $P$.  The  end  $P$  is 
released from  rest at a point $0\!\cdot\!5$ metres above the floor and vertically
below $A$.  Show that the string becomes slack, but that $P$ does not
reach the ceiling.

Show also that while the string is in tension, $P$ executes simple 
harmonic motion,
and that the time in seconds that elapses from the instant when $P$ is released
to the instant when $P$ first returns to its original position is
$$
\left(\frac8{3g}\right)^{\!\frac12}+
\left(\frac3 {5g}\right)^{\!\frac12}
{\Big(\pi - \arccos (3/7)\Big)}.
$$  

\noindent
[Note that $\arccos x$ is another notation for $\cos^{-1} x$.]
\end{question}
	

	
	\newpage
\section*{Section C: \ \ \ Probability and Statistics}


%%%%%%%%%% Q12
\begin{question}
{\it Tabulated values of ${\Phi}(\cdot)$,
the cumulative distribution function of a standard normal variable,
should not be used in this question.} 

 Henry the  commuter lives in Cambridge and his
working day starts at his office in London at 0900. He
 catches the 0715
train to King's Cross with probability $p$, or the 0720 to 
Liverpool Street with probability $1-p$.
 Measured in minutes, journey times for the first
train are $N(55,25)$ and for the second are $N(65,16)$. Journey times
from King's Cross and Liverpool Street to his office are $N(30,144)$ and
$N(25,9)$, respectively. Show that Henry is more likely
to be late for work if he catches the first train.
\par
Henry makes $M$ journeys, where $M$ is large.
Writing $A$ for $1-{\Phi}(20/13)$  and $B$ for $1-{\Phi}(2)$,  
find, in terms of $A$, $B$, $M$ and $p$,
the expected number, $L$,
of times that Henry will be late and show that for all possible
values of $p$,
$$
BM \le L \le AM.
$$ 

Henry noted that in 3/5 of the occasions when he was late,
he had caught the King's Cross train. Obtain an estimate of $p$
in terms of $A$ and $B$.

[A random variable is said to be $N\left({{\mu}, {\sigma}^2}\right)$
if it has a normal distribution with mean ${\mu}$ 
and variance ${\sigma}^2$.]

\end{question}

%%%%%%%%%% Q13
\begin{question}
A group of biologists attempts to estimate the magnitude, $N$, 
of an island population of voles ({\it Microtus agrestis}).   
Accordingly, the biologists  capture a random sample of 200 voles, mark them
and release them. A second random sample of 200 voles is then taken
of which 11 are found to be marked.  Show that the probability, $p_N$, of
this occurrence is given by
$$
p_N = k{{{\big((N-200)!\big)}^2}
\over
{N!(N-389)!}},
$$
where $k$ is independent of $N$.

The biologists  then estimate $N$ by calculating the value of
$N$ for which $p_N$ is a maximum. Find this estimate.

All unmarked voles in the second sample  
are marked and then the entire sample is released.
Subsequently a third random sample of 200
voles is taken. Write down the probability
that this sample contains exactly $j$ marked voles, leaving your 
answer in terms of binomial coefficients.

Deduce that 
$$
      \sum_{j=0}^{200}{389 \choose j}{3247 \choose {200-j}}
= {3636 \choose 200}.
$$
\end{question}

%%%%%%%%%% Q14
\begin{question}
The random variables $X_1$, $X_2$, $\ldots$ , $X_{2n+1}$ are 
independently and uniformly distributed on the interval
$0 \le x \le 1$.  The random variable $Y$ is defined to be the
median  of  $X_1$, $X_2$, $\ldots$ , $X_{2n+1}$. 
Given that the probability density function of $Y$ is $\g(y)$, where
\[
\mathrm{g}(y)=\begin{cases}
ky^{n}(1-y)^{n} & \mbox{ if }0\leqslant y\leqslant1\\
0 & \mbox{ otherwise}
\end{cases}
\]
use the result
$$
\int_0^1 {y^{r}}{{(1-y)}^{s}}\,\d y = 
\frac{r!s!}{(r+s+1)!}
$$
to show that $k={(2n+1)!}/{{(n!)}^2}$, and evaluate
$\E(Y)$ and ${\rm Var}\,(Y)$.
Hence show that, 
for any given positive number $d$, the inequality 
$$
{\P\left({\vert {Y - 1/2} \vert} < {d/{\sqrt {n}}} \right)} <
{\P\left({\vert {{\bar X} - 1/2} \vert} < {d/{\sqrt {n}}} \right)}
$$
holds provided $n$ is large enough, where
${\bar X}$ is the mean of  $X_1$, $X_2$, $\ldots$ , $X_{2n+1}$. 

[You may assume that $Y$ and $\bar X$ are normally distributed 
for large $n$.]
\end{question}
	
\end{document}
