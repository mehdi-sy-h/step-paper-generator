
\documentclass[a4, 11pt]{report}


\pagestyle{myheadings}
\markboth{}{Paper I, 1987
\ \ \ \ \ }               

\RequirePackage{amssymb}
\RequirePackage{amsmath}
\RequirePackage{graphicx}
\RequirePackage{color}
\RequirePackage[flushleft]{paralist}[2013/06/09]



\RequirePackage{geometry}
\geometry{%
  a4paper,
  lmargin=2cm,
  rmargin=2.5cm,
  tmargin=3.5cm,
  bmargin=2.5cm,
  footskip=12pt,
  headheight=24pt}


\newcommand{\comment}[1]{{\bf Comment} {\it #1}}
%\renewcommand{\comment}[1]{}

\newcommand{\bluecomment}[1]{{\color{blue}#1}}
%\renewcommand{\comment}[1]{}
\newcommand{\redcomment}[1]{{\color{red}#1}}



\usepackage{epsfig}
\usepackage{pstricks-add}
\usepackage{tgheros} %% changes sans-serif font to TeX Gyre Heros (tex-gyre)
\renewcommand{\familydefault}{\sfdefault} %% changes font to sans-serif
%\usepackage{sfmath}  %%%% this makes equation sans-serif
%\input RexFigs


\setlength{\parskip}{10pt}
\setlength{\parindent}{0pt}

\newlength{\qspace}
\setlength{\qspace}{20pt}


\newcounter{qnumber}
\setcounter{qnumber}{0}

\newenvironment{question}%
 {\vspace{\qspace}
  \begin{enumerate}[\bfseries 1\quad][10]%
    \setcounter{enumi}{\value{qnumber}}%
    \item%
 }
{
  \end{enumerate}
  \filbreak
  \stepcounter{qnumber}
 }


\newenvironment{questionparts}[1][1]%
 {
  \begin{enumerate}[\bfseries (i)]%
    \setcounter{enumii}{#1}
    \addtocounter{enumii}{-1}
    \setlength{\itemsep}{5mm}
    \setlength{\parskip}{8pt}
 }
 {
  \end{enumerate}
 }



\DeclareMathOperator{\cosec}{cosec}
\DeclareMathOperator{\Var}{Var}

\def\d{{\rm d}}
\def\e{{\rm e}}
\def\g{{\rm g}}
\def\h{{\rm h}}
\def\f{{\rm f}}
\def\p{{\rm p}}
\def\s{{\rm s}}
\def\t{{\rm t}}


\def\A{{\rm A}}
\def\B{{\rm B}}
\def\E{{\rm E}}
\def\F{{\rm F}}
\def\G{{\rm G}}
\def\H{{\rm H}}
\def\P{{\rm P}}


\def\bb{\mathbf b}
\def \bc{\mathbf c}
\def\bx {\mathbf x}
\def\bn {\mathbf n}

\newcommand{\low}{^{\vphantom{()}}}
%%%%% to lower suffices: $X\low_1$ etc


\newcommand{\subone}{ {\vphantom{\dot A}1}}
\newcommand{\subtwo}{ {\vphantom{\dot A}2}}




\def\le{\leqslant}
\def\ge{\geqslant}


\def\var{{\rm Var}\,}

\newcommand{\ds}{\displaystyle}
\newcommand{\ts}{\textstyle}




\begin{document}
\setcounter{page}{2}

 
\section*{Section A: \ \ \ Pure Mathematics}

%%%%%%%%%%Q1
\begin{question}
Find the stationary points of the function $\mathrm{f}$ given by
\[
\mathrm{f}(x)=\mathrm{e}^{ax}\cos bx,\mbox{ }(a>0,b>0).
\]
Show that the values of $\mathrm{f}$ at the stationary points with
$x>0$ form a geometric progression with common ratio $-\mathrm{e}^{a\pi/b}$. 


Give a rough sketch of the graph of $\mathrm{f}$. 
\end{question}

%%%%%%%%%%Q2
\begin{question}
$\ $

\vspace{-0.5cm}
\noindent \begin{center}
\newrgbcolor{aqaqaq}{0.5 0.5 0.5} \psset{xunit=1.0cm,yunit=1.0cm,algebraic=true,dotstyle=o,dotsize=3pt 0,linewidth=0.5pt,arrowsize=3pt 2,arrowinset=0.25} \begin{pspicture*}(-0.45,1.84)(8.6,7.17) \pspolygon[linecolor=white,fillcolor=black,fillstyle=solid,opacity=0.05](1.82,3.17)(1.91,3.21)(2.04,3.27)(2.16,3.32)(2.28,3.36)(2.45,3.42)(2.58,3.45)(2.75,3.48)(2.93,3.51)(3.06,3.52)(3.24,3.53)(3.33,3.53)(3.23,3.54)(3.13,3.57)(2.99,3.62)(2.88,3.67)(2.76,3.75)(2.66,3.84)(2.56,3.95)(2.47,4.08)(2.41,4.19)(2.35,4.35)(2.38,4.16)(2.38,4.02)(2.36,3.92)(2.33,3.8)(2.3,3.71)(2.24,3.58)(2.18,3.49)(2.1,3.38)(2.02,3.3)(1.93,3.23) \pspolygon[linecolor=white,fillcolor=black,fillstyle=solid,opacity=0.05](4.51,4.24)(4.49,4.1)(4.48,3.99)(4.49,3.88)(4.5,3.76)(4.53,3.65)(4.58,3.52)(4.64,3.4)(4.7,3.3)(4.78,3.2)(4.85,3.12)(5.05,2.96)(4.83,3.1)(4.71,3.17)(4.51,3.26)(4.34,3.33)(4.19,3.39)(4.03,3.43)(3.89,3.46)(3.76,3.49)(3.67,3.5)(3.62,3.51)(3.4,3.52)(3.52,3.52)(3.68,3.54)(3.82,3.58)(3.95,3.64)(4.08,3.71)(4.18,3.79)(4.28,3.88)(4.38,4)(4.45,4.12) \pspolygon[linecolor=white,fillcolor=black,fillstyle=solid,opacity=0.05](0.77,5.07)(0.94,5.17)(1.14,5.27)(1.3,5.34)(1.47,5.41)(1.59,5.46)(1.74,5.51)(1.86,5.55)(2.02,5.6)(2.2,5.64)(2.35,5.68)(2.55,5.72)(2.7,5.74)(2.84,5.76)(2.97,5.77)(3.13,5.79)(3.27,5.8)(3.44,5.8)(3.34,5.79)(3.2,5.77)(3.09,5.74)(2.97,5.69)(2.87,5.64)(2.76,5.56)(2.66,5.48)(2.58,5.39)(2.5,5.29)(2.43,5.17)(2.37,5.03)(2.34,4.9)(2.32,4.77)(2.31,4.6)(2.34,4.41)(2.27,4.59)(2.21,4.7)(2.14,4.8)(2.04,4.9)(1.93,5)(1.84,5.05)(1.75,5.1)(1.62,5.15)(1.51,5.18)(1.37,5.2)(1.25,5.2)(1.16,5.2)(1.02,5.17)(0.89,5.13) \pspolygon[linecolor=white,fillcolor=black,fillstyle=solid,opacity=0.05](6.31,5.02)(6.1,5.15)(5.92,5.24)(5.77,5.31)(5.61,5.38)(5.39,5.46)(5.12,5.56)(4.88,5.62)(4.64,5.68)(4.49,5.71)(4.24,5.75)(4.02,5.78)(3.8,5.79)(3.6,5.8)(3.51,5.79)(3.67,5.77)(3.79,5.74)(3.9,5.7)(4.03,5.64)(4.13,5.57)(4.24,5.48)(4.36,5.34)(4.46,5.17)(4.52,5.05)(4.56,4.9)(4.58,4.75)(4.59,4.59)(4.57,4.43)(4.63,4.57)(4.71,4.7)(4.78,4.79)(4.9,4.9)(5.03,5)(5.16,5.08)(5.3,5.14)(5.43,5.17)(5.57,5.19)(5.69,5.2)(5.82,5.19)(5.97,5.17)(6.15,5.11) \parametricplot{0.0}{3.141592653589793}{1*5.5*cos(t)+0*5.5*sin(t)+3.5|0*5.5*cos(t)+1*5.5*sin(t)+0.3} \parametricplot{0.07153894626263729}{3.2131315998524306}{1*3.08*cos(t)+0*3.08*sin(t)+3.27|0*3.08*cos(t)+1*3.08*sin(t)+0.45} \pscircle(1.29,4.12){1.09} \pscircle(3.45,4.66){1.14} \pscircle(5.69,3.99){1.21} \psline(1.29,4.12)(2.35,4.38) \psline(3.45,4.66)(3.46,5.8) \psline(5.69,3.99)(4.93,3.04) \psline{->}(2.14,6.66)(2.16,5.1) \psline{->}(4.72,6.7)(4.72,5.14) \psline{->}(2.56,2.5)(2.56,3.74) \psline{->}(4.4,2.42)(4.4,3.62) \rput[tl](1.9,7.1){$X/n$} \rput[tl](4.5,7.1){$X/n$} \rput[tl](2.34,2.43){$Y/n$} \rput[tl](4.16,2.41){$Y/n$} \rput[tl](3.7,5.2){$r$} \rput[tl](5.54,3.53){$r$} \rput[tl](1.74,4.1){$r$} \rput[tl](7.55,4.43){$R+r$} \rput[tl](5.95,2.41){$R-r$}  \end{pspicture*}
\par\end{center}


The region $A$ between concentric circles of radii $R+r$, $R-r$
contains $n$ circles of radius $r$. Each circle of radius $r$ touches
both of the larger circles as well as its two neighbours of radius
$r$, as shown in the figure. Find the relationship which must hold
between $n,R$ and $r$. 


Show that $Y$, the total area of $A$ outside the circle of radius
$r$ and adjacent to the circle of radius $R-r$, is given by 
\[
Y=nr\sqrt{R^{2}-r^{2}}-\pi(R-r)^{2}-n\pi r^{2}\left(\frac{1}{2}-\frac{1}{n}\right).
\]
Find similar expressions for $X$, the total area of $A$ outside
the circles of radius $r$ and adjacent to the circle of radius $R+r$,
and for $Z$, the total area inside the circle of radius $r$. 


What value does $(X+Y)/Z$ approach when $n$ becomes large?
\end{question}

%%%%%%%%% Q3
\begin{question}
By substituting $y(x)=xv(x)$ in the differential equation 
\[
x^{3}\frac{\mathrm{d}v}{\mathrm{d}x}+x^{2}v=\frac{1+x^{2}v^{2}}{\left(1+x^{2}\right)v},
\]
or otherwise, find the solution $v(x)$ that satisfies $v=1$ when
$x=1$. 


What value does this solution approach when $x$ becomes large?
\end{question}


%%%%%% Q4 

\begin{question}$\ $
Show that the sum of the infinite series 
\[
\log_{2}\mathrm{e}-\log_{4}\mathrm{e}+\log_{16}\mathrm{e}-\ldots+(-1)^{n}\log\low_{2^{2^{n}}}\mathrm{e}+\ldots
\]
is 
\[
\frac{1}{\ln(2\sqrt{2})}.
\]
{[}$\log_{a}b=c$ is equivalent to $a^{c}=b$.{]} 

\end{question}


%%%%%%%%% Q5
\begin{question}
	Using the substitution $x=\alpha\cos^{2}\theta+\beta\sin^{2}\theta,$
	show that, if $\alpha<\beta$, 
	\[
	\int_{\alpha}^{\beta}\frac{1}{\sqrt{(x-\alpha)(\beta-x)}}\,\mathrm{d}x=\pi.
	\]
	What is the value of the above integral if $\alpha>\beta$?


	Show also that, if $0<\alpha<\beta$, 
	\[
	\int_{\alpha}^{\beta}\frac{1}{x\sqrt{(x-\alpha)(\beta-x)}}\,\mathrm{d}x=\frac{\pi}{\sqrt{\alpha\beta}}.
	\]

	\end{question}
	
	%%%%%%%%% Q6
	\begin{question}
Let $y=\mathrm{f}(x)$, $(0\leqslant x\leqslant a)$, be a continuous
curve lying in the first quadrant and passing through the origin.
Suppose that, for each non-negative value of $y$ with $0\leqslant y\leqslant\mathrm{f}(a)$,
there is \textit{exactly} one value of $x$ such that $\mathrm{f}(x)=y$;
thus we may write $x=\mathrm{g}(y)$, for a suitable \phantom{} function $\mathrm{g}.$ 


For $0\leqslant s\leqslant a,$ $0\leqslant t\leqslant\mathrm{f}(a)$,
define 
\[
\mathrm{F}(s)=\int_{0}^{s}\mathrm{f}(x)\,\mathrm{d}x,\qquad\mathrm{G}(t)=\int_{0}^{t}\mathrm{g}(y)\,\mathrm{d}y.
\]
By a geometrical argument, show that 
\[
\mathrm{F}(s)+\mathrm{G}(t)\geqslant st.\tag{\ensuremath{*}}
\]
When does equality occur in $(*)$?


Suppose that $y=\sin x$ and that the ranges of $x,y,s,t$ are restricted
to $0\leqslant x\leqslant s\leqslant\frac{1}{2}\pi,$ $0\leqslant y\leqslant t\leqslant1$.
By considering $s$ such that the equality holds in $(*)$, show that
\[
\int_{0}^{t}\sin^{-1}y\,\mathrm{d}y=t\sin^{-1}t-\left(1-\cos(\sin^{-1}t)\right).
\]
Check this result by differentiating both sides with respect to $t$. 

	 \end{question}
	 
	 %%%%%%%%% Q7
\begin{question}
Sum each of the series
\[
\sin\left(\frac{2\pi}{23}\right)+\sin\left(\frac{6\pi}{23}\right)+\sin\left(\frac{10\pi}{23}\right)+\cdots+\sin\left(\frac{38\pi}{23}\right)+\sin\left(\frac{42\pi}{23}\right)
\]
and 
\[
\sin\left(\frac{2\pi}{23}\right)-\sin\left(\frac{6\pi}{23}\right)+\sin\left(\frac{10\pi}{23}\right)-\cdots-\sin\left(\frac{38\pi}{23}\right)+\sin\left(\frac{42\pi}{23}\right),
\]
giving each answer in terms of the tangent of a single angle. 


{[}No credit will be given for a numerical answer obtained purely
by use of a calculator.{]}  
	\end{question}
	
	%%%%%%%%% Q8
	\begin{question}

Explain why the use of the substitution $x=\dfrac{1}{t}$ does not
demonstrate that the integrals 
\[
\int_{-1}^{1}\frac{1}{(1+x^{2})^{2}}\,\mathrm{d}x\quad\mbox{ and }\quad\int_{-1}^{1}\frac{-t^{2}}{(1+t^{2})^{2}}\,\mathrm{d}t
\]
are equal. 


Evaluate both integrals correctly.
		
		\end{question}
		
		
%%%%%%%%% Q9
\begin{question}
$ABC$ is a triangle whose vertices have position vectors $\mathbf{a,b,c}$
respectively, relative to an origin in the plane $ABC$. Show that
an arbitrary point $P$ on the segment $AB$ has position vector 
\[
\rho\mathbf{a}+\sigma\mathbf{b},
\]
where $\rho\geqslant0$, $\sigma\geqslant0$ and $\rho+\sigma=1$. 


Give a similar expression for an arbitrary point on the segment $PC$,
and deduce that any point inside $ABC$ has position vector 
\[
\lambda\mathbf{a}+\mu\mathbf{b}+\nu\mathbf{c},
\]
where $\lambda\geqslant0$, $\mu\geqslant0$, $\nu\geqslant0$ and
$\lambda+\mu+\nu=1$. 


Sketch the region of the plane in which the point $\lambda\mathbf{a}+\mu\mathbf{b}+\nu\mathbf{c}$
lies in each of the following cases: 
\begin{questionparts}
\item $\lambda+\mu+\nu=-1$, $\lambda\leqslant0$, $\mu\leqslant0$, $\nu\leqslant0$;
\item $\lambda+\mu+\nu=1$, $\mu\leqslant0$, $\nu\leqslant0$.
\end{questionparts}
\end{question}

\newpage
\section*{Section B: \ \ \ Mechanics}


	
%%%%%%%%%% Q10
\begin{question}
A rubber band band of length $2\pi$ and modulus of elasticity $\lambda$
encircles a smooth cylinder of unit radius, whose axis is horizontal.
A particle of mass $m$ is attached to the lowest point of the band,
and hangs in equilibrium at a distance $x$ below the axis of the
cylinder. Obtain an expression in terms of $x$ for the stretched
length of the band in equilibrium. 


What is the value of $\lambda$ if $x=2$? 

	\end{question}
	
%%%%%%%%%% Q11
\begin{question}	
A smooth sphere of radius $r$ stands fixed on a horizontal floor.
A particle of mass $m$ is displaced gently from equilibrium on top
of the sphere. Find the angle its velocity makes with the horizontal
when it loses contact with the sphere during the subsequent motion. 


By energy considerations, or otherwise, find the vertical component
of the momentum of the particle as it strikes the floor. 
\end{question}

%%%%%%%%%% Q12

\begin{question}$\,$
	\vspace{-1cm}
\begin{center}
\psset{xunit=1.3cm,yunit=1.3cm,algebraic=true,dotstyle=o,dotsize=3pt 0,linewidth=0.4pt,arrowsize=3pt 2,arrowinset=0.25}
\begin{pspicture*}(-1.09,-1.66)(7.83,4.42)
\psline(-0.59,4)(0,4)
\psline(0,4)(0,2.5)
\psline(2.5,2.5)(0,2.5)
\psline(2.5,2.5)(2.5,1)
\psline(2.5,1)(5,1)
\psline(5,1)(5,-0.5)
\psline(5,-0.5)(7.5,-0.5)
\psline(7.5,-0.5)(7.49,-0.87)
\psline[linestyle=dashed,dash=3pt 3pt](-0.59,4)(-1,4)
\psline[linestyle=dashed,dash=3pt 3pt](7.49,-0.87)(7.49,-1.37)
\psline{->}(-0.31,4)(-0.31,2.51)
\psline{->}(-0.31,2.51)(-0.31,4)
\psline{->}(0,2.19)(2.5,2.19)
\psline{->}(2.5,2.19)(0,2.19)
\psline{->}(2.72,2.5)(2.72,1)
\psline{->}(2.72,1)(2.72,2.5)
\psline{->}(2.5,0.69)(5,0.69)
\psline{->}(5,0.69)(2.5,0.69)
\psline{->}(5.27,1)(5.27,-0.5)
\psline{->}(5.27,-0.5)(5.27,1)
\psline{->}(-0.08,4.09)(0.46,4.09)
\rput[tl](-0.55,3.55){$h$}
\rput[tl](2.81,2.02){$h$}
\rput[tl](5.43,0.47){$h$}
\rput[tl](0.95,2.05){$2d$}
\rput[tl](3.66,0.58){$2d$}
\begin{scriptsize}
\psdots[dotsize=7pt 0,dotstyle=*](-0.08,4.09)
\end{scriptsize}
\end{pspicture*}
\end{center}
A particle is placed at the edge of the top step of a flight of steps.
Each step is of width $2d$ and height $h$. The particle is kicked
horizontally perpendicular to the edge of the top step. On its first
and second bounces it lands exactly in the middle of each of the first
and second steps from the top. Find the coefficient of restitution
between the particle and the steps. 


Determine whether it is possible for the particle to continue bouncing
down the steps, hitting the middle of each successive step.
\end{question}
	
%%%%%%%%%% Q13
\begin{question}
A particle of mass $m$ moves along the $x$-axis. At time $t=0$
it passes through $x=0$ with velocity $v_{0}>0$. The particle is
acted on by a force $\mathrm{F}(x)$, directed along the $x$-axis
and measured in the direction of positive $x$, which is given by
\[
\mathrm{F}(x)=\begin{cases}
-m\mu^{2}x & \qquad(x\geqslant0),\\
-m\kappa\dfrac{\mathrm{d}x}{\mathrm{d}t} & \qquad(x<0),
\end{cases}
\]
where $\mu$ and $\kappa$ are positive constants. Obtain the particle's
subsequent position as a function of time, and give a rough sketch
of the $x$-$t$ graph. 
\end{question}
	
	\newpage
\section*{Section C: \ \ \ Probability and Statistics}


%%%%%%%%%% Q14
\begin{question}
$A,B$ and $C$ play a table tennis tournament. The winner of the
tournament will be the first person to win two games in a row. In
any game, whoever is not playing acts as a referee, and each player
has equal chance of winning the game. The first game of the tournament
is played between $A$ and $B$, with $C$ as referee. Thereafter,
if the tournament is still undecided at the end of any game, the winner
and referee of that game play the next game. The tournament is recorded
by listing in order the winners of each game, so that, for example,
$ACC$ records a three-game tournament won by $C$, the first game
having been won by $A$. Determine which of the following sequences
of letters could be the record of a complete tournament, giving brief
reasons for your answers: 

\begin{questionparts}
\item $ACB$, 
\item $ABB$, 
\item $ACBB$. 
\end{questionparts}

Find the probability that the tournament is still undecided after
5 games have been played. Find also the probabilities that each of
$A,B$ and $C$ wins in 5 or fewer games. 


Show that the probability that $A$ wins eventually is $\frac{5}{14}$,
and find the corresponding probabilities for $B$ and $C$. 

\end{question}

%%%%%%%%%% Q15
\begin{question}
A point $P$ is chosen at random (with uniform distribution) on the
circle $x^{2}+y^{2}=1$. The random variable $X$ denotes the distance
of $P$ from $(1,0)$. Find the mean and variance of $X$. 


Find also the probability that $X$ is greater than its mean. 
\end{question}

%%%%%%%%%% Q16
\begin{question}
	The parliament of Laputa consists of 60 Preservatives and 40 Progressives.
	Preservatives never change their mind, always voting the same way
	on any given issue. Progressives vote at random on any given issue. 

	\begin{questionparts}
	\item A randomly selected member is known to have voted the same way twice
	on a given issue. Find the probability that the member will vote the
	same way a third time on that issue. 
	\item Following a policy change, a proportion $\alpha$ of Preservatives
	now consistently votes against Preservative policy. The Preservative
	leader decides that an election must be called if the value of $\alpha$
	is such that, at any vote on an item of Preservative policy, the chance
	of a simple majority would be less than 80\%. By making a suitable
	normal approximation, estimate the least value of $\alpha$ which
	will result in an election being called.
\end{questionparts}
	\end{question}
\end{document}
