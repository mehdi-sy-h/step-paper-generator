
\documentclass[a4, 11pt]{report}


\pagestyle{myheadings}
\markboth{}{Paper III, 1989
\ \ \ \ \ 
\today 
}               

\RequirePackage{amssymb}
\RequirePackage{amsmath}
\RequirePackage{graphicx}
\RequirePackage{color}
\RequirePackage[flushleft]{paralist}[2013/06/09]



\RequirePackage{geometry}
\geometry{%
  a4paper,
  lmargin=2cm,
  rmargin=2.5cm,
  tmargin=3.5cm,
  bmargin=2.5cm,
  footskip=12pt,
  headheight=24pt}


\newcommand{\comment}[1]{{\bf Comment} {\it #1}}
%\renewcommand{\comment}[1]{}

\newcommand{\bluecomment}[1]{{\color{blue}#1}}
%\renewcommand{\comment}[1]{}
\newcommand{\redcomment}[1]{{\color{red}#1}}



\usepackage{epsfig}
\usepackage{pstricks-add}
\usepackage{tgheros} %% changes sans-serif font to TeX Gyre Heros (tex-gyre)
\renewcommand{\familydefault}{\sfdefault} %% changes font to sans-serif
%\usepackage{sfmath}  %%%% this makes equation sans-serif
%\input RexFigs


\setlength{\parskip}{10pt}
\setlength{\parindent}{0pt}

\newlength{\qspace}
\setlength{\qspace}{20pt}


\newcounter{qnumber}
\setcounter{qnumber}{0}

\newenvironment{question}%
 {\vspace{\qspace}
  \begin{enumerate}[\bfseries 1\quad][10]%
    \setcounter{enumi}{\value{qnumber}}%
    \item%
 }
{
  \end{enumerate}
  \filbreak
  \stepcounter{qnumber}
 }


\newenvironment{questionparts}[1][1]%
 {
  \begin{enumerate}[\bfseries (i)]%
    \setcounter{enumii}{#1}
    \addtocounter{enumii}{-1}
    \setlength{\itemsep}{5mm}
    \setlength{\parskip}{8pt}
 }
 {
  \end{enumerate}
 }



\DeclareMathOperator{\cosec}{cosec}
\DeclareMathOperator{\Var}{Var}

\def\d{{\rm d}}
\def\e{{\rm e}}
\def\g{{\rm g}}
\def\h{{\rm h}}
\def\f{{\rm f}}
\def\p{{\rm p}}
\def\s{{\rm s}}
\def\t{{\rm t}}


\def\A{{\rm A}}
\def\B{{\rm B}}
\def\E{{\rm E}}
\def\F{{\rm F}}
\def\G{{\rm G}}
\def\H{{\rm H}}
\def\P{{\rm P}}


\def\bb{\mathbf b}
\def \bc{\mathbf c}
\def\bx {\mathbf x}
\def\bn {\mathbf n}

\newcommand{\low}{^{\vphantom{()}}}
%%%%% to lower suffices: $X\low_1$ etc


\newcommand{\subone}{ {\vphantom{\dot A}1}}
\newcommand{\subtwo}{ {\vphantom{\dot A}2}}




\def\le{\leqslant}
\def\ge{\geqslant}


\def\var{{\rm Var}\,}

\newcommand{\ds}{\displaystyle}
\newcommand{\ts}{\textstyle}




\begin{document}
\setcounter{page}{2}

 
\section*{Section A: \ \ \ Pure Mathematics}

%%%%%%%%%%Q1
\begin{question}
Prove that the area of the zone of the surface of a sphere between
two parallel planes cutting the sphere is given by 
\[
2\pi\times(\mbox{radius of sphere})\times(\mbox{perpendicular distance between the planes}).
\]
A tangent from the origin $O$ to the curve with cartesian equation
\[
(x-c)^{2}+y^{2}=c^{2},
\]
where $a$ and $c$ are positive constants with $c>a,$ touches the
curve at $P$. The $x$-axis cuts the curve at $Q$ and $R$, the
points lying in the order $OQR$ on the axis. The line $OP$ and the
arc $PR$ are rotated through $2\pi$ radians about the line $OQR$
to form a surface. Find the area of this surface.
\end{question}
\vspace{-1cm}
%%%%%%%%%%Q2
\begin{question}
The points $A,B$ and $C$ lie on the surface of the ground, which
is an inclined plane. The point $B$ is 100m due north of $A,$ and
$C$ is 60m due east of $B$. The vertical displacements from $A$
to $B,$ and from $B$ to $C$, are each 5m downwards. A plane coal
seam lies below the surface and is to be located by making vertical
bore-holes at $A,B$ and $C$. The bore-holes strike the coal seam
at 95m, 45m and 76m below $A,B$ and $C$ respectively. Show that
the coal seam is inclined at $\cos^{-1}(\frac{4}{5})$ to the horizontal. 


The coal seam comes to the surface along a line. Find the bearing
of this line. 
\end{question}

\vspace{-1cm}
%%%%%%%%% Q3
\begin{question}
The matrix $\mathbf{M}$ is given by 
\[
\mathbf{M}=\begin{pmatrix}\cos(2\pi/m) & -\sin(2\pi/m)\\
\sin(2\pi/m) & \cos(2\pi/m)
\end{pmatrix},
\]
where $m$ is an integer greater than $1.$ Prove that 
\[
\mathbf{M}^{m-1}+\mathbf{M}^{m-2}+\cdots+\mathbf{M}^{2}+\mathbf{M}+\mathbf{I}=\mathbf{O},
\]
where $\mathbf{I}=\begin{pmatrix}1 & 0\\
0 & 1
\end{pmatrix}$ and $\mathbf{O}=\begin{pmatrix}0 & 0\\
0 & 0
\end{pmatrix}.$


The sequence $\mathbf{X}_{0},\mathbf{X}_{1},\mathbf{X}_{2},\ldots$
is defined by 
\[
\mathbf{X}_{k+1}=\mathbf{PX}_{k}+\mathbf{Q},
\]
where $\mathbf{P,Q}$ and $\mathbf{X}_{0}$ are given $2\times2$
matrices. Suggest a suitable expression for $\mathbf{X}_{k}$ in terms
of $\mathbf{P},$ $\mathbf{Q}$ and $\mathbf{X}_{0},$ and justify
it by induction. 


The binary operation $*$ is defined as follows: 
\[
\mathbf{X}_{i}*\mathbf{X}_{j}\mbox{ is the result of substituting \ensuremath{\mathbf{X}_{j}}for \ensuremath{\mathbf{X}_{0}}in the expression for \ensuremath{\mathbf{X}_{i}}. }
\]
Show that if $\mathbf{P=M},$ the set $\{\mathbf{X}_{1},\mathbf{X}_{2},\mathbf{X}_{3},\ldots\}$
forms a finite group under the operation $*$. 
\end{question}


%%%%%% Q4 

\begin{question}
Sketch the curve whose cartesian equation is 
\[
y=\frac{2x(x^{2}-5)}{x^{2}-4},
\]
and give the equations of the asymptotes and of the tangent to the
curve at the origin. 


Hence, or otherwise, determine (giving reasons) the number of real
roots of the following equations: 
\begin{itemize}
\setlength{\itemsep}{3mm}
\item[\bf (i)] $4x^{2}(x^{2}-5)=(5x-2)(x^{2}-4)$; 
\item[\bf (ii)] $4x^{2}(x^{2}-5)^{2}=(x^{2}-4)^{2}(x^{2}+1)$; 
\item[\bf (iii)] $4z^{2}(z-5)^{2}=(z-4)^{2}(z+1)$. 
	\end{itemize}

\end{question}


%%%%%%%%% Q5
\begin{question}
Given that $y=\cosh(n\cosh^{-1}x),$ for $x\geqslant1,$ prove that
\[
y=\frac{(x+\sqrt{x^{2}-1})^{n}+(x-\sqrt{x^{2}-1})^{n}}{2}.
\]
Explain why, when $n=2k+1$ and $k\in\mathbb{Z}^{+},$ $y$ can also
be expressed as the polynomial 
\[
a_{0}x+a_{1}x^{3}+a_{2}x^{5}+\cdots+a_{k}x^{2k+1}.
\]
Find $a_{0},$ and show that 

\begin{itemize}
\setlength{\itemsep}{3mm}
\item[\bf (i)] $a_{1}=(-1)^{k-1}2k(k+1)(2k+1)/3$; 
\item[\bf (ii)] $a_{2}=(-1)^{k}2(k-1)k(k+2)(2k+1)/15.$
	\end{itemize}

Find also the value of ${\displaystyle \sum_{r=0}^{k}a_{r}.}$ 
	\end{question}
	
	%%%%%%%%% Q6
	\begin{question}
Show that, for a given constant $\gamma$ $(\sin\gamma\neq0)$ and
with suitable choice of the constants $A$ and $B$, the line with
cartesian equation $lx+my=1$ has polar equations 
\[
\frac{1}{r}=A\cos\theta+B\cos(\theta-\gamma).
\]
The distinct points $P$ and $Q$ on the conic with polar equations
\[
\frac{a}{r}=1+e\cos\theta
\]
correspond to $\theta=\gamma-\delta$ and $\theta=\gamma+\delta$
respectively, and $\cos\delta\neq0.$ Obtain the polar equation of
the chord $PQ.$ Hence, or otherwise, obtain the equation of the tangent
at the point where $\theta=\gamma.$


The tangents at $L$ and $M$ to a conic with focus $S$ meet at $T.$
Show that $ST$ bisects the angle $LSM$ and find the position of
the intersection of $ST$ and $LM$ in terms of your chosen parameters
for $L$ and $M.$ 
	 \end{question}
	 
	 %%%%%%%%% Q7
\begin{question}
The linear transformation $\mathrm{T}$ is a shear which transforms
a point $P$ to the point $P'$ defined by 

\begin{itemize}
\setlength{\itemsep}{3mm}
\item[\bf (i)] $\overrightarrow{PP'}$ makes an acute angle $\alpha$ (anticlockwise)
with the $x$-axis, 
\item[\bf (ii)] $\angle POP'$ is clockwise (i.e. the rotation from $OP$ to $OP'$
clockwise is less than $\pi),$ 
\item[\bf (iii)] $PP'=k\times PN,$ where $PN$ is the perpendicular onto the line
$y=x\tan\alpha,$ where $k$ is a given non-zero constant. 
\end{itemize}

If $\mathrm{T}$ is represented in matrix form by $\begin{pmatrix}x'\\
y'
\end{pmatrix}=\mathbf{M}\begin{pmatrix}x\\
y
\end{pmatrix},$ show that 
\[
\mathbf{M}=\begin{pmatrix}1-k\sin\alpha\cos\alpha & k\cos^{2}\alpha\\
-k\sin^{2}\alpha & 1+k\sin\alpha\cos\alpha
\end{pmatrix}.
\]
Show that the necessary and sufficient condition for $\begin{pmatrix}p & q\\
r & t
\end{pmatrix}$ to commute with $\mathbf{M}$ is 
\[
t-p=2q\tan\alpha=-2r\cot\alpha.
\]
	\end{question}
	
	%%%%%%%%% Q8
	\begin{question}
Given that 
\[
\frac{\mathrm{d}x}{\mathrm{d}t}=4(x-y)\qquad\mbox{ and }\qquad\frac{\mathrm{d}y}{\mathrm{d}t}=x-12(\mathrm{e}^{2t}+\mathrm{e}^{-2t}),
\]
obtain a differential equation for $x$ which does not contain $y$.
Hence, or otherwise, find $x$ and $y$ in terms of $t$ given that
$x=y=0$ when $t=0$. 
		\end{question}
		
		
%%%%%%%%% Q9
		\begin{question}
Obtain the sum to infinity of each of the following series. 

\begin{questionparts}
\item $1{\displaystyle +\frac{2}{2}+\frac{3}{2^{2}}+\frac{4}{2^{3}}+\cdots+\frac{r}{2^{r-1}}+\cdots;}$
\item $1{\displaystyle +\frac{1}{2}\times\frac{1}{2}+\frac{1}{3}\times\frac{1}{2^{2}}+\cdots+\frac{1}{r}\times\frac{1}{2^{r-1}}+\cdots;}$
\item ${\displaystyle \dfrac{1\times3}{2!}\times\frac{1}{3}+\frac{1\times3\times5}{3!}\frac{1}{3^{2}}+\cdots+\frac{1\times3\times\cdots\times(2k-1)}{k!}\times\frac{1}{3^{k-1}}+\cdots.}$
\end{questionparts}

[Questions of convergence need not be considered.]
		\end{question}
		
	
%%%%%%%%%% 10
\begin{question}
\begin{questionparts}
	 \item Prove that 
\[
\sum_{r=1}^{n}r(r+1)(r+2)(r+3)(r+4)=\tfrac{1}{6}n(n+1)(n+2)(n+3)(n+4)(n+5)
\]
and deduce that 
\[
\sum_{r=1}^{n}r^{5}<\tfrac{1}{6}n(n+1)(n+2)(n+3)(n+4)(n+5).
\]
\item Prove that, if $n>1,$ 
\[
\sum_{r=0}^{n-1}r^{5}>\tfrac{1}{6}(n-5)(n-4)(n-3)(n-2)(n-1)n.
\]
\item Let $\mathrm{f}$ be an increasing function. If the limits
\[
\lim_{n\rightarrow\infty}\sum_{r=0}^{n-1}\frac{a}{n}\mathrm{f}\left(\frac{ra}{n}\right)\qquad\mbox{ and }\qquad\lim_{n\rightarrow\infty}\sum_{r=1}^{n}\frac{a}{n}\mathrm{f}\left(\frac{ra}{n}\right)
\]
both exist and are equal, the definite integral ${\displaystyle \int_{0}^{a}\mathrm{f}(x)\,\mathrm{d}x}$
is defined to be their common value. Using this definition, prove
that 
\[
\int_{0}^{a}x^{5}\,\mathrm{d}x=\tfrac{1}{6}a^6.
\]
\end{questionparts}
\end{question}
			
		
		
		
	
\newpage
\section*{Section B: \ \ \ Mechanics}


	
%%%%%%%%%% Q11
\begin{question}
A smooth uniform sphere, with centre $A$, radius $2a$ and mass $3m,$
is suspended from a fixed point $O$ by means of a light inextensible
string, of length $3a,$ attached to its surface at $C$. A second
smooth unifom sphere, with centre $B,$ radius $3a$ and mass $25m,$
is held with its surface touching $O$ and with $OB$ horizontal.
The second sphere is released from rest, falls and strikes the first
sphere. The coefficient of restitution between the spheres is $3/4.$
Find the speed $U$ of $A$ immediately after the impact in terms
of the speed $V$ of $B$ immediately before impact. 


The same system is now set up with a light rigid rod replacing the
string and rigidly attached to the sphere so that $OCA$ is a straight
line. The rod can turn freely about $O$. The sphere with centre $B$
is dropped as before. Show that the speed of $A$ immediately after
impact is $125U/127.$ 
	\end{question}
	
%%%%%%%%%% Q12
\begin{question}	
A smooth horizontal plane rotates with constant angular velocity $\Omega$
about a fixed vertical axis through a fixed point $O$ of the plane.
The point $A$ is fixed in the plane and $OA=a.$ A particle $P$
lies on the plane and is joined to $A$ by a light rod of length $b(>a)$
freely pivoted at $A$. Initially $OAP$ is a straight line and $P$
is moving with speed $(a+2\sqrt{ab})\Omega$ perpendicular to $OP$
in the same sense as $\Omega.$ At time $t,$ $AP$ makes an angle
$\phi$ with $OA$ produced. Obtain an expression for the component
of the acceleration of $P$ perpendicular to $AP$ in terms of $\dfrac{\mathrm{d}^{2}\phi}{\mathrm{d}t^{2}},\phi,a,b$
and \nolinebreak $\Omega.$


Hence find $\dfrac{\mathrm{d}\phi}{\mathrm{d}t}$, in terms of $\phi,a,b$
and $\Omega,$ and show that $P$ never crosses $OA.$
\end{question}

%%%%%%%%%% Q13

\begin{question}
The points $A,B,C,D$ and $E$ lie on a thin smooth horizontal table
and are equally spaced on a circle with centre $O$ and radius $a$.
At each of these points there is a small smooth hole in the table.
Five elastic strings are threaded through the holes, one end of each
beging attached at $O$ under the table and the other end of each
being attached to a particle $P$ of mass $m$ on top of the table.
Each of the string has natural length $a$ and modulus of elasticity
$\lambda.$ If $P$ is displaced from $O$ to any point $F$ on the
table and released from rest, show that $P$ moves with simple harmonic
motion of period $T$, where 
\[
T=2\pi\sqrt{\frac{am}{5\lambda}}.
\]
The string $PAO$ is replaced by one of natural length $a$ and modulus
$k\lambda.$ $P$ is displaced along $OA$ from its equilibrium position
and released. Show that $P$ still moves in a straight line with simple
harmonic motion, and, given that the period is $T/2,$ find $k$.
\end{question}
	
%%%%%%%%%% Q14
\begin{question}
\begin{questionparts} \item A solid circular disc has radius $a$ and
mass $m.$ The density is proportional to the distance from the centre
$O$. Show that the moment of inertia about an axis through $C$ perpendicular
to the plane of the disc is $\frac{3}{5}ma^{2}.$


\item A light inelastic string has one end fixed at $A$. It passes
under and supports a smooth pulley $B$ of mass $m.$ It then passes
over a rough pulley $C$ which is a disc of the type described in
\textbf{(i)}, free to turn about its axis which is fixed and horizontal.
The string carries a particle $D$ of mass $M$ at its other end.
The sections of the string which are not in contact with the pulleys
are vertical. The system is released from rest and moves under gravity
for $t$ seconds. At the end of this interval the pulley $B$ is suddenly
stopped. Given that $m<2M$, find the resulting impulse on $D$ in
terms of $m,M,g$ and $t$. 


{[}You may assume that the string is long enough for there to be no
collisions between the elements of the system, and that the pulley
$C$ is rough enough to prevent slipping throughout.{]} \end{questionparts}

\end{question}
	
	\newpage
\section*{Section C: \ \ \ Probability and Statistics}


%%%%%%%%%% Q15
\begin{question}
The continuous random variable $X$ is uniformly distributed over
the interval $[-c,c].$ Write down expressions for the probabilities
that: 

\begin{itemize}
\setlength{\itemsep}{3mm}
\item[\bf (i)] $n$ independently selected values of $X$ are all greater than $k$, 
\item[\bf (ii)] $n$ independently selected values of $X$ are all less than $k$,
\end{itemize}

where $k$ lies in $[-c,c]$. 


A sample of $2n+1$ values of $X$ is selected at random and $Z$
is the median of the sample. Show that $Z$ is distributed over $[-c,c]$
with probability density function 
\[
\frac{(2n+1)!}{(n!)^{2}(2c)^{2n+1}}(c^{2}-z^{2})^{n}.
\]
Deduce the value of ${\displaystyle \int_{-c}^{c}(c^{2}-z^{2})^{n}\,\mathrm{d}z.}$


Evaluate $\mathrm{E}(Z)$ and $\mathrm{var}(Z).$  
\end{question}

%%%%%%%%%% Q16
\begin{question}
It is believed that the population of Ruritania can be described as
follows: 

\begin{itemize}[indent]
\setlength{\itemsep}{3mm}

\item[\bf (i)] $25\%$ are fair-haired and the rest are dark-haired; 
\item[\bf (ii)] $20\%$ are green-eyed and the rest hazel-eyed; 
\item[\bf (iii)] the population can also be divided into narrow-headed and broad-headed; 
\item[\bf (iv)] no arrow-headed person has green eyes and fair hair; 
\item[\bf (v)] those who are green-eyed are as likely to be narrow-headed as broad-headed; 
\item[\bf (vi)] those who are green-eyed and broad-headed are as likely to be fair-headed
as dark-haired; 
\item[\bf (vii)] half of the population is broad-headed and dark-haired; 
\item[\bf (viii)] a hazel-haired person is as likely to be fair-haired and broad-headed
as dark-haired and narrow-headed. 
\end{itemize}

Find the proportion believed to be narrow-headed. 


I am acquainted with only six Ruritanians, all of whom are broad-headed.
Comment on this observation as evidence for or against the given model. 


A random sample of 200 Ruritanians is taken and is found to contain
50 narrow-heads. On the basis of the given model, calculate (to a
reasonable approximation) the probability of getting 50 or fewer narrow-heads.
Comment on the result. 
\end{question}
\end{document}
