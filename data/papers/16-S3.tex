\documentclass[a4, 11pt]{report}


%\pagestyle{myheadings}
%%\markboth{}{Paper III, 2006  final draft  
%\ \ \ \ \ 
%\today 
%}               




\RequirePackage{amssymb}
\RequirePackage{amsmath}
\RequirePackage{graphicx}
\RequirePackage{color}
\RequirePackage[flushleft]{paralist}[2013/06/09]



\RequirePackage{geometry}
\geometry{%
  a4paper,
  lmargin=2cm,
  rmargin=2.5cm,
  tmargin=3.5cm,
  bmargin=2.5cm,
  footskip=12pt,
  headheight=24pt}


\newcommand{\comment}[1]{{\bf Comment} {\it #1}}
%\renewcommand{\comment}[1]{}

\newcommand{\bluecomment}[1]{{\color{blue}#1}}
%\renewcommand{\comment}[1]{}
\newcommand{\rct}[1]{{\color{red}#1}}



\usepackage{epsfig}

%\input RexFigs


\setlength{\parskip}{10pt}
\setlength{\parindent}{0pt}

\newlength{\qspace}
\setlength{\qspace}{20pt}


\newcounter{qnumber}
\setcounter{qnumber}{0}

\newenvironment{question}%
 {\vspace{\qspace}
  \begin{enumerate}[\bfseries 1\quad][10]%
    \setcounter{enumi}{\value{qnumber}}%
    \item%
 }
{
  \end{enumerate}
  \filbreak
  \stepcounter{qnumber}
 }


\newenvironment{questionparts}[1][1]%
 {
  \begin{enumerate}[\bfseries (i)]%
    \setcounter{enumii}{#1}
    \addtocounter{enumii}{-1}
    \setlength{\itemsep}{5mm}
    \setlength{\parskip}{8pt}
 }
 {
  \end{enumerate}
 }


\DeclareMathOperator{\Var}{Var}

\DeclareMathOperator{\cosec}{cosec}
\DeclareMathOperator{\arcosh}{arcosh}
\DeclareMathOperator{\arctanh}{arctanh}
\DeclareMathOperator{\cosech}{cosech}
\DeclareMathOperator{\sech}{sech}
\DeclareMathOperator{\arsinh}{arsinh}



\def\d{{\rm d}}
\def\e{{\rm e}}
\def\g{{\rm g}}
\def\h{{\rm h}}
\def\f{{\rm f}}
\def\p{{\rm p}}
\def\s{{\rm s}}
\def\t{{\rm t}}


\def\A{{\rm A}}
\def\B{{\rm B}}
\def\E{{\rm E}}
\def\F{{\rm F}}
\def\G{{\rm G}}
\def\H{{\rm H}}
\def\P{{\rm P}}
\def\Q{{\rm Q}}


\def\bb{\mathbf b}
\def \bc{\mathbf c}
\def\bx {\mathbf x}
\def\bn {\mathbf n}

\makeatletter
\newcommand{\raisemath}[1]{\mathpalette{\raisem@th{#1}}}
\newcommand{\raisem@th}[3]{\raisebox{#1}{$#2#3$}}
\makeatother
%%%To raise suffice: e.g.  $\Pi_{\raisemath{2pt}{-}}$.



\def\le{\leqslant}
\def\ge{\geqslant}


\def\var{{\rm Var}\,}

\newcommand{\ds}{\displaystyle}
\newcommand{\ts}{\textstyle}




\begin{document}

\setcounter{page}{2}




\section*{Section A: \ \ \ Pure Mathematics}


%%%%%%%%%%%%%% Q1
\begin{question}
Let
\[
\displaystyle
I_n=
 \int_{-\infty}^\infty
\frac 1 {(x^2+2ax+b)^n}
\, \d x
\,,
\]
 where $a$ and $b$ are constants with $b > a^2$, and $n$ is a positive
integer.
\begin{questionparts}
\item
 By using the substitution $x + a = \sqrt{b-   a^2} \, \tan u\,$,
 or otherwise,
  show that 
\[
I_1 = \dfrac \pi {\sqrt{b-a^2}}\, .
\] 
\item  Show that $2n(b - a^2)\, I_{n+1} =(2n - 1) \, I_n\,$.
\item   Hence prove by induction that 
\[
I_n =
 \frac{\pi}{2^{2n-2}( b - a^2)^{n-\frac12}} 
\, \binom {2n\! - \!2}{n\!-\!1}
\,.
\]
\end{questionparts}
\end{question}

%%%%%%%%%%%%%% Q2
\begin{question}
 The distinct points $P(ap^2 , 2ap)$,  $Q(aq^2 , 2aq)$ and $R(ar^2,2ar)$
 lie on the parabola  $y^2 = 4ax$, where $a>0$.
The points are such that the normal to the parabola at $Q$ and 
the normal to the parabola at $R$ both pass through $P$.


\begin{questionparts}
\item Show that $q^2 +qp + 2 = 0$\,.
\item 
 Show that 
$QR$ passes through a certain point that is independent of the 
choice of $P$.
\item
Let $T$ be
the point of intersection of  $OP$ and $QR$, where
$O$ is the coordinate origin. Show that $T$
 lies on a line that is independent of the choice of $P$. 

Show further 
that the distance  from the $x$-axis to $T$ is less
than $\dfrac {\;a}{\sqrt2}\,$.
\end{questionparts}

\end{question} 



%%%%%%%%%%%%%% Q3
\begin{question}
\begin{questionparts}
\item
Given that 
\[
\int 
\frac {x^3-2}{(x+1)^2}\, \e ^x \d x = 
\frac{\P(x)}{\Q(x)}\,\e^x + \text{constant}
\,,
\]
 where $\P(x)$and $\Q(x)$ are polynomials,
show that
$\Q(x)$
has a factor of $x + 1$. 

Show also that
the degree of $\P(x)$ is exactly one more than the degree of $\Q(x)$,
and  find $\P(x)$ in the case $\Q(x) =x+1$.


\item
Show that there are no  polynomials $\P(x)$ and $\Q(x)$
such that 
\[
\int \frac 1 {x+1} \, \, \e^x \d x 
=
\frac{\P(x)}{\Q(x)}\,\e^x +\text{constant}
\,.
\]
You need consider only the case when $\P(x)$ and $\Q(x)$ 
have no common factors. 
\end{questionparts}






\end{question}





%%%%%%%%%%%%%% Q4
\begin{question}
\begin{questionparts}
\item
By considering
\ $\displaystyle
\frac1
{1+ x^r} 
-
\frac1
{1+ x^{r +1}}
$ 
\ for $\vert x \vert \ne 1$, 
simplify
\[
\sum_{r=1}^N
\frac{x^r}{(1+x^r)(1+x^{r+1})}
\,.
\]
Show that, for $\vert x \vert <1$,
\[
\sum_{r=1}^\infty
\frac{x^r}{(1+x^r)(1+x^{r+1})}
=
\frac x {1-x^2}
\,.
\]

\item
Deduce     that
\[
\sum_{r=1}^\infty
\sech(ry)\sech((r +  1)y) = 2\e^{-y} \cosech (2 y)
\]
 for $y > 0$.



Hence simplify  
\[
\sum_{r=-\infty}^\infty  \sech(ry) \sech((r + 1)y) 
\,,\]
for $y>0$.
\end{questionparts}


\end{question}




%%%%%%%%%%%%%% Q5
\begin{question}
\begin{questionparts}
\item By considering the binomial expansion of $(1+x)^{2m+1}$,
 prove that 
\[
\binom{ 2m \! +\!  1}{ m} < 2^{2m}\,,
\]
for any positive integer $m$.

\item
For any positive integers $r$ and  $s$ with $r<s$, 
$P_{r,s}$ is defined as follows:
$P_{r,s}$ is 
the product of all the prime numbers greater than  $r$ 
and less than or
equal to $s$, if there are any such primes numbers; 
if there are no such primes numbers,
then $P_{r,s}=1\,$.

For example, $P_{3,7}=35$, $P_{7,10}=1$ and $P_{14,18}=17$.
 
Show that, for any positive integer $m$,
$
P_{m+1\,,\, 2m+1} 
$    
divides 
$\displaystyle \binom{ 2m \! +\!  1}{ m} \,,$
and deduce that
\[
P_{m+1\,,\, 2m+1} < 2^{2m}
\,.
\]

\item  Show that, if $P_{1,k} < 4^k$ for $k = 2$, $3$,  $\ldots$, $2m$,
then  
$ P_{1,2m+1} < 4^{2m+1}\,$.

\item
Prove that $\P_{1,n}  < 4^n$ for 
 $n\ge2$.
\end{questionparts}

\end{question} 
%%%%%%%%%%%%%% Q6
\begin{question}
Show, by finding $R$ and $\gamma$, that 
$A \sinh x + B\cosh x $ can be written in the 
form $R\cosh (x+\gamma)$ if $B>A>0$. Determine the corresponding forms
in the other cases that arise, for $A>0$, according to the value of $B$. 

Two curves have equations  
$y = \sech x$  
and $y = a\tanh x + b\,$, where $a>0$.
\begin{questionparts}
\item
In the case $b>a$, show that if the curves intersect then the $x$-coordinates
of the points of intersection can be written in the form
\[
\pm\arcosh \left( \frac 1 {\sqrt{b^2-a^2}}\right) - {\rm artanh \,} \frac a b
.\]

\item 
Find the corresponding result 
in the case $a>b>0\,$. 

\item Find necessary and sufficient conditions on $a$ and $b$ for  the curves to intersect
at two distinct points.
\item Find necessary and sufficient conditions on $a$ and $b$ for  the curves to touch
 and, given that they touch, 
 express the $y$-coordinate of the point of contact 
 in terms of $a$.

\end{questionparts}
\end{question}



%%%%%%%%%%%%%% Q7
\begin{question}
Let $\omega = \e^{2\pi {\rm i}/n}$, where $n$ is a  positive integer.
Show that, for any complex number $z$,
\[
(z-1)(z-\omega) \cdots (z - \omega^{n-1}) = z^n -1\,.
\]

The points $X_0$, $X_1$, \ldots\,, $X_{n-1}$ lie on a circle 
with centre $O$ and radius 1,  and are the vertices of 
a regular polygon. 
\begin{questionparts}
\item  The point  $P$ is equidistant from  $X_0$ and $X_1$.
Show that, if $n$ is even, 
\[
|PX_0| \times  |PX_1 |\times \,\cdots\, \times |PX_{n-1}| = |OP|^n +1\,
,\]
where  $|PX_ k|$ denotes
the distance from $P$ to $X_k$.

Give the corresponding result  when $n$ is odd. (There are
two cases to consider.)


\item Show that
\[
|X_0 X_1|\times  |X_0 X_2|\times  \,\cdots\, \times |X_0 X_{n-1}| =n\,.
\] 
\end{questionparts}


\end{question} 

%%%%%%%%%%%%%% Q8
\begin{question}

\begin{questionparts}

\item
The function f satisfies, for all $x$,  the equation 
\[
\f(x) + (1- x)\f(-x) =  x^2\, .
\]
Show that $\f(-x) + (1 + x)\f(x) =  x^2\,$.
 Hence find $\f(x)$ in terms of $x$. You should
verify that your function satisfies the original equation.

\item 
The function ${\rm K}$ is defined, for $x\ne 1$, by 
\[{\rm K}(x) = \dfrac{x+1}{x-1}\,.\] 
Show that, for $x\ne1$,
${\rm K(K(}x)) =x\,$.

The function g satisfies the equation 
\[
\g(x)+ x\, \g\Big(\frac{ x+1   }{x-1}\Big)
 = x \ \ \ \ \ \ \ \ \ \ \  
( x\ne 1)
\,.
\]
 Show
that, for $x\ne1$, $\g(x)= \dfrac{2x}{x^2+1}\,$.

\item
Find $\h(x)$, for $x\ne0$, $x\ne1$, given that 
\[
\h(x)+ \h\Big(\frac 1 {1-x}\Big)=  1-x -\frac1{1-x}
\ \ \ \ \ \ (
x\ne0, \ \ x\ne1 )
\,.
\]
\end{questionparts}

\end{question} 

\newpage
\section*{Section B: \ \ \ Mechanics}
%%%%%%%%%%%%%% Q9
\begin{question}
Three pegs $P$, $Q$ and $R$
 are fixed on a smooth horizontal table in such a way that they
form the vertices of an equilateral triangle of side $2a$. A particle $X$
 of mass $m$ lies on the table.
It is attached to the  pegs by three springs, $PX$, $QX$ and $RX$, each  
of modulus of elasticity
$\lambda$ and
 natural length $l$,
 where $l < \frac{ \ 2 }{\sqrt3}\, a$.
Initially the particle is in equilibrium. 
Show that the extension in each spring is
$\frac{\ 2}{\sqrt3}\,a -l\,$.

The particle is then pulled a small distance directly towards 
$P$ and released. Show that the tension $T$ in the spring 
$RX$ 
is given by
\[
T= \frac {\lambda} l 
\left( \sqrt{\frac {4a^2}3 + \frac{2ax}{\sqrt3} +x^2\; }\; -l\right)
,
\]
 where $x$ is the displacement of $X$ from its equilibrium position. 

Show further that the particle performs approximate 
simple harmonic motion with period
\[
2\pi \sqrt{ \frac{4mla}{3  (4a-\sqrt3 \, l)\lambda } \; }\,.
\]

\end{question}


%%%%%%%%%%%%%% Q10
\begin{question}
A smooth plane is inclined
at an angle $\alpha$ to the horizontal. A particle $P$ of mass $m$
 is attached to a fixed point $A$
 above the plane by a light inextensible string of length $a$.
The particle rests in equilibrium on the plane,
and the string makes an angle $\beta$ with the plane.

The particle is given a horizontal impulse parallel 
to the plane so that it has an initial speed
of $u$. 
Show that the particle will not immediately leave the plane 
if $ag\cos(\alpha + \beta)> u^2 \tan\beta$.

Show further that a necessary condition  for the 
particle to perform a complete circle whilst in
contact with the plane is
$6\tan\alpha  \tan \beta < 1$.

\end{question}
%%%%%%%%%%%%%% Q11
\begin{question}
A car of mass $m$ travels along a straight horizontal road with 
its engine working  at a
constant rate $P$. The  resistance to its motion 
is such that
the  acceleration of the car is zero 
 when it is moving with speed
$4U$. 

\begin{questionparts}
\item
Given that the  resistance is proportional to the car's speed, 
show that 
the distance~$X_1$ travelled by the car 
while 
it accelerates from speed $U$ to speed $2U$, 
is given by
\[
\lambda X_1 = 2\ln \tfrac 9 5 - 1
\,,
\]
where $\lambda= P/(16mU^3)$.

\item
Given instead that  the  resistance is proportional to the square 
of the car's speed, show that 
the distance $X_2$ travelled  by the car while it accelerates from 
speed  $U$ to speed $2U$
is given by
\[
\lambda X_2 = \tfrac43 \ln \tfrac 98
\,.
\] 
\item
Given that  $3.17<\ln 24 < 3.18$ and $1.60<\ln 5 < 1.61$,
determine which is the larger of 
$X_1$ and $X_2$.
\end{questionparts} 

\end{question}


\newpage
\section*{Section C: \ \ \ Probability and Statistics}
%%%%%%%%%%%%%% Q12
\begin{question}
Let $X$ be a random variable with mean $\mu$ and standard deviation 
$\sigma$. {\em Chebyshev's inequality}, which you may use without proof,
is 
\[
\P\left(\vert X-\mu\vert > k\sigma\right) \le \frac 1 {k^2}
\,,
\]
where $k$ is any positive number.
\begin{questionparts}
\item

The probability of a biased coin landing heads up is $0.2$. It 
is thrown $100n$ times,  where~$n $ is an integer greater than 1. 
Let $\alpha $ be the probability that
the coin lands heads up $N$ times,  where  $16n \le N \le 24n$.

Use Chebyshev's inequality to show that
\[
\alpha \ge  1-\frac 1n
\,.
\]
\item
Use Chebyshev's inequality to show that
\[
1+ n + \frac{n^2}{ 2!} + \cdots + \frac {n^{2n}}{(2n)!} \ge 
\left(1-\frac1n\right)  \e^n
\,.
\]
\end{questionparts}

\end{question}




%%%%%%%%%%%%%% Q13
\begin{question}

Given a random variable $X$
 with mean $\mu$ and standard deviation $\sigma$, 
we define the {\em kurtosis},~ $\kappa$, of $X$
by
\[
\kappa = \frac{ \E\big((X-\mu)^4\big)}{\sigma^4} -3 \,.
\]
Show 
that the random variable $X-a$, where $a$ is a constant, has the same
kurtosis as $X$.

 

\begin{questionparts}   
\item
Show 
by integration
that a random variable which
is Normally distributed   with mean~0 has kurtosis~0.

\item Let $Y_1$, $Y_2$, $\ldots$\,, $Y_n$ be $n$ independent, identically 
distributed, random variables with mean 0, and let 
$T = \sum\limits_{r=1}^n Y_r$. Show that
\[
\E(T^4) =  \sum_{r=1}^n \E(Y_r^4) + 
6 \sum_{r=1}^{n-1}  \sum_{s=r+1}^{n} \E(Y^2_s)
\E(Y^2_r)
\,.
\]

\item
 Let $X_1$, $X_2$, $\ldots$\,, $X_n$ be $n$ independent, identically 
distributed, random variables each with kurtosis $\kappa$. Show that the
kurtosis of their sum is $\dfrac\kappa n\,$.

\end{questionparts}

\end{question}


\end{document}




\end{document}










\nq
For $k>m>0$, express $\dfrac 1 {(k+m)k(k-m)}$ in the form $\dfrac{A}{k(k-m)} +
\dfrac{B} {k(k+m)} $,  where $A$ and $B$ are independent of $k$.

By taking $m=1$,  show that 
\[
\frac 1 {1^3}+\frac 1 {2^3} + \frac 1 {3^3}  + \cdots 
\ < \frac 54
\, .
\]

Show also that
\[
\frac 1{1^3} + \frac 1 {2^3}+ \frac 1 {3^3}  + \cdots \ < \frac {119}{96} \,.
\]
\nq
For $k>m>0$, express $\dfrac 1 {(k+m)k(k-m)}$ in the form $\dfrac{A}{k(k-m)} +
\dfrac{B} {k(k+m)} $, where $A$ and $B$ are independent of $k$.

By taking $m=1$,  show that 
\[
\frac 1 {1^3}+\frac 1 {2^3} + \frac 1 {3^3}  + \cdots 
\ < \frac 54
\, .
\]

Show also that
\[
\frac 1{1^3} + \frac 1 {2^3}+ \frac 1 {3^3}  + \cdots \ < \frac {119}{96} \,.
\]

