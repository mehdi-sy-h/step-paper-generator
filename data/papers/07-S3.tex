\documentclass[a4, 11pt]{report}


\pagestyle{myheadings}
\markboth{}{Paper III, 2007
\ \ \ \ \ 
\today 
}               

\RequirePackage{amssymb}
\RequirePackage{amsmath}
\RequirePackage{graphicx}
\RequirePackage{color}
\RequirePackage[flushleft]{paralist}[2013/06/09]



\RequirePackage{geometry}
\geometry{%
  a4paper,
  lmargin=2cm,
  rmargin=2.5cm,
  tmargin=3.5cm,
  bmargin=2.5cm,
  footskip=12pt,
  headheight=24pt}


\newcommand{\comment}[1]{{\bf Comment} {\it #1}}
%\renewcommand{\comment}[1]{}

\newcommand{\bluecomment}[1]{{\color{blue}#1}}
%\renewcommand{\comment}[1]{}
\newcommand{\redcomment}[1]{{\color{red}#1}}



\usepackage{epsfig}
\usepackage{pstricks-add}
\usepackage{tgheros} %% changes sans-serif font to TeX Gyre Heros (tex-gyre)
\renewcommand{\familydefault}{\sfdefault} %% changes font to sans-serif
%\usepackage{sfmath}  %%%% this makes equation sans-serif
%\input RexFigs


\setlength{\parskip}{10pt}
\setlength{\parindent}{0pt}

\newlength{\qspace}
\setlength{\qspace}{20pt}


\newcounter{qnumber}
\setcounter{qnumber}{0}

\newenvironment{question}%
 {\vspace{\qspace}
  \begin{enumerate}[\bfseries 1\quad][10]%
    \setcounter{enumi}{\value{qnumber}}%
    \item%
 }
{
  \end{enumerate}
  \filbreak
  \stepcounter{qnumber}
 }


\newenvironment{questionparts}[1][1]%
 {
  \begin{enumerate}[\bfseries (i)]%
    \setcounter{enumii}{#1}
    \addtocounter{enumii}{-1}
    \setlength{\itemsep}{5mm}
    \setlength{\parskip}{8pt}
 }
 {
  \end{enumerate}
 }



\DeclareMathOperator{\cosec}{cosec}
\DeclareMathOperator{\Var}{Var}

\def\d{{\mathrm d}}
\def\e{{\mathrm e}}
\def\g{{\mathrm g}}
\def\h{{\mathrm h}}
\def\f{{\mathrm f}}
\def\p{{\mathrm p}}
\def\s{{\mathrm s}}
\def\t{{\mathrm t}}


\def\A{{\mathrm A}}
\def\B{{\mathrm B}}
\def\E{{\mathrm E}}
\def\F{{\mathrm F}}
\def\G{{\mathrm G}}
\def\H{{\mathrm H}}
\def\P{{\mathrm P}}


\def\bb{\mathbf b}
\def \bc{\mathbf c}
\def\bx {\mathbf x}
\def\bn {\mathbf n}

\newcommand{\low}{^{\vphantom{()}}}
%%%%% to lower suffices: $X\low_1$ etc


\newcommand{\subone}{ {\vphantom{\dot A}1}}
\newcommand{\subtwo}{ {\vphantom{\dot A}2}}




\def\le{\leqslant}
\def\ge{\geqslant}


\def\var{{\rm Var}\,}

\newcommand{\ds}{\displaystyle}
\newcommand{\ts}{\textstyle}
\def\half{{\textstyle \frac12}}
\def\l{\left(}
\def\r{\right)}



\begin{document}
\setcounter{page}{2}

 
\section*{Section A: \ \ \ Pure Mathematics}

%%%%%%%%%%Q1
\begin{question}
{\sl In this question, 
do not consider the special cases in which the denominators
of any of your expressions are zero.}

Express $\tan(\theta_1+\theta_2+\theta_3+\theta_4)$ in terms 
of $t_i$, where  $t_1=\tan\theta_1\,$, etc.

Given that $\tan\theta_1$, $\tan\theta_2$, $\tan\theta_3$ and $\tan\theta_4$ 
are the four roots
of the equation \[at^4+bt^3+ct^2+dt+e=0
\]
(where $a\ne0$), find an expression in terms
of $a$, $b$, $c$,  $d$ and $e$ for
$\tan(\theta_1+\theta_2+\theta_3+\theta_4)$.

The four real numbers  $\theta_1$, $\theta_2$, $\theta_3$ and $\theta_4$
lie in the range $0\le \theta_i<2\pi$ and satisfy
the  equation 
\[
p\cos2\theta+\cos(\theta-\alpha)+p=0\,,\]
where $p$  and 
$\alpha$ are independent of~$\theta$. 
Show that
$\theta_1+\theta_2+\theta_3+\theta_4=n\pi$ for some integer $n$.
\end{question}

%%%%%%%%%%Q2
\begin{question}
\begin{questionparts}
\item Show that $1.3.5.7. \;\ldots \;.(2n-1)=\dfrac {(2n)!}{2^n n!}\;$
and  that, for $\vert x \vert
  < \frac14$,
\[
\frac{1}{\sqrt{1-4x\;}\;}
=1+\sum_{n=1}^\infty \frac {(2n)!}{(n!)^2} \, x^n \,.
\]

\item By differentiating the above result, deduce that 
\[
\sum _{n=1}^\infty \frac{(2n)!}{n!\,(n-1)!} 
\left(\frac6{25}\right)^{\!\!n} = 
60
\,.
\]

\item Show that 
\[
\sum _{n=1}^\infty \frac{2^{n+1}(2n)!}{3^{2n}(n+1)!\,n!} 
 = 
1
\,.
\]

\end{questionparts}
\end{question}

%%%%%%%%% Q3
\begin{question}
A sequence of numbers, $F_1$, $F_2$, $\ldots$, is defined by
$
F_1=1$, $F_2=1$, and
\[ 
F_n=F_{n-1}+F_{n-2}\, \text{ \ \ \ for $n\ge 3$}.
\]

\begin{questionparts}
\item    Write down the values of $F_3$, $F_4$, $\ldots$ , $F_8$.
\item  Prove that 
$F^{\vphantom{2}}_{2k+3}F^{\vphantom{2}}_{2k+1} -F_{2k+2}^2 
= -F^{\vphantom{2}}_{2k+2}F^{\vphantom{2}}_{2k}+F_{2k+1}^2\,$.
\item Prove by induction or otherwise that
$
F^{\vphantom{2}}_{2n+1}
F^{\vphantom{2}}_{2n-1}-
F^2_{2n}=1\,$
and deduce that $F^2_{2n}+1\,$ is divisible by $F^{\vphantom{2}}_{2n+1}\,.$
\item  Prove that  $F^2_{2n-1}+1\,$ is divisible by 
$F^{\vphantom{2}}_{2n+1}\,.$
\end{questionparts}
\end{question}

%%%%%% Q4 
\begin{question}
A curve is given parametrically by
\begin{align*}
x&= a\big( \cos t +\ln \tan \tfrac12 t\big)\,,\\
y&= a\sin t\,,
\end{align*}
where $0<t<\frac12 \pi$ and $a$ is a positive constant. Show that 
$\ds \frac{\d y}{\d x} = \tan t$ and sketch the curve.

Let  $P$ be the  point with parameter $t$ and let $Q$ be the 
point where  the tangent 
to the curve at $P$ meets the $x$-axis.
Show that $PQ=a$.

The {\sl radius of curvature}, $\rho$, at $P$ 
 is defined by
\[
\rho=
\frac 
{\big(\dot x ^2+\dot y^2\big)^{\frac32}}
{\vert \dot x \ddot y - \dot y \ddot x\vert   \ \ }
\,,
\]
where the dots denote differentiation with respect to $t$. Show that
$\rho =a\cot t$.

The point  $C$ 
lies on the normal to the curve at $P$, a distance $\rho$ from $P$
and  above the curve.
Show that $CQ$ is parallel to the $y$-axis.
\end{question}

%%%%%%%%% Q5
\begin{question}
Let $y = \ln (x^2-1)\,$, where $x  >1$,  and let
 $r$ and $\theta$ be functions of $x$ determined by
$r= \sqrt{x^2-1}$ and $\coth\theta=  x$.
 Show that 
\[
\frac {\d y}{\d x} = \frac {2\cosh \theta}{r}
\text{ \ \ and \ \ }
\frac {\d^2 y}{\d x^2} = -\frac {2 \cosh 2\theta}{r^2}\,,
\]
and find an expression in terms of $r$ and $\theta$ for 
$\dfrac {\d^3 y}{\d x^3}\,$.
 

Find, with proof, 
 a similar formula for $\dfrac{\d^n y}{\d x^n}$ in terms of 
$r$ and $\theta$.
	\end{question}
	
%%%%%%%%% Q6
\begin{question}
The distinct points $P$, $Q$, $R$ and $S$ in the Argand diagram
lie on a circle of radius $a$ centred at the origin and are represented
by the complex numbers $p$, $q$, $r$ and $s$, respectively.
Show that 
\[
pq = -a^2 \frac {p-q}{p^*-q^*}\,.
\]
Deduce that, if the chords
$PQ$ and $RS$ are perpendicular, then $pq+rs=0$.

The distinct
points $A_1$, $A_2$, $\ldots$, $A_n$ (where $n\ge3$) lie on a circle. The 
points
\hbox{$B_1$, $B_2$, $\ldots$, $B_{n}$} lie on the same circle and are chosen 
so that the chords $B_1B_2$, $B_2B_3$,  $\ldots$, $B_nB_{1}$
are perpendicular, respectively, to the chords 
$A_1A_2$, $A_2A_3$,  $\ldots$, $A_nA_1$.
Show that, for  $n=3$, there are only two choices of $B_1$
for which this is possible. What is the corresponding result for $n=4$?
State the corresponding results  for values of $n$ greater than 4.
\end{question}
	
%%%%%%%%% Q7
\begin{question}
The functions $\s(x)$ ($0\le x<1$) and $\t(x)$ ($x\ge0$), 
and the real number $p$, are defined 
by
\[
\s(x) = \int_0^x \frac 1  {\sqrt{1-u^2}}\, \d u\;, \ \ \ \ 
\t(x) = \int_0^x \frac 1  {1+u^2}\, \d u\;, \ \ \ \ 
p= 2 \int_0^\infty \frac 1 {1+u^2}\, \d u \;.
\]  
For this question,
do not evaluate any of the above integrals explicitly in terms of 
inverse trigonometric functions or the number $\pi$.

\begin{questionparts}
\item  Use the substitution $u=v^{-1}$ to show that
$\ds \t(x) =\int_{1/x}^\infty\frac 1 {1+v^2}\, \d v \, $. Hence evaluate
$\t(1/x) + \t(x)$ in terms of $p$ and deduce that 
$2\t(1)= \frac12 p\,$.

\item 
Let 
$y=\dfrac{u}{\sqrt{1+u^2}}$\;. Express $u$ in terms  of $y$,
and show that 
$
\ds
\frac{\d u}{\d y} = \frac 1 {\sqrt{(1-y^2)^3}}\,.
$

By making a substitution  in the integral for $\t(x)$, show that
\[
\t(x) = \s\left(\frac{x}{\sqrt{1+x^2}}\right)\!.
\]
Deduce that $\s\big(\frac1{\sqrt2}\big) =\frac1 4 p\,$.

\item Let
$z= \dfrac{u+ \frac1{\sqrt3}}{1-\frac 1{\sqrt3}u}\,$.
  Show
that 
$\ds
\t(\tfrac1{\sqrt3}) = \int_{\frac1{\sqrt3}}^{\sqrt3} \frac1 {1+z^2} \,\d z\;,
$
and
hence  that $3\t(\frac1{\sqrt3}) = \frac12 p\,$.
\end{questionparts}
\end{question}
		
%%%%%%%%% Q8
\begin{question}	
\begin{questionparts}
\item Find functions ${\rm a}(x)$ and ${\rm b}(x)$ such that $u=x$ and
  $u=\e^{-x}$
both satisfy the equation 
$$\dfrac{\d^2u}{\d x^2} +{\rm a}(x) \dfrac{\d u}{\d x} + {\rm b} (x)u=0\,.$$ 
For these functions ${\rm a}(x)$ and ${\rm b}(x)$, write 
down the  general solution of the equation.

Show that  the substitution $y= \dfrac 1 {3u} \dfrac {\d u}{\d x}$
transforms the equation
\[
\frac{\d y}{\d x} +3y^2 + \frac {x} {1+x} y = \frac 1 {3(1+x)}
\tag{$*$}
\]
into
\[
\frac{\d^2 u}{\d x^2} +\frac x{1+x} \frac{\d u}{\d x} - \frac 1 {1+x}
u=0
\]
and hence show that the solution of equation ($*$) that satisfies
$y=0$ at $x=0$ is given by
 $y = \dfrac{1-\e^{-x}}{3(x+\e^{-x})}$.

\item
Find the solution of the equation
$$
\frac{\d y}{\d x} +y^2 + \frac x {1-x} y = \frac 1 {1-x}
$$
that satisfies $y=2$ at $x=0$.



\end{questionparts}
\end{question}	
		

		
	
\newpage
\section*{Section B: \ \ \ Mechanics}


	
%%%%%%%%%% Q9
\begin{question}
Two small beads, $A$ and $B$, each of mass $m$, are threaded  
on a smooth  horizontal circular hoop of radius $a$ and centre $O$. 
The angle $\theta$ is the acute angle determined by
$2\theta = \angle AOB$.

The beads are connected by a light  straight spring. 
The energy 
stored in the spring 
is 
\[
mk^2 a^2(\theta - \alpha)^2,
\]
where 
$k$ and $\alpha$ are constants satisfying 
$k>0$ and $\frac \pi 4< \alpha<\frac\pi2$.


The spring is held in compression with $\theta =\beta$
and then
released.
Find the period of oscillations in the two cases that arise according
to the value of $\beta$ and state the value of $\beta$ for which 
oscillations do not occur.
	\end{question}
	
%%%%%%%%%% Q10 
\begin{question}	
A particle is projected  
from a point on  a plane that is inclined
at an angle~$\phi$ to the horizontal. 
The position of the particle
at time $t$ after it is projected is $(x,y)$, where $(0,0)$
is the point of projection,
 $x$ measures 
distance up the line of greatest slope and
 $y$ measures
perpendicular distance from  the plane.
Initially, the velocity of the
particle is given by $(\dot x, \dot y) = (V\cos\theta, V\sin\theta)$, where
$V>0$ and $\phi+\theta<\pi/2\,$. 
Write down expressions
for $x$ and $y$.


The particle  bounces 
on the plane and  returns along the same path  to the
point of projection. Show that
\[2\tan\phi\tan\theta =1\]
and that 
\[
R= \frac{V^2\cos^2\theta}{2g\sin\phi}\,,
\]
where $R$ is the range
along the plane. 

Show further that 
\[
\frac{2V^2}{gR} = 3\sin\phi + {\rm cosec}\,\phi
\]
and
deduce that the largest possible value of $R$
is $V^2/ (\sqrt{3}\,g)\,$.
\end{question}

%%%%%%%%%% Q11

\begin{question}
\begin{questionparts}
\item
A wheel consists of a thin 
light circular rim attached by light spokes of length $a$
to a small hub of mass
$m$. The wheel  rolls without slipping
on a rough horizontal table
  directly towards a straight edge of the table.
The plane of the wheel is vertical throughout the motion.
The speed of the wheel is  $u$, where 
 $u^2<ag\,$.


Show  that, after the wheel reaches the edge of the
table and while it is still in contact with the table,
the frictional force on the wheel is zero.
Show also that the hub
 will fall  a vertical distance $(ag-u^2)/(3g)$
before the rim loses contact with the table. 



\item  Two particles, each of mass $m/2$, are attached
to a light circular hoop of radius $a$, at the ends
of a diameter. The hoop rolls without slipping
on a rough horizontal table
 directly towards a straight edge of the table.
The plane of the hoop is vertical throughout the motion.
When the centre of the hoop is vertically above
the edge of the table it has speed $u$, where 
 $u^2<ag\,$, and 
one particle is vertically above the other. 

Show that,  
 after the hoop  reaches the edge of the
table and while it is still in contact with the table,
the frictional force on the hoop 
is non-zero and  deduce that 
the hoop will slip before it loses contact with the table.

\end{questionparts}
\end{question}
	

	
	\newpage
\section*{Section C: \ \ \ Probability and Statistics}


%%%%%%%%%% Q12
\begin{question}
I choose a number from the integers $1, 2, \ldots$, $(2n-1)$ and
the outcome is the random variable~$N$. Calculate $ \E(N)$ and $\E(N^2)$.

I then  repeat a certain experiment $N$ times, the outcome of the 
$i$th experiment being the random variable $X_i$ ($1\le i \le
N$). For each $i$,
the random variable $X_i$ has mean $\mu$ and variance~$\sigma^2$, and 
$X_i$ is independent of $X_j$ for $i\ne j$ and also independent of $N$.
The random variable $Y$ is defined by $Y= \sum\limits_{i=1}^NX_i$.
Show that $\E(Y)=n\mu$ and that $\mathrm{Cov}(Y,N) = \frac13n(n-1)\mu$. Find 
$\var(Y) $ in terms of $n$, $\sigma^2$ and $\mu$.
\end{question}

%%%%%%%%%% Q13
\begin{question}
A frog  jumps towards a large pond.
  Each jump takes the frog either $1\,$m or $2\,$m nearer to the pond.
The probability of a $1\,$m jump is $p$ and the probability of a
$2\,$m jump is
$q$, where $p+q=1$, the occurence of long and short jumps being 
independent.

\begin{questionparts}
\item
Let $p_n(j)$ be the probability that the frog,
 starting at a point
$(n-\frac12)\,$m away from 
the edge of the pond,
 lands in the pond
for the first time on its $j$th jump.
Show that $p_2(2)=p$.

\item Let $u_n$ be the expected number of jumps,
 starting at a point
$(n-\frac12)\,$m away from 
the edge of the pond,
 required to land in the pond
for the first time. Write down the value of $u_1$.
 By finding first
the relevant values of $p_n(m)$, calculate $u_2$ and show that $u_3=
3-2q+q^2$.

\item
Given that $u_n$ can be expressed in the form $u_n= A(-q)^{n-1} +B +Cn$,
where $A$, $B$ and $C$ are constants
(independent of $n$), show that
 $C= (1+q)^{-1}$ and find $A$ and $B$ in terms of  $q$.
Hence show that, for large $n$, $u_n \approx \dfrac n{p+2q}$ and 
explain carefully why this result is to be expected.
 \end{questionparts}
\end{question}

%%%%%%%%%% Q14
\begin{question}
\begin{questionparts}
\item
My favourite dartboard is  a disc of unit radius and centre $O$. 
I never miss the board, and the probability of my hitting 
any given area of the dartboard is proportional to the area.
Each throw is independent of any other throw.
I throw a dart $n$ times (where $n>1$). Find the expected area of the 
smallest circle, with centre $O$, that encloses all
the $n$ holes made by my dart.


 Find also the expected area of the 
smallest circle, with centre $O$, that encloses all
the  $(n-1)$ holes nearest to $O$.

\item
My other dartboard is a square of side 2 units, with centre
$Q$. I never miss the board, and the probability of my hitting 
any given area of the dartboard is proportional to the area.
Each throw is independent of any other throw.
I 
throw a dart $n$ times (where $n>1$).  
 Find the expected area of the 
smallest square, with centre $Q$, that encloses all
the $n$ holes made by my dart.

\item
Determine,
without detailed calculations, whether 
the expected area of the 
smallest circle, with centre $Q$, on my square dartboard 
that encloses all
the $n$ holes made by my darts
is larger or
smaller than that for my circular dartboard.
\end{questionparts}
\end{question}
	
\end{document}
