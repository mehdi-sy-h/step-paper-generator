
\documentclass[a4, 11pt]{report}


\pagestyle{myheadings}
\markboth{}{Paper III, 1990
\ \ \ \ \ 
\today 
}               

\RequirePackage{amssymb}
\RequirePackage{amsmath}
\RequirePackage{graphicx}
\RequirePackage{color}
\RequirePackage[flushleft]{paralist}[2013/06/09]



\RequirePackage{geometry}
\geometry{%
  a4paper,
  lmargin=2cm,
  rmargin=2.5cm,
  tmargin=3.5cm,
  bmargin=2.5cm,
  footskip=12pt,
  headheight=24pt}


\newcommand{\comment}[1]{{\bf Comment} {\it #1}}
%\renewcommand{\comment}[1]{}

\newcommand{\bluecomment}[1]{{\color{blue}#1}}
%\renewcommand{\comment}[1]{}
\newcommand{\redcomment}[1]{{\color{red}#1}}



\usepackage{epsfig}
\usepackage{pstricks-add}
\usepackage{tgheros} %% changes sans-serif font to TeX Gyre Heros (tex-gyre)
\renewcommand{\familydefault}{\sfdefault} %% changes font to sans-serif
%\usepackage{sfmath}  %%%% this makes equation sans-serif
%\input RexFigs


\setlength{\parskip}{10pt}
\setlength{\parindent}{0pt}

\newlength{\qspace}
\setlength{\qspace}{20pt}


\newcounter{qnumber}
\setcounter{qnumber}{0}

\newenvironment{question}%
 {\vspace{\qspace}
  \begin{enumerate}[\bfseries 1\quad][10]%
    \setcounter{enumi}{\value{qnumber}}%
    \item%
 }
{
  \end{enumerate}
  \filbreak
  \stepcounter{qnumber}
 }


\newenvironment{questionparts}[1][1]%
 {
  \begin{enumerate}[\bfseries (i)]%
    \setcounter{enumii}{#1}
    \addtocounter{enumii}{-1}
    \setlength{\itemsep}{5mm}
    \setlength{\parskip}{8pt}
 }
 {
  \end{enumerate}
 }



\DeclareMathOperator{\cosec}{cosec}
\DeclareMathOperator{\Var}{Var}

\def\d{{\rm d}}
\def\e{{\rm e}}
\def\g{{\rm g}}
\def\h{{\rm h}}
\def\f{{\rm f}}
\def\p{{\rm p}}
\def\s{{\rm s}}
\def\t{{\rm t}}


\def\A{{\rm A}}
\def\B{{\rm B}}
\def\E{{\rm E}}
\def\F{{\rm F}}
\def\G{{\rm G}}
\def\H{{\rm H}}
\def\P{{\rm P}}


\def\bb{\mathbf b}
\def \bc{\mathbf c}
\def\bx {\mathbf x}
\def\bn {\mathbf n}

\newcommand{\low}{^{\vphantom{()}}}
%%%%% to lower suffices: $X\low_1$ etc


\newcommand{\subone}{ {\vphantom{\dot A}1}}
\newcommand{\subtwo}{ {\vphantom{\dot A}2}}




\def\le{\leqslant}
\def\ge{\geqslant}


\def\var{{\rm Var}\,}

\newcommand{\ds}{\displaystyle}
\newcommand{\ts}{\textstyle}




\begin{document}
\setcounter{page}{2}

 
\section*{Section A: \ \ \ Pure Mathematics}

%%%%%%%%%%Q1
\begin{question}
Show, using de Moivre's theorem, or otherwise, that 
\[
\tan9\theta=\frac{t(t^{2}-3)(t^{6}-33t^{4}+27t^{2}-3)}{(3t^{2}-1)(3t^{6}-27t^{4}+33t^{2}-1)},\qquad\mbox{ where }t=\tan\theta.
\]
By considering the equation $\tan9\theta=0,$ or otherwise, obtain
a cubic equation with integer coefficients whose roots are 
\[
\tan^{2}\left(\frac{\pi}{9}\right),\qquad\tan^{2}\left(\frac{2\pi}{9}\right)\qquad\mbox{ and }\qquad\tan^{2}\left(\frac{4\pi}{9}\right).
\]
Deduce the value of 
\[
\tan\left(\frac{\pi}{9}\right)\tan\left(\frac{2\pi}{9}\right)\tan\left(\frac{4\pi}{9}\right).
\]
Show that 
\[
\tan^{6}\left(\frac{\pi}{9}\right)+\tan^{6}\left(\frac{2\pi}{9}\right)+\tan^{6}\left(\frac{4\pi}{9}\right)=33273.
\]
\end{question}

%%%%%%%%%%Q2
\begin{question}
The distinct points $O\,(0,0,0),$ $A\,(a^{3},a^{2},a),$ $B\,(b^{3},b^{2},b)$
and $C\,(c^{3},c^{2},c)$ lie in 3-dimensional space. 

\begin{itemize}
\setlength{\itemsep}{3mm}
\item[\bf (i)] Prove that the lines $OA$ and $BC$ do not intersect. 
\item[\bf (ii)] Given that $a$ and $b$ can vary with $ab=1,$ show that $\angle AOB$
can take any value in the interval $0<\angle AOB<\frac{1}{2}\pi$,
but no others. 
\end{itemize}
\end{question}

%%%%%%%%% Q3
\begin{question}
The elements $a,b,c,d$ belong to the group $G$ with binary operation
$*.$ Show that

\begin{itemize}
\setlength{\itemsep}{3mm}
\item[\bf (i)] if $a,b$ and $a*b$ are of order 2, then $a$ and $b$ commute; 
\item[\bf (ii)] $c*d$ and $d*c$ have the same order; 
\item[\bf (iii)] if $c^{-1}*b*c=b^{r},$ then $c^{-1}*b^{s}*c=b^{sr}$ and $c^{-n}*b^{s}*c^{n}=b^{sr^{n}}.$
\end{itemize}
\end{question}


%%%%%% Q4 

\begin{question}
Given that $\sin\beta\neq0,$ sum the series 
\[
\cos\alpha+\cos(\alpha+2\beta)+\cdots+\cos(\alpha+2r\beta)+\cdots+\cos(\alpha+2n\beta)
\]
and 
\[
\cos\alpha+\binom{n}{1}\cos(\alpha+2\beta)+\cdots+\binom{n}{r}\cos(\alpha+2r\beta)+\cdots+\cos(\alpha+2n\beta).
\]
Given that $\sin\theta\neq0,$ prove that 
\[
1+\cos\theta\sec\theta+\cos2\theta\sec^{2}\theta+\cdots+\cos r\theta\sec^{r}\theta+\cdots+\cos n\theta\sec^{n}\theta=\frac{\sin(n+1)\theta\sec^{n}\theta}{\sin\theta}.
\]

\end{question}


%%%%%%%%% Q5
\begin{question}
Prove that, for any integers $n$ and $r$, with $1\leqslant r\leqslant n,$
\[
\binom{n}{r}+\binom{n}{r-1}=\binom{n+1}{r}.
\]
Hence or otherwise, prove that 
\[
(uv)^{(n)}=u^{(n)}v+\binom{n}{1}u^{(n-1)}v^{(1)}+\binom{n}{2}u^{(n-2)}v^{(2)}+\cdots+uv^{(n)},
\]
where $u$ and $v$ are functions of $x$ and $z^{(r)}$ means $\dfrac{\mathrm{d}^{r}z}{\mathrm{d}x^{r}}$. 


Prove that, if $y=\sin^{-1}y,$ then $(1-x^{2})y^{(n+2)}-(2n+1)xy^{(n+1)}-n^{2}y^{(n)}=0.$  
	\end{question}
	
	%%%%%%%%% Q6
	\begin{question}
The transformation $T$ from $\begin{pmatrix}x\\
y
\end{pmatrix}$ to $\begin{pmatrix}X\\
Y
\end{pmatrix}$ is given by 
\[
\begin{pmatrix}X\\
Y
\end{pmatrix}=\frac{2}{5}\begin{pmatrix}9 & -2\\
-2 & 6
\end{pmatrix}\begin{pmatrix}x\\
y
\end{pmatrix}.
\]
Show that $T$ leaves the vector $\begin{pmatrix}1\\
2
\end{pmatrix}$ unchanged in direction but multiplied by a scalar, and that $\begin{pmatrix}2\\
-1
\end{pmatrix}$ is similarly transformed. 


The circle $C$ whose equation is $x^{2}+y^{2}=1$ transforms under
$T$ to a curve $E$. Show that $E$ has equation 
\[
8X^{2}+12XY+17Y^{2}=80,
\]
and state the area of the region bounded by $E$. Show also that the
greatest value of $X$ on $E$ is $2\sqrt{17/5}.$ 


Find the equation of the tangent to $E$ at the point which corresponds
to the point $\frac{1}{5}(3,4)$ on $C$.
	 \end{question}
	 
	 %%%%%%%%% Q7
\begin{question}
The points $P\,(0,a),$ $Q\,(a,0)$ and $R\,(a,-a)$ lie on the curve
$C$ with cartesian equation 
\[
xy^{2}+x^{3}+a^{2}y-a^{3}=0,\qquad\mbox{ where }a>0.
\]
At each of $P,Q$ and $R$, express $y$ as a Taylor series in $h$,
where $h$ is a small increment in $x$, as far as the term in $h^{2}.$
Hence, or otherwise, sketch the shape of $C$ near each of these points. 


Show that, if $(x,y)$ lies on $C$, then 
\[
4x^{4}-4a^{3}x-a^{4}\leqslant0.
\]
Sketch the graph of $y=4x^{4}-4a^{3}-a^{4}.$


Given that the $y$-axis is an asymptote to $C$, sketch the curve
$C$.
	\end{question}
	
	%%%%%%%%% Q8
	\begin{question}
Let $P,Q$ and $R$ be functions of $x$. Prove that, for any function
$y$ of $x$, the function 
\[
Py''+Qy'+Ry
\]
can be written in the form
$\dfrac{\mathrm{d}}{\mathrm{d}x}(py'+qy),$ where $p$ and $q$ are
functions of $x$, if and only if $P''-Q'+R=0.$ 


Solve the differential equation 
\[
(x-x^{4})y''+(1-7x^{3})y'-9x^{2}y=(x^{3}+3x)\mathrm{e}^{x},
\]
given that when $x=2,y=2\mathrm{e}^{2}$ and $y'=0.$ 
		\end{question}
		
		
%%%%%%%%% Q9
		\begin{question}
The real variables $\theta$ and $u$ are related by the equation
$\tan\theta=\sinh u$ and $0\leqslant\theta<\frac{1}{2}\pi.$ Let
$v=\mathrm{sech}u.$ Prove that 

\begin{itemize}
\setlength{\itemsep}{3mm}
\item[\bf (i)]  $v=\cos\theta;$
\item[\bf (ii)] $\dfrac{\mathrm{d}\theta}{\mathrm{d}u}=v;$ 
\item[\bf (iii)] $\sin2\theta=-2\dfrac{\mathrm{d}v}{\mathrm{d}u}\quad$ and $\quad\cos2\theta=-\cosh u\dfrac{\mathrm{d}^{2}v}{\mathrm{d}u^{2}};$
\item[\bf (iv)] ${\displaystyle \frac{\mathrm{d}u}{\mathrm{d}\theta}\frac{\mathrm{d}^{2}u}{\mathrm{d}\theta^{2}}+\frac{\mathrm{d}v}{\mathrm{d}\theta}\frac{\mathrm{d}^{2}u}{\mathrm{d}\theta^{2}}+\left(\frac{\mathrm{d}u}{\mathrm{d}\theta}\right)^{2}=0.}$ 
\end{itemize}
\end{question}
		
	
%%%%%%%%%% 10
\begin{question}
By considering the graphs of $y=kx$ and $y=\sin x,$ show that the
equation $kx=\sin x,$ where $k>0,$ may have $0,1,2$ or $3$ roots
in the interval $(4n+1)\frac{\pi}{2}<x<(4n+5)\frac{\pi}{2},$
where $n$ is a positive integer. 


For a certain given value of $n$, the equation has exactly one root
in this interval. Show that $k$ lies in an interval which may be
written $\sin\delta<k<\dfrac{2}{(4n+1)\pi},$ where $0<\delta<\frac{1}{2}\pi$
and 
\[
\cos\delta=\left((4n+5)\frac{\pi}{2}-\delta\right)\sin\delta.
\]
Show that, if $n$ is large, then $\delta\approx\dfrac{2}{(4n+5)\pi}$
and obtain a second, improved, approximation. 
\end{question}
			
		
		
		
	
\newpage
\section*{Section B: \ \ \ Mechanics}


	
%%%%%%%%%% Q11
\begin{question}
The points $O,A,B$ and $C$ are the vertices of a uniform square
lamina of mass $M.$ The lamina can turn freely under gravity about
a horizontal axis perpendicular to the plane of the lamina through
$O$. The sides of the lamina are of length $2a.$ When the lamina
is haning at rest with the diagonal $OB$ vertically downwards it
is struck at the midpoint of $OC$ by a particle of mass $6M$ moving
horizontally in the plane of the lamina with speed $V$. The particle
adheres to the lamina. Find, in terms of $a,M$ and $g$, the value
which $V^{2}$ must exceed for the lamina and particle to make complete
revolutions about the axis.  
	\end{question}
	
%%%%%%%%%% Q12
\begin{question}	
A uniform smooth wedge of mass $m$ has congruent triangular end faces
$A_{1}B_{1}C_{1}$ and $A_{2}B_{2}C_{2},$ and $A_{1}A_{2},B_{1}B_{2}$
and $C_{1}C_{2}$ are perpendicular to these faces. The points $A,B$
and $C$ are the midpoints of $A_{1}A_{2},B_{1}B_{2}$ and $C_{1}C_{2}$
respectively. The sides of the triangle $ABC$ have lengths $AB=AC=5a$
and $BC=6a.$ The wedge is placed with $BC$ on a smooth horizontal
table, a particle of mass $2m$ is placed at $A$ on $AC,$ and the
system is released from rest. The particle slides down $AC,$ strikes
the table, bounces perfectly elastically and lands again on the table
at $D$. At this time the point $C$ of the wedge has reached the
point $E$. 

Show that $DE=\frac{192}{19}a.$ 
\end{question}

%%%%%%%%%% Q13

\begin{question}
A particle $P$ is projected, from the lowest point, along the smooth
inside surface of a fixed sphere with centre $O$. It leaves the surface
when $OP$ makes an angle $\theta$ with the upward vertical. Find
the smallest angle that must be exceeded by $\theta$ to ensure that
$P$ will strike the surface below the level of $O$. 


{[}You may find it helpful to find the time at which the particle
strikes the sphere.{]} 
\end{question}
	
%%%%%%%%%% Q14
\begin{question}
	The edges $OA,OB,OC$ of a rigid cube are taken as coordinate axes
	and $O',A',B',C'$ are the vertices diagonally opposite $O,A,B,C,$
	respectively. The four forces acting on the cube are 
	\[
	\begin{pmatrix}\alpha\\
	\beta\\
	\gamma
	\end{pmatrix}\mbox{ at }O\ (0,0,0),\ \begin{pmatrix}\lambda\\
	0\\
	1
	\end{pmatrix}\mbox{ at }O'\ (a,a,a),\ \begin{pmatrix}-1\\
	0\\
	2
	\end{pmatrix}\mbox{ at }B\ (0,a,0),\ \mbox{ and }\begin{pmatrix}1\\
	\mu\\
	\nu
	\end{pmatrix}\mbox{ at }B'\ (a,0,a).
	\]
	The moment of the system about $O$ is zero: find $\lambda,\mu$ and
	$\nu$.
\begin{itemize}
\setlength{\itemsep}{3mm}
\item[\bf (i)]  Given that $\alpha=\beta=\gamma=0,$ find the system consisting of
a single force at $B$ together with a couple which is equivalent
to the given system. 
\item[\bf (ii)] Given that $\alpha=2,\beta=3$ and $\gamma=2,$ find the equation
of the locus about each point of which the moment of the system is
zero. Find the number of units of work done on the cube when it moves
(without rotation) a distance in the direction of this line under
the action of the given forces only.  
\end{itemize}

\end{question}
	
	\newpage
\section*{Section C: \ \ \ Probability and Statistics}


%%%%%%%%%% Q15
\begin{question}
An unbiased twelve-sided die has its faces marked $A,A,A,B,B,B,B,B,B,B,B,B.$
In a series of throws of the die the first $M$ throws show $A,$
the next $N$ throws show $B$ and the $(M+N+1)$th throw shows $A$.
Write down the probability that $M=m$ and $N=n$, where $m\geqslant0$
and $n\geqslant1.$ Find

\begin{itemize}
\setlength{\itemsep}{3mm}
\item[\bf (i)]  the marginal distributions of $M$ and $N$, 
\item[\bf (ii)] the mean values of $M$ and $N$. 
\end{itemize}
Investigate whether $M$ and $N$ are independent. 


Find the probability that $N$ is greater than a given integer $k$,
where $k\geqslant1,$ and find $\mathrm{P}(N>M).$ Find also $\mathrm{P}(N=M)$
and show that $\mathrm{P}(N<M)=\frac{1}{52}.$ 
\end{question}

%%%%%%%%%% Q16
\begin{question}

\begin{questionparts}
\item A rod of unit length is cut into pieces of
length $X$ and $1-X$; the latter is then cut in half. The random
variable $X$ is uniformly distributed over $[0,1].$ For some values
of $X$ a triangle can be formed from the three pieces of the rod.
Show that the conditional probability that, if a triangle can be formed,
it will be obtuse-angled is $3-2\sqrt{2.}$


\item The bivariate distribution of the random variables $X$ and
$Y$ is uniform over the triangle with vertices $(1,0),(1,1)$ and
$(0,1).$ A pair of values $x,y$ is chosen at random from this distribution
and a (perhaps degenerate) triangle $ABC$ is constructed with $BC=x$
and $CA=y$ and $AB=2-x-y.$ Show that the construction is always
possible and that $\angle ABC$ is obtuse if and only if 
\[
y>\frac{x^{2}-2x+2}{2-x}.
\]
Deduce that the probability that $\angle ABC$ is obtuse is $3-4\ln2.$
\end{questionparts}

\end{question}
\end{document}
