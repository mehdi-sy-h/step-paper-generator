\documentclass[a4, 11pt]{report}


\pagestyle{myheadings}
\markboth{}{Paper II, 1992
\ \ \ \ \ 
\today 
}               

\RequirePackage{amssymb}
\RequirePackage{amsmath}
\RequirePackage{graphicx}
\RequirePackage{color}
\RequirePackage[flushleft]{paralist}[2013/06/09]



\RequirePackage{geometry}
\geometry{%
  a4paper,
  lmargin=2cm,
  rmargin=2.5cm,
  tmargin=3.5cm,
  bmargin=2.5cm,
  footskip=12pt,
  headheight=24pt}


\newcommand{\comment}[1]{{\bf Comment} {\it #1}}
%\renewcommand{\comment}[1]{}

\newcommand{\bluecomment}[1]{{\color{blue}#1}}
%\renewcommand{\comment}[1]{}
\newcommand{\redcomment}[1]{{\color{red}#1}}



\usepackage{epsfig}
\usepackage{pstricks-add}
\usepackage{tgheros} %% changes sans-serif font to TeX Gyre Heros (tex-gyre)
\renewcommand{\familydefault}{\sfdefault} %% changes font to sans-serif
%\usepackage{sfmath}  %%%% this makes equation sans-serif
%\input RexFigs


\setlength{\parskip}{10pt}
\setlength{\parindent}{0pt}

\newlength{\qspace}
\setlength{\qspace}{20pt}


\newcounter{qnumber}
\setcounter{qnumber}{0}

\newenvironment{question}%
 {\vspace{\qspace}
  \begin{enumerate}[\bfseries 1\quad][10]%
    \setcounter{enumi}{\value{qnumber}}%
    \item%
 }
{
  \end{enumerate}
  \filbreak
  \stepcounter{qnumber}
 }


\newenvironment{questionparts}[1][1]%
 {
  \begin{enumerate}[\bfseries (i)]%
    \setcounter{enumii}{#1}
    \addtocounter{enumii}{-1}
    \setlength{\itemsep}{5mm}
    \setlength{\parskip}{8pt}
 }
 {
  \end{enumerate}
 }



\DeclareMathOperator{\cosec}{cosec}
\DeclareMathOperator{\Var}{Var}

\def\d{{\rm d}}
\def\e{{\rm e}}
\def\g{{\rm g}}
\def\h{{\rm h}}
\def\f{{\rm f}}
\def\p{{\rm p}}
\def\s{{\rm s}}
\def\t{{\rm t}}


\def\A{{\rm A}}
\def\B{{\rm B}}
\def\E{{\rm E}}
\def\F{{\rm F}}
\def\G{{\rm G}}
\def\H{{\rm H}}
\def\P{{\rm P}}


\def\bb{\mathbf b}
\def \bc{\mathbf c}
\def\bx {\mathbf x}
\def\bn {\mathbf n}

\newcommand{\low}{^{\vphantom{()}}}
%%%%% to lower suffices: $X\low_1$ etc


\newcommand{\subone}{ {\vphantom{\dot A}1}}
\newcommand{\subtwo}{ {\vphantom{\dot A}2}}




\def\le{\leqslant}
\def\ge{\geqslant}


\def\var{{\rm Var}\,}

\newcommand{\ds}{\displaystyle}
\newcommand{\ts}{\textstyle}




\begin{document}
\setcounter{page}{2}

 
\section*{Section A: \ \ \ Pure Mathematics}

%%%%%%%%%%Q1
\begin{question}
Find the limit, as $n\rightarrow\infty,$ of each of the following.
You should explain your reasoning briefly. 
\begin{alignat*}{4}
\mathbf{(i)\ \ } & \dfrac{n}{n+1}, & \qquad & \mathbf{(ii)\ \ } & \dfrac{5n+1}{n^{2}-3n+4}, & \qquad & \mathbf{(iii)\ \ } & \dfrac{\sin n}{n},\\
\\
\mathbf{(iv)\ \ } & \dfrac{\sin(1/n)}{(1/n)}, &  & \mathbf{(v)}\ \  & (\arctan n)^{-1}, &  & \mathbf{(vi)\ \ } & \dfrac{\sqrt{n+1}-\sqrt{n}}{\sqrt{n+2}-\sqrt{n}}.
\end{alignat*}
\end{question}

%%%%%%%%%%Q2
\begin{question}
Suppose that $y$ satisfies the differential equation 
\[
y=x\frac{\mathrm{d}y}{\mathrm{d}x}-\cosh\left(\frac{\mathrm{d}y}{\mathrm{d}x}\right).\tag{\ensuremath{*}}
\]
By differentiating both sides of $(*)$ with respect to $x$, show
that either 
\[
\frac{\mathrm{d}^{2}y}{\mathrm{d}x^{2}}=0\qquad\mbox{ or }\qquad x-\sinh\left(\frac{\mathrm{d}y}{\mathrm{d}x}\right)=0.
\]
Find the general solutions of each of these two equations. Determine
the solutions of $(*)$.
\end{question}

%%%%%%%%% Q3
\begin{question}
In the figure, the large circle with centre $O$ has radius $4$ and
the small circle with centre $P$ has radius $1$. The small circle
rolls around the inside of the larger one. When $P$ was on the line
$OA$ (before the small circle began to roll), the point $B$ was
in contact with the point $A$ on the large circle. 


\noindent \begin{center}
\psset{xunit=0.7cm,yunit=0.7cm,algebraic=true,dotstyle=o,dotsize=3pt 0,linewidth=0.5pt,arrowsize=3pt 2,arrowinset=0.25} \begin{pspicture*}(-4.56,-4.33)(4.85,4.44) \pscircle(0,0){2.8} \pscircle(2.16,2.08){0.7} \psline(0,0)(4,0) \psline(0,0)(3.66,3.54) \rput[tl](-0.61,-0.05){$O$} \rput[tl](4.3,0.1){$A$} \pscustom{\parametricplot{0.0}{0.7687330396835076}{0.65*cos(t)+0|0.65*sin(t)+0}\lineto(0,0)\closepath} \rput[tl](2.19,2.03){$P$} \parametricplot{1.9074711461816238}{3.0492633188581943}{1*0.71*cos(t)+0*0.71*sin(t)+2.2|0*0.71*cos(t)+1*0.71*sin(t)+2.11} \psline{->}(1.97,2.78)(2.33,2.87) \rput[tl](0.5,2.34){$B$} \rput[tl](0.8,0.71){$\phi$} \begin{scriptsize} \psdots[dotstyle=*](0,0) \psdots[dotstyle=*](2.16,2.08) \psdots[dotstyle=*](1.16,2.12) \end{scriptsize} \end{pspicture*}
\par\end{center}


Sketch the curve $C$ traced by $B$ as the circle rolls. Show that
if we take $O$ to be the origin of cartesian coordinates and the
line $OA$ to be the $x$-axis (so that $A$ is the point $(4,0)$)
then $B$ is the point 
\[
(3\cos\phi+\cos3\phi,3\sin\phi-\sin3\phi).
\]
It is given that the area of the region enclosed by the curve $C$
is 
\[
\int_{0}^{2\pi}x\frac{\mathrm{d}y}{\mathrm{d}\phi}\,\mathrm{d}\phi,
\]
where $B$ is the point $(x,y).$ Calculate this area.
\end{question}

%%%%%% Q4 
\begin{question}
$\lozenge$ is an operation which take polynomials in $x$ to polynomials
in $x$; that is, given a polynomial $\mathrm{h}(x)$ there is another
polynomial called $\lozenge\mathrm{h}(x)$. It is given that, if $\mathrm{f}(x)$
and $\mathrm{g}(x)$ are any two polynomials in $x$, the following
are always true: 
\begin{itemize}
\setlength{\itemsep}{3mm}
\item[\bf (i)] $\lozenge(\mathrm{f}(x)\mathrm{g}(x))=\mathrm{g}(x)\lozenge\mathrm{f}(x)+\mathrm{f}(x)\lozenge\mathrm{g}(x),$ 
\item[\bf (ii)] $\lozenge(\mathrm{f}(x)+\mathrm{g}(x))=\lozenge\mathrm{f}(x)+\lozenge\mathrm{g}(x),$ 
\item[\bf (iii)] $\lozenge x=1$
\item[\bf (iv)] if $\lambda$ is a constant then $\lozenge(\lambda\mathrm{f}(x))=\lambda\lozenge\mathrm{f}(x).$

\end{itemize}

Show that, if $\mathrm{f}(x)$ is a constant (i.e., a polynomial of
degree zero), then $\lozenge\mathrm{f}(x)=0.$ 

Calculate $\lozenge x^{2}$ and $\lozenge x^{3}.$ Prove that $\lozenge\mathrm{h}(x)=\dfrac{\mathrm{d}}{\mathrm{d}x}(\mathrm{h}(x))$
for any polynomial $\mathrm{h}(x)$. 
	\end{question}

%%%%%%%%% Q5
\begin{question}
	Explain what is meant by the order of an element $g$ of a group $G$. 


	The set $S$ consists of all $2\times2$ matrices whose determinant
	is $1$. Find the inverse of the element $\mathbf{A}$ of $S$, where
	\[
	\mathbf{A}=\begin{pmatrix}w & x\\
	y & x
	\end{pmatrix}.
	\]
	Show that $S$ is a group under matrix multiplication (you may assume
	that matrix multiplication is associative). For which elements $\mathbf{A}$
	is $\mathbf{A}^{-1}=\mathbf{A}$? Which element or elements have order
	2? Show that the element $\mathbf{A}$ of $S$ has order 3 if, and
	only if, $w+z+1=0.$ Write down one such element.
	\end{question}
	
	%%%%%%%%% Q6
	\begin{question}
Sketch the graphs of $y=\sec x$ and $y=\ln(2\sec x)$ for $0\leqslant x\leqslant\frac{1}{2}\pi$.
Show graphically that the equation 
\[
kx=\ln(2\sec x)
\]
has no solution with $0\leqslant x<\frac{1}{2}\pi$ if $k$ is a small
positive number but two solutions if $k$ is large. 


Explain why there is a number $k_{0}$ such that 
\[
k_{0}x=\ln(2\sec x)
\]
has exactly one solution with $0\leqslant x<\frac{1}{2}\pi$. Let
$x_{0}$ be this solution, so that $0\leqslant x_{0}<\frac{1}{2}\pi$
and $k_{0}x_{0}=\ln(2\sec x)$. Show that 
\[
x_{0}=\cot x_{0}\ln(2\sec x_{0}).
\]
Use any appropriate method to find $x_{0}$ correct to two decimal
places. Hence find an approximate value for $k_{0}$. 
	\end{question}
	
	%%%%%%%%% Q7
	\begin{question}
The cubic equation 
\[
x^{3}-px^{2}+qx-r=0
\]
has roots $a,b$ and $c$. Express $p,q$ and $r$ in terms of $a,b$
and $c$. 


\begin{questionparts}
\item If $p=0$ and two of the roots are equal to each other, show
that 
\[
4q^{3}+27r^{2}=0.
\]
\item Show that, if two of the roots of the original equation are
equal to each other, then 
\[
4\left(q-\frac{p^{2}}{3}\right)^{3}+27\left(\frac{2p^{3}}{27}-\frac{pq}{3}+r\right)^{2}=0.
\]
\end{questionparts}
\end{question}
		
%%%%%%%%% Q8
	\begin{question}	
Calculate the following integrals


\begin{questionparts}
\item ${\displaystyle \int\frac{x}{(x-1)(x^{2}-1)}\,\mathrm{d}x}$; 
\item ${\displaystyle \int\frac{1}{3\cos x+4\sin x}\,\mathrm{d}x}$; 
\item ${\displaystyle \int\frac{1}{\sinh x}\,\mathrm{d}x}.$ 
\end{questionparts}
\end{question}	
		
%%%%%%%%%% Q9
\begin{question}
Let $\mathbf{a},\mathbf{b}$ and $\mathbf{c}$ be the position vectors
of points $A,B$ and $C$ in three-dimensional space. Suppose that
$A,B,C$ and the origin $O$ are not all in the same plane. Describe
the locus of the point whose position vector $\mathbf{r}$ is given
by 
\[
\mathbf{r}=(1-\lambda-\mu)\mathbf{a}+\lambda\mathbf{b}+\mu\mathbf{c},
\]
where $\lambda$ and $\mu$ are scalar parameters. By writing this
equation in the form $\mathbf{r}\cdot\mathbf{n}=p$ for a suitable
vector $\mathbf{n}$ and scalar $p$, show that 
\[
-(\lambda+\mu)\mathbf{a}\cdot(\mathbf{b}\times\mathbf{c})+\lambda\mathbf{b}\cdot(\mathbf{c}\times\mathbf{a})+\mu\mathbf{c}\cdot(\mathbf{a}\times\mathbf{b})=0
\]
for all scalars $\lambda,\mu.$ 

Deduce that 
\[
\mathbf{a}\cdot(\mathbf{b}\times\mathbf{c})=\mathbf{b}\cdot(\mathbf{c}\times\mathbf{a})=\mathbf{c}\cdot(\mathbf{a}\times\mathbf{b}).
\]
Say briefly what happens if $A,B,C$ and $O$ are all in the same
plane. 
\end{question}
			
		
			
%%%%%%%%%% Q10
\begin{question}
Let $\alpha$ be a fixed angle, $0<x\leqslant\frac{1}{2}\pi.$ In
each of the following cases, sketch the locus of $z$ in the Argand
diagram (the complex plane): 


\begin{questionparts}
\item ${\displaystyle \arg\left(\frac{z-1}{z}\right)=\alpha,}$
\item ${\displaystyle \arg\left(\frac{z-1}{z}\right)=\alpha-\pi,}$
\item $\left|\dfrac{z-1}{z}\right|=1.$ 
\end{questionparts}


Let $z_{1},z_{2},z_{3}$ and $z_{4}$ be four points lying (in that
order) on a circle in the Argand diagram. If 
\[
w=\frac{(z_{1}-z_{2})(z_{3}-z_{4})}{(z_{4}-z_{1})(z_{2}-z_{3})}
\]
show, by considering $\arg w$, that $w$ is real. 
\end{question}
		
	
\newpage
\section*{Section B: \ \ \ Mechanics}


	
%%%%%%%%%% Q11
\begin{question}
I am standing next to an ice-cream van at a distance $d$ from the
top of a vertical cliff of height $h$. It is not safe for me to go
any nearer to the top of the cliff. My niece Padma is on the broad
level beach at the foot of the cliff. I have just discovered that
I have left my wallet with her, so I cannot buy her an ice-cream unless
she can throw the wallet up to me. She can throw it at speed $V$,
at any angle she chooses and from anywhere on the beach. Air resistance
is negligible; so is Padma's height compared to that of the cliff.
Show that she can throw the wallet to me if and only if 
\[
V^{2}\geqslant g(2h+d).
\]
	\end{question}
	
%%%%%%%%%% Q12
\begin{question}	
In the figure, $W_{1}$ and $W_{2}$ are wheels, both of radius $r$.
Their centres $C_{1}$ and $C_{2}$ are fixed at the same height,
a distance $d$ apart, and each wheel is free to rotate, without friction,
about its centre. Both wheels are in the same vertical plane. Particles
of mass $m$ are suspended from $W_{1}$ and $W_{2}$ as shown, by
light inextensible strings would round the wheels. A light elastic
string of natural length $d$ and modulus elasticity $\lambda$ is
fixed to the rims of the wheels at the points $P_{1}$ and $P_{2}.$
The lines joining $C_{1}$ to $P_{1}$ and $C_{2}$ to $P_{2}$ both
make an angle $\theta$ with the vertical. The system is in equilibrium. 


\noindent \begin{center}
\psset{xunit=1.0cm,yunit=1.0cm,algebraic=true,dimen=middle,dotstyle=o,dotsize=3pt 0,linewidth=0.5pt,arrowsize=3pt 2,arrowinset=0.25} \begin{pspicture*}(-4.94,-1.8)(5.1,3.96) \psline(-4,1)(4,1) \pscircle(-3,2){1.41} \pscircle(3,2){1.41} \psline(-3,2)(-4,1) \psline(-3,2)(-3,0.59) \psline(3,2)(4,1) \psline(3,2)(3,0.59) \psline(1.59,1.98)(1.58,-1) \psline(-1.59,1.96)(-1.6,-1.06) \parametricplot{-2.356194490192345}{-1.5707963267948966}{0.4*cos(t)+-3|0.4*sin(t)+2} \parametricplot{-1.5707963267948966}{-0.7853981633974483}{0.4*cos(t)+3|0.4*sin(t)+2} \rput[tl](-2.88,2.24){$C_1$} \rput[tl](3.22,2.24){$C_2$} \rput[tl](-4.58,1.04){$P_1$} \rput[tl](4.32,1.02){$P_2$} \rput[tl](-1.7,-1.36){$m$} \rput[tl](1.44,-1.4){$m$} \rput[tl](-3.24,3.88){$W_1$} \rput[tl](2.76,3.86){$W_2$} \rput[tl](-3.42,1.5){$\theta$} \rput[tl](3.14,1.52){$\theta$} \begin{scriptsize} \psdots[dotstyle=*](-4,1) \psdots[dotstyle=*](4,1) \psdots[dotstyle=*](1.58,-1) \psdots[dotstyle=*](-1.6,-1.06) \end{scriptsize} \end{pspicture*}
\par\end{center}


\vspace{-0.5cm}
Show that
\[ \sin2\theta=\frac{mgd}{\lambda r}.
\]For what value or values of
$\lambda$ (in terms of $m,d,r$ and $g$) are there


\begin{itemize}
\setlength{\itemsep}{3mm}
\item[\bf (i)]  no equilibrium positions, 
\item[\bf (ii)] just one equilibrium position, 
\item[\bf (iii)] exactly two equilibrium positions, 
\item[\bf (iv)] more than two equilibrium positions?
\end{itemize}
\end{question}

%%%%%%%%%% Q13

\begin{question}
Two particles $P_{1}$ and $P_{2}$, each of mass $m$, are joined
by a light smooth inextensible string of length $\ell.$ $P_{1}$
lies on a table top a distance $d$ from the edge, and $P_{2}$ hangs
over the edge of the table and is suspended a distance $b$ above
the ground. The coefficient of friction between $P_{1}$ and the table
top is $\mu,$ and $\mu<1$. The system is released from rest. Show
that $P_{1}$ will fall off the edge of the table if and only if 
\[
\mu<\frac{b}{2d-b}.
\]
Suppose that $\mu>b/(2d-b)$ , so that $P_{1}$ comes to rest on the
table, and that the coefficient of restitution between $P_{2}$ and
the floor is $e$. Show that, if $e>1/(2\mu),$ then $P_{1}$ comes
to rest before $P_{2}$ bounces a second time. 




\end{question}
	
%%%%%%%%%% Q14
\begin{question}
\noindent \begin{center}
\psset{xunit=1.0cm,yunit=1.0cm,algebraic=true,dimen=middle,dotstyle=o,dotsize=3pt 0,linewidth=0.5pt,arrowsize=3pt 2,arrowinset=0.25} \begin{pspicture*}(-3.36,-3.71)(5.32,4.49) \pspolygon[linewidth=0pt,linecolor=white,hatchcolor=black,fillstyle=hlines,hatchangle=45.0,hatchsep=0.19](-3,4.22)(-3,4)(5,4)(5,4.22) \pscircle(-1,2){1} \pscircle(3,2){1} \pscircle(1,-1){1} \psline(0,2)(0,-1) \psline(2,2)(2,-1) \psline(-2,2)(-2,-1) \psline(4,2)(4,-1) \psline{->}(-2,-1.44)(-2,-2) \rput[tl](-2.25,-2.31){$m_1g$} \psline{->}(4,-1.44)(4,-2) \rput[tl](3.74,-2.25){$m_2g$} \psline{->}(1,-1)(1.02,-2.78) \rput[tl](0.72,-3.06){$m_3g$} \psline(-1,2)(-1,4) \psline(3,2)(3,4) \psline(-3,4)(5,4) \rput[tl](-1.19,1.67){$P_1$} \rput[tl](2.83,1.64){$P_2$} \rput[tl](0.83,-0.5){$P_3$} \begin{scriptsize} \psdots[dotstyle=*](-1,2) \psdots[dotstyle=*](3,2) \psdots[dotstyle=*](1,-1) \psdots[dotstyle=*](-2,-1) \psdots[dotstyle=*](4,-1) \end{scriptsize} \end{pspicture*}
\par\end{center}


\noindent In the diagram $P_{1}$ and $P_{2}$ are smooth light pulleys
fixed at the same height, and $P_{3}$ is a third smooth light pulley,
freely suspended. A smooth light inextensible string runs over $P_{1},$
under $P_{3}$ and over $P_{2},$ as shown: the parts of the string
not in contact with any pulley are vertical. A particle of mass $m_{3}$
is attached to $P_{3}.$ There is a particle of mass $m_{1}$ attached
to the end of the string below $P_{1}$ and a particle of mass $m_{2}$
attached to the other end, below $P_{2}.$ The system is released
from rest. Find the tension in the string, and show that the pulley
$P_{3}$ will remain at rest if 
\[
4m_{1}m_{2}=m_{3}(m_{1}+m_{2}).
\]
\end{question}
	
	\newpage
\section*{Section C: \ \ \ Probability and Statistics}


%%%%%%%%%% Q15
\begin{question}
A point moves in unit steps on the $x$-axis starting from the origin.
At each step the point is equally likely to move in the positive or
negative direction. The probability that after $s$ steps it is at
one of the points $x=2,x=3,x=4$ or $x=5$ is $\mathrm{P}(s).$ Show
that $\mathrm{P}(5)=\frac{3}{16},$ $\mathrm{P}(6)=\frac{21}{64}$
and 
\[
\mathrm{P}(2k)=\binom{2k+1}{k-1}\left(\frac{1}{2}\right)^{2k}
\]
where $k$ is a positive integer. Find a similar expression for $\mathrm{P}(2k+1).$ 


Determine the values of $s$ for which $\mathrm{P}(s)$ has its greatest
value. 
\end{question}

%%%%%%%%%% Q16
\begin{question}

A taxi driver keeps a packet of toffees and a packet of mints in her
taxi. From time to time she takes either a toffee (with probability
$p$) or mint (with probability $q=1-p$). At the beginning of the
week she has $n$ toffees and $m$ mints in the packets. On the $N$th
occasion that she reaches for a sweet, she discovers (for the first
time) that she has run out of that kind of sweet. What is the probability
that she was reaching for a toffee?
\end{question}
\end{document}
