\documentclass[a4, 11pt]{report}


\pagestyle{myheadings}
%\markboth{}{Paper I, 2018 second vetter draft  
%\ \ \ \ \ \
%\today
%}               


\usepackage{pstricks-add}
\usepackage{epsfig}

\RequirePackage{amssymb}
\RequirePackage{amsmath}
\RequirePackage{graphicx}
\RequirePackage{color}
\RequirePackage{xcolor}


\RequirePackage[flushleft]{paralist}[2013/06/09]



\RequirePackage{geometry}
\geometry{%
  a4paper,
  lmargin=2cm,
  rmargin=2.5cm,
  tmargin=3.5cm,
  bmargin=2.5cm,
  footskip=12pt,
  headheight=24pt}
\usepackage{verbatim}


\newcommand{\bct}[1]{{\color{blue}#1}}
%\renewcommand{\comment}[1]{}
\newcommand{\rct}[1]{{\color{red}#1}}

\setlength{\parskip}{10pt}
\setlength{\parindent}{0pt}

\newlength{\qspace}
\setlength{\qspace}{20pt}
\newcounter{qnumber}
\setcounter{qnumber}{0}

\newenvironment{question}%
 {\vspace{\qspace}
  \begin{enumerate}[\bfseries 1\quad][10]%
    \setcounter{enumi}{\value{qnumber}}%
    \item%
 }
{
  \end{enumerate}
  \filbreak
  \stepcounter{qnumber}
 }

\newenvironment{questionparts}[1][1]%
 {
  \begin{enumerate}[\bfseries (i)]%
    \setcounter{enumii}{#1}
    \addtocounter{enumii}{-1}
    \setlength{\itemsep}{5mm}
    \setlength{\parskip}{8pt}
 }
 {
  \end{enumerate}
 }


\DeclareMathOperator{\cosec}{cosec}
\DeclareMathOperator{\Var}{Var}

\def\c{{\rm c}}
\def\d{{\rm d}}
\def\e{{\rm e}}
\def\g{{\rm g}}
\def\h{{\rm h}}
\def\f{{\rm f}}
\def\p{{\rm p}}
\def\q{{\rm q}}
\def\s{{\rm s}}
\def\t{{\rm t}}
\def\z{{\rm z}}


\def\A{{\rm A}}
\def\B{{\rm B}}
\def\E{{\rm E}}
\def\F{{\rm F}}
\def\G{{\rm G}}
\def\H{{\rm H}}
\def\P{{\rm P}}


\def\bb {\mathbf b}
\def\bc {\mathbf c}
\def\bx {\mathbf x}
\def\bn {\mathbf n}

\makeatletter
\newcommand{\raisemath}[1]{\mathpalette{\raisem@th{#1}}}
\newcommand{\raisem@th}[3]{\raisebox{#1}{$#2#3$}}
\makeatother
%%%To raise suffices: e.g.  $\Pi_{\raisemath{2pt}{-}}$.


\def\le{\leqslant}
\def\ge{\geqslant}


\def\var{{\rm Var}\,}

\newcommand{\ds}{\displaystyle}
\newcommand{\ts}{\textstyle}









\begin{document}

\setcounter{page}{2}

\section*{Section A: \ \ \ Pure Mathematics}

%%%%%%%%%%%%%%%%Q1
\begin{question}


The line $y=a^2 x$ 
and  the curve 
 $y=x(b-x)^2$, where $0<a<b\,$, 
intersect at the origin 
$O$ and at points $P$ and $Q $. The $x$-coordinate of 
$P$ is less than the $x$-coordinate  of $Q$. 
Find the coordinates of $P$  and $Q$, 
and sketch the line and the
curve   on the same axes. 

Show that the equation
of the tangent to the curve at $P$ is
\[
y = a(3a-2b)x + 2a(b-a)^2
.
\]

This tangent meets the $y$-axis at $R$. 
The area of the region between the curve and the line 
segment $OP$ is denoted by $S$.
Show that 
\[
S= \frac1{12}(b-a)^3(3a+b)\,.
\] 

The area of triangle 
$OPR$ is denoted by~$T$. 
Show  that $S>\frac{1}{3}T\,$.


\end{question}

%%%%%%%%%%%%%%%%%Q2

\begin{question}
If $x=\log_bc\,$, express $c$ in terms of $b$ and $x$
and prove that
$
\dfrac{\log_a c}{\log_a b}
=
\ds \log_b c 
\,$.

\begin{questionparts}
\item Given that $\pi^2 < 10\,$, prove that
\[
\frac{1}{\log_2 \pi}+\frac{1}{\log_5 \pi} > 2\,.
\]
\item Given that $\ds \log_2  \frac{\pi}{\e}  > \frac{1}{5}$ 
and that $\e^2 < 8$, prove that $\ln \pi > \frac{17}{15}\,$.

\item Given that $\e^3 >20$, \,
$\pi^2 < 10\,$ 
 and  $\log_{10}2 >\frac{3}{10}\,$, 
prove that $\ln \pi < \frac{15}{13}\,$.
\end{questionparts}

\end{question}


%%%%%%%%%%%%%%%%Q3
\begin{question}
The points $R$ and $S$ have coordinates $(-a\,,\, 0)$ and $(2a\,,\, 0)$, 
respectively, where $a > 0\,$. 
The point $P$ has coordinates $(x\,,\, y)$ where $y > 0$ and $x < 2a\,$. 
Let $\angle PRS = \alpha $ and $\angle PSR = \beta\,$.

\begin{questionparts}
\item
 Show that, if $\beta = 2 \alpha\,$, then $P$
 lies on the curve $y^2=3(x^2-a^2)\,$.

\item
Find the possible relationships between $\alpha$ and $\beta$ when 
$0 < \alpha < \pi\,$ and
$P$ lies on the curve $y^2=3(x^2-a^2)\,$.

\end{questionparts}

\end{question}




%%%%%%%%%%%%%%%%Q4
\begin{question}
The function  $\f$ is defined  by
\[
\phantom{\ \ \ \ \ \ \ \ \ \ \ \ (x>0, \  \ x\ne1)}
\f(x) 
= 
\frac{1}{x\ln x} 
\left(1 -  (\ln x)^2 \right)^2
\ \ \ \ \ \ \ \ \ \ \  \ (x>0, \  \ x\ne1)
\,.\]
Show that,\, 
when $( \ln x )^2 = 1\,$,\, 
both $\f(x)=0$ and $\f'(x)=0\,$. 

The function $\F$ is defined by
\begin{equation*}
\F(x) 
= 
\begin{cases} 
     \ds \int_{ 1/\text{\footnotesize e}}^x   
\f(t) \; \mathrm{d}t \hfill 
& \text{ for } 0<x<1\,, 
\\[7mm]
      \ds \int_{\text{\footnotesize e}}^x \f(t) \; \mathrm{d}t  \hfill 
& \text{ for } x>1\,. \\
  \end{cases} 
\end{equation*}

\begin{questionparts}
\item Find $\F(x)$ explicitly and hence 
show that $\F(x^{-1})=\F(x)\,$.
\item Sketch 
the curve with equation $y=\F(x)\,$.
%You may assume that $\dfrac{ (\ln x)^k} x\to 0$ as $x\to\infty$ for 
%any constant $k$.

\end{questionparts}

\end{question}

%%%%%%%%%%%%%%%%%%%%Q5


\begin{question}
\begin{questionparts}
\item Write down the most general polynomial of degree 4 that 
leaves a remainder of 1 when divided by any of 
$x-1\,$, $x-2\,$, $x-3\,$  or $x-4\,$.


\item
The polynomial $\P(x)$ has degree $N$, where $N\ge1\,$,  
and satisfies
\[
\P(1) = \P(2) = \cdots = \P(N) =1\,.
\]
Show that $\P(N+1) \ne 1\,$. 

Given that $\P(N+1)= 2\,$, find $\P(N+r)$ where $r$ is a positive integer.
 Find a positive integer $r$, independent of $N$, 
such that  $\P(N+r) = N+r\,$.

\item
The polynomial ${\rm S}(x)$ has degree 4. It has 
integer coefficients and the coefficient of $x^4$ is 1. It
satisfies
\[
{\rm S}(a)    =
{\rm S}(b)    =
{\rm S}(c)    =
{\rm S}(d)    = 2001\,,
\]
where $a$, $b$, $c$ and $d$ are distinct (not necessarily positive)
integers.

\begin{itemize}
\item[{\bf (a)}]
Show that there is no integer $e$
such that ${\rm S}(e) = 2018\,$.

\medskip
\item[{\bf (b)}]
Find  the number of  ways the  (distinct) integers $a$, $b$, $c$ and  $d$ 
can be chosen  such that   
${\rm S}(0) = 2017$ 
 and 
 $a < b<  c< d\,$. 

\end{itemize}
\end{questionparts}
\end{question}



%%%%%%%%%%%%%%Q6
\begin{question}
Use the identity
\[
2 \sin P\,\sin Q = \cos(Q-P)-\cos(Q+P)\,
\] 
to show that
 \[
2\sin\theta
\,\big  (\sin\theta + \sin 3\theta + \cdots + \sin (2n-1)\theta\,\big ) =
 1-\cos 2n\theta
\,.
\]        
\begin{questionparts}
\item Let $A_n$ be the approximation to the area under the 
curve $y=\sin x$ from $x=0$ to $x=\pi$, 
using $n$ rectangular strips each of width 
$\frac{{\displaystyle \pi}}{\displaystyle n}$, such that the
midpoint of the top of each strip lies on the curve.
Show that
\[
A_n \sin \left( \frac{\pi}{2n} \right) = \frac \pi n\,.
\]

\item Let $B_n$ be the approximation to the 
area under the curve $y=\sin x$ 
from $x=0$ to $x=\pi\,$, using the trapezium rule with $n$ 
strips each of width $\frac{\displaystyle \pi}{ \displaystyle n}$.
Show that 
\[B_n \sin \left( \frac{\pi}{2n} \right) 
= 
\frac{\pi}{n} \cos \left( \frac{\pi}{2n} \right)
.
\]
\item Show that
\[
\frac{1}{2}(A_n + B_n) = B_{2n}\,,
\]
and that
\[
 A_n B_{2n} = A^2_{2n}\, .
\]

\end{questionparts}

\end{question}




%%%%%%%%%%Q7
\begin{question}

\begin{questionparts}
\item
 In the cubic equation $x^3-3pqx+pq(p+q)=0\,$, 
where $p$ and $q$ are distinct real numbers, 
use the substitution
\[
x=\frac{pz+q}{z+1}
\]
to show that the equation reduces to $az^3+b = 0\,$, 
where $a$ and $b$ are to be expressed in terms of $p$ and $q$.

\item
Show further that the equation $x^3 - 3cx + d = 0\,$, where $c$ and $d$ 
are non-zero real numbers,  
can be written in the form 
$x^3-3pqx+pq(p+q)=0\,$, 
where $p$ and $q$ are distinct real numbers, provided $d^2 > 4c^3\,$.

\item Find the real root of the cubic equation $x^3+6x-2=0\,$.

\item  Find the roots of the equation $x^3 - 3p^2x +2p^3=0\,$, 
and hence show how the equation $x^3 - 3cx + d = 0$ 
can be solved in the case $d^2 = 4c^3\,$.


\end{questionparts}

\end{question}


%%%%%%%%%%%%%%%Q8
                                                              
\begin{question}
The functions $\s$ and $\c$ satisfy $\s(0)= 0\,$, $\c(0)=1\,$ and
\[
\s'(x) = \c(x)^2
,\]
\[
\c'(x)=-\s(x)^2.
\]
You may assume that $\s$ and $\c$  are uniquely defined  by these conditions.

\begin{questionparts}
\item Show that $\s(x)^3+\c(x)^3$ is constant, and deduce
that
 $\s(x)^3+\c(x)^3=1\,$.

\item
Show that 
\[
\frac{\d   }{\d x} \, \Big( \s(x) \c(x) \Big) = 2\c(x)^3-1
\]
and find (and simplify) an expression in terms of $\c(x)$ for 
$\dfrac{\d  }{\d x} \left( \dfrac{\s(x)}{\c(x)} \right)
$.

\item Find the integrals
\[
\int \s(x)^2 \, \d x 
\ \ \ \ \ \
\text{and}
\ \ \ \ \ \
\int \s(x)^5 \, \d x
\,.
\]
 
\item 
Given that $\s$ has an inverse function, $\s^{-1}$,
use the substitution $u = \s(x)$ to show that
\[
\int \frac{1}{(1-u^3)^{\frac{2}{3}}} \, \d u = \s^{-1}(u) 
\, + 
\text{\,constant}.
\]

\item Find, in terms of $u$, the integrals
\[
\int \frac{1}{{(1-u^3)}^{\frac{4}{3}}} \, \d u 
\ \ \ \ \ \
\text{and}
\ \ \ \ \ \
\int {(1-u^3)}^{\frac{1}{3}} \, \d u \,.
\]

\end{questionparts}

\end{question}




\newpage
\section*{Section B: \ \ \ Mechanics}


%%%%%%%%%%%%%%%%%%Q9
\begin{question}
A straight
road leading to my house consists of two sections. 
The first section is inclined downwards
at a constant angle $\alpha$
 to the horizontal and ends in traffic lights; 
the second section is inclined
upwards at an angle $\beta$ to the horizontal and ends at my house. 
The distance between the traffic lights and my house
is $d$. 

I have a go-kart which I start from rest, pointing downhill,
 a distance $x$ 
from the traffic lights on the downward-sloping section. 
The go-kart is not powered in any way,  
all resistance forces are negligible, and
there is no sudden change of speed as I pass the traffic lights.
Given that I  reach my house,
 show that 
$x \sin \alpha\ge d \sin\beta\,$.

Let $T$ be the total time taken  to 
 reach my house.
Show that
\[
\left(\frac{g\sin\alpha}2 \right)^{\!\frac12} T = 
(1+k) \sqrt{x} - \sqrt{k^2 x -kd\;}  
\,,
\]
where $k = \dfrac{\sin\alpha}{\sin\beta}\,$.

Hence determine, in terms of $d$ and $k$, the value of $x$
 which minimises $T$. [You need not 
justify the fact that the  stationary value
is a minimum.]

\end{question}





%%%%%%%%%%%%%%%%%%Q9 
\setcounter{qnumber}{9}
\begin{question}
A train is made up of two engines, each of mass $M$, and $n$ carriages, 
each of mass $m$. One of the engines is at 
the front of the train, 
and the other is coupled between the $k$th and $(k+1)$th carriages. 
When the train is accelerating along a straight, horizontal track, 
the resistance to the motion of each carriage is $R$ 
 and the driving force on each engine is $D$, 
where $2D >nR\,$.
The tension in the coupling between the 
engine at the front 
and the first carriage is $T$.

\begin{questionparts}
\item Show that  
\[
T = \frac{n(mD+MR)}{nm+2M}\,.
\]

\item Show that $T$ 
is greater than  the tension in any other 
coupling provided that $k> \frac12n\,$.
\item
Show also that, if $k> \frac12n\,$, then at least one of the 
couplings is in compression (that is, there is a negative tension
 in the 
coupling).



\end{questionparts}
\end{question}




%%%%%%%%%%%%%%%%%%Q11

\begin {question}
The point $O$ lies on a rough plane that is inclined at an angle 
$\alpha$
 to the horizontal,
where $\alpha<45^\circ$.
 The point~$A$ lies on the plane
a distance $d$ from $O$ up the line $L$ of greatest slope through $O$.
The point $B$, which is not on the rough plane, 
 lies in the same vertical plane as $O$ and~$A$, 
and
 $AB$ is horizontal.
The distance from $O$ to $B$ is $d$.

A particle $P$ of mass $m$ rests on $L$ between $O$ and $A$. 
One end of a  light inelastic string is attached to $P$. The string
passes over a smooth light pulley fixed at $B$ 
and its other end is attached to a freely 
hanging particle of mass $\lambda m$.

\begin{questionparts}
\item
Show that  the acute angle, $\theta$, between the string and the line $L$ 
satisfies $\alpha \le \theta \le 2\alpha\,$.

\item
Given that $P$ can rest in equilibrium at 
every point on $L$ between  $O$ and $A$, 
show that $2 \lambda \sin \alpha\le   1\,$.
\item
The coefficient of friction  
between $P$ and the plane is $\mu$, and the acute angle 
$\beta$    is given by   $\mu = \tan\beta\,$.
  Show that if $\beta \ge 2 \alpha\,$,
then a necessary condition 
for equilibrium  
to be possible for every position of~$P$ on $L$ between $O$ and $A$
is
\[
\lambda \le
 \frac {\sin(\beta -\alpha)}{\cos(\beta-2\alpha)}
\,.
\]
 Obtain the corresponding result if  
 $\alpha\le  \beta\le 2\alpha\,$.


\end{questionparts}

\end{question}
\newpage
\section*{Section C: \ \ \ Probability and Statistics}


%%%%%%%Q12
\begin{question} A bag contains 
three coins. 
The probabilities of their showing  heads when 
tossed are $p_1$, $p_2$ and $p_3$. 

\begin{questionparts}

\item
A coin is taken at random from the bag and tossed. 
What is the probability that
it shows a  head?
 

\item A coin is taken at random from the bag (containing three coins)
and tossed; 
the coin is returned to the bag and  again a coin is taken at
random from the bag and tossed.
 Let $N_1$ be the random 
variable whose value is the number of heads shown 
on the two tosses. Find the expectation  
of~$N_1$ in terms of $p$, where $p = \frac{1}{3}(p_1+p_2+p_3)\,$,
and show that  $\var(N_1) =2p(1-p)\,$. 

  
\item Two of the coins are taken at random from the 
bag (containing three coins) and tossed. Let $N_2$ be the random variable 
whose value is the number of heads showing on the two coins. 
Find $\E(N_2)$ and $\var(N_2)$.  

\item
Show  that $\var(N_2)\le  \var(N_1)$,  with equality if and only if
 $p_1=p_2=p_3\,$.
 
\end{questionparts}
\end{question}

%%%%%%%%%%%%%%%%%%Q13
\begin{question}
A multiple-choice test consists of five questions. 
For each question, $n$ answers are given ($n\ge2$) only
one of which is correct and  candidates either
attempt the question by choosing one of the $n$ 
given answers or do
not attempt it. 

For each question attempted, candidates receive two marks for the 
correct answer
and  lose one mark for an incorrect answer. 
No marks are gained or lost for questions that are  not attempted.
The pass mark is five.

Candidates A, B and C  don't understand any of the questions 
so, for any question which they attempt, 
they each choose one of the $n$ given answers at random, 
independently of their choices for any other question.



\begin{questionparts}
\item
Candidate A  chooses in advance to attempt exactly $k$ of the five
questions, where $k=0$, 1, 2, 3, 4~or~5. 
Show that, in order to have the greatest probability of passing the test,
 she should choose $k=4\,$.                      

\item
Candidate B chooses at random  the number of questions he will attempt, 
the  six possibilities being equally likely.
Given that  Candidate B passed the test find, in terms of~$n$,  
the probability that 
he attempted exactly four questions. 

%Show that this probability is an increasing function of $n$.

\item For each of the five questions
 Candidate C decides whether to attempt the question
 by tossing a biased  coin. The coin has a probability
of $\frac n{n+1}$ of showing a head, and she attempts the
question if it shows a head. Find the probability, in terms of $n$,
that Candidate C passes the test.

 


\end{questionparts}
\end{question}





%%%%%%%%%%Q13







\end{document}
