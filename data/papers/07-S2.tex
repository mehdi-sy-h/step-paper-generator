\documentclass[a4, 11pt]{report}


\pagestyle{myheadings}
\markboth{}{Paper II, 2007
\ \ \ \ \ 
\today 
}               

\RequirePackage{amssymb}
\RequirePackage{amsmath}
\RequirePackage{graphicx}
\RequirePackage{color}
\RequirePackage[flushleft]{paralist}[2013/06/09]



\RequirePackage{geometry}
\geometry{%
  a4paper,
  lmargin=2cm,
  rmargin=2.5cm,
  tmargin=3.5cm,
  bmargin=2.5cm,
  footskip=12pt,
  headheight=24pt}


\newcommand{\comment}[1]{{\bf Comment} {\it #1}}
%\renewcommand{\comment}[1]{}

\newcommand{\bluecomment}[1]{{\color{blue}#1}}
%\renewcommand{\comment}[1]{}
\newcommand{\redcomment}[1]{{\color{red}#1}}



\usepackage{epsfig}
\usepackage{pstricks-add}
\usepackage{tgheros} %% changes sans-serif font to TeX Gyre Heros (tex-gyre)
\renewcommand{\familydefault}{\sfdefault} %% changes font to sans-serif
%\usepackage{sfmath}  %%%% this makes equation sans-serif
%\input RexFigs


\setlength{\parskip}{10pt}
\setlength{\parindent}{0pt}

\newlength{\qspace}
\setlength{\qspace}{20pt}


\newcounter{qnumber}
\setcounter{qnumber}{0}

\newenvironment{question}%
 {\vspace{\qspace}
  \begin{enumerate}[\bfseries 1\quad][10]%
    \setcounter{enumi}{\value{qnumber}}%
    \item%
 }
{
  \end{enumerate}
  \filbreak
  \stepcounter{qnumber}
 }


\newenvironment{questionparts}[1][1]%
 {
  \begin{enumerate}[\bfseries (i)]%
    \setcounter{enumii}{#1}
    \addtocounter{enumii}{-1}
    \setlength{\itemsep}{5mm}
    \setlength{\parskip}{8pt}
 }
 {
  \end{enumerate}
 }



\DeclareMathOperator{\cosec}{cosec}
\DeclareMathOperator{\Var}{Var}

\def\d{{\mathrm d}}
\def\e{{\mathrm e}}
\def\g{{\mathrm g}}
\def\h{{\mathrm h}}
\def\f{{\mathrm f}}
\def\p{{\mathrm p}}
\def\s{{\mathrm s}}
\def\t{{\mathrm t}}


\def\A{{\mathrm A}}
\def\B{{\mathrm B}}
\def\E{{\mathrm E}}
\def\F{{\mathrm F}}
\def\G{{\mathrm G}}
\def\H{{\mathrm H}}
\def\P{{\mathrm P}}


\def\bb{\mathbf b}
\def \bc{\mathbf c}
\def\bx {\mathbf x}
\def\bn {\mathbf n}

\newcommand{\low}{^{\vphantom{()}}}
%%%%% to lower suffices: $X\low_1$ etc


\newcommand{\subone}{ {\vphantom{\dot A}1}}
\newcommand{\subtwo}{ {\vphantom{\dot A}2}}




\def\le{\leqslant}
\def\ge{\geqslant}


\def\var{{\rm Var}\,}

\newcommand{\ds}{\displaystyle}
\newcommand{\ts}{\textstyle}
\def\half{{\textstyle \frac12}}
\def\l{\left(}
\def\r{\right)}



\begin{document}
\setcounter{page}{2}

 
\section*{Section A: \ \ \ Pure Mathematics}

%%%%%%%%%%Q1
\begin{question}
{\sl In this question, 
you are not required to  justify the accuracy of the approximations.}

\begin{questionparts}
\item Write down the binomial expansion of 
$\ds \left( 1+\frac k {100} \right)^{\!\frac12}$
in ascending powers of $k$, up to and including the
$k^3$ term.

\begin{questionparts}
\item[\bf {(a)}]
Use  the value $k=8$ to find an approximation to five decimal
  places
for $\sqrt{3}\,$.

\item[\bf {(b)}] By choosing a suitable integer value of $k$,
find an approximation to five decimal places for $\sqrt6\,$.
\end{questionparts}
\item
By considering the first two terms of the binomial expansion of 
$\ds \left( 1+\frac k {1000} \right)^{\!\frac13}$, show that
$\dfrac{3029}{2100}$ is an approximation to $\sqrt[3]{3}\,$.
\end{questionparts}
\end{question}

%%%%%%%%%%Q2
\begin{question}
A curve has equation $y=2x^3-bx^2+cx$. It  has a
maximum point at $(p,m)$ and a minimum point at $(q,n)$ where $p>0$ and $n>0$.
Let $R$ be the 
 region enclosed by the curve, the line $x=p$ and the line 
$y=n$.
\begin{questionparts}
\item Express $b$ and $c$ in terms of $p$ and $q$.
\item Sketch the curve. Mark on your sketch the point of inflection
  and shade the region $R$. Describe the symmetry of the curve.
\item  Show that $m-n=(q-p)^3$.
\item  Show that the area of $R$ is $\frac12 (q-p)^4$.
\end{questionparts}
\end{question}

%%%%%%%%% Q3
\begin{question}
By writing $x=a\tan\theta$, show that, for $a\ne0$, 
 $\ds \int \frac 1 {a^2+x^2}\,
\d x =\frac 1 a \arctan \frac x a + \text{constant}\,$.

%[Note: $\arctan$ is another notation for $\tan^{-1}$.]

\begin{questionparts}
\item Let $\ds I=\int_0^{\frac{1}{2}\pi} 
\frac {\cos x}{1+\sin^2 x} \, \d x\,$.

\vspace*{2mm}
{\bf (a)} 
 Evaluate $I$.

{\bf (b)} Use the substitution $t=\tan \frac12 x$ to show that
$\ds \int_0^1 \frac {1-t^2}{1+6t^2+t^4} \, \d t = \tfrac12 I\,$.

\item  Evaluate $\ds \int_0^1 \frac {1-t^2}{1+14t^2+t^4} \, \d t \,$.
\end{questionparts}
\end{question}

%%%%%% Q4 
\begin{question}
Given that $\cos A$, $\cos B$ and $\beta$ are non-zero, show
that
the equation
\[
\alpha \sin(A-B) + \beta \cos(A+B) = \gamma \sin(A+B)
\]
reduces to the form
\[
(\tan A-m)(\tan B-n)=0\,,
\]
where $m$ and $n$ are independent of $A$ and $B$,
if and only if $\alpha^2=\beta^2+\gamma^2$. 

Determine all values of $x$, in the range $0\le x <2\pi$, for which: 
\begin{questionparts}
\item  $2\sin(x-\frac14\pi) + \sqrt 3 \cos(x+\frac14\pi) =
\sin(x+\frac14\pi)\, $;
\item $2\sin(x-\frac16\pi) + \sqrt 3 \cos(x+\frac16\pi) =
\sin(x+\frac16\pi)\, $;
\item  $2\sin(x+\frac13\pi) + \sqrt 3 \cos(3x) =
\sin(3x)\, $.

\end{questionparts}
\end{question}

%%%%%%%%% Q5
\begin{question}
In this question, 
$\f^2(x)$ denotes $\f(\f(x))$, $\f^3(x)$ denotes $\f( \f (\f(x)))\,$,
and so on.

\begin{questionparts}
\item 
The function $\f$ is defined, for $x\ne \pm 1/ \sqrt3\,$,  
by 
$$ \f(x) = \ds \frac{x+\sqrt3} {1-\sqrt3\, x }\,.
$$
Find by direct calculation $\f^2(x) $ and $\f^3(x)$, and determine
$\f^{2007}(x)\,$.
\item Show that 
$\f^n(x) = \tan(\theta + \frac 13 n\pi)$, where $x=\tan\theta$
and $n$ is any positive integer.



\item The function $\g(t)$ is defined, for $\vert t\vert\le1$ by 
$\g(t) = \frac {\sqrt3}2 t + \frac 12 \sqrt {1-t^2}\,$.
Find an expression for $\g^n(t)$ for any positive integer $n$.

\end{questionparts}
	\end{question}
	
%%%%%%%%% Q6
\begin{question}
\begin{questionparts}
\item  Differentiate $\ln\big (x+\sqrt{3+x^2}\,\big)$
and $x\sqrt{3+x^2}$ and simplify your answers.

Hence find
$\int \! \sqrt{3+x^2}\, \d x$.

\item  Find the two  solutions of the differential equation
\[
3\left(\frac{\d y}{\d x}\right)^{\!2} + 2 x \frac {\d y}{\d x} =1
\]
that satisfy $y=0$ when $x=1$.
\end{questionparts}
\end{question}
	
%%%%%%%%% Q7
\begin{question}
A function $\f(x)$ is said to be concave on some interval if 
$\f''(x)<0$ in that interval. Show that 
$\sin x$ is concave for $0<x<\pi$ and
that $\ln x$ is concave for $x>0$.

Let $\f(x)$ be concave on a given  
interval and let $x_1$, $x_2$, $\ldots$, $x_n$ lie in the 
interval.
{\sl Jensen's inequality} states that
\[
\frac1 n \sum_{k=1}^n\f(x_k) \le \f \bigg (\frac1 n
\sum_{k=1}^n x_k\bigg)
\]
and that equality holds if and only if $x_1=x_2= \cdots =x_n$. You may
use this result without proving it.

\begin{questionparts}
\item 
Given that $A$, $B$ and $C$ are angles of a triangle, show that
\[
\sin A + \sin B + \sin C \le \frac{3\sqrt3}2 \,.
\]
\item 
By choosing a suitable function $\f$, prove that 
\[
\sqrt[n]{t_1t_2\cdots t_n}\; \le \; \frac{t_1+t_2+\cdots+t_n}n 
\]
for any positive integer $n$ and for any positive numbers $t_1$,
$t_2$,
$\ldots$, $t_n$. 

Hence:
\begin{questionparts}
\item[\bf (a)] show that 
$x^4+y^4+z^4 +16 \ge 8xyz$, where $x$, $y$ and $z$ are any 
positive numbers;

\item[\bf (b)] find the minimum value of  $x^5+y^5+z^5 -5xyz$, 
where $x$, $y$ and $z$ are any 
positive numbers.
\end{questionparts}

\end{questionparts}
\end{question}
		
%%%%%%%%% Q8
\begin{question}	
The points $B$ and $C$ have position vectors $\bb$ and $\bc$, 
respectively, relative to the origin $A$, and $A$, $B$ and $C$
are not collinear.

\begin{questionparts}
\item  The point $X$ has position vector 
$s \bb+t\bc$. Describe the locus of $X$ when $s+t=1$.
\item  The point $P$
has position vector $\beta \bb+\gamma\bc$, where $\beta$ and $\gamma$
are 
non-zero, and $\beta+\gamma\ne1$.
The line $AP$ cuts the line $BC$ at $D$. Show that $BD:DC
=\gamma:\beta$.
\item  The line $BP$ cuts the line $CA$ at $E$, and the
  line $CP$ cuts the line $AB$ at $F$. Show that 
\[
\frac{AF}{FB} \times \frac{BD}{DC} \times \frac{CE}{EA}=1\,.
\]

\end{questionparts}
\end{question}	
		

		
	
\newpage
\section*{Section B: \ \ \ Mechanics}


	
%%%%%%%%%% Q9
\begin{question}
A  solid right circular cone, of mass $M$, has 
semi-vertical angle $\alpha$ and smooth surfaces. 
It stands with its base on
a smooth horizontal table. 
A particle of mass $m$ 
is projected 
 so that it strikes the curved surface of the cone
at speed $u$. 
The  coefficient of restitution
 between the particle  and the cone is $e$.
The impact has no rotational effect on the cone
and the cone has no vertical velocity
after the impact.

\begin{questionparts}
\item The particle strikes the cone in the direction of the normal at
  the point of impact. Explain why the trajectory of the particle
  immediately
after the impact is parallel to the normal to the surface of the cone.
Find an expression, in terms of $M$, $m$,
$\alpha$, $e$ and $u$, for the speed at which the cone slides
along the table immediately after impact.

\item
If instead the particle falls vertically onto the cone, show that the speed $w$
  at which the cone slides
along the table immediately after impact is given by
\[
w= \frac{mu(1+e)\sin\alpha\cos\alpha}{M+m\cos^2\alpha}\,.
\]

Show also that the value of $\alpha$ for which $w$ is greatest is given by
\[
\cos \alpha = \sqrt{ \frac{M}{2M+m}}\ .
\]
\end{questionparts}
	\end{question}
	
%%%%%%%%%% Q10 
\begin{question}	
A solid figure is composed of a uniform solid cylinder 
of density $\rho$ 
and a uniform solid hemisphere of density $3\rho$. 
The cylinder has  circular
cross-section, with radius 
$r$, and height $3r$, and the  hemisphere has radius
$r$. The flat face of the 
hemisphere is joined to one end of the cylinder, so that their
centres coincide.

The figure  is held in equilibrium by a force $P$ so that one point of 
its flat base  is in contact with a rough horizontal plane
and its base is  inclined
at an angle $\alpha$ to the horizontal. The force $P$
is horizontal and  acts through the highest point 
of the base. The coefficient of friction between the solid and the
plane is $\mu$.
Show that 
\[\mu \ge \left\vert \tfrac98 -\tfrac12 \cot\alpha\right\vert\,.
\]
\end{question}

%%%%%%%%%% Q11

\begin{question}
{\sl In 
this question take the acceleration due to gravity to
  be
$10\,{\rm m \,s}^{-2}$ and neglect  air resistance.}

The  point $O$ lies in a horizontal field. The point $B$
lies $50\,$m east of $O$. A 
particle is projected from  $B$ at speed $25\,{\rm m\,s}^{-1}$ at an angle
$\arctan \frac12$ above the horizontal and in a direction 
that makes an angle $60^\circ$
with $OB$; it passes to the north of $O$.

\begin{questionparts}
\item Taking unit vectors $\mathbf i$, $\mathbf j$ and
$\mathbf k$ in the directions east, north and vertically 
upwards, respectively, find the position vector of the particle relative to 
$O$ at time $t$~seconds after the particle was  projected, and show that
its distance from $O$ is 
\[
5(t^2- \sqrt5 t +10)\, {\rm m}.
\]

When this distance is shortest, the  particle is at point $P$.
Find  the position vector of $P$ and its horizontal bearing from $O$.


\item Show that the particle reaches its maximum height at $P$.
\item When the particle is at $P$, a marksman fires a bullet from $O$
directly at $P$.
The initial speed of the bullet is $350\,{\rm m\,s}^{-1}$. Ignoring the
effect of gravity on the bullet show that, when it passes
through
$P$, the distance between $P$ and the 
particle is approximately~$3\,$m.
 
\end{questionparts}
\end{question}
	

	
	\newpage
\section*{Section C: \ \ \ Probability and Statistics}


%%%%%%%%%% Q12
\begin{question}
I have two identical dice. When I throw either one of them, the
probability
of it showing a 6 is $p$ and the probability of it not showing a 6 is
$q$,
where $p+q=1$.
As an experiment to determine $p$, I throw the dice simultaneously
until at least one die shows a 6. If both dice show a six on this 
throw, I stop. If just one die shows a six, I throw the other die 
until it shows a 6 and then stop.


\begin{questionparts}
\item Show that the probability that I stop after $r$ throws
is $pq^{r-1}(2-q^{r-1}-q^r)$, and find an expression for the
expected number of throws.

[{\bf Note:} You may use the result $\ds \sum_{r=0}^\infty rx^r =
  x(1-x)^{-2}$.]

\item In a large number of such experiments, the mean number
of throws was $m$. Find an estimate for $p$ in terms of $m$.
\end{questionparts}
\end{question}

%%%%%%%%%% Q13
\begin{question}
Given that $0<r<n$ and $r$ is much smaller than $n$, show that 
$\dfrac {n-r}n \approx \e^{-r/n}$.

There are $k$ guests at a party. Assuming that there are exactly 365
days 
in the year, and that the birthday of any guest is equally likely to 
fall on any of these days, show that the probability that there are at
least
two guests with the same birthday is approximately $1-\e^{-k(k-1)/730}$.

Using the approximation $ \frac{253}{365} \approx \ln 2$, find the
smallest value of $k$ such that
the probability that at least two guests share the
same birthday is at least  $\frac12$.

How many guests must there be at the party for  the probability 
that at least one guest has the same birthday as the host to be 
at least $\frac12$?


\end{question}

%%%%%%%%%% Q14
\begin{question}
The random variable $X$ has a  continuous probability density
function $\f(x)$ given by
\begin{equation*}
\f(x) =
\begin{cases}
0     & \text{for } x \le 1 \\
\ln x & \text{for } 1\le x \le k\\
\ln k & \text{for } k\le x \le 2k\\
a-bx  & \text{for } 2k \le x \le 4k \\
0 & \text{for } x\ge 4k
\end{cases}
\end{equation*}
where $k$, $a$ and $b$ are constants. 
\begin{questionparts}
\item Sketch the graph of $y=\f(x)$.
\item Determine $a$ and  $b$ in terms of $k$ and find the numerical 
values of $k$, $a$ and $b$.
\item Find the median value of $X$.
\end{questionparts}
\end{question}
	
\end{document}
