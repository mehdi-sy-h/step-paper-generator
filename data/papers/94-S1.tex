\documentclass[a4, 11pt]{report}


\pagestyle{myheadings}
\markboth{}{Paper I, 1994
\ \ \ \ \ 
\today 
}               

\RequirePackage{amssymb}
\RequirePackage{amsmath}
\RequirePackage{graphicx}
\RequirePackage{color}
\RequirePackage[flushleft]{paralist}[2013/06/09]



\RequirePackage{geometry}
\geometry{%
  a4paper,
  lmargin=2cm,
  rmargin=2.5cm,
  tmargin=3.5cm,
  bmargin=2.5cm,
  footskip=12pt,
  headheight=24pt}


\newcommand{\comment}[1]{{\bf Comment} {\it #1}}
%\renewcommand{\comment}[1]{}

\newcommand{\bluecomment}[1]{{\color{blue}#1}}
%\renewcommand{\comment}[1]{}
\newcommand{\redcomment}[1]{{\color{red}#1}}



\usepackage{epsfig}
\usepackage{pstricks-add}
\usepackage{pst-coil}
\usepackage{tgheros} %% changes sans-serif font to TeX Gyre Heros (tex-gyre)
\renewcommand{\familydefault}{\sfdefault} %% changes font to sans-serif
%\usepackage{sfmath}  %%%% this makes equation sans-serif
%\input RexFigs


\setlength{\parskip}{10pt}
\setlength{\parindent}{0pt}

\newlength{\qspace}
\setlength{\qspace}{20pt}


\newcounter{qnumber}
\setcounter{qnumber}{0}

\newenvironment{question}%
 {\vspace{\qspace}
  \begin{enumerate}[\bfseries 1\quad][10]%
    \setcounter{enumi}{\value{qnumber}}%
    \item%
 }
{
  \end{enumerate}
  \filbreak
  \stepcounter{qnumber}
 }


\newenvironment{questionparts}[1][1]%
 {
  \begin{enumerate}[\bfseries (i)]%
    \setcounter{enumii}{#1}
    \addtocounter{enumii}{-1}
    \setlength{\itemsep}{5mm}
    \setlength{\parskip}{8pt}
 }
 {
  \end{enumerate}
 }



\DeclareMathOperator{\cosec}{cosec}
\DeclareMathOperator{\Var}{Var}

\def\d{{\rm d}}
\def\e{{\rm e}}
\def\g{{\rm g}}
\def\h{{\rm h}}
\def\f{{\rm f}}
\def\p{{\rm p}}
\def\s{{\rm s}}
\def\t{{\rm t}}


\def\A{{\rm A}}
\def\B{{\rm B}}
\def\E{{\rm E}}
\def\F{{\rm F}}
\def\G{{\rm G}}
\def\H{{\rm H}}
\def\P{{\rm P}}


\def\bb{\mathbf b}
\def \bc{\mathbf c}
\def\bx {\mathbf x}
\def\bn {\mathbf n}

\newcommand{\low}{^{\vphantom{()}}}
%%%%% to lower suffices: $X\low_1$ etc


\newcommand{\subone}{ {\vphantom{\dot A}1}}
\newcommand{\subtwo}{ {\vphantom{\dot A}2}}




\def\le{\leqslant}
\def\ge{\geqslant}


\def\var{{\rm Var}\,}

\newcommand{\ds}{\displaystyle}
\newcommand{\ts}{\textstyle}




\begin{document}
\setcounter{page}{2}

 
\section*{Section A: \ \ \ Pure Mathematics}

%%%%%%%%%%Q1
\begin{question}
My house has an attic consisting of a horizontal rectangular base
of length $2q$ and breadth $2p$ (where $p<q$) and four plane roof
sections each at angle $\theta$ to the horizontal. Show that the
length of the roof ridge is independent of $\theta$ and find the
volume of the attic and the surface area of the roof.  
\end{question}

%%%%%%%%%%Q2
\begin{question}
Given that $a$ is constant, differentiate the following expressions
with respect to $x$: 

\begin{itemize}
\setlength{\itemsep}{3mm}
\item[\bf (i)]  $x^{a}$; 
\item[\bf (ii)] $a^{x}$; 
\item[\bf (iii)] $x^{x}$; 
\item[\bf (iv)] $x^{(x^{x})}$;
\item[\bf (v)] $(x^{x})^{x}.$
\end{itemize}
\end{question}

%%%%%%%%% Q3
\begin{question}
By considering the coefficient of $x^{n}$ in the identity $(1-x)^{n}(1+x)^{n}=(1-x^{2})^{2n},$
or otherwise, simplify 
\[
\binom{n}{0}^{2}-\binom{n}{1}^{2}+\binom{n}{2}^{2}-\binom{n}{3}^{2}+\cdots+(-1)^{n}\binom{n}{n}^{2}
\]
in the cases \textbf{(i) }when $n$ is even, \textbf{(ii) }when $n$
is odd. 
\end{question}

%%%%%% Q4 
\begin{question}
Show that 
\begin{questionparts}
\item $\dfrac{1-\cos\alpha}{\sin\alpha}=\tan\frac{1}{2}\alpha,$ 
\item  if $\left|k\right|<1$ then ${\displaystyle \int\frac{\mathrm{d}x}{1-2kx+x^{2}}=\frac{1}{\sqrt{1-k^{2}}}\tan^{-1}\left(\frac{x-k}{\sqrt{1-k^{2}}}\right)+C,}$
where $C$ is a constant of integration. 
\end{questionparts}
Hence, or otherwise, show that if $0<\alpha<\pi$ then 
\[
\int_{0}^{1}\frac{\sin\alpha}{1-2x\cos\alpha+x^{2}}\,\mathrm{d}x=\frac{\pi-\alpha}{2}.
\]
\end{question}

%%%%%%%%% Q5
\begin{question}
A parabola has the equation $y=x^{2}.$ The points $P$ and $Q$ with
coordinates $(p,p^{2})$ and $(q,q^{2})$ respectively move on the
parabola in such a way that $\angle POQ$ is always a right angle. 
\begin{questionparts}
\item Find and sketch the locus of the midpoint $R$ of the chord
$PQ.$ 
\item Find and sketch the locus of the point $T$ where the tangents
to the parabola at $P$ and $Q$ intersect. 
\end{questionparts}
\end{question}
	
%%%%%%%%% Q6
\begin{question}
The function $\mathrm{f}$ is defined, for any complex number $z$,
by 
\[
\mathrm{f}(z)=\frac{\mathrm{i}z-1}{\mathrm{i}z+1}.
\]
Suppose throughout that $x$ is a real number. 

\begin{questionparts}
\item Show that 
\[
\mathrm{Re}\,\mathrm{f}(x)=\frac{x^{2}-1}{x^{2}+1}\qquad\mbox{ and }\qquad\mathrm{Im}\,\mathrm{f}(x)=\frac{2x}{x^{2}+1}.
\]
\item Show that $\mathrm{f}(x)\mathrm{f}(x)^{*}=1,$ where $\mathrm{f}(x)^{*}$
is the complex conjugate of $\mathrm{f}(x)$. 
\item Find expressions for $\mathrm{Re}\,\mathrm{f}(\mathrm{f}(x))$
and $\mathrm{Im}\,\mathrm{f}(\mathrm{f}(x)).$ 
\item Find $\mathrm{f}(\mathrm{f}(\mathrm{f}(x))).$ 
\end{questionparts}
\end{question}
	
%%%%%%%%% Q7
\begin{question}
From the facts 
\begin{alignat*}{2}
1 & \quad=\quad &  & 0\\
2+3+4 & \quad=\quad &  & 1+8\\
5+6+7+8+9 & \quad=\quad &  & 8+27\\
10+11+12+13+14+15+16 & \quad=\quad &  & 27+64
\end{alignat*}
guess a general law. Prove it. 


Hence, or otherwise, prove that 
\[
1^{3}+2^{3}+3^{3}+\cdots+N^{3}=\tfrac{1}{4}N^{2}(N+1)^{2}
\]
for every positive integer $N$. 

{[}\textbf{Hint. }\textit{You may
assume that $1+2+3+\cdots+n=\frac{1}{2}n(n+1)$.}{]} 
\end{question}
		
%%%%%%%%% Q8
\begin{question}	
By means of the change of variable $\theta=\frac{1}{4}\pi-\phi,$
or otherwise, show that 
\[
\int_{0}^{\frac{1}{4}\pi}\ln(1+\tan\theta)\,\mathrm{d}\theta=\tfrac{1}{8}\pi\ln2.
\]
Evaluate 
\[
{\displaystyle \int_{0}^{1}\frac{\ln(1+x)}{1+x^{2}}\,\mathrm{d}x}\qquad\mbox{ and }\qquad{\displaystyle \int_{0}^{\frac{1}{2}\pi}\ln\left(\frac{1+\sin x}{1+\cos x}\right)\,\mathrm{d}x}.
\]
\end{question}	
	
	
\newpage
\section*{Section B: \ \ \ Mechanics}

%%%%%%%%%% Q9
\begin{question}
A cannon-ball is fired from a cannon at an initial speed $u$. After
time $t$ it has reached height $h$ and is at a distance $\sqrt{x^{2}+h^{2}}$
from the cannon. Ignoring air resistance, show that 
\[
\tfrac{1}{4}g^{2}t^{4}-(u^{2}-gh)t^{2}+h^{2}+x^{2}=0.
\]
Hence show that if $u^{2}>2gh$ then the horizontal range for a given
height $h$ and initial speed $u$ is less than or equal to 
\[
\frac{u\sqrt{u^{2}-2gh}}{g}.
\]
Show that there is always an angle of firing for which this value
is attained. 
\end{question}

%%%%%%%%%% Q10
\begin{question}
One end $A$ of a light elastic string of natural length $l$ and
modulus of elasticity $\lambda$ is fixed and a particle of mass $m$
is attached to the other end $B$. The particle moves in a horizontal
circle with centre on the vertical through $A$ with angular velocity
$\omega.$ If $\theta$ is the angle $AB$ makes with the downward
vertical, find an expression for $\cos\theta$ in terms of $m,g,l,\lambda$
and $\omega.$


Show that the motion described is possible only if 
\[
\frac{g\lambda}{l(\lambda+mg)}<\omega^{2}<\frac{\lambda}{ml}.
\]
\end{question}
	
%%%%%%%%%% Q11
\begin{question}
 $\,$
\begin{center}
\psset{xunit=1.0cm,yunit=1.0cm,algebraic=true,dotstyle=o,dotsize=3pt 0,linewidth=0.3pt,arrowsize=3pt 2,arrowinset=0.25}
\begin{pspicture*}(-2.2,-0.26)(7.1,4.4)
\pscircle[fillcolor=black,fillstyle=solid,opacity=0.4](5,4){0.19}
\psline(-2,0)(7,0)
\psline(0,0)(5,4)
\psline(5,4)(5,0)
\psline(-0.08,0.26)(4.88,4.15)
\rput[tl](-0.4,0.98){$A$}
\rput[tl](5.59,3){$B$}
\psline(5.18,4.02)(5.18,3)
\begin{scriptsize}
\psdots[dotsize=13pt 0,dotstyle=*](-0.08,0.23)
\psdots[dotsize=11pt 0,dotstyle=*](5.18,3)
\end{scriptsize}
\end{pspicture*}
\par
\end{center}


The diagram shows a small railway wagon $A$ of mass $m$ standing
at the bottom of a smooth railway track of length $d$ inclined at
an angle $\theta$ to the horizontal. A light inextensible string,
also of length $d$, is connected to the wagon and passes over a light
frictionless pulley at the top of the incline. On the other end of
the string is a ball $B$ of mass $M$ which hangs freely. The system
is initially at rest and is then released. 


\begin{questionparts}


\item Find the condition which $m,M$ and $\theta$ must satisfy
to ensure that the ball will fall to the ground. Assuming that this
condition is satisfied, show that the velocity $v$ of the ball when
it hits the ground satisfies 
\[
v^{2}=\frac{2g(M-m\sin\theta)d\sin\theta}{M+m}.
\]



\item Find the condition which $m,M$ and $\theta$ must satisfy
if the wagon is not to collide with the pulley at the top of the incline.
\end{questionparts}
	\end{question}
	

	

	
	\newpage
\section*{Section C: \ \ \ Probability and Statistics}

%%%%%%%%%% Q12
\begin{question}	
There are 28 colleges in Cambridge, of which two (New Hall and Newnham)
are for women only; the others admit both men and women. Seven women,
Anya, Betty, Celia, Doreen, Emily, Fariza and Georgina, are all applying
to Cambridge. Each has picked three colleges at random to enter on
her application form. 


\begin{itemize}[indent]
\setlength{\itemsep}{3mm}
\item[\bf (i)]  What is the probability that Anya's first choice college is
single-sex?
\item[\bf (ii)] What is the probability that Betty has picked Newnham?
\item[\bf (iii)] What is the probability that Celia has picked at least one
single-sex college?
\item[\bf (iv)] Doreen's first choice is Newnham. What is the probability that
one of her other two choices is New Hall?
\item[\bf (v)] Emily has picked Newnham. What is the probability that she
has also picked New Hall?
\item[\bf (vi)] Fariza's first choice college is single-sex. What is the probability
that she has also chosen the other single-sex college?
\item[\bf (vii)] One of Georgina's choices is a single-sex college. What is
the probability that she has also picked the other single-sex college?
\end{itemize}
\end{question}

%%%%%%%%%% Q13

\begin{question}
I have a bag containing $M$ tokens, $m$ of which are red. I remove
$n$ tokens from the bag at random without replacement. Let 
\[
X_{i}=\begin{cases}
1 & \mbox{ if the \ensuremath{i}th token I remove is red;}\\
0 & \mbox{ otherwise.}
\end{cases}
\]
Let $X$ be the total number of red tokens I remove. 
\begin{itemize}[indent]
\setlength{\itemsep}{3mm}

\item[\bf (i)] Explain briefly why $X=X_{1}+X_{2}+\cdots+X_{n}.$
\item[\bf (ii)] Find the expectation $\mathrm{E(}X_{i}).$
\item[\bf (iii)] Show that $\mathrm{E}(X)=mn/M$. 
\item[\bf (iv)] Find $\mathrm{P}(X=k)$ for $k=0,1,2,\ldots,n$. 
\item[\bf (v)] Deduce that 
\[
\sum_{k=1}^{n}k\binom{m}{k}\binom{M-m}{n-k}=m\binom{M-1}{n-1}.
\]
\end{itemize}
\end{question}

%%%%%%%%%% Q14
\begin{question}
Each of my $n$ students has to hand in an essay to me. Let $T_{i}$
be the time at which the $i$th essay is handed in and suppose that
$T_{1},T_{2},\ldots,T_{n}$ are independent, each with probability
density function $\lambda\mathrm{e}^{-\lambda t}$ ($t\geqslant0$).
Let $T$ be the time I receive the first essay to be handed in and
let $U$ be the time I receive the last one. 
\begin{questionparts}
\item Find the mean and variance of $T_{i}.$
\item Show that $\mathrm{P}(U\leqslant u)=(1-\mathrm{e}^{-\lambda u})^{n}$
for $u\geqslant0,$ and hence find the probability density function
of $U$. 
\item Obtain $\mathrm{P}(T>t),$ and hence find the probability density
function of $T$. 
\item Write down the mean and variance of $T$. 
\end{questionparts}
\end{question}
\end{document}
