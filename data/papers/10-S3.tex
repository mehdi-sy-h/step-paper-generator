\documentclass[a4, 11pt]{report}


\pagestyle{myheadings}
\markboth{}{Paper III, 2010
\ \ \ \ \ 
\today 
}               

\RequirePackage{amssymb}
\RequirePackage{amsmath}
\RequirePackage{graphicx}
\RequirePackage{color}
\RequirePackage[flushleft]{paralist}[2013/06/09]



\RequirePackage{geometry}
\geometry{%
  a4paper,
  lmargin=2cm,
  rmargin=2.5cm,
  tmargin=3.5cm,
  bmargin=2.5cm,
  footskip=12pt,
  headheight=24pt}


\newcommand{\comment}[1]{{\bf Comment} {\it #1}}
%\renewcommand{\comment}[1]{}

\newcommand{\bluecomment}[1]{{\color{blue}#1}}
%\renewcommand{\comment}[1]{}
\newcommand{\redcomment}[1]{{\color{red}#1}}



\usepackage{epsfig}
\usepackage{pstricks-add}
\usepackage{tgheros} %% changes sans-serif font to TeX Gyre Heros (tex-gyre)
\renewcommand{\familydefault}{\sfdefault} %% changes font to sans-serif
%\usepackage{sfmath}  %%%% this makes equation sans-serif
%\input RexFigs


\setlength{\parskip}{10pt}
\setlength{\parindent}{0pt}

\newlength{\qspace}
\setlength{\qspace}{20pt}


\newcounter{qnumber}
\setcounter{qnumber}{0}

\newenvironment{question}%
 {\vspace{\qspace}
  \begin{enumerate}[\bfseries 1\quad][10]%
    \setcounter{enumi}{\value{qnumber}}%
    \item%
 }
{
  \end{enumerate}
  \filbreak
  \stepcounter{qnumber}
 }


\newenvironment{questionparts}[1][1]%
 {
  \begin{enumerate}[\bfseries (i)]%
    \setcounter{enumii}{#1}
    \addtocounter{enumii}{-1}
    \setlength{\itemsep}{5mm}
    \setlength{\parskip}{8pt}
 }
 {
  \end{enumerate}
 }



\DeclareMathOperator{\cosec}{cosec}
\DeclareMathOperator{\Var}{Var}

\def\d{{\mathrm d}}
\def\e{{\mathrm e}}
\def\g{{\mathrm g}}
\def\h{{\mathrm h}}
\def\f{{\mathrm f}}
\def\p{{\mathrm p}}
\def\s{{\mathrm s}}
\def\t{{\mathrm t}}


\def\A{{\mathrm A}}
\def\B{{\mathrm B}}
\def\E{{\mathrm E}}
\def\F{{\mathrm F}}
\def\G{{\mathrm G}}
\def\H{{\mathrm H}}
\def\P{{\mathrm P}}


\def\bb{\mathbf b}
\def \bc{\mathbf c}
\def\bx {\mathbf x}
\def\bn {\mathbf n}

\newcommand{\low}{^{\vphantom{()}}}
%%%%% to lower suffices: $X\low_1$ etc


\newcommand{\subone}{ {\vphantom{\dot A}1}}
\newcommand{\subtwo}{ {\vphantom{\dot A}2}}




\def\le{\leqslant}
\def\ge{\geqslant}
\def\arcosh{{\rm arcosh}\,}


\def\var{{\rm Var}\,}

\newcommand{\ds}{\displaystyle}
\newcommand{\ts}{\textstyle}
\def\half{{\textstyle \frac12}}
\def\l{\left(}
\def\r{\right)}



\begin{document}
\setcounter{page}{2}

 
\section*{Section A: \ \ \ Pure Mathematics}

%%%%%%%%%%Q1
\begin{question}
Let $x_{\low1}$, $x_{\low2}$, \ldots, $x_n$ and 
$x_{\vphantom {\dot A} n+1}$ be any fixed real numbers.
The numbers $A$ and $B$ are defined by
\[ 
A = \frac 1 n 
\sum_{k=1}^n x_{ \low k}
 \,, \ \ \ 
B= \frac 1 n 
 \sum_{k=1}^n (x_{\low k}-A)^2
\,,  \ \ \
\]
and the numbers $C$ and $D$ are defined by
\[
 C = \frac 1 {n+1}
\sum\limits_{k=1}^{n+1} x_{\low k} 
\,, 
\ \ \
 D = \frac1{n+1} 
\sum_{k=1}^{n+1} (x_{\low k}-C)^2
\,.
\]              
\begin{questionparts}
\item
 Express $ C$  in terms of $A$,  $x_{\low n+1}$ and $n$. 
\item Show that $ \displaystyle 
B= \frac 1 n
 \sum_{k=1}^n x_{\low k}^2
- A^2\,$. 
\item  Express $D $  in terms of $B$, $A$,  $x_{\low n+1}$ and $n$.

Hence show that $(n + 1)D  \ge nB$ 
for all values of $x_{\low n+1}$, but that $D  < B$
if and only if
\[
A-\sqrt{\frac{(n+1)B}{n}}<x_{\low n+1}<    
A+\sqrt{\frac{(n+1)B}{n}}\,.
\]  
\end{questionparts}
\end{question}

%%%%%%%%%%Q2
\begin{question}
In this question, $a$ is a positive constant.
\begin{questionparts}
\item  Express $\cosh a$ in terms of exponentials. 

By using partial fractions, prove that
\[
\int_0^1 \frac 1{ x^2 +2x\cosh a +1} \, \d x = \frac a {2\sinh a}\,. 
\]

\item Find, expressing your answers in terms of hyperbolic functions, 
\[
\int_1^\infty \frac 1 {x^2 +2x \sinh a -1} \,\d x
\,
\]

and 

\[
\int_0^\infty \frac 1 {x^4 +2x^2\cosh a +1} \,\d x
\,.\]

\end{questionparts}

\end{question}

%%%%%%%%% Q3
\begin{question}
For any given positive integer $n$,
a number $a$ (which may be complex) is said to be 
a {\sl primitive $n$th root of unity} if $a^n=1$
and there is no integer $m$ such that $0 < m < n$ and $a^m = 1$.
Write down the two primitive 4th roots of unity.


Let ${\rm C}_n(x)$ be the  polynomial such that
 the roots of the equation ${\rm C}_n(x)=0$ are the primitive
$n$th roots of unity,
the coefficient of the highest power of $x$ is one and the equation
has no repeated roots.
Show that ${\rm C}_4(x) = x^2+1\,$.

\begin{questionparts}
\item Find ${\rm C}_1(x)$, ${\rm C}_2(x)$, ${\rm C}_3(x)$, 
${\rm C}_5(x)$ and   ${\rm C}_6(x)$, giving your answers as
unfactorised polynomials.
\item Find the value of $n$ for which ${\rm C}_n(x) = x^4 + 1$.
\item Given that $p$ is prime, find an
expression for ${\rm C}_p(x)$, giving your answer as an unfactorised
polynomial.
\item Prove that there are no positive integers $q$, $r$ and $s$ such
that ${\rm C}_q(x) \equiv {\rm C}_r(x) {\rm C}_s(x)\,$.

\end{questionparts}
\end{question}

%%%%%% Q4 
\begin{question}
\begin{questionparts}
\item
The number $\alpha$ is a common root of the 
equations $x^2 +ax +b=0$ and  $x^2+cx+d=0$
(that is, $\alpha$ satisfies both equations). Given that $a\ne c$, show that
\[
\alpha =- \frac{b-d}{a-c}\,.
\]
Hence, or otherwise, show that the equations have at least one
common root if and only if
\[
(b-d)^2 -a(b-d)(a-c) + b(a-c)^2 =0\,.
\]

Does this result still hold if the condition $a\ne c$ is not imposed?

\item Show that  the equations 
$x^2+ax+b=0$ and  $x^3+(a+1)x^2+qx+r=0$
have at least one common root if and only if
\[
(b-r)^2-a(b-r)(a+b-q) +b(a+b-q)^2=0\,.
\]

Hence, or otherwise, find the values of $b$ for which the equations
$2x^2+5 x+2 b=0$ and $2x^3+7x^2+5x+1=0$
have at least one common root.


\end{questionparts}
\end{question}

%%%%%%%%% Q5
\begin{question}
The vertices $A$, $B$, $C$ and $D$ of a square have coordinates
$(0,0)$, $(a,0)$, $(a,a)$ and $(0,a)$, respectively.
The points $P$ and $Q$ have coordinates $(an,0)$ and $(0,am)$ respectively, 
where $0<m<n<1$.  
The line $CP$ produced
meets $DA$ produced at $R$ and the line $CQ$ produced meets $BA$ produced
at $S$.  The line $PQ$ produced meets the line $RS$ produced at $T$.
Show that $TA$ is perpendicular to $AC$.

Explain how, given a square of area $a^2$, a square of area $2a^2$ may
be constructed using only a straight-edge.
 
[{\bf Note}: a straight-edge is a ruler with no markings on it;
no measurements (and no use of  compasses) are allowed in the construction.]

	\end{question}
	
%%%%%%%%% Q6
\begin{question}
The points $P$, $Q$ and  $R$ 
lie on a sphere of unit radius centred at the origin, $O$,
which is fixed. 
Initially, $P$ is at $P_0(1, 0, 0)$, $Q$ is at $Q_0(0, 1, 0)$ 
and $R$ is at
$R_0(0, 0, 1)$. 

\begin{questionparts}
\item
The sphere is then rotated about the $z$-axis,
so that the line $OP$ turns directly
towards the positive 
$y$-axis through an angle $\phi$. The position of $P$ after this
rotation is denoted by $P_1$.
Write down the coordinates of $P_1$. 

\item
The sphere is now rotated about the line in the $x$-$y$ plane 
perpendicular to $OP_1$, so that the line $OP$
turns directly towards the positive $z$-axis through an angle $\lambda$. 
The position of $P$
after this rotation is denoted by $P_2$.
Find the coordinates of $P_2$. 
 Find also
the coordinates of the points $Q_2$ and $R_2$, which are
the positions of $Q$ and $R$ after
the two rotations.
\item                  
The sphere is now  rotated for a third time,
 so that $P$ returns from $P_2$ to its
original position~$P_0$. During the rotation, $P$ remains in the 
plane containing $P_0$, $P_2$ and $O$.
Show that the  angle of this
rotation, $\theta$, satisfies 
\[
\cos\theta  = \cos\phi  \cos\lambda\,,
\]
and find a vector in the direction of the axis 
about which this rotation takes place.
\end{questionparts}
\end{question}
	
%%%%%%%%% Q7
\begin{question}
Given that $y = \cos(m \arcsin x)$, for $\vert x \vert <1$,
prove that
\[
(1-x^2) \frac {\d^2 y}{\d x^2} -x \frac {\d y}{\d x} +m^2y=0\,.
\]

Obtain a similar equation relating $\dfrac{\d^3y}{\d x^3}\,$,\; 
 $\dfrac{\d^2y}{\d x^2}\, $ and $\, \dfrac{\d y}{\d x}\,$, and a
similar equation
relating 
$\dfrac{\d^4y}{\d x^4}\,$,~~$\dfrac{\d^3y}{\d x^3}\,$ and 
$\,\dfrac{\d^2 y}{\d x^2}\,$.

Conjecture and prove 
 a relation between  
$\dfrac{\d^{n+2}y}{\d x^{n+2}}\,$,  \ 
$\dfrac{\d^{n+1}y}{\d x^{n+1}}\;$ and $\;\dfrac{\d^n y}{\d x^n}\,$.



Obtain the first three non-zero terms of the 
Maclaurin series for $y$. Show that, if $m$ is an even
integer, $\cos m\theta$  may be written as a polynomial in $\sin\theta$
beginning
\[
1 - \frac{m^2\sin^2\theta}{2!}+ \frac{m^2(m^2-2^2)\sin^4\theta}{4!} -\cdots \,.
\, \tag{$\vert\theta\vert < \tfrac12 \pi$}
\]
State the degree of the polynomial. 
\end{question}
		
%%%%%%%%% Q8
\begin{question}
Given that ${\rm P} (x) = {\rm Q} (x){\rm R}'(x) - {\rm Q}'(x){\rm R}(x)$, 
write down an expression for
\[
\int   \frac{{\rm P} ( x)}{ \big( {\rm Q} ( x)\big )^ 2}\, \d x\, .
\]

\begin{questionparts}
\item
By choosing  the function ${\rm R}(x)$ to be of the form
$a +bx+c x^2$,
find  
\[
\int \frac{5x^2 - 4x - 3} {(1 + 2x + 3x^2 )^2 } \, \d x
\,.\]
Show that the choice of ${\rm R}(x)$ is not unique and, by comparing 
the two functions  ${\rm R}(x)$ corresponding to two different values of 
$a$, explain how the different
choices are related.

\item
Find  the general solution of 
\[
(1+\cos x +2 \sin x) \frac {\d y}{\d x} 
+(\sin x -2 \cos x)y = 5 - 3 \cos x + 4 \sin x\,.
\]



\end{questionparts}
\end{question}	
		

		
	
\newpage
\section*{Section B: \ \ \ Mechanics}


	
%%%%%%%%%% Q9
\begin{question}$\,$
\begin{center}
\newrgbcolor{wwwwww}{0.4 0.4 0.4}
\psset{xunit=1.0cm,yunit=1.0cm,algebraic=true,dotstyle=o,dotsize=3pt 0,linewidth=0.3pt,arrowsize=3pt 2,arrowinset=0.25}
\begin{pspicture*}(1.92,2.15)(7.25,6.21)
\pspolygon[linecolor=wwwwww,fillcolor=wwwwww,fillstyle=solid,opacity=0.75](2.27,2.85)(2.27,2.52)(6.48,2.52)(6.48,2.85)
\psline[linecolor=wwwwww](2.27,2.85)(2.27,2.52)
\psline[linecolor=wwwwww](2.27,2.52)(6.48,2.52)
\psline[linecolor=wwwwww](6.48,2.52)(6.48,2.85)
\psline[linewidth=1.2pt,linecolor=wwwwww](6.48,2.85)(2.27,2.85)
\psline(6.48,2.85)(6.48,5.88)
\pscustom[linewidth=0.5pt]{\parametricplot{1.5707963267948966}{3.141592653589793}{1*3.02*cos(t)+0*3.02*sin(t)+6.48|0*3.02*cos(t)+1*3.02*sin(t)+2.85}\lineto(6.48,2.85)\closepath}
\parametricplot{1.5450288353258959}{2.629484171074415}{1*3.1*cos(t)+0*3.1*sin(t)+6.48|0*3.1*cos(t)+1*3.1*sin(t)+2.85}
\psline(6.56,3.94)(6.56,5.96)
\psline[linestyle=dashed,dash=1pt 1pt](3.78,4.37)(6.48,2.85)
\rput[tl](5.07,4.02){$a$}
\rput[tl](5.76,3.17){$\theta $}
\parametricplot{2.629484171074415}{3.141592653589793}{1*0.87*cos(t)+0*0.87*sin(t)+6.48|0*0.87*cos(t)+1*0.87*sin(t)+2.85}
\rput[tl](6.61,3.14){$O$}
\rput[tl](3.31,4.89){$P$}
\rput[tl](6.85,4.39){$Q$}
\begin{scriptsize}
\psdots[dotsize=6pt 0,dotstyle=*](3.78,4.37)
\psdots[dotsize=6pt 0,dotstyle=*](6.56,3.94)
\end{scriptsize}
\end{pspicture*}
\end{center}
The diagram shows
two particles, $P$ and $Q$, 
connected by a light inextensible string which passes over a
smooth block fixed to a horizontal table.
 The
cross-section of the block is a quarter circle with centre $O$, which 
is at the edge of the table, and radius $a$. The angle between
$OP$ and the table is $\theta$.
The masses of $P$ and $Q$ are $m$ and $M  $, respectively,
where $m < M$. 

Initially, $P$ is held at rest on the table and in contact with the block,
  $Q$ is 
 vertically above $O$, and the string is taut.
Then $P$ is released. Given that, in the subsequent motion,
$P$ 
remains in contact with the block as $\theta$ 
increases from $0$ to $\frac12\pi$,
find an expression, in terms of $m$, $M$, $\theta$ and $g$,
 for the normal reaction of the block on $P$ and show 
   that
 \[
\frac{m}{M} \ge \frac{\pi-1}3\,.
\]
	\end{question}
	
%%%%%%%%%% Q10 
\begin{question}	
A small bead $B$, of mass $m$,
 slides without friction on a fixed horizontal ring of
radius $a$. The centre of the ring is at $O$. The bead is attached by a light
elastic string to a fixed
point $P$ in the plane of the ring such that  $OP = b$, where $b > a$.
The natural length of the elastic string is $c$, where $c < b - a$, and its
modulus of elasticity is $\lambda$. 
Show that the equation of motion of the bead is
\[
ma\ddot \phi = 
-\lambda\left( \frac{a\sin\phi}{c\sin\theta}-1\right)\sin(\theta+\phi)
\,,
\]
where
$\theta=\angle BPO$ and $\phi=\angle BOP$. 


Given that $\theta$ and $\phi$ are small, show that
 $a(\theta+\phi)\approx
b\theta$. Hence find the period of  
small oscillations about the
equilibrium position $\theta=\phi =0$. 
\end{question}

%%%%%%%%%% Q11

\begin{question}
A bullet of mass $m$ is fired horizontally with speed $u$ into a 
wooden block of
mass $M$ at rest on a horizontal surface. The coefficient of friction
between the block and the surface is $\mu$. While
the bullet is moving through the block, it 
experiences a constant force of resistance to its motion
of magnitude $R$, where $R>(M+m)\mu g$.
The bullet moves horizontally in the block and does not emerge from the
other side of the block.
 
\begin{questionparts}
\item
Show that 
the magnitude, $a$, of the deceleration
of the bullet relative to the block
while the bullet is moving through the block is given by
\[
a= 
\frac R m + \frac {R-(M+m)\mu g}{M}\,
.
\]

\item Show that the common speed, $v$,  of the block and bullet when the 
bullet stops moving through the block satisfies
\[
av = \frac{Ru-(M+m)\mu gu}M\,.
\]

\item Obtain an expression, in terms of $u$, $v$ and $a$, for the 
distance moved by the block while the bullet is moving through the block.

\item Show that the total distance moved by the block is 
\[
\frac{muv}{2(M+m)\mu g}\,.
\]
\end{questionparts}

Describe briefly what happens if $R< (M+m)\mu g$.
\end{question}
	

	
	\newpage
\section*{Section C: \ \ \ Probability and Statistics}


%%%%%%%%%% Q12
\begin{question}
The infinite series $S$ is given by
 \[
      S = 1 + (1 + d)r + (1 + 2d)r^2 + \cdots + (1+nd)r^n +\cdots\; 
,\]   
for $\vert r \vert <1\,$.
By considering $S - rS$, or otherwise, prove that 
\[
S = \frac 1{1-r} + \frac {rd}{(1-r)^2}
\,.\]

Arthur and Boadicea shoot arrows at a target. The probability that an
arrow shot by Arthur hits the target is $a$; the probability that an arrow
shot by Boadicea hits the target is $b$. Each shot is independent of all
others. Prove that the expected number of shots it takes Arthur to hit
the target is $1/a$.

Arthur and Boadicea now have a contest. They take alternate shots,
with Arthur going first. The winner is the one who hits the target first.
The probability
that Arthur wins the contest is  $\alpha$ and 
the probability that Boadicea wins is
$\beta$. Show that
\[
\alpha = \frac a {1-a'b'}\,,
\]
where $a' = 1-a$ and $b'=1-b$, and find $\beta$.
 

Show that the expected number of shots in the contest is
$\displaystyle \frac \alpha a + \frac \beta b\,.$
\end{question}

%%%%%%%%%% Q13
\begin{question}
In this question, ${\rm Corr}(U,V)$ denotes the product moment
correlation coefficient between the random variables
$U$ and $V$, defined by
\[
\mathrm{Corr}(U,V) \equiv \frac{\mathrm{Cov}(U,V)}{\sqrt{\var(U)\var(V)}}\,.
\]

The independent random variables $Z_1$, $Z_2$ and  $Z_3$
each have expectation 0
and variance 1. What is  the value of 
$\mathrm{Corr} (Z_1,Z_2)$?

Let $Y_1 = Z_1$ and let
\[
Y_2 = \rho^{\vphantom 2} _{12} Z_1 + 
(1 - {\rho_{12}^2)_{\vphantom A}}^{\! \frac12}  Z_ 2\,,
\]
where $\rho^{\vphantom 2}_{12}$ is a given constant with $-1<\rho^{\vphantom 2}
_{12}<1$.
Find $\E(Y_2)$, $\var(Y_2)$ and
$\mathrm{Corr}(Y_1, Y_2)$.

Now let $Y_3 =  aZ_1 + bZ_2 + cZ_3$, where $a$, $b$ and $c$ are real
constants and $c\ge0$. Given that 
$\E(Y_3) = 0$,  $\var(Y_3) = 1$, 
$ \mathrm{Corr}(Y_1, Y_3) 
=\rho^{{\vphantom 2}}_{13} 
$
and 
$ \mathrm{Corr}(Y_2, Y_3)= 
\rho^{\vphantom {2}} _{23}$, express
 $a$, $b$ and
$c$ in terms of $\rho^{\vphantom 2} _{23}$, $\rho^{\vphantom 2}_{13}$ 
and $\rho^{\vphantom 2} _{12}$.

Given constants $\mu_i$ and $\sigma_i$, for $i=1$, $2$ and $3$, give
 expressions in terms of the $Y_i$ for random variables $X_i$ 
such that
$\E(X_i) = \mu_i$, $\var(X_i) = \sigma_ i^2$ and 
$\mathrm{Corr}(X_i,X_j) = \rho_{ij}$.
\end{question}

\end{document}
