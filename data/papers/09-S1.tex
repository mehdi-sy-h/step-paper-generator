\documentclass[a4, 11pt]{report}


\pagestyle{myheadings}
\markboth{}{Paper I, 2009
\ \ \ \ \ 
\today 
}               

\RequirePackage{amssymb}
\RequirePackage{amsmath}
\RequirePackage{graphicx}
\RequirePackage{color}
\RequirePackage[flushleft]{paralist}[2013/06/09]



\RequirePackage{geometry}
\geometry{%
  a4paper,
  lmargin=2cm,
  rmargin=2.5cm,
  tmargin=3.5cm,
  bmargin=2.5cm,
  footskip=12pt,
  headheight=24pt}


\newcommand{\comment}[1]{{\bf Comment} {\it #1}}
%\renewcommand{\comment}[1]{}

\newcommand{\bluecomment}[1]{{\color{blue}#1}}
%\renewcommand{\comment}[1]{}
\newcommand{\redcomment}[1]{{\color{red}#1}}



\usepackage{epsfig}
\usepackage{pstricks-add}
\usepackage{tgheros} %% changes sans-serif font to TeX Gyre Heros (tex-gyre)
\renewcommand{\familydefault}{\sfdefault} %% changes font to sans-serif
%\usepackage{sfmath}  %%%% this makes equation sans-serif
%\input RexFigs


\setlength{\parskip}{10pt}
\setlength{\parindent}{0pt}

\newlength{\qspace}
\setlength{\qspace}{20pt}


\newcounter{qnumber}
\setcounter{qnumber}{0}

\newenvironment{question}%
 {\vspace{\qspace}
  \begin{enumerate}[\bfseries 1\quad][10]%
    \setcounter{enumi}{\value{qnumber}}%
    \item%
 }
{
  \end{enumerate}
  \filbreak
  \stepcounter{qnumber}
 }


\newenvironment{questionparts}[1][1]%
 {
  \begin{enumerate}[\bfseries (i)]%
    \setcounter{enumii}{#1}
    \addtocounter{enumii}{-1}
    \setlength{\itemsep}{5mm}
    \setlength{\parskip}{8pt}
 }
 {
  \end{enumerate}
 }



\DeclareMathOperator{\cosec}{cosec}
\DeclareMathOperator{\Var}{Var}

\def\d{{\mathrm d}}
\def\e{{\mathrm e}}
\def\g{{\mathrm g}}
\def\h{{\mathrm h}}
\def\f{{\mathrm f}}
\def\p{{\mathrm p}}
\def\s{{\mathrm s}}
\def\t{{\mathrm t}}


\def\A{{\mathrm A}}
\def\B{{\mathrm B}}
\def\E{{\mathrm E}}
\def\F{{\mathrm F}}
\def\G{{\mathrm G}}
\def\H{{\mathrm H}}
\def\P{{\mathrm P}}


\def\bb{\mathbf b}
\def \bc{\mathbf c}
\def\bx {\mathbf x}
\def\bn {\mathbf n}

\newcommand{\low}{^{\vphantom{()}}}
%%%%% to lower suffices: $X\low_1$ etc


\newcommand{\subone}{ {\vphantom{\dot A}1}}
\newcommand{\subtwo}{ {\vphantom{\dot A}2}}




\def\le{\leqslant}
\def\ge{\geqslant}
\def\arcosh{{\rm arcosh}\,}


\def\var{{\rm Var}\,}

\newcommand{\ds}{\displaystyle}
\newcommand{\ts}{\textstyle}
\def\half{{\textstyle \frac12}}
\def\l{\left(}
\def\r{\right)}



\begin{document}
\setcounter{page}{2}

 
\section*{Section A: \ \ \ Pure Mathematics}

%%%%%%%%%%Q1
\begin{question}
A {\em proper factor} of an integer  $N$ is a positive integer,
not $1$ or $N$, that divides $N$.

\begin{questionparts}
\item Show that $3^2\times 5^3$ has exactly $10$ proper factors. Determine
how  many other integers of the form $3^m\times5^n$ (where $m$ and $n$
are integers) have exactly 10 proper factors.


\item Let $N$ be the smallest positive 
integer that has exactly $426$ proper
factors. Determine~$N$, giving your answer in terms of its prime factors.
\end{questionparts}
\end{question}

%%%%%%%%%%Q2
\begin{question}
A curve has the equation                           
\[
y^3 = x^3 +a^3+b^3\,,
\]
where $a$ and $b$ are positive constants. Show that the tangent
to the curve 
at the point $(-a,b)$~is 
\[
b^2y-a^2x = a^3+b^3\,.
\]

In the case $a=1$ and $b=2$, show that the $x$-coordinates of the 
points where the tangent meets the curve satisfy
\[
7x^3 -3x^2 -27x-17 =0\,.
\]
Hence find  positive integers $p$, $q$, $r$ and $s$ such that
\[
p^3 = q^3 +r^3 +s^3\,.
\]
\end{question}

%%%%%%%%% Q3
\begin{question}
\begin{questionparts}
\item
 By considering the equation $x^2+x-a=0\,$, 
show that the equation
$x={(a-x)\vphantom M}^{\frac12}$ has one real solution when $a\ge0$ and no
real solutions when $a<0\,$.

Find the number of distinct real solutions of the equation
\[
x={\big((1+a)x-a\big)}^{\!\frac13}
\]
in the cases that arise according to the value of $a$.
\item Find the number of distinct real solutions of the equation
\[
x={(b+x)\vphantom M}^{\frac12}
\]
in the cases that arise according to the value of $b\,$.
\end{questionparts}
\end{question}

%%%%%% Q4 
\begin{question}
The sides of a triangle have lengths
$p-q$, $p$ and $p+q$, where $p>q>  0\,$.
The largest
and smallest angles of the triangle are $\alpha$ and $\beta$,
respectively.
Show by means of the cosine rule that
\[ 4(1-\cos\alpha)(1-\cos\beta) = \cos\alpha + \cos\beta
\,.
\]

In the case $\alpha = 2\beta$, 
show that $\cos\beta=\frac34$ and hence find the
ratio of the lengths of the sides of the triangle.
\end{question}

%%%%%%%%% Q5
\begin{question}
A right circular cone has base radius $r$, height $h$ and
slant length $\ell$. Its volume $V$, and the area~$A$ of its curved
surface,
are given by
\[
V= \tfrac13 \pi r^2 h \,, \ \ \ \ \ \ \ 
A = \pi r\ell\,.
\]
\vspace*{-1cm}
\begin{questionparts}
\item Given that
$A$ is fixed and $r$ is chosen so that 
  $V$ is at its stationary value, show that $A^2 = 3\pi^2r^4$ and that
$
\ell  =\sqrt3\,r$.

\item Given, instead, that  $V$ is fixed and $r$ is chosen so that
$A$ is at its stationary value, find~$h$ in terms of $r$.
\end{questionparts}
	\end{question}
	
%%%%%%%%% Q6
\begin{question}
\begin{questionparts}
\item
Show that, for $m>0\,$, 
\[
\int_{1/m}^m \frac{x^2}{x+1} \, \d x 
= \frac{(m-1)^3(m+1)}{2m^2}+ \ln m\,.
\]
\item
Show by means of a substitution that
\[
\int_{1/m}^m \frac1 {x^n(x+1)}\,\d x
 = \int_{1/m}^m \frac {u^{n-1}}{u+1}\,\d u \,.
\]
\item
Evaluate:
\begin{itemize}
\item[\bf (a)]  \quad
$\displaystyle \int_{1/2}^2 \frac {x^5+3}{x^3(x+1)}\,\d x \;;$
\\
\item[\bf (b)]
\quad
 $\displaystyle \int_1^2 \frac{x^5+x^3 +1}{x^3(x+1)}\, \d x\;. $
\end{itemize}
\end{questionparts}
\end{question}
	
%%%%%%%%% Q7
\begin{question}
Show that, for any integer $m$, 
\[
\int_0^{2\pi} \e^x \cos mx \, \d x = \frac
    {1}{m^2+1}\big(\e^{2\pi}-1\big)\,.
\]

\begin{questionparts}
\item
Expand 
 $\cos(A+B) +\cos(A-B)$.  Hence show that
\[\displaystyle 
\int_0^{2\pi} \e^x \cos x \cos 6x \, \d x\, 
= \tfrac{19}{650}\big( \e^{2\pi}-1\big)\,.
\]
\item Evaluate $\displaystyle 
\int_0^{2\pi} \e^x \sin 2x \sin 4x \cos x \, \d x\,$.
\end{questionparts}
\end{question}
		
%%%%%%%%% Q8
\begin{question}
\begin{questionparts}
\item
 The equation of the circle $C$ is 
\[
(x-2t)^2 +(y-t)^2 =t^2,
\]
where $t$ is a positive number.
Show that $C$ touches the line $y=0\,$.

Let $\alpha$ be the acute angle between the $x$-axis and the 
line joining the origin to the centre of $C$. Show that $\tan2\alpha
=\frac43$ and deduce that $C$ touches the line $3y=4x\,$.

\item Find the equation of the incircle of the triangle formed by 
the lines $y=0$, $3y=4x$ and $4y+3x=15\,$.

{\bf Note:} The {\em incircle} of a triangle is the circle,
lying totally inside the
triangle, that touches all three sides.


\end{questionparts}
\end{question}	
		

		
	
\newpage
\section*{Section B: \ \ \ Mechanics}


	
%%%%%%%%%% Q9
\begin{question}
Two particles $P$ and $Q$ are projected simultaneously from points
$O$ and $D$, respectively, where~$D$ is a distance $d$ directly
above $O$. The initial speed of $P$ is $V$ and its angle of projection
{\em above} the horizontal is $\alpha$. The initial speed of $Q$ is 
$kV$, where $k>1$, and its angle of projection {\em below} the horizontal
is $\beta$. The particles collide at time $T$ after projection.

Show that $\cos\alpha = k\cos\beta$ and that  
$T$ satisfies the equation
\[
(k^2-1)V^2T^2 +2dVT\sin\alpha -d^2 =0\,.
\]

Given that the particles collide when $P$ reaches its maximum height,
find an expression for~$\sin^2\alpha$ in terms of $g$, $d$, $k$ and
$V$,
and deduce that 
\[
gd\le (1+k)V^2\,.
\]

	\end{question}
	
%%%%%%%%%% Q10 
\begin{question}	
A triangular wedge is fixed to a horizontal surface. The 
 base angles
of the wedge are $\alpha$ and $\frac\pi 2-\alpha$.
Two particles, of masses $M$ and $m$, lie on different faces of 
the wedge, and are connected by a light inextensible string
which passes over a smooth pulley at the apex of the wedge, as
shown in the diagram. 
The contacts between the particles and the wedge are smooth.
  
\begin{center} \psset{xunit=1.0cm,yunit=1.0cm,algebraic=true,dimen=middle,dotstyle=o,dotsize=3pt 0,linewidth=0.3pt,arrowsize=3pt 2,arrowinset=0.25}
\begin{pspicture*}(-0.12,-0.34)(8.27,3.81)
\pspolygon(5.74,3.35)(5.89,3.08)(6.15,3.24)(6,3.5)
\pscircle[linewidth=0.4pt,fillcolor=black,fillstyle=solid,opacity=0.4](5.99,3.5){0.23}
\psline(0,0)(6,3.5)
\psline(8,0)(6,3.5)
\psline(6.2,3.59)(7.03,2.13)
\psline(5.87,3.69)(1.99,1.44)
\psline(0,0)(8,0)
\rput[tl](0.91,0.44){$\alpha$}
\rput[tl](6.54,0.63){$\frac{\pi}{2}-\alpha$}
\rput[tl](1.38,1.99){$M$}
\rput[tl](7.31,2.6){$m$}
\begin{scriptsize}
\psdots[dotsize=12pt 0,dotstyle=*](7.03,2.13)
\psdots[dotsize=12pt 0,dotstyle=*](1.99,1.4)
\end{scriptsize}
\end{pspicture*}
  \end{center}
  
\begin{questionparts}
\item
Show that if $\tan \alpha> \dfrac m M $
the particle of mass $M$ will slide down the face of the wedge.

\item 

Given that $\tan \alpha = \dfrac{2m}M$, show that the magnitude of the
acceleration of the
particles is
\[
\frac{g\sin\alpha}{\tan\alpha +2}
\]
and that this             is maximised at
$4m^3=M^3\,$.

\end{questionparts}
\end{question}

%%%%%%%%%% Q11

\begin{question}
Two particles 
move on a smooth horizontal table and collide. 
The  masses of the particles are $m$ and $M$.  Their 
velocities before the collision are $u{\bf i}$ and $v{\bf i}\,$, 
respectively, where $\bf i$ is a unit vector and $u>v$.  
Their velocities  after the collision are $p{\bf i}$ and $q{\bf i}\,$, 
respectively.
The coefficient of 
restitution between the two particles is $e$, where $e<1$. 

\begin{questionparts}
\item Show that the 
loss of kinetic energy due to the collision is
\[
\tfrac12 m (u-p)(u-v)(1-e)\,,
\]
and deduce that $u\ge p$.

\item
Given that each particle loses the same (non-zero)
amount of kinetic energy in
the collision, show that
\[
u+v+p+q=0\,,
\]
and that, if $m\ne M$, 
\[
e= \frac{(M+3m)u + (3M+m)v}{(M-m)(u-v)}\,.
\]

\end{questionparts}
\end{question}
	

	
	\newpage
\section*{Section C: \ \ \ Probability and Statistics}


%%%%%%%%%% Q12
\begin{question}
Prove that, for any real numbers $x$ and $y$, $x^2+y^2\ge2xy\,$.

\begin{questionparts}
\item Carol has two bags of sweets. The first bag contains $a$ red sweets  
and $b$ blue sweets, whereas the second bag contains $b$ red sweets   and
$a$ blue  sweets, where $a$ and $b$ are positive integers. Carol shakes
the bags and picks
one sweet from each bag without looking. Prove that the probability
that
the sweets  are of the same colour 
cannot exceed  the probability that
they are of different colours.
\item Simon has three bags of sweets. The first bag
contains
$a$ red sweets, $b$ white sweets and $c$ yellow sweets, where $a$, $b$ and
$c$ are positive integers. The second
bag contains
$b$ red sweets, $c$ white sweets and $a$ yellow sweets. The third
bag contains
$c$ red sweets, $a$ white sweets and $b$ yellow sweets.
Simon shakes the bags and 
picks one sweet from each bag without looking.
Show that the probability that exactly two of the sweets are of the 
same colour is
\[
\frac {3(a^2b+b^2c+c^2a+ab^2 + bc^2 +ca^2)}{(a+b+c)^3}\,,
\]
and find the probability that the sweets are all of the same colour.
Deduce that the probability that exactly two of 
the sweets are of the same colour is at least 6 times the probability
that the sweets are all of the same colour.
\end{questionparts}
\end{question}

%%%%%%%%%% Q13
\begin{question}
I seat $n$ boys and $3$ girls in a line    at random, so 
that each order of the $n+3$ children
is as likely to occur as any other. Let $K$ be the maximum number of 
consecutive girls in the line so, for example, $K=1$ if  
there is at least one boy between 
 each pair of girls.
\begin{questionparts}
\item Find $\P(K=3)$.
\item Show that 
\[\P(K=1)=
\frac{n(n-1)}{(n+2)(n+3)}\,.
\]
\item Find $\E(K)$.
\end{questionparts}
\end{question}

\end{document}
