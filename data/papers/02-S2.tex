\documentclass[a4, 11pt]{report}


\pagestyle{myheadings}
\markboth{}{Paper II, 2002
\ \ \ \ \ 
\today 
}               

\RequirePackage{amssymb}
\RequirePackage{amsmath}
\RequirePackage{graphicx}
\RequirePackage{color}
\RequirePackage[flushleft]{paralist}[2013/06/09]



\RequirePackage{geometry}
\geometry{%
  a4paper,
  lmargin=2cm,
  rmargin=2.5cm,
  tmargin=3.5cm,
  bmargin=2.5cm,
  footskip=12pt,
  headheight=24pt}


\newcommand{\comment}[1]{{\bf Comment} {\it #1}}
%\renewcommand{\comment}[1]{}

\newcommand{\bluecomment}[1]{{\color{blue}#1}}
%\renewcommand{\comment}[1]{}
\newcommand{\redcomment}[1]{{\color{red}#1}}



\usepackage{epsfig}
\usepackage{pstricks-add}
\usepackage{tgheros} %% changes sans-serif font to TeX Gyre Heros (tex-gyre)
\renewcommand{\familydefault}{\sfdefault} %% changes font to sans-serif
%\usepackage{sfmath}  %%%% this makes equation sans-serif
%\input RexFigs


\setlength{\parskip}{10pt}
\setlength{\parindent}{0pt}

\newlength{\qspace}
\setlength{\qspace}{20pt}


\newcounter{qnumber}
\setcounter{qnumber}{0}

\newenvironment{question}%
 {\vspace{\qspace}
  \begin{enumerate}[\bfseries 1\quad][10]%
    \setcounter{enumi}{\value{qnumber}}%
    \item%
 }
{
  \end{enumerate}
  \filbreak
  \stepcounter{qnumber}
 }


\newenvironment{questionparts}[1][1]%
 {
  \begin{enumerate}[\bfseries (i)]%
    \setcounter{enumii}{#1}
    \addtocounter{enumii}{-1}
    \setlength{\itemsep}{5mm}
    \setlength{\parskip}{8pt}
 }
 {
  \end{enumerate}
 }



\DeclareMathOperator{\cosec}{cosec}
\DeclareMathOperator{\Var}{Var}

\def\d{{\mathrm d}}
\def\e{{\mathrm e}}
\def\g{{\mathrm g}}
\def\h{{\mathrm h}}
\def\f{{\mathrm f}}
\def\p{{\mathrm p}}
\def\s{{\mathrm s}}
\def\t{{\mathrm t}}


\def\A{{\mathrm A}}
\def\B{{\mathrm B}}
\def\E{{\mathrm E}}
\def\F{{\mathrm F}}
\def\G{{\mathrm G}}
\def\H{{\mathrm H}}
\def\P{{\mathrm P}}


\def\bb{\mathbf b}
\def \bc{\mathbf c}
\def\bx {\mathbf x}
\def\bn {\mathbf n}

\newcommand{\low}{^{\vphantom{()}}}
%%%%% to lower suffices: $X\low_1$ etc


\newcommand{\subone}{ {\vphantom{\dot A}1}}
\newcommand{\subtwo}{ {\vphantom{\dot A}2}}




\def\le{\leqslant}
\def\ge{\geqslant}


\def\var{{\rm Var}\,}

\newcommand{\ds}{\displaystyle}
\newcommand{\ts}{\textstyle}
\def\half{{\textstyle \frac12}}
\def\l{\left(}
\def\r{\right)}



\begin{document}
\setcounter{page}{2}

 
\section*{Section A: \ \ \ Pure Mathematics}

%%%%%%%%%%Q1
\begin{question}
Show that
\[
\int_{\frac{1}{6}\pi}^{\frac{1}{4}\pi} \frac 1{1-\cos2\theta} \;\d\theta = \frac{\sqrt3}2 - \frac12\;.
\]

By using the substitution $x=\sin2\theta$, or otherwise, show that
\[
\int_{\sqrt3/2}^1 \frac 1 {1-\sqrt{1-x^2}} \, \d x = \sqrt 3 -1 -\frac\pi 6 \;.
\]

Hence evaluate the integral
\[
\int_1^{2/\sqrt3} \frac  1{y ( y - \sqrt{y^2-1^2})} \, \d y \;.
\]
\end{question}

%%%%%%%%%%Q2
\begin{question}
Show that setting $z - z^{-1}=w$ in the quartic equation
\[
z^4 +5z^3 +4z^2 -5z +1=0
\]
results in the quadratic equation $w^2+5w+6=0$. Hence  
solve the above quartic equation.

Solve similarly the equation
\[
2z^8 -3z^7-12z^6 +12z^5 +22z^4-12z^3 -12 z^2 +3z +2=0 \;.
\]
\end{question}

%%%%%%%%% Q3
\begin{question}
The $n$th  Fermat number, $F_n$, is defined by
\[
F_n = 2^{2^n} +1\, ,
\ \ \ \ \ \ \ n=0, \ 1, \ 2, \ \ldots \ ,
\]    
where $\ds  2^{2^n}$ means $2$ raised to the power $2^n\,$.
Calculate $F_0$, $F_1$, $F_2$ and $F_3\,$. Show that,
for $k=1$, $k=2$ and $k=3\,$,
$$
F_0F_1 \ldots F_{k-1} = F_k-2 \;. \eqno(*)
$$


Prove, by induction, or otherwise, that 
$(*)$ holds for all  $k\ge1$.  Deduce that no two Fermat numbers
have a common factor greater than $1$.

Hence
show that there are infinitely many  prime numbers.

\end{question}

%%%%%% Q4 
\begin{question}
Give a sketch to show that, if $\f(x)>0$ for $p<x<q\,$, then
$\int_p^{ \raisebox{2pt}{$\scriptstyle \hspace {1pt} q$}} \f(x) \d x >0\,$.

\begin{questionparts}
\item By considering $\f(x) = ax^2-bx+c\,$ show that,
if $a>0$ and $b^2<4ac$, then $3b<2a+6c\,$.
\item By considering $\f(x)= a\sin^2x - b\sin x + c\,$
show that, if $a>0$ and $b^2<4ac$, then 
$
4b<(a+2c)\pi \;.
$
\item Show that,  if $a>0$, $b^2<4ac$ and $q>p>0\,$, then
$$
 b\ln(q/p) < a\left(\frac1p -\frac1q\right) +c(q-p)\;.
$$
\end{questionparts}

\end{question}

%%%%%%%%% Q5
\begin{question}
The numbers $x_n$,  where $n=0$, $1$, $2$, $\ldots$ , satisfy 
\[
x_{n+1} = kx_n(1-x_n) \;.
\]

\begin{questionparts}
\item Prove that, if $0<k<4$ and $0<x_0<1$, then $0<x_n<1$ for all $n\,$.
\item Given that $x_0=x_1=x_2 = \cdots =a\,$, with  $a\ne0$ and $a\ne1$, find
$k$ in terms of $a\,$.
\item Given instead that $x_0=x_2=x_4 = \cdots = a\,$, with   $a\ne0$ and $a\ne1$,
show that $ab^3 -b^2 +(1-a)=0$, where $b=k(1-a)\,$. Given, in addition, that
$x_1 \ne a$, find the possible values of $k$ in terms of $a\,$.
\end{questionparts}
	\end{question}
	
%%%%%%%%% Q6
\begin{question}
The lines $l_1$, $l_2$ and $l_3$ lie in an inclined plane $P$ and    pass through 
a common point $A$.  The line $l_2$ is a
line of greatest slope in $P$.  The line $l_1$ is  perpendicular to $l_3$ and
makes an acute angle $\alpha$ with $l_2$.
The angles between the horizontal and 
$l_1$, $l_2$  and $l_3$ are  $\pi/6$, $\beta$  and $\pi/4$, respectively. 
Show that $\cos\alpha\sin\beta = \frac12\,$
and find the value of $\sin\alpha \sin\beta\,$. Deduce that 
$\beta = \pi/3\,$.

The lines $l_1$ and $l_3$ are rotated in $P$ about 
$A$ so that $l_1$ and $l_3$ remain perpendicular to each other.
The new acute angle between
$l_1$ and $l_2$ is $\theta$.  The new angles which $l_1$ and $l_3$
 make with the horizontal are $\phi$ and $2\phi$, respectively.  Show that
\[
             \tan^2\theta = \frac{3+\sqrt{13}}2\;.
\]

\end{question}
	
%%%%%%%%% Q7
\begin{question}
In 3-dimensional space, the lines $m_1$ and $m_2$ pass through the origin and 
have directions $\bf i + j$ and $\bf i +k $, respectively. Find the directions  of
the two lines $m_3$ and $m_4$  that  pass through the origin and make
angles of $\pi/4$   with both $m_1$ and $m_2$. Find also the cosine of the 
acute angle between $m_3$ and $m_4$.

The points $A$ and $B$ lie on $m_1$ and $m_2$ respectively, and are each at 
distance $\lambda \surd2$ units from~$O$. The points $P$ and $Q$ 
lie on $m_3$ and $m_4$ respectively, and are each at 
distance $1$ unit from~$O$. 
If all the coordinates (with respect to axes $\bf i$, $\bf j$ and $\bf k$)
 of $A$, $B$, $P$ and $Q$ are non-negative, prove that:
\begin{questionparts}
\item there are only two values of $\lambda$ for which $AQ$ is perpendicular
to $BP\,$;
\item there are no non-zero values of $\lambda$ for which $AQ$ and $BP$
intersect.
\end{questionparts}
\end{question}
		
%%%%%%%%% Q8
\begin{question}	
Find $y$ in terms of $x$, given that:
\begin{eqnarray*}
    \mbox{for $x < 0\,$}, && \frac{\d y}{\d x} = -y \mbox{ \ \ and \ \ } 
y = a \mbox{ when } x = -1\;;
\\
    \mbox{for $x > 0\,$}, && \frac{\d y}{\d x} = y \mbox{ \ \  \ \ and \ \ }
y = b  \ \mbox{ when } x = 1\;.
\end{eqnarray*}


Sketch a solution curve.  Determine the condition on $a$ and $b$
for the solution curve  to be continuous (that is,  for there to be no `jump' in the value of 
$y$) at $x = 0$.


Solve the differential equation
\[
             \frac{\d y}{\d x} = \left\vert \e^x-1\right\vert y
\]
given that $y=\e^{\e}$ when $x=1$ and that $y$ is continuous at $x=0\,$.
Write down  the following limits:

\
\[
  \text{\textbf{(i)}} \ \   \lim_ {x \to +\infty} y\exp(-\e^x)\;; 
 \ \ \ \ \ \ \ \ \ 
\text{\textbf{(ii)}} \ \ \lim_{x \to -\infty}y \e^{-x}\,.
\]
\end{question}	
		

		
	
\newpage
\section*{Section B: \ \ \ Mechanics}


	
%%%%%%%%%% Q9
\begin{question}
 A particle is projected from a point $O$ on a horizontal plane
with speed $V$ and at an angle
of elevation $\alpha$. The vertical plane in which the motion takes place
is perpendicular to two vertical walls, both of height $h$, at distances
$a$ and $b$ from $O$. Given that the particle just passes over the
walls, find $\tan\alpha$ in terms of $a$, $b$ and $h$ and
show that
\[
\frac{2V^2} g = \frac {ab} h +\frac{ (a+b)^2 h}{ab} \;.
\]

The heights of the walls  are now increased  by the same  small positive
amount $\delta h\,$. 
A  second particle is projected so that it just passes over
both walls,  and  the new angle and speed of projection 
are  $\alpha +\delta \alpha $ and $V+\delta V$, respectively.
Show that 
\[
\sec^2 \alpha \, \delta \alpha  \approx \frac {a+b}{ab}\,\delta h \;,
\]
and deduce that $\delta \alpha >0\,$. Show also that 
$\delta V$ is positive if $h> ab/(a+b)$ and negative if $h<ab/(a+b)\,$.
	\end{question}
	
%%%%%%%%%% Q10 
\begin{question}	
A competitor in a Marathon of $42 \frac38$ km 
 runs the first $t$ hours of the race at a constant speed of 13 km h$^{-1}$
and the remainder at a constant speed of $14 + 2t/T$ km h$^{-1}$, where $T$ hours
is her time for the race. Show that the minimum possible value of $T$ over
all possible values of $t$ is 3.

The speed of another competitor decreases linearly with respect
to time from 16~km~h$^{-1}$ at the start of the race.  If both of these
competitors have a run time of 3 hours, find the maximum distance between
them at any stage of the race.
\end{question}

%%%%%%%%%% Q11

\begin{question}
A rigid straight beam $AB$ has length $l$ and weight $W$. Its 
weight per unit length at a distance $x$ from $B$ is 
$\alpha Wl^{-1} (x/l)^{\alpha-1}\,$, where $\alpha$ is a positive 
constant. Show that the centre of mass of the beam is at a distance 
$\alpha l/(\alpha+1)$ from $B$.

The beam is placed with the end $A$ on a rough horizontal floor and the
end $B$ resting against a rough vertical wall. The beam is in a vertical
plane at right angles to the plane of the wall and makes an angle of
$\theta$ with the floor. The coefficient of friction
between the floor and the beam is $\mu$ and the coefficient of friction 
between the wall and the beam is also $\mu\,$.
Show that, if the equilibrium is limiting at both $A$ and $B$, then
\[
\tan\theta = \frac{1-\alpha \mu^2}{(1+\alpha)\mu}\;.
\]
Given that $\alpha =3/2\,$ and given also that the beam slides for any $\theta<\pi/4\,$
find the greatest  possible value of $\mu\,$. 
\end{question}
	

	
	\newpage
\section*{Section C: \ \ \ Probability and Statistics}


%%%%%%%%%% Q12
\begin{question}
On $K$ consecutive days each of $L$ identical coins
 is thrown $M$ times.  For each coin, the probability
of throwing a head in any one throw is $p$ (where $0 < p < 1$).
Show that the
probability that on exactly $k$ of these days more than $l$ of the coins
will each produce fewer than $m$ heads can be approximated by
\[
          {K \choose  k}q^k(1-q)^{K-k},
\]
where 
\[
q=\Phi\left( \frac{2h-2l-1}{2\sqrt{h} }\right),
\ \ \ \ \ \ 
h=L\Phi\left( \frac{2m-1-2Mp}{2\sqrt{ Mp(1-p)}}\right)
\]
and $\Phi(.)$ is the cumulative distribution function of a standard
normal variate.

Would you expect this approximation to be accurate in the case
$K=7$, $k=2$, $L=500$, $l=4$, $M=100$, $m=48$ and $p=0.6\;$?

\end{question}

%%%%%%%%%% Q13
\begin{question}
Let $\F(x)$ be the cumulative distribution function of a random variable
$X$, which satisfies $\F(a)=0$ and $\F(b)=1$,  where $a>0$.  Let
\[
\G(y) = \frac{\F(y)}{2-\F(y)}\;.
\]
Show that $\G(a)=0\,$, $\G(b)=1\,$ and that $\G'(y)\ge0\,$.
Show also that
\[
\frac12 \le \frac2{(2-\F(y))^2} \le 2\;.
\]

The random variable $Y$ has cumulative distribution function $\G(y)\,$. Show that
\[
{\ts \frac12} \,\E(X) \le \E(Y) \le 2 \E(X) \;,
\]
and that
 \[
\var(Y) \le 2\var(X) +{\ts \frac 74} \big(\E(X)\big)^2\;.
\]
\end{question}

%%%%%%%%%% Q14
\begin{question}
A densely populated circular island is divided into $N$ concentric 
regions $R_1$, $R_2$, $\ldots\,$, $R_N$, such that the inner and outer
radii of $R_n$ are $n-1$ km and $n$ km, respectively. The average number
of road accidents  that occur in any one day in $R_n$ is $2-n/N\,$, 
independently of the number of accidents in any other region.

Each day an observer selects a region at random, with a probability that
is proportional to the area of the region, and records the number of 
road accidents, $X$, that occur in it. Show that, in the long term, the
average number of recorded accidents per day will be
\[
2-\frac16\left(1+\frac1N\right)\left(4-\frac1N\right)\;.
\]

[Note: $\sum\limits_{n=1}^N n^2 = \frac16 N(N+1)(2N+1) \;$.]

Show also that 
\[
\P(X=k) = \frac{\e^{-2}N^{-k-2}}{k!}\sum_{n=1}^N (2n-1)(2N-n)^k\e^{n/N} \;.
\]


Suppose now that $N=3$ and that, on a particular day, two accidents  were recorded.
Show that the probability that $R_2$ had been selected is 
\[
\frac{48}{48 + 45\e^{1/3} +25 \e^{-1/3}}\;.
\]
\end{question}
	
\end{document}
