\documentclass[a4, 11pt]{report}


\pagestyle{myheadings}
\markboth{}{Paper I, 1998
\ \ \ \ \ 
\today 
}               

\RequirePackage{amssymb}
\RequirePackage{amsmath}
\RequirePackage{graphicx}
\RequirePackage{color}
\RequirePackage[flushleft]{paralist}[2013/06/09]



\RequirePackage{geometry}
\geometry{%
  a4paper,
  lmargin=2cm,
  rmargin=2.5cm,
  tmargin=3.5cm,
  bmargin=2.5cm,
  footskip=12pt,
  headheight=24pt}


\newcommand{\comment}[1]{{\bf Comment} {\it #1}}
%\renewcommand{\comment}[1]{}

\newcommand{\bluecomment}[1]{{\color{blue}#1}}
%\renewcommand{\comment}[1]{}
\newcommand{\redcomment}[1]{{\color{red}#1}}



\usepackage{epsfig}
\usepackage{pstricks-add}
\usepackage{tgheros} %% changes sans-serif font to TeX Gyre Heros (tex-gyre)
\renewcommand{\familydefault}{\sfdefault} %% changes font to sans-serif
%\usepackage{sfmath}  %%%% this makes equation sans-serif
%\input RexFigs


\setlength{\parskip}{10pt}
\setlength{\parindent}{0pt}

\newlength{\qspace}
\setlength{\qspace}{20pt}


\newcounter{qnumber}
\setcounter{qnumber}{0}

\newenvironment{question}%
 {\vspace{\qspace}
  \begin{enumerate}[\bfseries 1\quad][10]%
    \setcounter{enumi}{\value{qnumber}}%
    \item%
 }
{
  \end{enumerate}
  \filbreak
  \stepcounter{qnumber}
 }


\newenvironment{questionparts}[1][1]%
 {
  \begin{enumerate}[\bfseries (i)]%
    \setcounter{enumii}{#1}
    \addtocounter{enumii}{-1}
    \setlength{\itemsep}{5mm}
    \setlength{\parskip}{8pt}
 }
 {
  \end{enumerate}
 }



\DeclareMathOperator{\cosec}{cosec}
\DeclareMathOperator{\Var}{Var}

\def\d{{\rm d}}
\def\e{{\rm e}}
\def\g{{\rm g}}
\def\h{{\rm h}}
\def\f{{\rm f}}
\def\p{{\rm p}}
\def\s{{\rm s}}
\def\t{{\rm t}}


\def\A{{\rm A}}
\def\B{{\rm B}}
\def\E{{\rm E}}
\def\F{{\rm F}}
\def\G{{\rm G}}
\def\H{{\rm H}}
\def\P{{\rm P}}


\def\bb{\mathbf b}
\def \bc{\mathbf c}
\def\bx {\mathbf x}
\def\bn {\mathbf n}

\newcommand{\low}{^{\vphantom{()}}}
%%%%% to lower suffices: $X\low_1$ etc


\newcommand{\subone}{ {\vphantom{\dot A}1}}
\newcommand{\subtwo}{ {\vphantom{\dot A}2}}




\def\le{\leqslant}
\def\ge{\geqslant}


\def\var{{\rm Var}\,}

\newcommand{\ds}{\displaystyle}
\newcommand{\ts}{\textstyle}




\begin{document}
\setcounter{page}{2}

 
\section*{Section A: \ \ \ Pure Mathematics}

%%%%%%%%%%Q1
\begin{question}
How many integers between $10\,000$ and $100\,000$
(inclusive) contain exactly two different digits?
($23\,332$ contains exactly two different digits but neither
of $33\,333$ and $12\,331$ does.)
\end{question}

%%%%%%%%%%Q2
\begin{question}
Show, by means of a suitable change of variable,
or otherwise, that
\[
\int_{0}^{\infty}\mathrm{f}((x^{2}+1)^{1/2}+x)\,{\mathrm d}x
=\frac{1}{2}
\int_{1}^{\infty}(1+t^{-2})\mathrm{f}(t)\,{\mathrm d}t.
\]

Hence, or otherwise, show that
\[
\int_{0}^{\infty}((x^{2}+1)^{1/2}+x)^{-3}\,{\mathrm d}x
=\frac{3}{8}.
\]        

\end{question}

%%%%%%%%% Q3
\begin{question}
Which of the following statements are true
and which are false? Justify your answers.
\begin{questionparts}
\item $a^{\ln b}=b^{\ln a}$ for all $a,b>0$.

\item $\cos(\sin\theta)=\sin(\cos\theta)$ for all real $\theta$.

\item There exists  a polynomial $\mathrm{P}$ such that
$|\mathrm{P}(\theta)-\cos\theta|\leqslant 10^{-6}$
for all real $\theta$.
\item $x^{4}+3+x^{-4}\geqslant 5$ for all $x>0$.
\end{questionparts}
\end{question}

%%%%%% Q4 
\begin{question}
Prove that the rectangle of greatest
perimeter which can be inscribed in a given
circle is a square. 

The result changes if, instead of maximising the sum
of lengths of sides of the rectangle, we seek to
maximise the sum of $n$th powers of the lengths of those
sides for $n\geqslant 2$. What happens if $n=2$? 
What happens if $n=3$? Justify your answers.


	\end{question}

%%%%%%%%% Q5
\begin{question}
\begin{questionparts}
\item In the Argand diagram, the points $Q$ and
$A$ represent the complex numbers $4+6i$ and
$10+2i$. If $A$, $B$, $C$, $D$, $E$, $F$
are the vertices, taken in clockwise order, of 
a regular hexagon (regular six-sided polygon)
with centre $Q$, find the complex number which
represents $B$.

\item Let $a$, $b$ and $c$ be real numbers. Find a condition 
of the form $Aa+Bb+Cc=0$,
where $A$, $B$ and $C$ are integers,  which ensures that
\[\frac{a}{1+i}+\frac{b}{1+2i}+\frac{c}{1+3i}\]
is real. 
\end{questionparts}

	\end{question}
	
%%%%%%%%% Q6
\begin{question}
Let $a_{1}=\cos x$ with $0<x<\pi/2$
and let $b_{1}=1$. Given that
\begin{eqnarray*}
a_{n+1}&=&{\textstyle \frac{1}{2}}(a_{n}+b_{n}),\\[2mm]
b_{n+1}&=&(a_{n+1}b_{n})^{1/2},
\end{eqnarray*}
find $a_{2}$ and $b_{2}$ and show that
\[a_{3}=\cos\frac{x}{2}\cos^{2}\frac{x}{4}
\ \quad\mbox{and}\quad
\ b_{3}=\cos\frac{x}{2}\cos\frac{x}{4}.\] 
Guess general expressions for $a_{n}$ and $b_{n}$ (for $n\ge2$)
as products of cosines
and verify that they satisfy the given equations.
\end{question}
	
%%%%%%%%% Q7
\begin{question}
My bank pays $\rho$\% interest at the end of
each year. I start with nothing in my account.
Then for $m$ years I deposit $\pounds a$
in my  account at the beginning of each year. After the end of the $m$th year, 
I neither deposit nor withdraw for $l$ years.
Show that the total amount in my account at the end
of this period is
\[\pounds a\frac{r^{l+1}(r^{m}-1)}{r-1}\]
where $r=1+{\displaystyle \frac{\rho}{100}}$.

At the beginning of each
of the  $n$ years following this period I withdraw $\pounds b$
and this leaves my account empty after the $n$th withdrawal.
Find an expression for $a/b$ in terms of $r$, $l$, $m$ and $n$.

\end{question}
		
%%%%%%%%% Q8
\begin{question}	
Fluid flows steadily under a constant pressure
gradient along a straight tube of circular 
cross-section of radius $a$. The velocity $v$
of a particle of the fluid is parallel to the axis of the tube 
and depends only on the distance $r$
from the axis. The equation satisfied by $v$ is
\[\frac{1}{r}\frac{{\mathrm d}\ }{{\mathrm d}r}
\left(r\frac{{\mathrm d}v}{{\mathrm d}r}\right)
=-k,\]
where $k$ is constant. 
Find the general solution for $v$. 

Show that $|v|\rightarrow\infty$ as $r\rightarrow 0$ 
unless one of the constants in your solution 
is chosen to be~$0$.
Suppose that this constant is, in fact, $0$ and
that $v=0$ when $r=a$.
Find $v$ in terms of $k$, $a$ and $r$. 

The volume $F$ flowing through the tube
per unit time is given by
\[F=2\pi\int_{0}^{a}rv\,{\mathrm d}r.
\]
Find $F$.
\end{question}	
		

		
	
\newpage
\section*{Section B: \ \ \ Mechanics}


	
%%%%%%%%%% Q9
\begin{question}
Two small spheres $A$ and $B$ of equal mass $m$
are suspended in contact by two light inextensible strings
of equal length so that
the strings are vertical and
the line of centres is horizontal.
The coefficient of restitution between the spheres is $e$.
The sphere $A$ is
drawn aside through a very small distance
in the plane of the strings
and allowed
to fall back and collide with the other sphere $B$, its speed
on impact being $u$.
Explain briefly why the succeeding collisions will all occur
at the lowest point. (Hint: Consider the periods of the
two pendulums involved.)

Show that the speed of sphere $A$
immediately after the second impact is 
$\frac{1}{2}u(1+e^{2})$
and find the speed, then, of sphere $B$.

	\end{question}
	
%%%%%%%%%% Q10
\begin{question}	
A shell explodes on the surface of horizontal ground.
Earth is scattered in all directions with varying velocities.
Show that particles of earth with initial speed $v$
landing a  distance $r$ from the centre of explosion
will do so at times $t$ given by
\[
{\textstyle \frac{1}{2}}
g^2t^2=v^{2}\pm\surd(v^{4}-g^{2}r^{2}).
\]

Find an expression in terms of $v$, $r$ and $g$ for 
the greatest height reached by such particles.

\end{question}

%%%%%%%%%% Q11

\begin{question}
Hank's Gold Mine has a very long vertical shaft
of height $l$.
A light chain of length $l$
passes over a 
small smooth
light fixed pulley at the top of the shaft.
To one end of the chain is attached a bucket $A$ of
negligible mass and to the other a bucket $B$ of mass
$m$. The system is used to raise ore from the mine as follows.
When bucket $A$ is at the top it is filled with mass $2m$
of water and bucket $B$ is filled with mass $\lambda m$
of ore, where $0<\lambda<1$. The buckets are then released,
so that bucket $A$ descends and bucket $B$ ascends. 
When bucket $B$
reaches the top both buckets are emptied and released,
so that bucket $B$ descends and bucket $A$ ascends. The time to
fill and empty the buckets is negligible. Find the
time taken from the moment bucket $A$ is released at the top
until the first time it reaches the top again.

This process
goes on for a very long time. Show that, if the greatest
amount of ore is to be raised in that time, then
$\lambda$ must satisfy the condition $\mathrm{f}'(\lambda)=0$
where
\[\mathrm{f}(\lambda)=\frac{\lambda(1-\lambda)^{1/2}}
{(1-\lambda)^{1/2}+(3+\lambda)^{1/2}}.\]

\end{question}
	

	
	\newpage
\section*{Section C: \ \ \ Probability and Statistics}


%%%%%%%%%% Q12
\begin{question}
Suppose that a solution $(X,Y,Z)$ of the equation
\[X+Y+Z=20,\]
with $X$, $Y$ and $Z$ non-negative
integers, is chosen at
random (each such solution being equally likely).
Are $X$ and $Y$ independent? Justify your answer.

Show that the probability that $X$ is divisible by $5$
is $5/21$.
What is the probability that $XYZ$ is divisible by 5?


\end{question}

%%%%%%%%%% Q13
\begin{question}
I have a bag initially containing $r$ red fruit pastilles
(my favourites)
and $b$ fruit pastilles of other colours. From time to time
I shake the bag thoroughly and remove a pastille at random.
(It may be assumed that all pastilles have an equal chance
of being selected.) If the pastille is red I eat it
but otherwise I replace it in the bag. After $n$ such
drawings, I find that I have only eaten one pastille.
Show that the probability that I ate it on my last drawing
is
\[\frac{(r+b-1)^{n-1}}{(r+b)^{n}-(r+b-1)^{n}}.\]

\end{question}

%%%%%%%%%% Q14
\begin{question}
To celebrate the opening of the financial year
the finance minister of Genland flings a Slihing, a circular
coin of radius $a$ cm, where $0<a<1$, onto a 
large board divided
into squares by two sets
of parallel lines 2 cm apart. If the coin does not
cross any line, or if the coin covers an intersection,
the tax on yaks remains unchanged. Otherwise
the tax is doubled. Show that, in order to raise most tax,
the value of $a$ should be
 \[\left(1+{\displaystyle \frac{\pi}{4}}\right)^{-1}.\]

If, indeed, $a=\left(1+{\displaystyle \frac{\pi}{4}}\right)^{-1}$
and 
the tax on yaks is 1 Slihing per yak this year, show that
its expected value after $n$ years will have passed is
\[ \left(\frac{8+\pi}{4+\pi}\right)^{n}.\]
\end{question}
	
\end{document}
