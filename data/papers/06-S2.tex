\documentclass[a4, 11pt]{report}


\pagestyle{myheadings}
\markboth{}{Paper II, 2006
\ \ \ \ \ 
\today 
}               

\RequirePackage{amssymb}
\RequirePackage{amsmath}
\RequirePackage{graphicx}
\RequirePackage{color}
\RequirePackage[flushleft]{paralist}[2013/06/09]



\RequirePackage{geometry}
\geometry{%
  a4paper,
  lmargin=2cm,
  rmargin=2.5cm,
  tmargin=3.5cm,
  bmargin=2.5cm,
  footskip=12pt,
  headheight=24pt}


\newcommand{\comment}[1]{{\bf Comment} {\it #1}}
%\renewcommand{\comment}[1]{}

\newcommand{\bluecomment}[1]{{\color{blue}#1}}
%\renewcommand{\comment}[1]{}
\newcommand{\redcomment}[1]{{\color{red}#1}}



\usepackage{epsfig}
\usepackage{pstricks-add}
\usepackage{tgheros} %% changes sans-serif font to TeX Gyre Heros (tex-gyre)
\renewcommand{\familydefault}{\sfdefault} %% changes font to sans-serif
%\usepackage{sfmath}  %%%% this makes equation sans-serif
%\input RexFigs


\setlength{\parskip}{10pt}
\setlength{\parindent}{0pt}

\newlength{\qspace}
\setlength{\qspace}{20pt}


\newcounter{qnumber}
\setcounter{qnumber}{0}

\newenvironment{question}%
 {\vspace{\qspace}
  \begin{enumerate}[\bfseries 1\quad][10]%
    \setcounter{enumi}{\value{qnumber}}%
    \item%
 }
{
  \end{enumerate}
  \filbreak
  \stepcounter{qnumber}
 }


\newenvironment{questionparts}[1][1]%
 {
  \begin{enumerate}[\bfseries (i)]%
    \setcounter{enumii}{#1}
    \addtocounter{enumii}{-1}
    \setlength{\itemsep}{5mm}
    \setlength{\parskip}{8pt}
 }
 {
  \end{enumerate}
 }



\DeclareMathOperator{\cosec}{cosec}
\DeclareMathOperator{\Var}{Var}

\def\d{{\mathrm d}}
\def\e{{\mathrm e}}
\def\g{{\mathrm g}}
\def\h{{\mathrm h}}
\def\f{{\mathrm f}}
\def\p{{\mathrm p}}
\def\s{{\mathrm s}}
\def\t{{\mathrm t}}


\def\A{{\mathrm A}}
\def\B{{\mathrm B}}
\def\E{{\mathrm E}}
\def\F{{\mathrm F}}
\def\G{{\mathrm G}}
\def\H{{\mathrm H}}
\def\P{{\mathrm P}}


\def\bb{\mathbf b}
\def \bc{\mathbf c}
\def\bx {\mathbf x}
\def\bn {\mathbf n}

\newcommand{\low}{^{\vphantom{()}}}
%%%%% to lower suffices: $X\low_1$ etc


\newcommand{\subone}{ {\vphantom{\dot A}1}}
\newcommand{\subtwo}{ {\vphantom{\dot A}2}}




\def\le{\leqslant}
\def\ge{\geqslant}


\def\var{{\rm Var}\,}

\newcommand{\ds}{\displaystyle}
\newcommand{\ts}{\textstyle}
\def\half{{\textstyle \frac12}}
\def\l{\left(}
\def\r{\right)}



\begin{document}
\setcounter{page}{2}

 
\section*{Section A: \ \ \ Pure Mathematics}

%%%%%%%%%%Q1
\begin{question}
The sequence of real numbers $u_1$, $u_2$, $u_3$, $\ldots$ is defined by
\begin{equation*}
u_1=2 \,,
\qquad\text{and} \qquad  u_{n+1} = k - \frac{36}{u_n} 
\quad \text{for } n\ge1,
\tag{$*$}
\end{equation*}
where $k$ is a constant.

\begin{questionparts}
\item Determine the values of $k$ for which the sequence $(*)$ is:

\textbf{(a)} constant;

\textbf{(b)} periodic with period 2;

\textbf{(c)} periodic with period 4.

\item
In the case $k=37$, show that $u_n\ge 2$ for all $n$. Given that  in this
case the sequence $(*)$ converges to a limit
$\ell$, find the value of $\ell$.

\end{questionparts}
\end{question}

%%%%%%%%%%Q2
\begin{question}
Using the series
\[
\e^x = 1 + x +\frac{x^2}{2!} + \frac{x^3}{3!} + \frac{x^4}{4!}+\cdots\,,
\]
show that $\e>\frac83$. 

Show that   $n!>2^n$ for $n\ge4$ and hence show that
$\e<\frac {67}{24}$.


Show that the curve with equation 
\[
y= 3\e^{2x} +14 \ln (\tfrac43-x)\,,
\qquad
{x<\tfrac43}
\]
has a minimum turning point between $x=\frac12$ and $x=1$ and 
give a  sketch to show the shape of the curve.
\end{question}

%%%%%%%%% Q3
\begin{question}
\begin{questionparts}
\item Show that 
$\displaystyle \big( 5 + \sqrt {24}\;\big)^4 
+ \frac{1 }{\big(5 + \sqrt {24}\;\big)^4} \ $ is an integer.

Show also 
that 
\[\displaystyle 0.1 < \frac{1}{  5 + \sqrt {24}} <\frac 2 {19}< 0.11\,.\] 

Hence determine, with clear reasoning, 
the value of $\l 5 + \sqrt {24}\r^4$ correct to four decimal places.

\item If $N$ is an integer greater than 1, 
show that  $( N + \sqrt {N^2 - 1} \,) ^k$, where $k$ is a positive
integer,  differs from 
the integer nearest to it by less than $\big( 2N - \frac12 \big)^{-k}$.
\end{questionparts}
\end{question}

%%%%%% Q4 
\begin{question}
By making the substitution $x=\pi-t\,$, show that
\[
 \! \int_0^\pi x\f(\sin x) \d x = \tfrac12 \pi \! \int_0^\pi \f(\sin x) \d x\,,
\]
where $\f(\sin x)$ is a given function of $\sin x$.

Evaluate the following integrals:
\begin{questionparts}
 \item $\displaystyle \int_0^\pi \frac {x \sin x}{3+\sin^2 x}\,\d x\,$;
\item  $\displaystyle \int_0^{2\pi} 
\frac {x \sin x}{3+\sin^2 x}\,\d x\,$;
\item   $\displaystyle \int_{0}^{\pi} 
\frac {x \big\vert\sin 2x\big\vert}{3+\sin^2 x}\,\d x\,$.

\end{questionparts}
\end{question}

%%%%%%%%% Q5
\begin{question}
The notation ${\boldsymbol \lfloor } x \rfloor$ 
denotes the greatest integer less than or equal to the real
number $x$. Thus, for example, $\lfloor \pi\rfloor =3\,$, $\lfloor
18\rfloor =18\,$ 
and $\lfloor-4.2\rfloor = -5\,$.

\begin{questionparts}
\item Two curves are given by $y= x^2+3x-1$ and 
$y=x^2 +3\lfloor  x\rfloor -1\,$.
Sketch the curves, for~$1\le x \le 3\,$, on the same axes.

Find the area between the two curves for $1\le x \le n$, where 
$n$ is  a positive integer.

\item
Two curves are given by $y= x^2+3x-1$ and $y=\lfloor x\rfloor ^2
+3\lfloor x\rfloor -1\,$.
Sketch the curves, for $1\le x \le 3\,$, on the same axes.

Show that  the area  between the two curves for $1\le x \le n$, where 
$n$ is  a positive integer,~is 
\[
\tfrac 16 (n-1)(3n+11)\,.
\]
\end{questionparts}
	\end{question}
	
%%%%%%%%% Q6
\begin{question}
By considering a suitable scalar product, prove that 
\[
(ax+by+cz)^2 \le (a^2+b^2+c^2)(x^2+y^2+z^2)
\]
for any real numbers $a$, $b$, $c$, $x$, $y$ and $z$. Deduce a necessary and 
sufficient condition on  $a$, $b$, $c$, $x$, $y$ and $z$ for
the following equation to hold:
\[
(ax+by+cz)^2 = (a^2+b^2+c^2)(x^2+y^2+z^2) \,.
\]

\begin{questionparts}
\item Show that $(x+2y+2z)^2 \le 9(x^2+y^2+z^2)$ for all real numbers
$x$, $y$ and $z$.
\item Find real numbers $p$, $q$ and $r$ that   satisfy both
\[
p^2+4q^2+9r^2 = 729
 \text{ \ and \ }
8p+8q+3r = 243\,.
\]
\end{questionparts}
\end{question}
	
%%%%%%%%% Q7
\begin{question}
An ellipse has equation $\dfrac{x^2}{a^2}  +\dfrac {y^2}{b^2} =
1$. Show that 
the equation of the tangent at the point $(a\cos\alpha, b\sin\alpha)$ is
\[
y=- \frac {b  \cot \alpha} a \, x + b\, {\rm cosec\,}\alpha\,.
\]


The point $A$ has coordinates $(-a,-b)$, where $a$ and $b$ are
positive. The point $E$ has coordinates $(-a,0)$
and the point $P$ has coordinates $(a,kb)$, where $0<k<1$.
The line through $E$ parallel to  $AP$ meets the line $y=b$ at
the point $Q$. Show that the line $PQ$ is tangent to the above ellipse
at the point given by $\tan(\alpha/2)=k$.

Determine by means of sketches, or otherwise, whether this result
holds also for $k=0$ and $k=1$.
\end{question}
		
%%%%%%%%% Q8
\begin{question}	
Show that the line through the points with position
vectors $\bf x$ and $\bf y$ has equation 
\[{\bf r} = (1-\alpha){\bf x} +\alpha {\bf y}\,,
\]
where $\alpha$ is a scalar parameter.

The sides $OA$ and $CB$ of a trapezium $OABC$ are parallel, and $OA>CB$.
The point $E$ on $OA$ is such that $OE : EA = 1:2$, and $F$ is the midpoint of 
$CB$. The point $D$ is the intersection of $OC$ produced and $AB$ produced;
the point $G$ is the intersection of $OB$ and $EF$; and the point $H$ 
is the intersection of $DG$ produced and $OA$. Let $\bf a$ and $\bf c$ be the 
position vectors of the points $A$ and $C$, respectively, with respect to
the origin $O$.

\begin{questionparts}
\item Show that $B$ has position vector $\lambda {\bf a} + {\bf c}$ for
some scalar parameter $\lambda$.

\item Find, in terms of $\bf a$, $\bf c$ and $\lambda$
only, the position vectors of $D$, $E$, $F$, $G$ and $H$.
Determine the ratio  $OH:HA$.
\end{questionparts}
\end{question}	
		

		
	
\newpage
\section*{Section B: \ \ \ Mechanics}


	
%%%%%%%%%% Q9
\begin{question}
A 
painter of weight $kW$ uses a ladder to reach the guttering on the
outside wall of  a house. The wall is vertical and the ground
is horizontal.
The ladder is modelled as a uniform rod of weight $W$ and length $6a$.


The ladder is not long enough, so the painter
stands the ladder on a uniform table. The table has weight $2W$ and a square 
top of side $\frac12 a$ with  a leg of length $a$ at each corner.
The foot of the ladder is at the centre of the table top and the ladder
is inclined at an angle $\arctan 2$ to the horizontal.
The edge of the table nearest the wall is parallel to the wall.

 The coefficient of friction
between the foot of the ladder and the table top is $\frac12$.
The contact between the ladder and the wall is sufficiently smooth
for the effects of friction to be ignored.

\begin{questionparts}
\item Show that, if the legs of the table are fixed to the ground,
 the ladder does not slip on the table
however high the painter stands on the ladder.
\item It is given that $k=9$  and
that the
coefficient of friction between each table leg 
and the ground is $\frac13$. If the legs of the table are not fixed 
to the ground, 
so that  the table can tilt or slip, determine which 
occurs first when
 the painter slowly climbs the ladder.
\end{questionparts}

[Note: $\arctan 2$ is another notation for $\tan^{-1}2$.]
	\end{question}
	
%%%%%%%%%% Q10 
\begin{question}	
Three particles, $A$, $B$ and $C$, of masses $m$, $km$ and $3m$ respectively,
are initially at rest lying in a straight line on a smooth horizontal surface.
Then $A$ is projected towards  $B$ at speed $u$. After the 
collision, $B$ collides with  $C$. The coefficient of 
restitution between  $A$ and $B$ is $\frac12$ and 
the coefficient of 
restitution between $B$ and $C$ is $\frac14$.

\begin{questionparts}
\item Find the range of values of $k$ for which $A$ and 
$B$ collide for a second time.
\item Given that $k=1$ and that $B$ and $C$ are initially 
a distance $d$ apart, show that 
 the time that elapses between the two collisions 
of $A$ and $B$ is $\dfrac{60d}{13u}\,$.
\end{questionparts}
\end{question}

%%%%%%%%%% Q11

\begin{question}
A projectile of unit mass
is fired in a northerly direction
from a point on a horizontal plain at speed $u$ and an angle 
$\theta$ above the horizontal.  It lands at a point $A$ on the plain.
In flight, the 
projectile experiences two forces: 
gravity, of magnitude $g$; and 
a horizontal force of
constant magnitude $f$ due to a wind blowing from North to South. Derive an
expression, in terms of $u$, $g$, $f$ and $\theta$ for the distance $OA$.

\begin{questionparts} 
\item Determine the angle $\alpha$ such that, for all $\theta>\alpha$,
the wind starts to blow the projectile back towards $O$ before it 
lands at $A$.
\item An identical projectile, which experiences the same forces,
 is fired from $O$ in a northerly direction
at speed $u$ and angle $45^\circ$ above the 
horizontal and lands at a point $B$ on the plain. Given that $\theta$ is
chosen to maximise $OA$, show that 
\[
\frac{OB}{OA} = \frac{ g-f}{\; \sqrt{g^2+f^2\;}- f \;\;}\;.
\]

Describe carefully the motion of the second  projectile when $f=g$.
\end{questionparts}
\end{question}
	

	
	\newpage
\section*{Section C: \ \ \ Probability and Statistics}


%%%%%%%%%% Q12
\begin{question}
A cricket team has only three bowlers, Arthur, Betty and 
Cuba, each of whom bowls 30
balls in any match. Past performance reveals that, on average, 
 Arthur  takes one wicket for every~36 balls bowled, 
Betty  takes one wicket for every 25 balls bowled,
and Cuba  takes one wicket for every 41 balls bowled.

\begin{questionparts}
\item In one match, the team took exactly one wicket, but 
the name of the bowler was
not recorded. Using a binomial model,
find the probability that
 Arthur  was the bowler.
\item Show that the  average 
number of wickets taken by the team in a match is approximately~3.
Give  with brief justification a suitable model for the number of
wickets taken by the team in a match and show that  
the probability of the team taking at least five wickets in a given match
is approximately $\frac15$. 

\noindent[You may use the approximation $\e^3 = 20$.]
\end{questionparts}
\end{question}

%%%%%%%%%% Q13
\begin{question}
I know that 
ice-creams come in $n$ different sizes, but I don't know what the sizes are.
I am offered one of each in 
succession, in random order.
I am certainly going to choose one - the bigger
the better -  but I
am not allowed more than one. My strategy is to reject the first  
ice-cream I am offered
 and choose the  first one thereafter  that is bigger than the first 
one I was offered; if the first ice-cream offered is in fact the biggest one,
then I have to put up with the last one, however small.

Let $\P_n(k)$ be the probability that I choose the $k$th biggest ice-cream,
where $k=1$ is the biggest and $k=n$ is the smallest.

\begin{questionparts}
\item Show that $\P_4(1) = \frac{11}{24}$ and find $\P_4(2)$, $\P_4(3)$
and $\P_4(4)$.
\item Find an expression  for $\P_n(1)$.
\end{questionparts}
\end{question}

%%%%%%%%%% Q14
\begin{question}
Sketch the graph of 
$
 y= \dfrac1 { x \ln x}
\text{ \ \ for $x>0$, \ $x\ne1$}.
$
You may assume that $x\ln x \to 0$ as $x\to 0$.


The continuous random variable $X$ has probability density function 
\[
\f(x) = 
\begin{cases}
\dfrac \lambda {x\ln x}& \text{for $a\le x \le b$}\;, \\[3mm]
\ \ \  0 & \text{otherwise },
\end{cases}
\]
where $a$, $b$ and $\lambda$ are suitably chosen constants.

\begin{questionparts}

\item In the case $a=1/4$ and $b=1/2$, find $\lambda\,$.
\item In the case $\lambda=1$ and $a>1$, show that $b=a^\e$.
\item In the case $\lambda =1$ and $a=\e$, show that 
$\P(\e^{3/2}\le X \le \e^2)\approx \frac {31}{108}\,$.
\item In the case $\lambda =1$ and $a=\e^{1/2}$, find  
$\P(\e^{3/2}\le X \le \e^2)\;$.

\end{questionparts}
\end{question}
	
\end{document}
