
\documentclass[a4, 11pt]{report}


\pagestyle{myheadings}
\markboth{}{Paper I, 1989
\ \ \ \ \ 
\today 
}               

\RequirePackage{amssymb}
\RequirePackage{amsmath}
\RequirePackage{graphicx}
\RequirePackage{color}
\RequirePackage[flushleft]{paralist}[2013/06/09]



\RequirePackage{geometry}
\geometry{%
  a4paper,
  lmargin=2cm,
  rmargin=2.5cm,
  tmargin=3.5cm,
  bmargin=2.5cm,
  footskip=12pt,
  headheight=24pt}


\newcommand{\comment}[1]{{\bf Comment} {\it #1}}
%\renewcommand{\comment}[1]{}

\newcommand{\bluecomment}[1]{{\color{blue}#1}}
%\renewcommand{\comment}[1]{}
\newcommand{\redcomment}[1]{{\color{red}#1}}



\usepackage{epsfig}
\usepackage{pstricks-add}
\usepackage{tgheros} %% changes sans-serif font to TeX Gyre Heros (tex-gyre)
\renewcommand{\familydefault}{\sfdefault} %% changes font to sans-serif
%\usepackage{sfmath}  %%%% this makes equation sans-serif
%\input RexFigs


\setlength{\parskip}{10pt}
\setlength{\parindent}{0pt}

\newlength{\qspace}
\setlength{\qspace}{20pt}


\newcounter{qnumber}
\setcounter{qnumber}{0}

\newenvironment{question}%
 {\vspace{\qspace}
  \begin{enumerate}[\bfseries 1\quad][10]%
    \setcounter{enumi}{\value{qnumber}}%
    \item%
 }
{
  \end{enumerate}
  \filbreak
  \stepcounter{qnumber}
 }


\newenvironment{questionparts}[1][1]%
 {
  \begin{enumerate}[\bfseries (i)]%
    \setcounter{enumii}{#1}
    \addtocounter{enumii}{-1}
    \setlength{\itemsep}{5mm}
    \setlength{\parskip}{8pt}
 }
 {
  \end{enumerate}
 }



\DeclareMathOperator{\cosec}{cosec}
\DeclareMathOperator{\Var}{Var}

\def\d{{\rm d}}
\def\e{{\rm e}}
\def\g{{\rm g}}
\def\h{{\rm h}}
\def\f{{\rm f}}
\def\p{{\rm p}}
\def\s{{\rm s}}
\def\t{{\rm t}}


\def\A{{\rm A}}
\def\B{{\rm B}}
\def\E{{\rm E}}
\def\F{{\rm F}}
\def\G{{\rm G}}
\def\H{{\rm H}}
\def\P{{\rm P}}


\def\bb{\mathbf b}
\def \bc{\mathbf c}
\def\bx {\mathbf x}
\def\bn {\mathbf n}

\newcommand{\low}{^{\vphantom{()}}}
%%%%% to lower suffices: $X\low_1$ etc


\newcommand{\subone}{ {\vphantom{\dot A}1}}
\newcommand{\subtwo}{ {\vphantom{\dot A}2}}




\def\le{\leqslant}
\def\ge{\geqslant}


\def\var{{\rm Var}\,}

\newcommand{\ds}{\displaystyle}
\newcommand{\ts}{\textstyle}




\begin{document}
\setcounter{page}{2}

 
\section*{Section A: \ \ \ Pure Mathematics}

%%%%%%%%%%Q1
\begin{question}
$\,$


\noindent \begin{center}
\psset{xunit=1.0cm,yunit=1.0cm,algebraic=true,dotstyle=o,dotsize=3pt 0,linewidth=0.5pt,arrowsize=3pt 2,arrowinset=0.25} \begin{pspicture*}(-2.7,0.1)(4.82,5.5) \parametricplot{0.0}{3.141592653589793}{1*3*cos(t)+0*3*sin(t)+1|0*3*cos(t)+1*3*sin(t)+1.8} \parametricplot{3.141592653589793}{6.283185307179586}{1*1*cos(t)+0*1*sin(t)+1|0*1*cos(t)+1*1*sin(t)+1.8} \parametricplot{0.0}{3.141592653589793}{1*1*cos(t)+0*1*sin(t)+-1|0*1*cos(t)+1*1*sin(t)+1.8} \parametricplot{0.0}{3.141592653589793}{1*1*cos(t)+0*1*sin(t)+3|0*1*cos(t)+1*1*sin(t)+1.8} \rput[tl](-2.38,1.92){$A$} \rput[tl](1.1,5.3){$H$} \rput[tl](-1.02,3.3){$B$} \rput[tl](-0.44,2.14){$C$} \rput[tl](0.74,0.72){$D$} \rput[tl](2.2,2.14){$E$} \rput[tl](2.96,3.3){$F$} \rput[tl](4.14,1.92){$G$} \end{pspicture*}
\par\end{center}


In the above diagram, $ABC,CDE,EFG$ and $AHG$ are semicircles and
$A,C,E,G$ lie on a straight line. The radii of $ABC,EFG,AHG$ are
$h$, $h$ and $r$ respectively, where $2h<r$. Show that the area
enclosed by $ABCDEFGH$ is equal to that of a circle with diameter
$HD$. 


Each semicircle is now replaced by a portion of a parabola, with vertex
at the midpoint of the semicircle arc, passing through the endpoints
(so, for example, $ABC$ is replaced by part of a parabola passing
through $A$ and $C$ and with vertex at $B$). Find a formula in
terms of $r$ and $h$ for the area enclosed by $ABCDEFGH$. 
\end{question}

%%%%%%%%%%Q2
\begin{question}
For $x>0$ find $\int x\ln x\,\mathrm{d}x$. 


By approximating the area corresponding to $\int_{0}^{1}x\ln(1/x)\, \d x$
by $n$ rectangles of equal width and with their top right-hand vertices
on the curve $y=x\ln(1/x)$, show that, as $n\rightarrow\infty$,
\[
\frac{1}{2}\left(1+\frac{1}{n}\right)\ln n-\frac{1}{n^{2}}\left[\ln\left(\frac{n!}{0!}\right)+\ln\left(\frac{n!}{1!}\right)+\ln\left(\frac{n!}{2!}\right)+\cdots+\ln\left(\frac{n!}{(n-1)!}\right)\right]\rightarrow\frac{1}{4}.
\]



{[}You may assume that $x\ln x\rightarrow0$ as $x\rightarrow0$.{]}
\end{question}

%%%%%%%%% Q3
\begin{question}
In the triangle $OAB,$ $\overrightarrow{OA}=\mathbf{a},$ $\overrightarrow{OB}=\mathbf{b}$
and $OA=OB=1$. Points $C$ and $D$ trisect $AB$ (i.e. $AC=CD=DB=\frac{1}{3}AB$).
$X$ and $Y$ lie on the line-segments $OA$ and $OB$ respectively,
in such a way that $CY$ and $DX$ are perpendicular, and $OX+OY=1$.
Denoting $OX$ by $x$, obtain a condition relating $x$ and $\mathbf{a\cdot b}$,
and prove that 
\[
\frac{8}{17}\leqslant\mathbf{a\cdot b}\leqslant1.
\]
If the angle $AOB$ is as large as possible, determine the distance
$OE,$ where $E$ is the point of intersection of $CY$ and $DX$. 
\end{question}


%%%%%% Q4 

\begin{question}
Six points $A,B,C,D,E$ and $F$ lie in three dimensional space and
are in general positions, that is, no three are collinear and no four
lie on a plane. All possible line segments joining pairs of points
are drawn and coloured either gold or silver. Prove that there is
a triangle whose edges are entirely of one colour. {[}\textbf{Hint}:
consider segments radiating from $A.${]}


Give a sketch showing that the result is false for five points in
general positions. 
\end{question}


%%%%%%%%% Q5
\begin{question}
Write down the binomial expansion of $(1+x)^{n}$, where $n$ is a
positive integer. 

\begin{questionparts}
\item By substituting particular values of $x$ in the above expression,
or otherwise, show that, if $n$ is an even positive integer, 
\[
\binom{n}{0}+\binom{n}{2}+\binom{n}{4}+\cdots+\binom{n}{n}=\binom{n}{1}+\binom{n}{3}+\binom{n}{5}+\cdots+\binom{n}{n-1}=2^{n-1}.
\]

\item Show that, if $n$ is any positive integer, then 
\[
\binom{n}{1}+2\binom{n}{2}+3\binom{n}{3}+\cdots+n\binom{n}{n}=n2^{n-1}.
\]

\end{questionparts}

Hence evaluate 
\[
\sum_{r=0}^{n}\left(r+(-1)^{r}\right)\binom{n}{r}\,.
\]
\end{question}
	
	%%%%%%%%% Q6
	\begin{question}
The normal to the curve $y=\mathrm{f}(x)$ at the point $P$ with
coordinates $(x,\mathrm{f}(x))$ cuts the $y$-axis at the point $Q$.
Derive an expression in terms of $x$, $\mathrm{f}(x)$ and $\mathrm{f}'(x)$
for the $y$-coordinate of $Q$. 


If, for all $x$, $PQ=\sqrt{\mathrm{e}^{x^{2}}+x^{2}}$, find a differential
equation satisfied by $\mathrm{f}(x)$. If the curve also has a minimum
point $(0,-2)$, find its equation.
	 \end{question}
	 
	 %%%%%%%%% Q7
\begin{question}
Sketch the curve $y^{2}=1-\left|x\right|$. A rectangle, with sides
parallel to the axes, is inscribed within this curve. Show that the
largest possible area of the rectangle is $8/\sqrt{27}$. 


Find the maximum area of a rectangle similarly inscribed within the
curve given by $y^{2m}=\left(1-\left|x\right|\right)^{n}$, where
$m$ and $n$ are positive integers, with $n$ odd. 
	\end{question}
	
	%%%%%%%%% Q8
	\begin{question}
By using de Moivre's theorem, or otherwise, show that 

\begin{itemize}
\setlength{\itemsep}{3mm}
\item[\bf (i)] $\cos4\theta=8\cos^{4}\theta-8\cos^{2}\theta+1;$
\item[\bf (ii)] $\cos6\theta=32\cos^{6}\theta-48\cos^{4}\theta+18\cos^{2}\theta-1.$
\end{itemize}

Hence, or otherwise, find all the real roots of the equation 
\[
16x^{6}-28x^{4}+13x^{2}-1=0.
\]
{[}No credit will be given for numerical approximations.{]} 
		\end{question}
		
		
%%%%%%%%% Q9
\begin{question}
Sketch the graph of $8y=x^{3}-12x$ for $-4\leqslant x\leqslant4$,
marking the coordinates of the turning points. Similarly marking the
turning points, sketch the corresponding graphs in the $(X,Y)$-plane,
if
\begin{alignat*}{3}
\rm{(a)} & \quad &  & X=\tfrac{1}{2}x, & \qquad & Y=y,\\
\rm{(b)} &  &  & X=x, &  & Y=\tfrac{1}{2}y,\\
\rm{(c)} &  &  & X=\tfrac{1}{2}x+1, &  & Y=y,\\
\rm{(d)} &  &  & X=x, &  & Y=\tfrac{1}{2}y+1.
\end{alignat*}
 Find values for $a,b,c,d$ such that, if $X=ax+b,$ $Y=cy+d$, then
the graph in the $(X,Y)$-plane corresponding to $8y=x^{3}-12x$ has
turning points at $(X,Y)=(0,0)$ and $(X,Y)=(1,1)$. 
\end{question}

\newpage
\section*{Section B: \ \ \ Mechanics}


	
%%%%%%%%%% Q10
\begin{question}
A spaceship of mass $M$ is travelling at constant speed $V$ in a
straight line when it enters a force field which applies a resistive
force acting directly backwards and of magnitude \phantom{ }$M\omega(v^{2}+V^{2})/v$,
where $v$ is the instantaneous speed of the spaceship, and $\omega$
is a positive constant. No other forces act on the spaceship. Find
the distance travelled from the edge of the force field until the
speed is reduced to $\frac{1}{2}V$. 


As soon as the spaceship has travelled this distance within the force
field, the field is altered to a constant resistive force, acting
directly backwards, whose magnitude is within 10\% of that of the
force acting on the spaceship immediately before the change. If $z$
is the extra distance travelled by the spaceship before coming instantaneously
to rest, determine limits between which $z$ must lie. 
	\end{question}
	
%%%%%%%%%% Q11
\begin{question}	
A shot-putter projects a shot at an angle $\theta$ above the horizontal,
releasing it at height $h$ above the level ground, with speed $v$.
Show that the distance $R$ travelled horizontally by the shot from
its point of release until it strikes the ground is given by 
\[
R=\frac{v^{2}}{2g}\sin2\theta\left(1+\sqrt{1+\frac{2hg}{v^{2}\sin^{2}\theta}}\right).
\]
The shot-putter's style is such that currently $\theta=45^{\circ}$.
Determine (with justification) whether a small decrease in $\theta$
will increase $R$. 


{[}Air resistance may be neglected.{]}
\end{question}

%%%%%%%%%% Q12

\begin{question}A regular tetrahedron $ABCD$ of mass $M$ is made of 6 identical
uniform rigid rods, each of length $2a.$ Four light elastic strings
$XA,XB,XC$ and $XD$, each of natural length $a$ and modulus of
elasticity $\lambda,$ are fastened together at $X$, the other end
of each string being attached to the corresponding vertex. Given that
$X$ lies at the centre of mass of the tetrahedron, find the tension
in each string. 


The tetrahedron is at rest on a smooth horizontal table, with $B,C$
and $D$ touching the table, and the ends of the strings at $X$ attached
to a point $O$ fixed in space. Initially the centre of mass of the
tetrahedron coincides with $O.$ Suddenly the string $XA$ breaks,
and the tetrahedron as a result rises vertically off the table. If
the maximum height subsequently attained is such that $BCD$ is level
with the fixed point $O,$ show that (to 2 significant figures) 
\[
\frac{Mg}{\lambda}=0.098.
\]
\end{question}
	
%%%%%%%%%% Q13
\begin{question}
A uniform ladder of mass $M$ rests with its upper end against a smooth
vertical wall, and with its lower end on a rough slope which rises
upwards towards the wall and makes an angle of $\phi$ with the horizontal.
The acute angle between the ladder and the wall is $\theta$. If the
ladder is in equilibrium, show that $N$ and $F$, the normal reaction
and frictional force at the foot of the ladder are given by 
\[
N=Mg\left(\cos\phi-\frac{\tan\theta\sin\phi}{2}\right),
\]
\[
F=Mg\left(\sin\phi+\frac{\tan\theta\cos\phi}{2}\right).
\]
If the coefficient of friction between the ladder and the slope is
$2$, and $\phi=45^{\circ}$, what is the largest value of $\theta$
for which the ladder can rest in equilibrium?
\end{question}
	
	\newpage
\section*{Section C: \ \ \ Probability and Statistics}


%%%%%%%%%% Q14
\begin{question}
The prevailing winds blow in a constant southerly direction from an
enchanted castle. Each year, according to an ancient tradiction, a
princess releases 96 magic seeds from the castle, which are carried
south by the wind before falling to rest. South of the castle lies
one league of grassy parkland, then one league of lake, then one league
of farmland, and finally the sea. If a seed falls on land it will
immediately grow into a fever tree. (Fever trees do not grow in water).
Seeds are blown independently of each other. The random variable $L$
is the distance in leagues south of the castle at which a seed falls
to rest (either on land or water). It is known that the probability
density function $\mathrm{f}$ of $L$ is given by 
\[
\mathrm{f}(x)=\begin{cases}
\frac{1}{2}-\frac{1}{8}x & \mbox{ for }0\leqslant x\leqslant4,\\
0 & \mbox{ otherwise.}
\end{cases}
\]
What is the mean number of fever trees which begin to grow each year?

\begin{questionparts}
\item The random variable $Y$ is defined as the distance in leagues south
of the castle at which a new fever tree grows from a seed carried
by the wind. Sketch the probability density function of $Y$, and
find the mean of $Y$. 
\item One year messengers bring the king the news that 23 new fever trees
have grown in the farmland. The wind never varies, and so the king
suspects that the ancient tradition have not been followed properly.
Is he justified in his suspicions?
\end{questionparts}
\end{question}

%%%%%%%%%% Q15
\begin{question}
I can choose one of three routes to cycle to school. Via Angle Avenue
the distance is 5$\,$km, and I am held up at a level crossing for
$A$ minutes, where $A$ is a continuous random variable uniformly
distributed between $0$ and 10. Via Bend Boulevard the distance is
4$\,$km, and I am delayed, by talking to each of $B$ friends for
3$\,$minutes, for a total of $3B$ minutes, where $B$ is a random
variable whose distribution is Poisson with mean 4. Via Detour Drive
the distance should be only 2$\,$km, but in addition, due to never-ending
road works, there are five places at each of which, with probability
$\frac{4}{5},$ I have to make a detour that increases the distance
by 1$\,$km. Except when delayed by talking to friends or at the level
crossing, I cycle at a steady 12$\,$km$\,$h$^{-1}$. For each of
the three routs, calculate the probability that a journey lasts at
least 27 minutes. 


Each day I choose one of the three routes at random, and I am equally
likely to choose any of the three alternatives. One day I arrive at
school after a journey of at least 27 minutes. What is the probability
that I came via Bend Boulevard?


Which route should I use all the time: 
\begin{itemize}
\setlength{\itemsep}{3mm}
\item[\bf (i)] if I wish my average journey time to be as small as possible; 
\item[\bf (ii)] if I wish my journey time to be less than 32 minutes as often as possible?
\end{itemize}

Justify your answers. 
\end{question}

%%%%%%%%%% Q16
\begin{question}
 A and B play a guessing game. Each simultaneously names one of the
numbers $1,2,3.$ If the numbers differ by 2, whoever guessed the
\textit{smaller }pays the opponent $\pounds 2$. If the numbers differ
by 1, whoever guessed the \textit{larger }pays the opponent $\pounds 1.$
Otherwise no money changes hands. Many rounds of the game are played. 

\begin{questionparts}
\item If A says he will always guess the same number $N$, explain (for
each value of $N$) how B can maximise his winnings. 
\item In an attempt to improve his play, A announces that he will guess
each number at random with probability $\frac{1}{3},$ guesses on
different rounds being independent. To counter this, B secretly decides
to guess $j$ with probability $b_{j}$ ($j=1,2,3,\, b_{1}+b_{2}+b_{3}=1$),
guesses on different rounds being independent. Derive an expression
for B's expected winnings on any round. How should the probabilities
$b_{j}$ be chosen so as to maximize this expression? 
\item A now announces that he will guess $j$ with probability $a_{j}$
($j=1,2,3,\, a_{1}+a_{2}+a_{3}=1$). If B guesses $j$ with probability
$b_{j}$ ($j=1,2,3,\, b_{1}+b_{2}+b_{3}=1$), obtain an expression
for his expected winnings in the form 
\[
Xa_{1}+Ya_{2}+Za_{3}.
\]
Show that he can choose $b_{1},b_{2}$ and $b_{3}$ such that $X,Y$
and $Z$ are all non-negative. Deduce that, whatever values for $a_{j}$
are chosen by A, B can ensure that in the long run he loses no money.
 \end{questionparts}
\end{question}
\end{document}
