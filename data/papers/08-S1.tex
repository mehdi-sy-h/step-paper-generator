\documentclass[a4, 11pt]{report}


\pagestyle{myheadings}
\markboth{}{Paper I, 2008
\ \ \ \ \ 
\today 
}               

\RequirePackage{amssymb}
\RequirePackage{amsmath}
\RequirePackage{graphicx}
\RequirePackage{color}
\RequirePackage[flushleft]{paralist}[2013/06/09]



\RequirePackage{geometry}
\geometry{%
  a4paper,
  lmargin=2cm,
  rmargin=2.5cm,
  tmargin=3.5cm,
  bmargin=2.5cm,
  footskip=12pt,
  headheight=24pt}


\newcommand{\comment}[1]{{\bf Comment} {\it #1}}
%\renewcommand{\comment}[1]{}

\newcommand{\bluecomment}[1]{{\color{blue}#1}}
%\renewcommand{\comment}[1]{}
\newcommand{\redcomment}[1]{{\color{red}#1}}



\usepackage{epsfig}
\usepackage{pstricks-add}
\usepackage{tgheros} %% changes sans-serif font to TeX Gyre Heros (tex-gyre)
\renewcommand{\familydefault}{\sfdefault} %% changes font to sans-serif
%\usepackage{sfmath}  %%%% this makes equation sans-serif
%\input RexFigs


\setlength{\parskip}{10pt}
\setlength{\parindent}{0pt}

\newlength{\qspace}
\setlength{\qspace}{20pt}


\newcounter{qnumber}
\setcounter{qnumber}{0}

\newenvironment{question}%
 {\vspace{\qspace}
  \begin{enumerate}[\bfseries 1\quad][10]%
    \setcounter{enumi}{\value{qnumber}}%
    \item%
 }
{
  \end{enumerate}
  \filbreak
  \stepcounter{qnumber}
 }


\newenvironment{questionparts}[1][1]%
 {
  \begin{enumerate}[\bfseries (i)]%
    \setcounter{enumii}{#1}
    \addtocounter{enumii}{-1}
    \setlength{\itemsep}{5mm}
    \setlength{\parskip}{8pt}
 }
 {
  \end{enumerate}
 }



\DeclareMathOperator{\cosec}{cosec}
\DeclareMathOperator{\Var}{Var}

\def\d{{\mathrm d}}
\def\e{{\mathrm e}}
\def\g{{\mathrm g}}
\def\h{{\mathrm h}}
\def\f{{\mathrm f}}
\def\p{{\mathrm p}}
\def\s{{\mathrm s}}
\def\t{{\mathrm t}}


\def\A{{\mathrm A}}
\def\B{{\mathrm B}}
\def\E{{\mathrm E}}
\def\F{{\mathrm F}}
\def\G{{\mathrm G}}
\def\H{{\mathrm H}}
\def\P{{\mathrm P}}


\def\bb{\mathbf b}
\def \bc{\mathbf c}
\def\bx {\mathbf x}
\def\bn {\mathbf n}

\newcommand{\low}{^{\vphantom{()}}}
%%%%% to lower suffices: $X\low_1$ etc


\newcommand{\subone}{ {\vphantom{\dot A}1}}
\newcommand{\subtwo}{ {\vphantom{\dot A}2}}




\def\le{\leqslant}
\def\ge{\geqslant}


\def\var{{\rm Var}\,}

\newcommand{\ds}{\displaystyle}
\newcommand{\ts}{\textstyle}
\def\half{{\textstyle \frac12}}
\def\l{\left(}
\def\r{\right)}



\begin{document}
\setcounter{page}{2}

 
\section*{Section A: \ \ \ Pure Mathematics}

%%%%%%%%%%Q1
\begin{question}
What does it mean to say that a number $x$ is {\em irrational}?


Prove by contradiction statements A and B below, where $p$ and $q$ are
real numbers.
\begin{itemize}
\item [\quad\bf A:] If $pq$ is irrational, then at least one
of $p$ and $q$ is irrational.
\item [\quad\bf B:] If $p+q$ is irrational, then at least one of 
$p$ and $q$ is irrational.
\end{itemize}
Disprove  by means of a counterexample  statement C below, 
where $p$ and $q$ are
real numbers.
\begin{itemize}
\item [\quad\bf C:]
If $p$ and $q$ are irrational, then $p+q$ is  irrational.
\end{itemize}


If the numbers $\e$, $\pi$, $\pi^2$, $\e^2$ and 
$\e\pi$ are irrational, prove  that at most one of the numbers
$\pi+\e$, $\pi -\e$, $\pi^2-\e^2$, $\pi^2+\e^2$ is rational.
\end{question}

%%%%%%%%%%Q2
\begin{question}
The variables $t$ and $x$ are related by
 $t=x+ \sqrt{x^2+2bx+c\;} \,$, where $b$ and $c$ are constants
and $b^2<c$.  Show that 
\[
\frac{\d x}{\d t} = \frac{t-x}{t+b}\;,
\]
and hence
 integrate
$ \ \ds \frac1 {\ \sqrt{x^2+2bx+c} \ }\,$.

Verify by direct integration that your result holds also 
in the case $b^2=c$ if  $x+b>0$ but that your result 
does not hold in the case
$b^2=c$ if $x+b<0\,$.
\end{question}

%%%%%%%%% Q3
\begin{question}
Prove that, if $c\ge a$ and $d\ge b$, then 
\[
ab+cd\ge bc+ad\,.
\tag{$*$}
\]

\begin{questionparts} 
\item If $x\ge y$, use $(*)$ to show that $x^2+y^2\ge 2xy\,$.

If, further, $x\ge z$ and $y\ge z$, use $(*)$ to
 show that $z^2+xy\ge xz+yz$ and deduce 
that $x^2+y^2+z^2\ge xy+yz+zx\,$. 

Prove that 
the inequality  $x^2+y^2+z^2\ge xy+yz+zx\,$
holds for all $x$, $y$ and $z$.



\item Show similarly that the inequality
\[\frac st +\frac tr +\frac rs 
+\frac ts +\frac rt +\frac sr 
\ge 6\]
 holds for all
positive $r$, $s$ and $t$.


\end{questionparts}
\end{question}

%%%%%% Q4 
\begin{question}
A function $\f(x)$ is said to be {\em convex} in the interval
 $a< x < b$
if $\f''(x)\ge0$ for all $x$ in this interval.

\begin{questionparts}
\item Sketch on the same axes the graphs of $y= \frac23 \cos^2 x$ and
  $y=\sin x$ in the interval $0\le x \le 2\pi$.

The function $\f(x)$ is defined for $0<x< 2\pi$ by
\[\f(x) = \e^{\frac23 \sin x}.
\]
Determine the intervals in which $\f(x)$ is convex.


 \item 
The function $\g(x)$ is defined for $0< x< \frac12\pi$ by
\[\g(x) = \e^{-k \tan x}.
\]
If $k=\sin 2 \alpha$ and $0<\alpha< \frac{1}{4}\pi$, show that 
$\g(x)$ is convex in the interval
 $0<x<\alpha$, 
and give one other interval in which $\g(x)$ is convex.


\end{questionparts}
\end{question}

%%%%%%%%% Q5
\begin{question}
The polynomial  $\p(x)$ is given by
\[
\ds \p(x)=  x^n +\sum\limits_{r=0}^{n-1}a_rx^r\,,
\] 
where $a_0$, $a_1$, $\ldots$ , $a_{n-1}$ are fixed real
numbers and $n\ge1$.
Let
$M$ be the greatest value of $\big\vert \p(x) \big\vert$ for $\vert x \vert\le
1$. 
Then {\em Chebyshev's theorem} states that  $M\ge 2^{1-n}$.

\begin{questionparts}
\item Prove Chebyshev's theorem in the case $n=1$
and verify that Chebyshev's theorem holds in the following cases:

{\bf (a)} $ \p(x) = x^2 - \frac12\,$;

{\bf (b)} $\p(x) = x^3 -  x \,$.

\item Use Chebyshev's theorem to show that the curve  
$ \ y= 64x^5+25x^4-66x^3-24x^2+3x+1
\ $
has at least one turning point in the
  interval
$-1\le x \le 1$.


\end{questionparts}
	\end{question}
	
%%%%%%%%% Q6
\begin{question}
The function $\f$ is defined by
\[
\f(x) = \frac{\e^x-1}{\e-1},
\ \ \ \ \
x\ge0,
\]
and the function $\g$ is the inverse function to $\f$, so that
$\g(\f(x))=x$. Sketch $\f(x)$ and $\g(x)$ on the same axes.

Verify, by evaluating each integral, that
\[
\int_0^\frac12 \f(x) \,\d x + \int_0^k \g(x) \,\d x = \frac1 {2(\sqrt \e
  +1)}\,,
\]
where $\displaystyle k= \frac 1{\sqrt\e+1}$, and explain this result by means of a diagram.
\end{question}
	
%%%%%%%%% Q7
\begin{question}
The point $P$ has coordinates $(x,y)$ with respect to
the origin $O$. By writing $x=r\cos\theta$ and $y=r\sin\theta$,
or otherwise,  show that, if the line $OP$ is rotated by $60^\circ$
clockwise about $O$, the  new $y$-coordinate of $P$
is $\frac12(y-\sqrt3\,x)$.
What is the new $y$-coordinate in the case of an anti-clockwise
rotation by $60^\circ\,$?

An equilateral triangle $OBC$ has vertices at $O$, $(1,0)$ and 
$(\frac12,\frac12 \sqrt3)$, respectively. 
The point $P$ has coordinates
$(x,y)$. 
The perpendicular distance from $P$ to the line
through $C$ and $O$ is $h_1$;  
the perpendicular distance from $P$ to the line
through $O$ and $B$ is $h_2$; 
and the perpendicular distance from $P$ to the line through $B$ and $C$ is
$h_3$.

Show that 
$h_1=\frac12 \big\vert y-\sqrt3\,x\big\vert$ and find expressions
for $h_2$ and $h_3$.


Show that $h_1+h_2+h_3=\frac12 \sqrt3$ if and only if $P$ lies on or
in the triangle $OBC$.
\end{question}
		
%%%%%%%%% Q8
\begin{question}	
\begin{questionparts}
\item
The gradient $y'$ of a curve at a point $(x,y)$ satisfies
\[
(y')^2 -xy'+y=0\,.
\tag{$*$}
\]
By differentiating $(*)$ with respect to $x$, show that either
$y''=0$ or $2y'=x\,$. 

Hence show that the curve is either a straight line of the form
$y=mx+c$, where $c=-m^2$, or the parabola $4y=x^2$.

\item
The gradient $y'$ of a curve at a point $(x,y)$ satisfies
\[
(x^2-1)(y')^2 -2xyy'+y^2-1=0\,.
\]
Show that the curve is either a straight line, the form
of which you should specify,
or a circle, the equation of which you should determine.


\end{questionparts}
\end{question}	
		

		
	
\newpage
\section*{Section B: \ \ \ Mechanics}


	
%%%%%%%%%% Q9
\begin{question}
Two identical particles $P$ and $Q$, each of 
mass $m$, are attached to the ends of a diameter
of a light thin circular hoop of radius $a$. 
The hoop rolls without slipping
 along a straight line
on a horizontal table with  the plane of the hoop vertical.
Initially, $P$ is in contact with 
the table. At time $t$, the hoop has rotated through an angle
$\theta$. Write down the position at time $t$ of $P$, relative to its starting
point, in cartesian coordinates, and  determine its speed in
terms
of $a$, $\theta$ and $\dot\theta$. 
Show that the total 
kinetic energy of the two particles is $2ma^2\dot\theta^2$.

Given that the only external forces on the system are gravity
and the vertical reaction of the table on the hoop,
show that the hoop rolls with constant speed.
	\end{question}
	
%%%%%%%%%% Q10 
\begin{question}	
On the (flat) planet Zog, the acceleration due to gravity 
is $g$ up to height $h$ above the surface and $g'$ at greater heights.
A particle is projected from the surface at speed $V$ and at an 
angle $\alpha$ to the surface, where $V^2 \sin^2\alpha > 2 gh\,$.
Sketch, on the same axes,  the trajectories in the cases $g'=g$
and $g'<g$.

Show that the particle lands a distance $d$ from the point of
projection given by
\[
d = \left(\frac {V-V'} g + \frac {V'}{  g'}  \right)
V\sin2\alpha\,,
\]
where $V' = \sqrt{V^2-2gh\,\rm{cosec}^2\alpha\,}\,$.
\end{question}

%%%%%%%%%% Q11

\begin{question}
A straight uniform rod has mass $m$. Its  ends $P_1$ and 
$P_2$ are attached to small light rings that are constrained to move
on a rough rigid circular wire with centre $O$
fixed in a vertical plane, and the angle
$P_1OP_2$ is a right angle. 
The rod
rests with $P_1$ lower than $P_2$, and with both ends lower than $O$.
The coefficient
of friction between each of the rings and the wire is $\mu$.  
 Given that the rod is in limiting equilibrium (i.e. \  on the point of
 slipping at both ends),
show that 
\[
\tan \alpha = \frac{1-2\mu -\mu^2}{1+2\mu -\mu^2}\,,
\]
where $\alpha$ is the angle between $P_1O$ and the vertical 
($0<\alpha<45^\circ$).

 Let $\theta$ be 
the acute angle between the rod and the horizontal.
Show that $\theta =2\lambda$, where
$\lambda $ is defined by 
 $\tan \lambda= \mu$ and   $0<\lambda<22.5^\circ$.
\end{question}
	

	
	\newpage
\section*{Section C: \ \ \ Probability and Statistics}


%%%%%%%%%% Q12
\begin{question}
In this question, you may use without proof the results:
\[
 \sum_{r=1}^n r = \tfrac12 n(n+1)
\qquad\text{and}\qquad
 \sum_{r=1}^n r^2 = \tfrac1 6 n(n+1)(2n+1)\,.
\]


The independent random variables $X_1$ and $X_2$ each take values
$1$, $2$, $\ldots$, $N$, each value being equally likely. The random
variable
$X$ is defined by
\[
X= 
\begin{cases}
X_1 & \text { if } X_1\ge X_2\\
X_2 & \text { if } X_2\ge X_1\;.
\end{cases}
\]

\begin{questionparts}
\item Show that $\P(X=r) = \dfrac{2r-1}{N^2}\,$ for $r=1$, $2$, $\ldots$, $N$.



\item Find an expression for the expectation, $\mu$,  of $X$ and show
  that $\mu=67.165$ in the case $N=100$. 
 \item The  median, $m$, of $X$ is defined to be the integer such that
$\P(X\ge m) \ge \frac 12$ and $\P(X\le m)\ge \frac12$. Find an expression
for $m$ in terms of $N$
and give an explicit value for $m$
 in the case $N=100$.

\item Show that when $N$ is very large,
\[
\frac \mu m
 \approx \frac {2\sqrt2}3\,.
\]
\end{questionparts}
\end{question}

%%%%%%%%%% Q13
\begin{question}
Three married couples sit down at a round table at which
there are six chairs. All of the possible seating  arrangements
 of the six people
are equally likely.

\begin{questionparts}
\item
Show that the probability that each husband sits next to his wife
is $\frac{2}{15}$.
\item
Find the probability that exactly two husbands sit next to their
wives.
\item Find the probability that no husband sits next to his wife.
\end{questionparts}
\end{question}

\end{document}
