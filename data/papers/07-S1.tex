\documentclass[a4, 11pt]{report}


\pagestyle{myheadings}
\markboth{}{Paper I, 2007
\ \ \ \ \ 
\today 
}               

\RequirePackage{amssymb}
\RequirePackage{amsmath}
\RequirePackage{graphicx}
\RequirePackage{color}
\RequirePackage[flushleft]{paralist}[2013/06/09]



\RequirePackage{geometry}
\geometry{%
  a4paper,
  lmargin=2cm,
  rmargin=2.5cm,
  tmargin=3.5cm,
  bmargin=2.5cm,
  footskip=12pt,
  headheight=24pt}


\newcommand{\comment}[1]{{\bf Comment} {\it #1}}
%\renewcommand{\comment}[1]{}

\newcommand{\bluecomment}[1]{{\color{blue}#1}}
%\renewcommand{\comment}[1]{}
\newcommand{\redcomment}[1]{{\color{red}#1}}



\usepackage{epsfig}
\usepackage{pstricks-add}
\usepackage{tgheros} %% changes sans-serif font to TeX Gyre Heros (tex-gyre)
\renewcommand{\familydefault}{\sfdefault} %% changes font to sans-serif
%\usepackage{sfmath}  %%%% this makes equation sans-serif
%\input RexFigs


\setlength{\parskip}{10pt}
\setlength{\parindent}{0pt}

\newlength{\qspace}
\setlength{\qspace}{20pt}


\newcounter{qnumber}
\setcounter{qnumber}{0}

\newenvironment{question}%
 {\vspace{\qspace}
  \begin{enumerate}[\bfseries 1\quad][10]%
    \setcounter{enumi}{\value{qnumber}}%
    \item%
 }
{
  \end{enumerate}
  \filbreak
  \stepcounter{qnumber}
 }


\newenvironment{questionparts}[1][1]%
 {
  \begin{enumerate}[\bfseries (i)]%
    \setcounter{enumii}{#1}
    \addtocounter{enumii}{-1}
    \setlength{\itemsep}{5mm}
    \setlength{\parskip}{8pt}
 }
 {
  \end{enumerate}
 }



\DeclareMathOperator{\cosec}{cosec}
\DeclareMathOperator{\Var}{Var}

\def\d{{\mathrm d}}
\def\e{{\mathrm e}}
\def\g{{\mathrm g}}
\def\h{{\mathrm h}}
\def\f{{\mathrm f}}
\def\p{{\mathrm p}}
\def\s{{\mathrm s}}
\def\t{{\mathrm t}}


\def\A{{\mathrm A}}
\def\B{{\mathrm B}}
\def\E{{\mathrm E}}
\def\F{{\mathrm F}}
\def\G{{\mathrm G}}
\def\H{{\mathrm H}}
\def\P{{\mathrm P}}


\def\bb{\mathbf b}
\def \bc{\mathbf c}
\def\bx {\mathbf x}
\def\bn {\mathbf n}

\newcommand{\low}{^{\vphantom{()}}}
%%%%% to lower suffices: $X\low_1$ etc


\newcommand{\subone}{ {\vphantom{\dot A}1}}
\newcommand{\subtwo}{ {\vphantom{\dot A}2}}




\def\le{\leqslant}
\def\ge{\geqslant}


\def\var{{\rm Var}\,}

\newcommand{\ds}{\displaystyle}
\newcommand{\ts}{\textstyle}
\def\half{{\textstyle \frac12}}
\def\l{\left(}
\def\r{\right)}



\begin{document}
\setcounter{page}{2}

 
\section*{Section A: \ \ \ Pure Mathematics}

%%%%%%%%%%Q1
\begin{question}
A positive integer with $2n$ digits (the first of which must not be $0$) is
called a {\sl balanced number} if the sum of the first
$n$ digits equals the sum of the last $n$ digits.
For example,
$1634$ is a  $4$-digit balanced number,
but $123401$ is not a balanced number.

\begin{questionparts}
\item
Show that seventy $4$-digit balanced numbers can 
be made using the digits $0, 1, 2, 3$ and~$4$.

\item
Show that 
\  $\frac16 {k \left( k+1 \right) \left( 4k+5 \right) }$  
\ $4$-digit balanced numbers can be made using the digits $0$ to $k$.


{\it You may use the identity 
$\ds \sum _{r=0}^{n} r^2 \equiv \tfrac{1}{6} {n \left( n+1 \right) 
\left( 2n+1 \right) } \;$.}

\end{questionparts}
\end{question}

%%%%%%%%%%Q2
\begin{question}
\begin{questionparts}
\item Given that $A = \arctan \frac12$
and that $B = \arctan\frac13\,$ (where $A$ and $B$
are acute) show, by considering $\tan \left( A + B \right)$, 
that $A + B = {\frac{1}{4}\pi  }$.


The non-zero integers $p$ and $q$ satisfy
\[
\displaystyle \arctan {\frac1 p} + \arctan {\frac1  q}
= {\frac\pi 4}\,.
\] 
Show that \mbox{$ \left ( p-1 \right) \left(q-1 \right) = 2$} and 
hence determine $p$ and $q$.



\item Let $r$, $s$ and $t$ be positive integers
such that  the highest common factor of $s$ and $t$ is $1$. Show that, if
\[
\arctan {\frac1  r} + \arctan \frac s {s+t} = {\frac\pi  4}\,,
\]
then there are only two possible values for $t$, and give $r$ in terms of $s$ in each case.

\end{questionparts}
\end{question}

%%%%%%%%% Q3
\begin{question}
Prove the identities
$\cos^4\theta -\sin^4\theta \equiv \cos 2\theta$ and
 $\cos^4 \theta + \sin^4 \theta \equiv 1 - {\frac12}
\sin^2 2 \theta$. Hence or otherwise evaluate
\[
\int_0^{\frac{1}{2}\pi} \cos^4 \theta \; \d \theta \;\;\;\;
\mbox{and}\;\;\;\; \int_0^{\frac{1}{2}\pi} \sin^4 \theta \; \d \theta \,.
\]
Evaluate also
\[
\int_0^{\frac{1}{2}\pi} \cos^6 \theta \; \d \theta \;\;\;\;
\mbox{and}\;\;\;\; \int_0^{\frac{1}{2}\pi} \sin^6 \theta \; \d \theta \,.
\]
\end{question}

%%%%%% Q4 
\begin{question}
Show that 
$x^3-3xbc + b^3 + c^3$ can be written in the form
$\left( x+ b+ c \right) {\rm Q}( x)$,
where ${\rm Q}( x )$ is a quadratic expression.
Show that $2{\rm Q }( x )$ can be written
as the sum of three expressions, each of which is a perfect square.


It is given that the equations
$ay^2 + by + c =0$ and $by^2 + cy + a = 0$
have a common root $k$. The coefficients $a$, $b$ and $c$ are
real, $a$ and $b$ are both non-zero, and $ac \neq b^2$. Show that
\[
\left( ac - b^2 \right) k = bc - a^2
\]
and determine a similar expression involving $k^2$. Hence show that
\[
\left( ac - b^2 \right) \left(ab-c^2 \right) = \left( bc - a^2 \right)^2
\]
and that $ a^3 -3abc + b^3 +c^3 = 0\,$. Deduce that either $k=1$
or  the two equations are identical.
\end{question}

%%%%%%%%% Q5
\begin{question}
{\it Note: a regular octahedron is a polyhedron with eight faces
each of which is an equilateral \hbox{triangle}.}
\begin{questionparts}
\item
Show that the angle between any two faces of a regular
octahedron is $\arccos \left( -{\frac1  3} \right)$.

\item
Find the  ratio of the volume of a regular octahedron to the volume of 
the cube whose vertices are the centres  of the faces of the 
octahedron.

\end{questionparts}
	\end{question}
	
%%%%%%%%% Q6
\begin{question}
\begin{questionparts}
\item
Given that 
$x^2 - y^2 = \left( x - y \right)^3$ 
and that $x-y = d$ (where $d \neq 0$), 
express each of $x$ and $y$ in terms of $d$. 
Hence find a pair of integers $m$ and $n$
satisfying $m-n = \left( \sqrt {m} - \sqrt{n} \right)^3$
where $m > n > 100$.

\item
Given that $x^3 - y^3 = \left( x - y \right)^4$ 
and that $x-y = d$ (where $d \neq 0$),
show that $3xy = d^3 - d^2$. Hence show that
\[
2x = d \pm d \sqrt {\frac{4d-1 }{3}}
\]
and determine a pair of distinct positive integers $m$ and $n$
such that $m^3 - n^3 = \left( m - n \right)^4$.

\end{questionparts}
\end{question}
	
%%%%%%%%% Q7
\begin{question}
\begin{questionparts}
\item The line $L_1$ has vector equation
$\displaystyle
{\bf r} =
\begin{pmatrix}
  1 \\
  0 \\
  2
\end{pmatrix}
+
\lambda
\begin{pmatrix}
\hphantom{-}  2 \\
 \hphantom{-} 2 \\
  -3
\end{pmatrix}
$.

The line $L_2$ has vector equation
$\displaystyle
{\bf r} =
\begin{pmatrix}
 \hphantom{-} 4 \\
  -2 \\
 \hphantom{-} 9
 \end{pmatrix}
+
\mu
\begin{pmatrix}
 \hphantom{-} 1 \\
 \hphantom{-} 2 \\
  -2
\end{pmatrix}
.
$

Show that the distance $D$
between a point on $L_1$ and a point on $L_2$
can be expressed in the form
\[
D^2 = \left(3\mu -4 \lambda-5 \right)^2 + \left( \lambda -1 \right)^2 + 36\,.
\]
Hence determine the minimum distance
between these two lines and find the coordinates
of the points on the two lines that are the minimum distance apart.


\item
The line $L_3$ has vector equation
${\bf r} =
\begin{pmatrix}
  2 \\
  3 \\
  5
\end{pmatrix}
+
\alpha
\begin{pmatrix}
  0 \\
  1 \\
  0
\end{pmatrix}
.
$

The line $L_4$ has vector equation
$
{\bf r} =
\begin{pmatrix}
\hphantom{-}  3 \\
\hphantom{-}  3 \\
  -2
\end{pmatrix}
+
\beta
\begin{pmatrix}
\,  4k\\
  1-k \\
 \!\!\! -3k
\end{pmatrix}
.
$

Determine the minimum distance between these two lines,
explaining geometrically the two different cases that arise
 according to the value of $k$.


\end{questionparts}
\end{question}
		
%%%%%%%%% Q8
\begin{question}	
A curve is given by the equation
\[
y = ax^3 - 6ax^2+ \left( 12a + 12 \right)x - \left( 8a + 16 \right)\,,
\tag{$*$}
\]
where $a$ is a real number. Show that this curve touches the curve 
with equation
\[
y=x^3
\tag{$**$}
\]
 at $\left( 2 \, , \, 8 \right)$. 
Determine the coordinates of any other point of
intersection of the two curves.

\begin{questionparts}
\item Sketch on the same axes the curves $(*)$
and $(**)$ when $a = 2$.

\item Sketch on the same axes the curves $(*)$
and $(**)$
when $a = 1$.

\item Sketch on the same axes the curves $(*)$
and $(**)$
when $a = -2$.
\end{questionparts}
\end{question}	
		

		
	
\newpage
\section*{Section B: \ \ \ Mechanics}


	
%%%%%%%%%% Q9
\begin{question}
A particle of weight $W$ is  placed on a rough plane
inclined at an angle of $\theta$ to the horizontal.
The coefficient of friction between the particle and
the plane is $\mu$. A horizontal force $X$ acting on
the particle is just sufficient to prevent the
particle from sliding down the plane; when a horizontal
force $kX$ acts on the particle, the particle is
about to slide up the plane. Both horizontal forces act in the vertical
plane containing the line of greatest slope.

Prove that
\[
\left( k-1 \right) \left( 1 + \mu^2 \right)
 \sin \theta \cos \theta = \mu \left( k + 1 \right)
\]
and hence that 
$\displaystyle k \ge \frac{ \left( 1+ \mu \right)^2}
{ \left( 1 - \mu \right)^2}$ .
	\end{question}
	
%%%%%%%%%% Q10 
\begin{question}	
The Norman army is advancing with constant speed $u$
towards the Saxon army, which is at rest.
When the armies are $d$ apart, a Saxon horseman rides from
the Saxon army directly towards the Norman army
at constant speed $x$.
Simultaneously a Norman horseman rides from the Norman
army directly towards the Saxon army at constant speed $y$, where $y >
u$. 
The horsemen ride their horses so that $y - 2x < u < 2y - x$.


When each horseman reaches the opposing army,
he immediately rides straight back to his own army
without changing his speed. Represent this information
on a displacement-time graph, and show that the two
horsemen pass each other at distances
\[
\frac{xd }{ x + y} \;\;
\mbox{and} \;\; \frac{xd(2y -x-u)}
{(u+x ) ( x + y )}
\]
from the Saxon army.


Explain briefly what will happen in the cases
(i) $u > 2y - x$
and (ii) $u < y - 2x$.
\end{question}

%%%%%%%%%% Q11

\begin{question}
A smooth, straight, narrow tube of length $L$
 is fixed at an angle of $30^\circ$ to the horizontal.
A~particle is fired up the tube, from the lower end,
with initial velocity $u$.
When the particle reaches the upper end of the tube,
it continues its motion until it returns to the same
level as the lower end of the tube, having travelled 
a horizontal distance $D$ after leaving the tube.
Show that $D$ satisfies the equation
\[
4gD^2 - 2 \sqrt{3} \left( u^2 - Lg \right)D
- 3L \left( u^2 - gL \right) = 0
\]
and hence that 
\[ 
\frac{{\rm d}D}{ {\rm d}L}
= - \frac{ 2\sqrt{3}gD - 3(u^2-2gL)}
{ 8gD - 2 \sqrt{3} \left(u^2 - gL \right)}.
\]



The final  horizontal displacement 
of the particle from the lower end of the tube is $R$. 
Show that $\dfrac{\d R}{\d L} = 0$ when $2D = L \sqrt 3$, 
and determine, in terms of $u$ and $g$,
 the corresponding value of $R$.
\end{question}
	

	
	\newpage
\section*{Section C: \ \ \ Probability and Statistics}


%%%%%%%%%% Q12
\begin{question}
\begin{questionparts}
\item A bag contains $N$ sweets (where $N \ge 2$),
of which $a$ are red. Two sweets are drawn
from the bag without replacement. Show that
the probability that the first sweet is red
is equal to the probability that the second sweet is red.

\item There are two bags, each containing $N$ sweets (where $N \ge 2$).
The first bag contains $a$ red sweets, and the
second bag contains $b$ red sweets. There is also a
biased coin, showing Heads with probability $p$ and Tails with probability $q$, where $p+q = 1$.

The coin is tossed. If it shows Heads then a
sweet is chosen from the first bag and transferred
to the second bag; if it shows Tails then a sweet
is chosen from the second bag and transferred
to the first bag. The coin is then tossed a second time:
if it shows Heads then a sweet is chosen from the first bag,
and if it shows Tails then a sweet is chosen from the second bag.

Show that the probability that the first sweet
is red is equal to the probability that the second sweet is red.
\end{questionparts}
\end{question}

%%%%%%%%%% Q13
\begin{question}
A bag contains eleven small discs, 
which are identical except that six of the discs are blank 
and five of the discs are numbered, 
using the numbers 1, 2, 3, 4 and 5. 
The bag is shaken, and four discs are taken one at a time without replacement.

Calculate the probability that:
\begin{questionparts}
\item all four discs taken are numbered;
\item all four discs taken are numbered, 
given that the disc numbered ``3'' is taken first;
\item exactly two numbered discs are taken, 
given that the disc numbered ``3'' is taken first;
\item exactly two numbered discs are taken, 
given that the disc numbered ``3'' is taken;
\item
exactly two numbered discs are taken, 
given that a numbered disc is taken first;
\item exactly two numbered discs are taken, 
given that a numbered disc is taken.
\end{questionparts}
\end{question}

%%%%%%%%%% Q14
\begin{question}
The discrete random variable $X$
has a Poisson distribution with mean $\lambda$.

\begin{questionparts}
\item Sketch the graph $y=\l x+1 \r \e^{-x}$,
stating the coordinates of the turning point
and the points of intersection with the axes. 

It is known that $\P(X \ge 2) = 1-p$,
where $p$ is a given number in the range $0 < p <1$.
Show that this information determines a unique value
(which you should not attempt to find) of $\lambda$.

\item  It is known (instead) that $\P \l X = 1 \r = q$,
where $q$ is a given number in the range $0 < q <1$.
Show that this information determines a unique value of
$\lambda$ (which you should find) for exactly
one value of $q$ (which you should also find).

\item It is known (instead) that 
$\P \l X = 1 \, \vert \, X \le 2 \r =  r$,
where $r$ is a given number in the range $0<r<1$.
 Show that this information determines a unique value of
$\lambda$ (which you should find) for exactly
one value of $r$ (which you should also find).
\end{questionparts}
\end{question}
	
\end{document}
