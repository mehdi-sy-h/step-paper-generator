
\documentclass[a4, 11pt]{report}


\pagestyle{myheadings}
\markboth{}{Paper III, 1993
\ \ \ \ \ 
\today 
}               

\RequirePackage{amssymb}
\RequirePackage{amsmath}
\RequirePackage{graphicx}
\RequirePackage{color}
\RequirePackage[flushleft]{paralist}[2013/06/09]



\RequirePackage{geometry}
\geometry{%
  a4paper,
  lmargin=2cm,
  rmargin=2.5cm,
  tmargin=3.5cm,
  bmargin=2.5cm,
  footskip=12pt,
  headheight=24pt}


\newcommand{\comment}[1]{{\bf Comment} {\it #1}}
%\renewcommand{\comment}[1]{}

\newcommand{\bluecomment}[1]{{\color{blue}#1}}
%\renewcommand{\comment}[1]{}
\newcommand{\redcomment}[1]{{\color{red}#1}}



\usepackage{epsfig}
\usepackage{pstricks-add}
\usepackage{tgheros} %% changes sans-serif font to TeX Gyre Heros (tex-gyre)
\renewcommand{\familydefault}{\sfdefault} %% changes font to sans-serif
%\usepackage{sfmath}  %%%% this makes equation sans-serif
%\input RexFigs


\setlength{\parskip}{10pt}
\setlength{\parindent}{0pt}

\newlength{\qspace}
\setlength{\qspace}{20pt}


\newcounter{qnumber}
\setcounter{qnumber}{0}

\newenvironment{question}%
 {\vspace{\qspace}
  \begin{enumerate}[\bfseries 1\quad][10]%
    \setcounter{enumi}{\value{qnumber}}%
    \item%
 }
{
  \end{enumerate}
  \filbreak
  \stepcounter{qnumber}
 }


\newenvironment{questionparts}[1][1]%
 {
  \begin{enumerate}[\bfseries (i)]%
    \setcounter{enumii}{#1}
    \addtocounter{enumii}{-1}
    \setlength{\itemsep}{5mm}
    \setlength{\parskip}{8pt}
 }
 {
  \end{enumerate}
 }



\DeclareMathOperator{\cosec}{cosec}
\DeclareMathOperator{\Var}{Var}

\def\d{{\rm d}}
\def\e{{\rm e}}
\def\g{{\rm g}}
\def\h{{\rm h}}
\def\f{{\rm f}}
\def\p{{\rm p}}
\def\s{{\rm s}}
\def\t{{\rm t}}


\def\A{{\rm A}}
\def\B{{\rm B}}
\def\E{{\rm E}}
\def\F{{\rm F}}
\def\G{{\rm G}}
\def\H{{\rm H}}
\def\P{{\rm P}}


\def\bb{\mathbf b}
\def \bc{\mathbf c}
\def\bx {\mathbf x}
\def\bn {\mathbf n}

\newcommand{\low}{^{\vphantom{()}}}
%%%%% to lower suffices: $X\low_1$ etc


\newcommand{\subone}{ {\vphantom{\dot A}1}}
\newcommand{\subtwo}{ {\vphantom{\dot A}2}}




\def\le{\leqslant}
\def\ge{\geqslant}


\def\var{{\rm Var}\,}

\newcommand{\ds}{\displaystyle}
\newcommand{\ts}{\textstyle}




\begin{document}
\setcounter{page}{2}

 
\section*{Section A: \ \ \ Pure Mathematics}

%%%%%%%%%%Q1
\begin{question}
The   curve $P$ has the parametric equations
$$
x= \sin\theta, \quad y=\cos2\theta
\qquad\hbox{ for }-\pi/2 \le \theta \le \pi/2.
$$
Show that $P$ is part of the parabola $y=1-2x^2$ and sketch $P$.

Show that the length of $P$ is $\surd (17) + {1\over 4} \sinh^{-1}4$.

Obtain the volume of the solid enclosed when  $P$ 
is rotated through $2\pi$ radians about the line $y=-1$.

\end{question}

%%%%%%%%%%Q2
\begin{question}
The curve $C$ has the equation $x^3+y^3 = 3xy$.
\begin{questionparts}
\item Show that there is no point of inflection on $C$. You may assume that
the origin is not a point of inflection.

\item The part of $C$ which lies in the first quadrant is a closed loop
touching the axes at the origin. By converting to polar coordinates,
or otherwise, evaluate the area of this loop. 
\end{questionparts}
\end{question}

%%%%%%%%% Q3
\begin{question}
The matrices $\mathbf{A},\mathbf{B}$ and $\mathbf{M}$ are given
by 
\[
\mathbf{A}=\begin{pmatrix}a & 0 & 0\\
b & c & 0\\
d & e & f
\end{pmatrix},\quad\mathbf{B}=\begin{pmatrix}1 & p & q\\
0 & 1 & r\\
0 & 0 & 1
\end{pmatrix},\quad\mathbf{M}=\begin{pmatrix}1 & 3 & 2\\
4 & 13 & 5\\
3 & 8 & 7
\end{pmatrix},
\]
where $a,b,\ldots,r$ are real numbers. Given that $\mathbf{M=AB},$
show that $a=1,b=4,c=1,d=3,e=1,f=-2,p=3,q=2$ and $r=-3$ gives the
\textit{unique }solution for $\mathbf{A}$ and $\mathbf{B}.$ Evaluate
$\mathbf{A}^{-1}$ and $\mathbf{B}^{-1},$


Hence, or otherwise, solve the simultaneous equations 
\begin{alignat*}{1}
x+3y+2z & =7\\
4x+13y+5z & =18\\
3x+8y+7z & =25.
\end{alignat*}
\end{question}


%%%%%% Q4 

\begin{question}
Sum the following infinite series.

\begin{questionparts}
	\item $\displaystyle 1 + 
{1 \over 3} \bigg({1\over 2}\bigg)^2 +{1 \over 5 }\bigg({1\over 2}\bigg)^4
+ \cdots + {1 \over 2n+1} \bigg({1 \over 2}\bigg)^{2n} + \cdots$ .

\item $\displaystyle 
2 -x -x^3 +2x^4 - \cdots + 2x^{4k} - x^{4k+1} - x^{4k+3} +\cdots$
where $|x|<1$.

\item  $\displaystyle  \sum _{r=2}^\infty
 {r\,  2^{r-2} \over 3^{r-1} }.$

\item $\displaystyle\sum_{r=2}^\infty
 {2 \over r(r^2-1) }$.
 \end{questionparts}
\end{question}


%%%%%%%%% Q5
\begin{question}
The set $S$ consists of ordered pairs of complex numbers
$(z_1,z_2)$ and a binary operation  $\circ$ on $S$  is defined by
$$
(z_1,z_2)\circ(w_1,w_2)=
(z_1w_1-z_2w^*_2, \; z_1w_2+z_2w^*_1).
$$
Show that the operation $\circ$ is associative and determine whether
it is commutative. Evaluate $(z,0)\circ(w,0)$, $(z,0)\circ(0,w)$,
$(0,z)\circ(w,0)$ and $(0,z)\circ(0,w)$.

The set $S_1$ is the subset of $S$ consisting of $A$, $B$, $\ldots\,$, $H$,
where $A=(1,0)$, $B=(0,1)$, $C=(i,0)$, $D=(0,i)$, $E=(-1,0)$, $F=(0,-1)$,
$G=(-i,0)$ and $H=(0,-i)$. Show that $S_1$ is closed under $\circ$ and 
that it has an identity element. Determine the inverse and order of
each element of $S_1$. Show that $S_1$ is a group under 
$\circ$.
\hfil\break
[You are not required to compute the multiplication table
in full.]

Show that $\{A,B,E,F\}$ is a subgroup of $S_1$ and determine whether
it is isomorphic to the group generated by the $2\times2$ matrix
$\begin{pmatrix}0 & 1\\
-1 & 0
\end{pmatrix}$ under 
matrix multiplication.
	\end{question}
	
	%%%%%%%%% Q6
	\begin{question}
The point in the Argand diagram representing the complex number
$z$ lies on the circle with centre $K$ and radius $r$, where $K$
represents the complex number $k$. Show that 
$$
zz^* -kz^* -k^*z +kk^* -r^2 =0.
$$

The points $P$, $Q_1$ and $Q_2$ represent the complex numbers
$z$, $w_1$ and $w_2$ respectively. The point $P$ lies on the circle
with $OA$ as diameter, where $O$ and $A$ represent $0$ and 
$2i$ respectively. Given that $w_1=z/(z-1)$, find the equation of the 
locus $L$ of $Q_1$ in terms of $w_1$ and describe 
the geometrical form of $L$.

Given that $w_2=z^*$, show that the locus of $Q_2$ is also $L$. Determine the
positions of $P$ for which $Q_1$ coincides with $Q_2$.
	 \end{question}
	 
	 %%%%%%%%% Q7
\begin{question}
The real numbers $x$ and $y$ satisfy the simultaneous equations
$$
\sinh (2x) = \cosh y
\qquad\hbox{and}\qquad
\sinh(2y) = 2 \cosh x.
$$
Show that $\sinh^2 y$ is a root of the equation
$$
4t^3 + 4t^2 -4t -1=0
$$
and demonstrate that this gives at most one valid solution for $y$. Show that
the relevant value of $t$ lies between $0.7$ and $0.8$, and use an
iterative process to find $t$ to 6 decimal places.

Find $y$ and hence find $x$, checking your answers and stating the final 
answers to four decimal places.
	\end{question}
	
	%%%%%%%%% Q8
	\begin{question}
A square pyramid has its base vertices at the points $A$ $(a,0,0)$,
$B$ $(0,a,0)$, $C$ $(-a,0,0)$ and $D$ $(0,-a,0)$, and its vertex at 
$E$ $(0,0,a)$. The point $P$ lies  on $AE$ with $x$-coordinate $\lambda a$,
where $0<\lambda<1$, and the point $Q$ lies on $CE$ with $x$-coordinate
$-\mu a$, where $0<\mu<1$. The plane $BPQ$ cuts $DE$
at $R$ and the $y$-coordinate of $R$ is $-\gamma a$. Prove that 
$$
\gamma = {\lambda \mu \over \lambda + \mu - \lambda \mu}.
$$

Show that the quadrilateral $BPRQ$ cannot be a parallelogram.
		\end{question}
		
		
%%%%%%%%% Q9
		\begin{question}
For the real numbers $a_1$, $a_2$, $a_3$, $\ldots$,
\begin{questionparts}
\item prove that $a_1^2+a_2^2 \ge 2a_1a_2$,
\item prove that $a_1^2+a_2^2 +a_3^2  \ge a_2a_3 + a_3a_1 +a_1a_2$,
\item prove that $3(a_1^2+a_2^2 +a_3^2 +a_4^2)
 \ge 2(a_1a_2+a_1a_3 + a_1a_4 +a_2a_3 + a_2a_4 +a_3a_4)$,
\item state and prove a generalisation of (iii) to the case of $n$ real
numbers,
\item prove that 
$$
\left(\sum_{i=1}^n a_i \right)^2 \ge {2n\over n-1} \sum_{i,j} a_ia_j,
$$
where the latter sum is taken over all pairs $(i,j)$ with $1\le i<j\le
n$. 
\end{questionparts}
		\end{question}
		
	
%%%%%%%%%% 10
\begin{question}
The transformation $T$ of the point $P$ in the $x$,$y$ plane to
the point $P'$ is constructed as follows:
\hfil\break
Lines are drawn through $P$ parallel to the lines $y=mx$ and $y=-mx$
to cut the line $y=kx$ at $Q$ and $R$ respectively, $m$ and $k$ being
given constants. $P'$ is the fourth vertex of the parallelogram
$PQP'R$. 

Show that if $P$ is $(x_1,y_1)$ then $Q$ is 
$$
\left( {mx_1-y_1 \over m-k}, {k(mx_1-y_1)\over m-k}\right).
$$
Obtain the coordinates of $P'$ in terms of 
$x_1$, $y_1$, $m$ and $k$, and express $T$ as a matrix transformation.
Show that  areas are transformed under $T$ into areas of the same 
magnitude.  
\end{question}
			
		
		
		
	
\newpage
\section*{Section B: \ \ \ Mechanics}


	
%%%%%%%%%% Q11
\begin{question}
\textit{In this question, all gravitational forces are to be neglected. }


A rigid frame is constructed from 12 equal uniform rods, each of length
$a$ and mass $m,$ forming the edges of a cube. Three of the edges
are $OA,OB$ and $OC,$ and the vertices opposite $O,A,B$ and $C$
are $O',A',B'$ and $C'$ respectively. Forces act along the lines
as follows, in the directions indicated by the order of the letters:
\begin{alignat*}{3}
2mg\mbox{ along }OA, & \qquad & mg\mbox{ along }AC', & \qquad & \sqrt{2}mg\mbox{ along }O'A,\\
\sqrt{2}mg\mbox{ along }OA', &  & 2mg\mbox{ along }C'B, &  & mg\mbox{ along }A'C.
\end{alignat*}

\begin{questionparts}
\item The frame is freely pivoted at $O$. Show that the direction of the line
about which it will start to rotate is
$\begin{pmatrix}1\\
1\\
2
\end{pmatrix}$ with respect to axes
along $OA$, $OB$ and $OC$ respectively.

\item Show that the moment of inertia of the rod 
$OA$ about the axis $OO'$ is $2ma^2/9$ and about a parallel axis through
 its mid-point is $ma^2/18$. Hence find the 
moment of inertia of $B'C$ about $OO'$ and show that the moment of inertia
of the frame about $OO'$ is $14ma^2/3$.  If the frame
 is freely pivoted about the line $OO'$ and the forces continue 
to act along the specified lines, find the initial angular 
acceleration of the frame.   
\end{questionparts}
	\end{question}
	
%%%%%%%%%% Q12
\begin{question}	
$ABCD$ is a horizontal line with 
$AB=CD=a$ and $BC=6a$. 
There are
fixed smooth pegs at $B$ and $C$.
A uniform string of natural length $2a$ and modulus of elasticity
$kmg$ is stretched from $A$ to $D$, passing over the pegs at 
$B$ and $C$. A particle of mass $m$ is attached to the midpoint $P$
of the string.  When the
system is in equilibrium, $P$ is a distance $a/4$ below $BC$. Evaluate
$k$.

The particle is pulled down to a point $Q$, which is
at a distance $pa$ below
the mid-point of $BC$, and is released from rest. $P$ rises to 
a point $R$, which is at 
 a distance $3a$ above $BC$. Show that $2p^2-p-17=0$.

Show also that the tension in the strings is less when the 
particle is at $R$ than when the particle is at $Q$.
\end{question}

%%%%%%%%%% Q13

\begin{question}
	$\ $\vspace{-1cm}
	
	\noindent
	\begin{center}
		\psset{xunit=1.0cm,yunit=1.0cm,algebraic=true,dotstyle=o,dotsize=3pt 0,linewidth=0.5pt,arrowsize=3pt 2,arrowinset=0.25}
		\begin{pspicture*}(-3.18,-6.26)(2.72,2.3)
		\pscircle(0,0){2}
		\psline[linewidth=1.2pt](-2,0)(-2,-4)
		\psline[linewidth=1.2pt](2,0)(2,-5)
		\rput[tl](-2.5,0.14){$R$}
		\rput[tl](0.2,0.2){$O$}
		\rput[tl](2.2,0.14){$Q$}
		\rput[tl](-2.1,-4.26){$S$}
		\rput[tl](1.86,-5.2){$P$}
		\parametricplot[linewidth=1.2pt]{0.0}{3.141592653589793}{1*2*cos(t)+0*2*sin(t)+0|0*2*cos(t)+1*2*sin(t)+0}
		\begin{scriptsize}
		\psdots[dotstyle=+,dotsize=6pt](0,0)
		\end{scriptsize}
		\end{pspicture*}
		\end{center}
		
		
A uniform circular disc with  radius $a$, mass $4m$ and centre $O$ is freely
mounted on a fixed horizontal axis which is
perpendicular to its plane and passes through $O$. A uniform heavy chain
$PS$ of length $(4+\pi)a$, mass $(4+\pi)m$ and negligible thickness is
hung over the rim of the disc as shown in the diagram: $Q$ and $R$ are
the points of the chain at the same level as $O$. The contact between the
chain and the rim of the disc is sufficiently rough to prevent slipping.
Initially, the system is at rest with $PQ=RS =2a$. A particle of mass 
$m$ is attached to the chain at $P$ and the system is released. 
By considering the energy of the system, show that when $P$ has descended
a distance $x$, its speed $v$ is given by
$$
(\pi+7)av^2 = 2g(x^2+ax).
$$

By considering the part $PQ$ of the chain as a body of variable mass, show 
that when $S$ reaches $R$ the tension in the chain at $Q$ is
$$
{5\pi -2 \over \pi +7} mg.
$$ 
\end{question}

%%%%%%%%%% Q14
\begin{question}
A particle rests at a point $A$ on a horizontal table and is joined to a
point $O$ on the table by a taut inextensible string of length $c$. The
particle is projected vertically upwards at a speed $64\surd(6gc)$. It
next strikes the table at a point $B$ and rebounds. The coefficient
of restitution for any impact between the particle and the table is
${1\over 2}$. After rebounding at $B$, the particle will rebound
alternately at $A$ and $B$ until the string becomes slack. Show that 
when the string  becomes slack the particle is at height $c/2$ above
the table.

Determine whether the first rebound {\it between} $A$ and $B$ is
nearer to $A$ or to $B$.
\end{question}

	
	\newpage
\section*{Section C: \ \ \ Probability and Statistics}


%%%%%%%%%% Q15
\begin{question}
The probability of throwing a head with a certain coin is $p$ and
the probability of throwing a tail is $q=1-p$. 
The coin is thrown until at least two heads and at least two tails have
been thrown; this happens when the coin has been thrown $N$ times.
Write down an expression for the probability that $N=n$.
                                                  
Show that the expectation of $N$ is 
$$
2\bigg({1\over pq} -1-pq\bigg).
$$  
\end{question}

%%%%%%%%%% Q16
\begin{question}
The time taken for me to set an acceptable examination question it $T$
hours. The distribution of $T$ is a truncated normal distribution with
probability density $\f$  where

\[
\mathrm{f}(t)=\begin{cases}
\dfrac{1}{k\sigma\sqrt{2\pi}}\exp\left(-\dfrac{1}{2}\left(\dfrac{t-\sigma}{\sigma}\right)^{2}\right) & \mbox{ for }t\geqslant0\\
0 & \mbox{ for }t<0.
\end{cases}
\]
Sketch the graph of $\f(t)$. Show that $k$ is approximately $0.841$
 and obtain the mean of $T$ as a multiple of
$\sigma$.

Over a period of years, I find that the mean setting time is 3 hours.
 \begin{questionparts}                  
\item Find the approximate probability that none of the 16
questions on next year's paper will take more than 4 hours to set.
                       
\item Given that a particular question is unsatisfactory after 2 hours work, 
find the probability that it will still be unacceptable after a further
2 hours work.
\end{questionparts}

\end{question}
\end{document}
