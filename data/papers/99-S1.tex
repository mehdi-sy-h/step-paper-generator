\documentclass[a4, 11pt]{report}


\pagestyle{myheadings}
\markboth{}{Paper I, 1999
\ \ \ \ \ 
\today 
}               

\RequirePackage{amssymb}
\RequirePackage{amsmath}
\RequirePackage{graphicx}
\RequirePackage{color}
\RequirePackage[flushleft]{paralist}[2013/06/09]



\RequirePackage{geometry}
\geometry{%
  a4paper,
  lmargin=2cm,
  rmargin=2.5cm,
  tmargin=3.5cm,
  bmargin=2.5cm,
  footskip=12pt,
  headheight=24pt}


\newcommand{\comment}[1]{{\bf Comment} {\it #1}}
%\renewcommand{\comment}[1]{}

\newcommand{\bluecomment}[1]{{\color{blue}#1}}
%\renewcommand{\comment}[1]{}
\newcommand{\redcomment}[1]{{\color{red}#1}}



\usepackage{epsfig}
\usepackage{pstricks-add}
\usepackage{tgheros} %% changes sans-serif font to TeX Gyre Heros (tex-gyre)
\renewcommand{\familydefault}{\sfdefault} %% changes font to sans-serif
%\usepackage{sfmath}  %%%% this makes equation sans-serif
%\input RexFigs


\setlength{\parskip}{10pt}
\setlength{\parindent}{0pt}

\newlength{\qspace}
\setlength{\qspace}{20pt}


\newcounter{qnumber}
\setcounter{qnumber}{0}

\newenvironment{question}%
 {\vspace{\qspace}
  \begin{enumerate}[\bfseries 1\quad][10]%
    \setcounter{enumi}{\value{qnumber}}%
    \item%
 }
{
  \end{enumerate}
  \filbreak
  \stepcounter{qnumber}
 }


\newenvironment{questionparts}[1][1]%
 {
  \begin{enumerate}[\bfseries (i)]%
    \setcounter{enumii}{#1}
    \addtocounter{enumii}{-1}
    \setlength{\itemsep}{5mm}
    \setlength{\parskip}{8pt}
 }
 {
  \end{enumerate}
 }



\DeclareMathOperator{\cosec}{cosec}
\DeclareMathOperator{\Var}{Var}

\def\d{{\mathrm d}}
\def\e{{\mathrm e}}
\def\g{{\mathrm g}}
\def\h{{\mathrm h}}
\def\f{{\mathrm f}}
\def\p{{\mathrm p}}
\def\s{{\mathrm s}}
\def\t{{\mathrm t}}


\def\A{{\mathrm A}}
\def\B{{\mathrm B}}
\def\E{{\mathrm E}}
\def\F{{\mathrm F}}
\def\G{{\mathrm G}}
\def\H{{\mathrm H}}
\def\P{{\mathrm P}}


\def\bb{\mathbf b}
\def \bc{\mathbf c}
\def\bx {\mathbf x}
\def\bn {\mathbf n}

\newcommand{\low}{^{\vphantom{()}}}
%%%%% to lower suffices: $X\low_1$ etc


\newcommand{\subone}{ {\vphantom{\dot A}1}}
\newcommand{\subtwo}{ {\vphantom{\dot A}2}}




\def\le{\leqslant}
\def\ge{\geqslant}


\def\var{{\rm Var}\,}

\newcommand{\ds}{\displaystyle}
\newcommand{\ts}{\textstyle}
\def\half{{\textstyle \frac12}}




\begin{document}
\setcounter{page}{2}

 
\section*{Section A: \ \ \ Pure Mathematics}

%%%%%%%%%%Q1
\begin{question}
How many integers greater than
or equal to zero and less than a million 
are
not divisible by 2 or 5? What is the average value
of these integers?

How many integers greater than or equal to zero and less than
4179 are not divisible by 3 or~7? What is the average value
of these integers?
\end{question}

%%%%%%%%%%Q2
\begin{question}
A point moves in the $x$-$y$ plane so that the sum of the squares
of its distances from the three fixed points $(x_{1},y_{1})$,
$(x_{2},y_{2})$, and $(x_{3},y_{3})$ is always $a^{2}$.
Find the equation of the locus of the point and interpret
it geometrically.
Explain why $a^2$ cannot be less 
than the sum of the squares of the 
distances of the three points from their centroid.

\noindent
[The {\sl centroid } has coordinates
$(\bar x, \bar y)$ where 
$
3\bar x = x_1+x_2+x_3,$
$
3\bar y = y_1+y_2+y_3.
\;$]
\end{question}

%%%%%%%%% Q3
\begin{question}
The $n$ positive  numbers  $x_{1},x_{2},\dots,x_{n}$, where  $n\ge3$,
satisfy
$$
x_{1}=1+\frac{1}{x_{2}}\, ,\ \ \ 
x_{2}=1+\frac{1}{x_{3}}\, , \ \ \
\dots\; ,
\ \ \ x_{n-1}=1+\frac{1}{x_{n}}\, ,
$$
and also
$$
\ x_{n}=1+\frac{1}{x_{1}}\, .
$$
Show that
\begin{itemize}
\setlength{\itemsep}{3mm}
\item[\bf (i)]  $x_{1},x_{2},\dots,x_{n}>1$,

\item[\bf (ii)] ${\displaystyle x_{1}-x_{2}=-\frac{x_{2}-x_{3}}{x_{2}x_{3}}}$,

\item[\bf (iii)] $x_{1}=x_{2}=\cdots=x_{n}$.
\end{itemize}
\vspace{2mm}
Hence find the value of $x_1$. 
\end{question}

%%%%%% Q4 
\begin{question}
Sketch the following subsets of the $x$-$y$ plane:

\begin{questionparts}
\item $|x|+|y|\le 1$ ;

\item $|x-1|+|y-1|\le 1 $ ;

\item $|x-1|-|y+1|\le 1 $ ;

\item $|x|\, |y-2|\le 1$ .

\end{questionparts}
	\end{question}

%%%%%%%%% Q5
\begin{question}
For this question, you may use the following 
approximations, valid if $\theta $ is  small: \ 
$\sin\theta \approx \theta$ and $\cos\theta \approx 1-\theta^2/2\,$.

A satellite $X$ is directly above the point $Y$
on the Earth's surface and can just be seen 
(on the horizon) from
another point $Z$ on the Earth's surface.
The radius of the Earth  is $R$ and the height of
the satellite above the Earth is $h$.  

\begin{questionparts}
\item Find the distance $d$
of $Z$ from $Y$ along the Earth's surface.
\item If the satellite is in low orbit (so that $h$ is
small compared with $R$),
 show that
$$d \approx  k(Rh)^{1/2},$$ where $k$ is to be found.
\item If the satellite is very distant from the Earth (so that $R$ is small
compared with $h$), show that 
$$d\approx aR+b(R^2/h),$$
where $a$ and $b$ are to be found.
\end{questionparts}
\end{question}
	
%%%%%%%%% Q6
\begin{question}
\begin{questionparts}
\item Find the greatest and least values of $bx+a$
for $-10\leqslant x \leqslant 10$, distinguishing
carefully between the cases $b>0$, $b=0$ and $b<0$.

\item Find the greatest and least values of $cx^{2}+bx+a$,
where $c\ge0$,
for $-10\leqslant x \leqslant 10$, distinguishing
carefully between the cases that can arise
for different values of $b$ and $c$.

\end{questionparts}
\end{question}
	
%%%%%%%%% Q7
\begin{question}
Show that $\sin(k\sin^{-1} x)$, 
where $k$ is a constant,
satisfies the  differential equation
$$
(1-x^{2})\frac {\d^2 y}{\d x^2} -x\frac{\d y}{\d x}
+k^{2}y=0.
\eqno(*)
$$

In the particular case when $k=3$, find the  solution
 of  equation
$(*)$ of the form
\[
y=Ax^{3}+Bx^{2}+Cx+D,
\]
that satisfies $y=0$ and $\displaystyle \frac{\d y}{\d x}=3$ at $x=0$.

Use this result to express $\sin 3\theta$ in terms of powers of $\sin\theta$.
\end{question}
		
%%%%%%%%% Q8
\begin{question}	
The function $\f$ satisfies  $0\leqslant\f(t)\leqslant K$
when $0\leqslant t\leqslant x$. Explain by means of a sketch, or
otherwise,  why  
\[0\leqslant\int_{0}^{x} \f (t)\,{\mathrm d}t
\leqslant Kx.\]
By considering 
$\displaystyle \int_{0}^{1}\frac{t}{n(n-t)}\,{\mathrm d}t$, or otherwise,
show that, if $n>1$,
\[
0\le \ln \left( \frac n{n-1}\right)  -\frac 1n \le \frac 1 {n-1} - \frac 1n
\]
and deduce that
\[
0\le \ln N -\sum_{n=2}^N \frac1n  \le 1.
\]
Deduce that as $N\to \infty$
\[
 \sum_{n=1}^N \frac1n \to\infty.
\]
Noting that $2^{10}=1024$, show also that  if $N<10^{30}$ then
\[
 \sum_{n=1}^N \frac1n <101.
\]

\end{question}	
		

		
	
\newpage
\section*{Section B: \ \ \ Mechanics}


	
%%%%%%%%%% Q9
\begin{question}
A tortoise and a hare have a race to the vegetable patch, a
distance $X$ kilometres from the starting post, and back.
The tortoise sets off immediately, at a steady $v$ kilometers per
hour. The hare goes to sleep for half an hour and then sets off
at a steady speed $V$ kilometres per hour. The hare overtakes
the tortoise half a kilometre from the starting post, and continues
on to the vegetable patch, where she has another half an hour's 
sleep before setting  off for the return journey
at her previous pace. One and quarter kilometres from the 
vegetable patch, she 
passes the tortoise, still plodding gallantly and steadily 
towards the vegetable 
patch. Show that 
\[
V= \frac{10}{4X-9}
\]
and find
$v$ in terms of $X$. 

Find $X$ if
the hare  arrives back at the starting post
one and a half  hours after the start of the race.
	\end{question}
	
%%%%%%%%%% Q10
\begin{question}	
A particle is attached to a point $P$ of an 
unstretched light uniform
spring $AB$ of modulus of elasticity $\lambda$ in such a way that
$AP$ has length $a$ and $PB$ has length $b$.
The ends $A$ and $B$ of the spring are
now fixed to points in a vertical line a distance $l$ apart,
The particle oscillates
along this line. Show that
the motion is simple harmonic. Show also that the period
is the same whatever the value of $l$
 and whichever end of the string is uppermost.
\end{question}

%%%%%%%%%% Q11

\begin{question}
The force of attraction between two
stars of masses $m_{1}$ and $m_{2}$ 
a distance $r$ apart is
$\gamma m_{1}m_{2}/r^{2}$. 
The Starmakers of Kryton
place three stars of equal mass $m$ at the
corners of an equilateral triangle of side $a$.
Show that it is possible for each star to revolve
round the centre of mass of the system with angular velocity
$(3\gamma m/a^{3})^{1/2}$.

Find a corresponding result if the Starmakers  place a fourth star,
of mass $\lambda m$, at the centre of mass of the system.
\end{question}
	

	
	\newpage
\section*{Section C: \ \ \ Probability and Statistics}


%%%%%%%%%% Q12
\begin{question}
\begin{questionparts}
\item Prove that if $x>0$ then $x+x^{-1}\ge2.\;$

I have a pair of six-faced dice, each with faces numbered from 1 to 6.
The probability of throwing $i$ with the first die
is $q_{i}$ and the  
probability of throwing $j$ with the second die
is $r_{j}$ ($1\le i,j \le 6$). The two dice are thrown independently
and the sum noted. By considering the
probabilities of throwing 2, 12 and 7, show
the sums 2, 3, \dots, 12 are not equally likely.

\item
The first die described above is thrown twice
and the two numbers on the die  noted.
Is it possible to find values of $q_{j}$
so that the probability that the numbers are the same
is less than~$1/36$?
\end{questionparts}
\end{question}

%%%%%%%%%% Q13
\begin{question}
Bar magnets are placed randomly end-to-end in a straight line.
If adjacent magnets have ends of opposite polarities facing each
other, they join together to form a single unit.
If they have ends of the same polarity facing each other, they
stand apart. Find the expectation
and variance of the number of separate units in terms of the total
number $N$ of magnets.
\end{question}

%%%%%%%%%% Q14
\begin{question}
When I throw a dart at a target, the probability that it lands a 
distance $X$ from the centre is a random variable with density
function
\[
\mathrm{f}(x)=\begin{cases}
2x & \text{ if }0\leqslant x\leqslant1;\\
0 & \text{ otherwise.}
\end{cases}
\]
I score points according to the position of the dart as follows:
%\begin{center}
%\begin{tabular}{c|c}
%Range of $X$ & my score \\[1mm]
%\hline\\
%$0\le X< \frac14$ & 4 \\[2mm]
%$\frac14\le X< \frac12$ & 3 \\[2mm]
%$\frac12\le X< \frac34$ & 2 \\[2mm]
%$\frac34\le X\le 1$ & 1 
%\end{tabular}
%\end{center}
%\newline\hspace*{10mm} 
if~$0\le X< \frac14$, my score is 4;  
%\newline\hspace*{10mm} 
if~$\frac14\le X< \frac12$, my score is  3;
%\newline\hspace*{10mm} 
if $\frac12\le X< \frac34$, my score is  2;
%\newline\hspace*{10mm} 
if $\frac34\le X\le 1$, my score is  1. 

\begin{questionparts}
\item Show that my expected score from one dart is 15/8.
\item I play a game with the following rules. 
I start off with a total score 0, and each time~I throw a dart
my score on that throw is added to my total. Then:
\newline
\hspace*{10mm} 
if my new total is greater than 3, I have lost and the game ends;
\newline
\hspace*{10mm} if my new total is 3, I have won and the game ends;
\newline
\hspace*{10mm} if my new total is less than 3, I throw again.

 Show that, if I
have won such a game, the probability that I threw the dart three
times is 343/2231.
\end{questionparts}
\end{question}
	
\end{document}
