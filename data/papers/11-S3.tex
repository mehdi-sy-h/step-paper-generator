\documentclass[a4, 11pt]{report}


\pagestyle{myheadings}
\markboth{}{Paper III, 2011
\ \ \ \ \ 
\today 
}               

\RequirePackage{amssymb}
\RequirePackage{amsmath}
\RequirePackage{graphicx}
\RequirePackage{color}
\RequirePackage[flushleft]{paralist}[2013/06/09]



\RequirePackage{geometry}
\geometry{%
  a4paper,
  lmargin=2cm,
  rmargin=2.5cm,
  tmargin=3.5cm,
  bmargin=2.5cm,
  footskip=12pt,
  headheight=24pt}


\newcommand{\comment}[1]{{\bf Comment} {\it #1}}
%\renewcommand{\comment}[1]{}

\newcommand{\bluecomment}[1]{{\color{blue}#1}}
%\renewcommand{\comment}[1]{}
\newcommand{\redcomment}[1]{{\color{red}#1}}



\usepackage{epsfig}
\usepackage{pstricks-add}
\usepackage{tgheros} %% changes sans-serif font to TeX Gyre Heros (tex-gyre)
\renewcommand{\familydefault}{\sfdefault} %% changes font to sans-serif
%\usepackage{sfmath}  %%%% this makes equation sans-serif
%\input RexFigs


\setlength{\parskip}{10pt}
\setlength{\parindent}{0pt}

\newlength{\qspace}
\setlength{\qspace}{20pt}


\newcounter{qnumber}
\setcounter{qnumber}{0}

\newenvironment{question}%
 {\vspace{\qspace}
  \begin{enumerate}[\bfseries 1\quad][10]%
    \setcounter{enumi}{\value{qnumber}}%
    \item%
 }
{
  \end{enumerate}
  \filbreak
  \stepcounter{qnumber}
 }


\newenvironment{questionparts}[1][1]%
 {
  \begin{enumerate}[\bfseries (i)]%
    \setcounter{enumii}{#1}
    \addtocounter{enumii}{-1}
    \setlength{\itemsep}{5mm}
    \setlength{\parskip}{8pt}
 }
 {
  \end{enumerate}
 }



\DeclareMathOperator{\cosec}{cosec}
\DeclareMathOperator{\Var}{Var}

\def\d{{\mathrm d}}
\def\e{{\mathrm e}}
\def\g{{\mathrm g}}
\def\h{{\mathrm h}}
\def\f{{\mathrm f}}
\def\p{{\mathrm p}}
\def\s{{\mathrm s}}
\def\t{{\mathrm t}}


\def\A{{\mathrm A}}
\def\B{{\mathrm B}}
\def\E{{\mathrm E}}
\def\F{{\mathrm F}}
\def\G{{\mathrm G}}
\def\H{{\mathrm H}}
\def\P{{\mathrm P}}


\def\bb{\mathbf b}
\def \bc{\mathbf c}
\def\bx {\mathbf x}
\def\bn {\mathbf n}

\newcommand{\low}{^{\vphantom{()}}}
%%%%% to lower suffices: $X\low_1$ etc


\newcommand{\subone}{ {\vphantom{\dot A}1}}
\newcommand{\subtwo}{ {\vphantom{\dot A}2}}




\def\le{\leqslant}
\def\ge{\geqslant}
\def\arcosh{{\rm arcosh}\,}


\def\var{{\rm Var}\,}

\newcommand{\ds}{\displaystyle}
\newcommand{\ts}{\textstyle}
\def\half{{\textstyle \frac12}}
\def\l{\left(}
\def\r{\right)}



\begin{document}
\setcounter{page}{2}

 
\section*{Section A: \ \ \ Pure Mathematics}

%%%%%%%%%%Q1
\begin{question}
\begin{questionparts}
\item
Find the general solution of the differential equation
\[
\frac{\d u}{\d x} - \left(\frac { x +2}{x+1}\right)u =0\,.
\]

\item Show that substituting
$y=z\e^{-x}$ (where $z$ is a function of $x$) into the 
second order differential equation
\[
(x+1) \frac{\d ^2 y}{\d x^2} + x \frac{\d y}{\d x} -y = 0
\tag{$*$}
\]
leads to a first order differential equation for $\dfrac{\d z}{\d x}\,$.
Find $z$ and hence show that the general solution of $(*)$ is 
\[
y= Ax + B\e^{-x}\,,
\]
where $A$ and $B$ are arbitrary constants.

\item Find the general solution of the differential equation
\[
(x+1) \frac{\d ^2 y}{\d x^2} + x \frac{\d y}{\d x} -y = 
(x+1)^2                                                             .
\]

\end{questionparts}
\end{question}

%%%%%%%%%%Q2
\begin{question}
The polynomial $\f(x)$ is defined by
\[
\f(x) =   x^n + a_{\low{n-1}}x^{n-1}
  + \cdots +     a_{\low2} x^2+  a_{\low1} x + a_{\low0}\,,
\]
where $n\ge2$ and the coefficients $a_{\low0}$, $\ldots,$ $a_{\low{n-1}}$ 
are integers, with $a_0\ne0$. 
 Suppose that the equation $\f(x)=0$ has a rational root
$p/q$, where $p$ and $q$ are integers
with no common factor greater than $1$, and $q>0$. 
By considering $q^{n-1}\f(p/q)$,
find the value of $q$ and deduce that any rational root of 
the equation $\f(x)=0$ must be an integer.

\begin{questionparts}
\item Show that the $n$th root of $2$ is irrational for $n\ge2$.

\item Show that 
 the cubic equation 
\[
x^3- x +1 =0
\]
has no rational roots.
\item
Show that 
the polynomial equation 
\[
x^n- 5x +7 =0
\]
has no rational roots for $n\ge2$.


\end{questionparts}
\end{question}

%%%%%%%%% Q3
\begin{question}
Show that, provided $q^2\ne 4p^3$,
 the polynomial 
\[
\hphantom{(p\ne0, \ q\ne0)\hspace{2cm}}
x^3-3px +q
\hspace {2cm} (p\ne0, \ q\ne0)
\]
 can be written in the 
form 
\[ a(x-\alpha)^3 + b(x-\beta)^3\,,
\]
where $\alpha$ and $\beta$ are the roots of the quadratic equation
$pt^2 -qt +p^2=0$, and $a$ and $b$ are constants which you should
express in terms of $\alpha$ and $\beta$.

Hence show that one solution of 
 the equation $x^3-24x+48=0\,$ is
\[
x= \frac{2 (2-2^{\frac13})}{1-2^{\frac13}}
\]
and obtain similar expressions for the other two solutions
in
terms of $\omega$,  where $\omega = \mathrm{e}^{2\pi\mathrm{i}/3}\,$.
 

Find also the roots of
$x^3-3px +q=0$
when 
$p=r^2$ and $q= 2r^3$ for some non-zero constant $r$.
\end{question}

%%%%%% Q4 
\begin{question}
The following result applies to any
 function $\f$ 
which is continuous, has positive  gradient  and satisfies
$\f(0)=0\,$:
\[
ab\le \int_0^a\f(x)\, \d x + \int_0^b \f^{-1}(y)\, \d y\,,
\tag{$*$}\]
where  $\f^{-1}$ denotes the inverse function of $\f$, and  
$a\ge  0$ and $b\ge  0$.



\begin{questionparts}
\item
By considering the graph of $y=\f(x)$, explain briefly why
the inequality $(*)$ holds.

In the case $a>0$ and $b>0$,
state a condition on $a$ and $b$ under which  equality  holds.
\item
By taking $\f(x) = x^{p-1}$ in $(*)$, where $p>1$, show that if
$\displaystyle \frac 1p + \frac 1q =1$ then 
\[
ab \le \frac{a^p}p + \frac{b^q}q\,.
\]
Verify that equality holds under the condition you stated above.

\item
Show that, for $0\le a \le  \frac12 \pi$ and $0\le b \le 1$,  
\[
ab \le  b\arcsin b + \sqrt{1-b^2} \, - \cos a\,.
\]
Deduce that,  
for $t\ge1$, 
\[
\arcsin (t^{-1})  \ge t - \sqrt{t^2-1}\,.
\]

\end{questionparts}
	\end{question}
	
%%%%%%%%% Q5
\begin{question}
A  movable  point $P$ has cartesian coordinates $(x,y)$, where 
$x$ and $y$ are functions of $t$.  The polar coordinates of 
$P$ with respect to the origin $O$ are $r$ and $\theta$. 
Starting with the expression 
\[
\tfrac12 \int r^2 \, \d \theta
\] for the area swept out by $OP$, obtain the equivalent expression
\[
\tfrac12 \int \left( x\frac{\d y}{\d t} - y \frac{\d x}{\d t}\right)\d t
\,.
\tag{$*$}
\]

The ends of a thin  straight rod $AB$ lie on a closed
convex curve $\cal C$. The point $P$ on the rod is a fixed 
distance $a$ from 
$A$ and a fixed distance $b$ from $B$. 
The angle between 
$AB$ and the positive $x$ direction  is $t$. 
As $A$ and $B$ move anticlockwise round $\cal C$, the angle $t$ increases
 from $0$ to $2\pi$
and  $P$ traces
a closed convex curve $\cal D$ inside $\cal C$, 
with the origin $O$ lying inside $\cal D$, as shown in the diagram. 

\noindent
\begin{center}
\psset{xunit=0.8cm,yunit=0.8cm,algebraic=true,dotstyle=o,dotsize=3pt 0,linewidth=0.3pt,arrowsize=3pt 2,arrowinset=0.25}
\begin{pspicture*}(-3.1,-3.45)(9.16,7.38)	
\psbezier(0.04,2.13)(-0.22, 3.77)(2.63, 4.87)(3.63, 4.35)
\psbezier(3.63, 4.35)(4.97, 3.75)(7.09, 1.81)(5.59, 0.23)
\psbezier(0.04, 2.13)(0.05, 0.29)(3.37, -1.84)(5.59, 0.23)
\rput{0.2}(3,1.95){\psellipse(0,0)(4.32,3.27)}
\psline[linewidth=1.6pt](-1.11,0.95)(1.98,5.12)
\psline{->}(-2.65,1.27)(8.54,1.27)
\psline{->}(3.79,-3.49)(3.81,6.8)
\rput[tl](1.31,3.78){$P$}
\rput[tl](1.24,4.89){$b$}
\rput[tl](-0.65,2.41){$a$}
\rput[tl](1.87,5.65){$B$}
\rput[tl](-1.6,0.95){$A$}
\rput[tl](3.26,1.11){$O$}
\rput[tl](5.65,3.17){$\mathcal{D}$}
\rput[tl](6.73,4.22){$\mathcal{C}$}
\pscustom[linewidth=0.2pt]{\parametricplot{-0.001127872300442774}{0.9332580339440653}{0.87*cos(t)+-0.87|0.87*sin(t)+1.28}\lineto(-0.87,1.28)\closepath}
\rput[tl](-0.45,1.7){$t$}
\rput[tl](3.72,7.25){$y$}
\rput[tl](8.6,1.4){$x$}
\begin{scriptsize}
\psdots[dotsize=5pt 0,dotstyle=*](1.09,3.93)
\end{scriptsize}
\end{pspicture*}
\end{center}

Let $(x,y)$ be the coordinates of $P$. Write down the 
coordinates of $A$ and $B$ in terms of $a$, $b$, $x$, $y$ and $t$.


The areas swept out by $OA$, $OB$ and $OP$ are
denoted by $[A]$, $[B]$ and $[P]$, respectively. 
Show, using $(*)$, that 
\[
[A] = [P] +\pi a^2 - af
\]
where 
\[
f = \tfrac12 \int _0^{2\pi} \left(
 \Big(x+\frac{\d y}{\d t}\Big)\cos t
+
  \Big(y- \frac{\d x}{\d t}\Big)\sin t
\right) \d t\,.
\]
Obtain a corresponding expression for $[B]$ involving $b$. Hence show that
the area between the curves $\cal C$ and $\cal D$ is $\pi ab$.
\end{question}
	
%%%%%%%%% Q6
\begin{question}
The definite integrals $T$, $U$, $V$ and $X$ are defined by
\begin{align*}
T&= \int_{\frac13}^{\frac12} \frac{{\rm artanh}\, t}t \,\d t\,, & 
U&=  \int _{\ln 2 }^{\ln 3 } \frac{u}{2\sinh u}\, \d u \,, \\[3mm]
V&= - \int_{\frac13}^{\frac12} \frac{\ln v}{1-v^2} \,\d v \,, &
X&=  \int _{\frac12\ln2}^{\frac12\ln3} \ln ({\coth x})\, \d x\,.
\end{align*}
Show, without evaluating any of them, that $T$, $U$, $V$ and $X$ are
all equal.
\end{question}
		
%%%%%%%%% Q7
\begin{question}
Let
\[
T _n    = 
\left( \sqrt{a+1} + \sqrt a\right)^n\,,
\]
where $n$ is a positive integer and $a$ is any given positive integer.
\begin{questionparts}
\item
In the case when $n$ is even, show
 by induction  that  
$T_n$ can be written in the form
\[
A_n +B_n \sqrt{a(a+1)}\,,
\]
 where
$A_n$ and $B_n$ are integers (depending on $a$ and $n$)
and $A_n^2 =a(a+1)B_n^2 +1$.

\item In the case when $n$ is odd, show by considering
$(\sqrt{a+1} +\sqrt a)T_m$ where $m$ is even, or otherwise,
that  $T_n$ 
can be written in the form
\[
C_n \sqrt {a+1} + D_n \sqrt a \,,
\]
where $C_n$ and $D_n$ are integers (depending on $a$ and $n$) and 
$ (a+1)C_n^2 = a D_n^2 +1\,$.

\item Deduce that, for each $n$, $T_n$ can be written
as the sum of the square roots of two consecutive integers.

\end{questionparts}
\end{question}	

%%%%%%%%% Q8
\begin{question}
The complex numbers $z$ and $w$ are related by
\[
w= \frac{1+\mathrm{i}z}{\mathrm{i}+z}\,.
\]
Let $z=x+\mathrm{i}y$ and $w=u+\mathrm{i}v$, 
where $x$, $y$, $u$ and $v$ are real. 
Express $u$ and $v$ in terms of $x$ and $y$.
\begin{questionparts}
\item 
By setting $x=\tan(\theta/2)$, or otherwise,
show that if the locus of $z$ is the real axis $y=0$, $-\infty<x<\infty$,
then the locus of $w$ is the circle $u^2+v^2=1$ with one point
omitted.
\item Find the locus of $w$ when the locus of $z$ is 
the line segment $y=0$, $-1<x<1\,$.
\item Find the locus of $w$ when the locus of $z$ is 
the line segment $x=0$, $-1<y<1\,$.
\item Find the locus of $w$ when the locus of $z$ is 
the line $y=1$, $-\infty<x<\infty\,$. 
\end{questionparts}
\end{question}	
		

		
	
\newpage
\section*{Section B: \ \ \ Mechanics}


	
%%%%%%%%%% Q9
\begin{question}
Particles $P$ and $Q$ have masses
$3m$ and $4m$, respectively. They
lie on the outer curved surface of a~smooth 
circular cylinder of radius~$a$
which is fixed with its axis horizontal.
They are connected by a light inextensible 
string of length $\frac12 \pi a$, which passes over the 
surface of the cylinder. The particles and the string all lie
in a vertical plane perpendicular to the axis of the cylinder,
and the axis intersects this plane at $O$.
Initially, the particles are in equilibrium.

Equilibrium is slightly disturbed and $Q$  begins to
move downwards. Show that while the two particles
are still in contact with the cylinder the angle $\theta$
between $OQ$ and
the vertical satisfies
\[
7a\dot\theta^2 +8g \cos\theta + 6 g\sin\theta =   10g\,.
\]

\begin{questionparts}
\item
Given that $Q$ loses contact with 
the cylinder first, show that  it does so when~$\theta=\beta$,
where $\beta$ satisfies 

\[
15\cos\beta +6\sin\beta =10.
\] 

\item
Show also that while $P$ and $Q$ are still in contact 
with the cylinder 
the tension in the string is $\frac {12}7 mg(\sin\theta
+\cos\theta)\,$.
\end{questionparts}
	\end{question}
	
%%%%%%%%%% Q10 
\begin{question}	
Particles $P$ and $Q$, each of mass $m$, lie initially at rest
 a distance $a$ apart on a smooth horizontal plane.
They are connected by a light elastic string of natural
length $a$ and modulus of elasticity
$\frac12 m a \omega^2$, where $\omega$ is a constant.
 
Then $P$ receives an impulse which gives it a 
velocity $u$ directly away from $Q$. Show that when the string 
next returns to length $a$, the particles have travelled
a distance $\frac12 \pi u/\omega\,$, and find the speed of 
each particle. 

Find also the total time between the impulse and the subsequent 
collision of the particles.
\end{question}

%%%%%%%%%% Q11

\begin{question}
A thin uniform circular disc of radius $a$ and mass $m$ is held
in equilibrium in a horizontal plane a distance $b$ below a horizontal
ceiling, where $b>2a$. It is held in this way by $n$ light
inextensible vertical strings, each of length $b$; one end of 
each string is attached to the edge of the disc and the other end is
attached to a point on the ceiling. The strings are equally 
spaced around the edge of the disc.
One of the strings is attached to the point $P$ on the disc
which has coordinates
$(a,0,-b)$ with respect to 
cartesian axes with origin on the ceiling directly above the
centre of the disc.


The disc is then rotated through an angle $\theta$ (where $\theta<\pi$)
about its vertical axis of symmetry and held at rest by a couple
acting in the plane of the disc.
Show that the  string attached to~$P$ now makes an angle $\phi$
with the vertical,
where
\[
b\sin\phi = 2a \sin\tfrac12 \theta\,.
\]
Show further that the magnitude of the couple is
\[
\frac {mga^2\sin\theta}{\sqrt{b^2-4a^2\sin^2 \frac12\theta \ } \ }\,.
\]

The disc is now released from rest. Show that 
its angular speed, $\omega$, when the strings
are vertical is given by
\[
\frac{a^2\omega^2}{4g} = b-\sqrt{b^2 - 4a^2\sin^2 \tfrac12\theta \;}\,.
\]
\end{question}
	

	
	\newpage
\section*{Section C: \ \ \ Probability and Statistics}


%%%%%%%%%% Q12
\begin{question}
The random variable $N$ takes positive integer values
and has pgf (probability generating function) $\G(t)$.
The random variables $X_i$, where $i=1$, $2$, $3$, $\ldots,$
 are independently and identically 
distributed, each  with pgf ${\H}(t)$. The random variables $X_i$
are also independent of $N$. The random variable $Y$ is defined by 
\[
Y=
\sum_{i=1}^N X_i \;.
\] 
Given  that the pgf of $Y$ is $\G(\H(t))$, 
show that  
\[
\E(Y) = \E(N)\E(X_i)
\text{ \ \ \ \ and \ \ \ \ }
\var(Y) = \var(N)\big(\E(X_i)\big)^2 + \E(N) \var(X_i)
\,.\]

A fair coin is tossed until a head occurs. The total number of tosses is
$N$. The coin is then tossed a further $N$ times and the total number of
heads in these $N$ tosses is $Y$. Find in this particular case 
the pgf of $Y$, $\E(Y)$, $\var(Y)$ and $\P(Y=r)$.
\end{question}

%%%%%%%%%% Q13
\begin{question}
In this question, the notation $\lfloor x \rfloor$
denotes the greatest integer less than or equal to $x$,
so for example $\lfloor \pi\rfloor = 3$ and $\lfloor 3 \rfloor =3$.
\begin{questionparts}
\item
A bag contains $n$ balls, of which $b$ are black. A sample of $k$ balls
is drawn, one after another, at random {\sl with} replacement. The random 
variable $X$ denotes the number of black balls in the sample.
By considering 
\[
\frac{\P(X=r+1)}{\P(X=r)}\,,
\]
show that, in the case that it is unique,   
the most probable number of black balls
in the sample is 
\[
\left\lfloor \frac{(k+1)b}{n}\right\rfloor.
\]
Under what circumstances is the answer not unique?

\item A bag contains 
$n$ balls, of which $b$ are black. A sample of $k$ balls
(where $k\le b$)
is drawn, one after another, at random {\sl without} replacement. 
Find, in the case that it is unique, the most probable number of black
balls in the sample.

Under what circumstances is the answer not unique?

\end{questionparts} 
\end{question}

\end{document}
