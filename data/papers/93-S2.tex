\documentclass[a4, 11pt]{report}


\pagestyle{myheadings}
\markboth{}{Paper II, 1993
\ \ \ \ \ 
\today 
}               

\RequirePackage{amssymb}
\RequirePackage{amsmath}
\RequirePackage{graphicx}
\RequirePackage{color}
\RequirePackage[flushleft]{paralist}[2013/06/09]



\RequirePackage{geometry}
\geometry{%
  a4paper,
  lmargin=2cm,
  rmargin=2.5cm,
  tmargin=3.5cm,
  bmargin=2.5cm,
  footskip=12pt,
  headheight=24pt}


\newcommand{\comment}[1]{{\bf Comment} {\it #1}}
%\renewcommand{\comment}[1]{}

\newcommand{\bluecomment}[1]{{\color{blue}#1}}
%\renewcommand{\comment}[1]{}
\newcommand{\redcomment}[1]{{\color{red}#1}}



\usepackage{epsfig}
\usepackage{pstricks-add}
\usepackage{tgheros} %% changes sans-serif font to TeX Gyre Heros (tex-gyre)
\renewcommand{\familydefault}{\sfdefault} %% changes font to sans-serif
%\usepackage{sfmath}  %%%% this makes equation sans-serif
%\input RexFigs


\setlength{\parskip}{10pt}
\setlength{\parindent}{0pt}

\newlength{\qspace}
\setlength{\qspace}{20pt}


\newcounter{qnumber}
\setcounter{qnumber}{0}

\newenvironment{question}%
 {\vspace{\qspace}
  \begin{enumerate}[\bfseries 1\quad][10]%
    \setcounter{enumi}{\value{qnumber}}%
    \item%
 }
{
  \end{enumerate}
  \filbreak
  \stepcounter{qnumber}
 }


\newenvironment{questionparts}[1][1]%
 {
  \begin{enumerate}[\bfseries (i)]%
    \setcounter{enumii}{#1}
    \addtocounter{enumii}{-1}
    \setlength{\itemsep}{5mm}
    \setlength{\parskip}{8pt}
 }
 {
  \end{enumerate}
 }



\DeclareMathOperator{\cosec}{cosec}
\DeclareMathOperator{\Var}{Var}

\def\d{{\rm d}}
\def\e{{\rm e}}
\def\g{{\rm g}}
\def\h{{\rm h}}
\def\f{{\rm f}}
\def\p{{\rm p}}
\def\s{{\rm s}}
\def\t{{\rm t}}


\def\A{{\rm A}}
\def\B{{\rm B}}
\def\E{{\rm E}}
\def\F{{\rm F}}
\def\G{{\rm G}}
\def\H{{\rm H}}
\def\P{{\rm P}}


\def\bb{\mathbf b}
\def \bc{\mathbf c}
\def\bx {\mathbf x}
\def\bn {\mathbf n}

\newcommand{\low}{^{\vphantom{()}}}
%%%%% to lower suffices: $X\low_1$ etc


\newcommand{\subone}{ {\vphantom{\dot A}1}}
\newcommand{\subtwo}{ {\vphantom{\dot A}2}}




\def\le{\leqslant}
\def\ge{\geqslant}


\def\var{{\rm Var}\,}

\newcommand{\ds}{\displaystyle}
\newcommand{\ts}{\textstyle}




\begin{document}
\setcounter{page}{2}

 
\section*{Section A: \ \ \ Pure Mathematics}

%%%%%%%%%%Q1
\begin{question}
In the game of ``Colonel Blotto'' there are two players, Adam and
Betty. First Adam chooses three non-negative integers $a_{1},a_{2}$
and $a_{3},$ such that $a_{1}+a_{2}+a_{3}=9,$ and then Betty chooses
non-negative integers $b_{1},b_{2}$ and $b_{3}$, such that $b_{1}+b_{2}+b_{3}=9.$
If $a_{1}>b_{1}$ then Adam scores one point; if $a_{1}<b_{1}$ then
Betty scores one point; and if $a_{1}=b_{1}$ no points are scored.
Similarly for $a_{2},b_{2}$ and $a_{3},b_{3}.$ The winner is the
player who scores the greater number of points: if the socres are
equal then the game is drawn. Show that, if Betty knows the numbers
$a_{1},a_{2}$ and $a_{3},$ she can always choose her numbers so
that she wins. Show that Adam can choose $a_{1},a_{2}$ and $a_{3}$
in such a way that he will never win no matter what Betty does. 


Now suppose that Adam is allowed to write down two triples of numbers
and that Adam wins unless Betty can find one triple that beats both
of Adam's choices (knowing what they are). Confirm that Adam wins
by writing down $(5,3,1)$ and $(3,1,5).$ 
\end{question}

%%%%%%%%%%Q2
\begin{question}
\begin{questionparts} 
	\item
Evaluate 
\[
\int_{0}^{2\pi}\cos(mx)\cos(nx)\,\mathrm{d}x,
\]
where $m,n$ are integers, taking into account any special cases that
arise. 


\item Find ${\displaystyle \int\sqrt{1+\frac{1}{x}}\,\mathrm{d}x}.$
\end{questionparts}
\end{question}

%%%%%%%%% Q3
\begin{question}
\begin{questionparts}
\item Solve the differential equation 
\[
\frac{\mathrm{d}y}{\mathrm{d}x}-y-3y^{2}=-2
\]
by making the substitution $y=-\dfrac{1}{3u}\dfrac{\mathrm{d}u}{\mathrm{d}x}.$


\item Solve the differential equation 
\[
x^{2}\frac{\mathrm{d}y}{\mathrm{d}x}+xy+x^{2}y^{2}=1
\]
by making the substitution 
\[
y=\frac{1}{x}+\frac{1}{v},
\]
where $v$ is a function of $x$. 
\end{questionparts}
\end{question}

%%%%%% Q4 
\begin{question}
Two non-parallel lines in 3-dimensional space are given by $\mathbf{r}=\mathbf{p}_{1}+t_{1}\mathbf{m}_{1}$
and $\mathbf{r}=\mathbf{p}_{2}+t_{2}\mathbf{m}_{2}$ respectively,
where $\mathbf{m}_{1}$ and $\mathbf{m}_{2}$ are unit vectors. Explain
by means of a sketch why the shortest distance between the two lines
is 
\[
\frac{\left|(\mathbf{p}_{1}-\mathbf{p}_{2})\cdot(\mathbf{m}_{1}\times\mathbf{m}_{2})\right|}{\left|(\mathbf{m}_{1}\times\mathbf{m}_{2})\right|}.
\]


\begin{questionparts}
\item Find the shortest distance between the lines in the case 
\[
\mathbf{p}_{1}=(2,1,-1)\qquad\mathbf{p}_{2}=(1,0,-2)\qquad\mathbf{m}_{1}=\tfrac{1}{5}(4,3,0)\qquad\mathbf{m}_{2}=\tfrac{1}{\sqrt{10}}(0,-3,1).
\]

\item Two aircraft, $A_{1}$ and $A_{2},$ are flying in the directions
given by the unit vectors $\mathbf{m}_{1}$ and $\mathbf{m}_{2}$
at constant speeds $v_{1}$ and $v_{2}.$ At time $t=0$ they pass
the points $\mathbf{p}_{1}$ and $\mathbf{p}_{2}$, respectively.
If $d$ is the shortest distance between the two aircraft during the
flight, show that 
\[
d^{2}=\frac{\left|\mathbf{p}_{1}-\mathbf{p}_{2}\right|^{2}\left|v_{1}\mathbf{m}_{1}-v_{2}\mathbf{m}_{2}\right|^{2}-[(\mathbf{p}_{1}-\mathbf{p}_{2})\cdot(v_{1}\mathbf{m}_{1}-v_{2}\mathbf{m}_{2})]^{2}}{\left|v_{1}\mathbf{m}_{1}-v_{2}\mathbf{m}_{2}\right|^{2}}.
\]

\item Suppose that $v_{1}$ is fixed. The pilot of $A_{2}$ has chosen $v_{2}$
so that $A_{2}$ comes as close as possible to $A_{1}.$ How close
is that, if $\mathbf{p}_{1},\mathbf{p}_{2},\mathbf{m}_{1}$ and $\mathbf{m}_{2}$
are as in \textbf{(i)}?
\end{questionparts}
	\end{question}

%%%%%%%%% Q5
\begin{question}
\noindent \begin{center}
\psset{xunit=1.0cm,yunit=1.0cm,algebraic=true,dotstyle=o,dotsize=3pt 0,linewidth=0.5pt,arrowsize=3pt 2,arrowinset=0.25} \begin{pspicture*}(-0.57,-0.63)(8.51,6.23) \psline(0,0)(7,5) \psline(7,5)(7.75,1.98) \psline(7.75,1.98)(0,0) \parametricplot{-0.6740818217636368}{0.22208190190547994}{1*5.52*cos(t)+0*5.52*sin(t)+1.48|0*5.52*cos(t)+1*5.52*sin(t)+4.9} \psline(7,5)(5.79,1.45) \rput[tl](-0.4,-0.02){$O$} \rput[tl](5.76,1.29){$P$} \rput[tl](8.1,2.01){$R$} \rput[tl](7.2,5.26){$Q$} \psline(7.67,2.29)(7.37,2.22) \psline(7.37,2.22)(7.45,1.91) \end{pspicture*}
\par\end{center}


In the diagram, $O$ is the origin, $P$ is a point of a curve $r=r(\theta)$
with coordinates $(r,\theta)$ and $Q$ is another point of the curve,
close to $P$, with coordinates $(r+\delta r,\theta+\delta\theta).$
The angle $\angle PRQ$ is a right angle. By calculating $\tan\angle QPR,$
show that the angle at which the curve cuts $OP$ is 
\[
\tan^{-1}\left({\displaystyle r\dfrac{\mathrm{d}\theta}{\mathrm{d}r}}\right).
\]



Let $\alpha$ be a constant angle, $0<\alpha<\frac{1}{2}\pi$. The
curve with the equation 
\[
r=\mathrm{e}^{\theta\cot\alpha}
\]
in polar coordinates is called an \textit{equiangular spiral}. Show
that it cuts every radius line at an angle $\alpha.$ Sketch the spiral. 


Find the length of the complete turn of the spiral beginning at $r=1$
and going outwards. What is the total length of the part of the spiral
for which $r\leqslant1$? 


{[}You may assume that the arc length $s$ of the curve satisfies
\[
{\displaystyle \left(\frac{\mathrm{d}s}{\mathrm{d}\theta}\right)^{2}=r^{2}+\left(\frac{\mathrm{d}r}{\mathrm{d}\theta}\right)^{2}.}]
\]
	\end{question}
	
	%%%%%%%%% Q6
	\begin{question}
\textit{In this question, }\textbf{\textit{$\mathbf{A},\mathbf{B}$
}}\textit{and $\mathbf{X}$ are non-zero $2\times2$ real matrices.}


Are the following assertions true or false? You must provide a proof
or a counterexample in each case. 
\begin{questionparts}
\item If $\mathbf{AB=0}$ then $\mathbf{BA=0}.$
\item $(\mathbf{A-B)(A+B)=}\mathbf{A}^{2}-\mathbf{B}^{2}.$
\item The equation $\mathbf{AX=0}$ has a non-zero solution $\mathbf{X}$
if and only if $\det\mathbf{A}=0.$
\item For any $\mathbf{A}$ and $\mathbf{B}$ there are at most two matrices
$\mathbf{X}$ such that $\mathbf{X}^{2}+\mathbf{AX}+\mathbf{B}=\mathbf{0}.$ 
\end{questionparts}

	\end{question}
	
%%%%%%%%% Q7
\begin{question}
 The integers $a,b$ and $c$ satisfy 
\[
2a^{2}+b^{2}=5c^{2}.
\]
By considering the possible values of $a\pmod5$ and $b\pmod5$, show
that $a$ and $b$ must both be divisible by $5$. 


By considering how many times $a,b$ and $c$ can be divided by $5$,
show that the only solution is $a=b=c=0.$ 
\end{question}
		
%%%%%%%%% Q8
\begin{question}	
Suppose that $a_{i}>0$ for all $i>0$. Show that 
\[
a_{1}a_{2}\leqslant\left(\frac{a_{1}+a_{2}}{2}\right)^{2}.
\]
Prove by induction that for all positive integers $m$ 
\[
a_{1}\cdots a_{2^{m}}\leqslant\left(\frac{a_{1}+\cdots+a_{2^{m}}}{2^{m}}\right)^{2^{m}}.\tag{\ensuremath{*}}
\]
If $n<2^{m}$, put $b_{1}=a_{2},$ $b_{2}=a_{2},\cdots,b_{n}=a_{n}$
and $b_{n+1}=\cdots=b_{2^{m}}=A$, where 
\[
A=\frac{a_{1}+\cdots+a_{n}}{n}.
\]
By applying $(*)$ to the $b_{i},$ show that 
\[
a_{1}\cdots a_{n}A^{(2^{m}-n)}\leqslant A^{2^{m}}
\]
(notice that $b_{1}+\cdots+b_{n}=nA).$ Deduce the (arithmetic mean)/(geometric
mean) inequality 
\[
\left(a_{1}\cdots a_{n}\right)^{1/n}\leqslant\frac{a_{1}+\cdots+a_{n}}{n}.
\]
\end{question}	
		
%%%%%%%%%% Q9
\begin{question}
\textit{In this question, the argument of a complex number is chosen
to satisfy $0\leqslant\arg z<2\pi.$}


Let $z$ be a complex number whose imaginary part is positive. What
can you say about $\arg z$?


The complex numbers $z_{1},z_{2}$ and $z_{3}$ all have positive
imaginary part and $\arg z_{1}<\arg z_{2}<\arg z_{3}.$ Draw a diagram
that shows why 
\[
\arg z_{1}<\arg(z_{1}+z_{2}+z_{3})<\arg z_{3}.
\]
Prove that $\arg(z_{1}z_{2}z_{3})$ is never equal to $\arg(z_{1}+z_{2}+z_{3}).$ 
\end{question}
			
		
			
%%%%%%%%%% Q10
\begin{question}
Verify that if 
\[
\mathbf{P}=\begin{pmatrix}1 & 2\\
2 & -1
\end{pmatrix}\qquad\mbox{ and }\qquad\mathbf{A}=\begin{pmatrix}-1 & 8\\
8 & 11
\end{pmatrix}
\]
then $\mathbf{PAP}$ is a diagonal matrix. 


Put $\mathbf{x}=\begin{pmatrix}x\\
y
\end{pmatrix}$ and $\mathbf{x}_{1}=\begin{pmatrix}x_{1}\\
y_{1}
\end{pmatrix}.$ By writing 
\[
\mathbf{x}=\mathbf{P}\mathbf{x}_{1}+\mathbf{a}
\]
for a suitable vector $\mathbf{a},$ show that the equation 
\[
\mathbf{x}^{\mathrm{T}}\mathbf{Ax}+\mathbf{b}^{\mathrm{T}}\mathbf{x}-11=0,
\]
where $\mathbf{b}=\begin{pmatrix}18\\
6
\end{pmatrix}$ and $\mathbf{x}^{\mathrm{T}}$ is the transpose of $\mathbf{x},$
becomes 
\[
3x_{1}^{2}-y_{1}^{2}=c
\]
for some constant $c$ (which you should find). 
\end{question}
		
	
\newpage
\section*{Section B: \ \ \ Mechanics}


	
%%%%%%%%%% Q11
\begin{question}
\textit{In this question, take the value of $g$ to be $10\ \mathrm{ms}^{-2}.$}


A body of mass $m$ kg is dropped vertically into a deep pool of liquid.
Once in the liquid, it is subject to gravity, an upward buoyancy force
of $\frac{6}{5}$ times its weight, and a resistive force of $2mv^{2}\mathrm{N}$
opposite to its direction of travel when it is travelling at speed
$v$ $\mathrm{ms}^{-1}.$ Show that the body stops sinking less than
$\frac{1}{4}\pi$ seconds after it enters the pool. 


Suppose now that the body enters the liquid with speed $1\ \mathrm{ms}^{-1}.$
Show that the body descends to a depth of $\frac{1}{4}\ln2$ metres
and that it returns to the surface with speed $\frac{1}{\sqrt{2}}\ \mathrm{ms}^{-1},$
at a time 
\[
\frac{\pi}{8}+\frac{1}{4}\ln\left(\frac{\sqrt{2}+1}{\sqrt{2}-1}\right)
\]
seconds after entering the pool.
	\end{question}
	
%%%%%%%%%% Q12
\begin{question}$\,$
	\vspace{-1cm}	
\noindent \begin{center}
\psset{xunit=1.0cm,yunit=1.0cm,algebraic=true,dotstyle=o,dotsize=3pt 0,linewidth=0.5pt,arrowsize=3pt 2,arrowinset=0.25} \begin{pspicture*}(-1.28,-0.87)(4.65,6.17) \pscircle(2.59,2.83){1.42} \psline(4,0)(-0.85,3.65) \pscustom{\parametricplot{1.5707963267948966}{2.4966538173979873}{0.61*cos(t)+4|0.61*sin(t)+0}\lineto(4,0)\closepath} \rput[tl](1.44,1.51){$Q_2$} \rput[tl](4.24,3.06){$Q_1$} \rput[tl](4.14,4.45){$W_1$} \rput[tl](-0.19,2.31){$W_2$} \rput[tl](0.76,2.95){$P$} \rput[tl](3.54,0.91){$\alpha$} \psline(4,6)(4,-0.7) \begin{scriptsize} \psdots[dotstyle=*](4,3) \psdots[dotstyle=*](1.65,1.77) \psdots[dotstyle=*](1.17,2.8) \end{scriptsize} \end{pspicture*}
\par\end{center}


A uniform sphere of mass $M$ and radius $r$ rests between a vertical
wall $W_{1}$ and an inclined plane $W_{2}$ that meets $W_{1}$ at
an angle $\alpha.$ $Q_{1}$ and $Q_{2}$ are the points of contact
of the sphere with $W_{1}$ and $W_{2}$ resectively, as shown in
the diagram. A particle of mass $m$ is attached to the sphere at
$P$, where $PQ_{1}$ is a diameter, and the system is released. The
sphere is on the point of slipping at $Q_{1}$ and at $Q_{2}.$ Show
that if the coefficients of friction between the sphere and $W_{1}$
and $W_{2}$ are $\mu_{1}$ and $\mu_{2}$ respectively, then 
\[
m=\frac{\mu_{2}+\mu_{1}\cos\alpha-\mu_{1}\mu_{2}\sin\alpha}{(2\mu_{1}\mu_{2}+1)\sin\alpha+(\mu_{2}-2\mu_{1})\cos\alpha-\mu_{2}}M.
\]
If the sphere is on the point of rolling about $Q_{2}$ instead of
slipping, show that 
\[
m=\frac{M}{\sec\alpha-1}.
\]
\end{question}

%%%%%%%%%% Q13

\begin{question}
The force $F$ of repulsion between two particles with positive charges
$Q$ and $Q'$ is given by $F=kQQ'/r^{2},$ where $k$ is a positive
constant and $r$ is the distance between the particles. Two small
beads $P_{1}$ and $P_{2}$ are fixed to a straight horizontal smooth
wire, a distance $d$ apart. A third bead $P_{3}$ of mass $m$ is
free to move along the wire between $P_{1}$ and $P_{3}.$ The beads
carry positive electrical charges $Q_{1},Q_{2}$ and $Q_{3}.$ If
$P_{3}$ is in equilibrium at a distance $a$ from $P_{1},$ show
that 
\[
a=\frac{d\sqrt{Q_{1}}}{\sqrt{Q_{1}}+\sqrt{Q_{2}}}.
\]
Suppose that $P_{3}$ is displaced slightly from its equilibrium position
and released from rest. Show that it performs approximate simple harmonic
motion with period 
\[
\frac{\pi d}{(\sqrt{Q_{1}}+\sqrt{Q_{2}})^{2}}\sqrt{\frac{2md\sqrt{Q_{1}Q_{2}}}{kQ_{3}}.}
\]
{[}You may use the fact that $\dfrac{1}{(a+y)^{2}}\approx\dfrac{1}{a^{2}}-\dfrac{2y}{a^{3}}$
for small $y.${]}  




\end{question}
	
%%%%%%%%%% Q14
\begin{question}
A ball of mass $m$ is thrown vertically upwards from the floor of
a room of height $h$ with speed $\sqrt{2kgh},$ where $k>1.$ The
coefficient of restitution between the ball and the ceiling or floor
is $a$. Both the ceiling and floor are level. Show that the kinetic
energy of the ball immediately before hitting the ceiling for the
$n$th time is 
\[
mgh\left(a^{4n-4}(k-1)+\frac{a^{4n-4}-1}{a^{2}+1}\right).
\]
Hence show that the number of times the ball hits the ceiling is at
most 
\[
1-\frac{\ln[a^{2}(k-1)+k]}{4\ln a}.
\]
\end{question}
	
	\newpage
\section*{Section C: \ \ \ Probability and Statistics}


%%%%%%%%%% Q15
\begin{question}
Two computers, LEP and VOZ are programmed to add numbers after first
approximating each number by an integer. LEP approximates the numbers
by rounding: that is, it replaces each number by the nearest integer.
VOZ approximates by truncation: that is, it replaces each number by
the largest integer less than or equal to the number. The fractional
parts of the numbers to be added are uniformly and independently distributed.
(The fractional part of a number $a$ is $a-\left\lfloor a\right\rfloor ,$
where $\left\lfloor a\right\rfloor $ is the largest integer less
than or equal to $a$.) Both computers approximate and add 1500 numbers.
For each computer, find the probability that the magnitude of error
in the answer will exceed 15. 


How many additions can LEP perform before the probability that the
magnitude of error is less than 10 drops below 0.9? 
\end{question}

%%%%%%%%%% Q16
\begin{question}

At the terminus of a bus route, passengers arrive at an average rate
of 4 per minute according to a Poisson process. Each minute, on the
minute, one bus arrives with probability $\frac{1}{4},$ independently
of the arrival of passengers or previous buses. Just after eight o'clock
there is no-one at the bus stop. 

\begin{questionparts}
\item What is the probability that the first bus arrives at $n$ minutes
past 8?
\item If the first bus arrives at 8:05, what is the probability that there
are $m$ people waiting for it?
\item Each bus can take 25 people and, since it is the terminus, the bus
arrive empty. Explain carefully how you would calculate, to two significant
figures, the probability that when the first bus arrives it is unable
to pick up all the passengers. Your method should need the use of
a calculator and standard tables only. There is no need to carry out
the calculation. 
\end{questionparts}
\end{question}
\end{document}
