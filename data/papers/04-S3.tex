\documentclass[a4, 11pt]{report}


\pagestyle{myheadings}
\markboth{}{Paper III, 2004
\ \ \ \ \ 
\today 
}               

\RequirePackage{amssymb}
\RequirePackage{amsmath}
\RequirePackage{graphicx}
\RequirePackage{color}
\RequirePackage[flushleft]{paralist}[2013/06/09]



\RequirePackage{geometry}
\geometry{%
  a4paper,
  lmargin=2cm,
  rmargin=2.5cm,
  tmargin=3.5cm,
  bmargin=2.5cm,
  footskip=12pt,
  headheight=24pt}


\newcommand{\comment}[1]{{\bf Comment} {\it #1}}
%\renewcommand{\comment}[1]{}

\newcommand{\bluecomment}[1]{{\color{blue}#1}}
%\renewcommand{\comment}[1]{}
\newcommand{\redcomment}[1]{{\color{red}#1}}



\usepackage{epsfig}
\usepackage{pstricks-add}
\usepackage{tgheros} %% changes sans-serif font to TeX Gyre Heros (tex-gyre)
\renewcommand{\familydefault}{\sfdefault} %% changes font to sans-serif
%\usepackage{sfmath}  %%%% this makes equation sans-serif
%\input RexFigs


\setlength{\parskip}{10pt}
\setlength{\parindent}{0pt}

\newlength{\qspace}
\setlength{\qspace}{20pt}


\newcounter{qnumber}
\setcounter{qnumber}{0}

\newenvironment{question}%
 {\vspace{\qspace}
  \begin{enumerate}[\bfseries 1\quad][10]%
    \setcounter{enumi}{\value{qnumber}}%
    \item%
 }
{
  \end{enumerate}
  \filbreak
  \stepcounter{qnumber}
 }


\newenvironment{questionparts}[1][1]%
 {
  \begin{enumerate}[\bfseries (i)]%
    \setcounter{enumii}{#1}
    \addtocounter{enumii}{-1}
    \setlength{\itemsep}{5mm}
    \setlength{\parskip}{8pt}
 }
 {
  \end{enumerate}
 }



\DeclareMathOperator{\cosec}{cosec}
\DeclareMathOperator{\Var}{Var}

\def\d{{\mathrm d}}
\def\e{{\mathrm e}}
\def\g{{\mathrm g}}
\def\h{{\mathrm h}}
\def\f{{\mathrm f}}
\def\p{{\mathrm p}}
\def\s{{\mathrm s}}
\def\t{{\mathrm t}}


\def\A{{\mathrm A}}
\def\B{{\mathrm B}}
\def\E{{\mathrm E}}
\def\F{{\mathrm F}}
\def\G{{\mathrm G}}
\def\H{{\mathrm H}}
\def\P{{\mathrm P}}


\def\bb{\mathbf b}
\def \bc{\mathbf c}
\def\bx {\mathbf x}
\def\bn {\mathbf n}

\newcommand{\low}{^{\vphantom{()}}}
%%%%% to lower suffices: $X\low_1$ etc


\newcommand{\subone}{ {\vphantom{\dot A}1}}
\newcommand{\subtwo}{ {\vphantom{\dot A}2}}




\def\le{\leqslant}
\def\ge{\geqslant}


\def\var{{\rm Var}\,}

\newcommand{\ds}{\displaystyle}
\newcommand{\ts}{\textstyle}
\def\half{{\textstyle \frac12}}
\def\l{\left(}
\def\r{\right)}



\begin{document}
\setcounter{page}{2}

 
\section*{Section A: \ \ \ Pure Mathematics}

%%%%%%%%%%Q1
\begin{question}
Show that
\[
\int_0^a \frac{\sinh x}{2\cosh^2 x -1} \, \mathrm{d} x = \frac{1}{2 \sqrt{2}} \ln \l \frac{\sqrt{2}\cosh a -1}{\sqrt{2}\cosh a +1}\r + \frac{1}{2 \sqrt{2}} \ln \l \frac{\sqrt{2}+1}{\sqrt{2}-1}\r
\]
and find
\[
\int_0^a \frac{\cosh x}{1+2\sinh^2 x} \, \mathrm{d} x \, .
\]

Hence show that 
\[
\int_0^\infty \frac{\cosh x - \sinh x}{1+2\sinh^2 x} \, \mathrm{d} x = \frac{\pi}{2\sqrt{2}} - \frac{1}{2 \sqrt{2}} \ln \l \frac{\sqrt{2}+1}{\sqrt{2}-1}\r \, .
\]

By substituting $u = \e^x$ in this result, or otherwise, find
\[
\int_1^\infty \frac{1}{1+u^4} \, \mathrm{d} u \, .
\]
\end{question}

%%%%%%%%%%Q2
\begin{question}
The equation of a curve is $y=\f ( x )$ where
\[
\f ( x ) = x-4-{16 \l 2x+1 \r^2 \over x^2 \l x - 4 \r} \;.
\]

\begin{questionparts}
\item
Write down the equations of the vertical and oblique asymptotes to the curve and 
show that the oblique asymptote is a tangent to the curve. 
\item
Show that the equation $\f ( x ) =0$ has a double root.
\item
Sketch the curve.
\end{questionparts}
\end{question}

%%%%%%%%% Q3
\begin{question}
Given that $\f''(x) > 0$ when $a \le x \le b\,$, 
explain with the aid of a sketch why
\[
(b-a) \, \f  \Big( {a+b \over 2} \Big)  
< \int^b_a \f(x) \, \mathrm{d}x 
< (b-a) \, \displaystyle \frac{\f(a) + \f(b)}{2} \;.
\]
By choosing suitable $a$, $b$ and $\f(x)\,$, show that
\[
{4 \over (2n-1)^2} < {1 \over n-1} - {1 \over n} 
< {1 \over 2} \l {1 \over n^2} + {1 \over (n-1)^2}\r \,,
\]
where $n$ is an integer greater than 1.

Deduce that
\[
4 \l {1 \over 3^2} +{1 \over 5^2} + {1 \over 7^2} + \cdots \r 
< 1 
< {1 \over 2} + 
\left( {1 \over 2^2} +{1 \over 3^2} + {1 \over 4^2} + \cdots \right)\,.
\]

Show that  
\[
{1 \over 2} \l {1 \over 3^2} 
+ {1 \over 4^2} + {1 \over 5^2} + \frac 1 {6^2} + \cdots \right)
<
{1 \over 3^2} +{1 \over 5^2} + {1 \over 7^2} 
+  \cdots
\]
 and hence show that
\[
 {3 \over 2} \displaystyle 
< \sum_{n=1}^\infty {1 \over n^2} <{7 \over 4}\;.
\]
\end{question}

%%%%%% Q4 
\begin{question}The triangle $OAB$ is isosceles, 
with $OA = OB$ and angle $AOB = 2 \alpha$ where $0< \alpha < {\pi \over 2}\,$. 
The semi-circle $\mathrm{C}_0$ has its centre at the midpoint of the base $AB$ of the triangle, 
and the sides $OA$ and $OB$ of the triangle are both tangent to the semi-circle. 
$\mathrm{C}_1, \mathrm{C}_2, \mathrm{C}_3, \ldots$ 
are circles such that $\mathrm{C}_n$ is tangent to $\mathrm{C}_{n-1}$ 
and to sides $OA$ and $OB$ of the triangle. 

Let $r_n$ be the radius of $\mathrm{C}_n\,$. Show that
\[
\frac{r_{n+1}}{r_n} = \frac{1-\sin\alpha}{1+\sin\alpha}\;.
\]

Let $S$ be the total area of the semi-circle $\mathrm{C}_0$ and the 
circles  $\mathrm{C}_1$, $\mathrm{C}_2$, $\mathrm{C}_3$, $\ldots\;$. 
Show that
\[
S = {1 + \sin^2 \alpha  \over 4 \sin \alpha} \, \pi r_0^2 \;.
\]

Show that there are values of $\alpha$ for which $S$ is more than four fifths 
of the area of triangle~$OAB$.
\end{question}

%%%%%%%%% Q5
\begin{question}
Show that if $\, \cos(x - \alpha) = \cos \beta \,$  
then either $\, \tan x = \tan ( \alpha + \beta)\,$ or
$\; \tan x = \tan ( \alpha - \beta)\,$. 
By choosing suitable values of $x$, $\alpha$ and $\beta\,$,
give an example to show that if 
$\,\tan x = \tan ( \alpha + \beta)\,$, 
then $\,\cos(x - \alpha) \, $ need not equal $ \cos \beta \,$.

Let $\omega$ be the acute angle such that $\tan \omega = \frac 43\,$.
\begin{questionparts}
\item For $0 \le x \le 2 \pi$, solve the equation
\[
\cos x -7 \sin x = 5
\]
giving both solutions in terms of $\omega\,$.
\item  For $0 \le x \le 2 \pi$, solve the equation
\[
2\cos x + 11 \sin x = 10
\]
showing that one solution is twice the other and giving both in terms of $\omega\,$.
\end{questionparts}
	\end{question}
	
%%%%%%%%% Q6
\begin{question}
Given a sequence $w_0$, $w_1$, $w_2$, $\ldots\,$, the sequence $F_1$, $F_2$, $\ldots$ is
defined by
 $$F_n = w_n^2 + w_{n-1}^2 - 4w_nw_{n-1} \,.$$
Show that
$\;
F_{n}-F_{n-1} = \l w_n-w_{n-2} \r \l w_n+w_{n-2}-4w_{n-1} \r \; 
$ for $n \ge 2\,$.

\begin{questionparts}
\item
The sequence $u_0$, $u_1$, $u_2$, $\ldots$ 
 has $u_0 = 1$, and $u_1 = 2$ and satisfies 
\[
u_n = 4u_{n-1} -u_{n-2} \quad (n \ge 2)\;.
\]
Prove  that
\ $
u_n^2 + u_{n-1}^2 = 4u_nu_{n-1}-3
\; $
for  $n \ge 1\,$.

\item
A sequence $v_0$, $v_1$, $v_2$, $\ldots\,$ has $v_0=1$ and satisfies
\begin{equation*}
v_n^2 + v_{n-1}^2 = 4v_nv_{n-1}-3   \quad (n \ge 1).  \tag{$\ast$}
\end{equation*}
\makebox[7mm]{(a) \hfill}Find $v_1$ and  prove that,  for each $n\ge2\,$, either 
$v_n= 4v_{n-1} -v_{n-2}$ or $v_n=v_{n-2}\,$.
 
\makebox[7mm]{(b) \hfill}Show that the sequence, with period 2,  defined  by
\begin{equation*}
v_n = 
\begin{cases}
1 & \mbox{for $n$ even} \\  
2 &  \mbox{for $n$   odd}
\end{cases}
\end{equation*}
\makebox[7mm]{\hfill}satisfies $(\ast)$.

\makebox[7mm]{(c) \hfill}Find a sequence $v_n$ with period 4 
which has $v_0=1\,$,  and satisfies~$(\ast)$.
\end{questionparts} 
\end{question}
	
%%%%%%%%% Q7
\begin{question}
For $n=1$, $2$, $3$, $\ldots\,$, let
\[
I_n = \int_0^1 {t^{n-1} \over \l t+1 \r^n} \, \mathrm{d} t \, .
\]
By considering the greatest value taken by 
$\ds {t \over t+1}$ for $0 \le t \le 1$ 
show that 
$I_{n+1} < {1 \over 2} I_{n}\,$.

Show also that 
$\; \ds I_{n+1}= - \frac 1{\; n\,  2^n}   + I_{n}\,$.

Deduce that 
$\; \ds I_n < \frac1 {\; n \, 2^{n-1}}\,$.

Prove that
\[
\ln 2 = \sum_{r=1}^n {1 \over \; r\, 2^r} + I_{n+1}
\]
and hence show that \hspace{2 pt} ${2 \over 3} < \ln 2 < {17 \over 24}\,$.
\end{question}
		
%%%%%%%%% Q8
\begin{question}	
 Show that if
\[
{\mathrm{d}y \over \mathrm{d} x}=\f(x)y + {\g(x) \over y}
\]
then the substitution $u = y^2$ gives a linear differential equation for $u(x)\,$.

Hence or otherwise solve the differential equation
\[
{\mathrm{d}y \over \mathrm{d} x}={y \over x} - {1 \over y}\;.
\]

Determine the solution curves of 
this equation which pass through 
$(1 \,,  1)\,$, $(2\, ,  2)$ and 
$(4 \, , 4)$ and sketch graphs of all three curves on the same axes.
\end{question}	
		

		
	
\newpage
\section*{Section B: \ \ \ Mechanics}


	
%%%%%%%%%% Q9
\begin{question}
A circular hoop of radius $a$ is free to rotate about a fixed horizontal 
axis passing through a point $P$ on its circumference. The plane of the hoop
is perpendicular to this axis. 
The hoop hangs in equilibrium with its centre, $O$, vertically below $P$.
The point $A$ on the hoop is vertically below $O$, so that $POA$ is a diameter of the hoop. 


A mouse $M$ runs at constant speed $u$ round the 
rough inner surface of the lower part of the hoop. 
Show that the mouse can choose its speed so that the hoop 
remains in equilibrium with diameter $POA$ vertical.

Describe what happens to the hoop when the mouse passes the point at which angle 
$AOM =  2 \arctan \mu\,$, 
where $\mu$ is the coefficient of friction between mouse and hoop.
	\end{question}
	
%%%%%%%%%% Q10 
\begin{question}	
A particle $P$ of mass $m$ is attached to points $A$ and $B$, where $A$ is a distance $9a$ 
vertically above $B$, by elastic strings, 
each of which has modulus of elasticity $6mg$. 
The string $AP$ has natural length $6a$ and the string 
$BP$ has natural length $2a$. Let $x$ be the distance $AP$.

The system is released from rest with $P$ 
on the vertical line $AB$ and $x = 6a$. 
Show that the acceleration $\ddot{x}$ of $P$ is 
$\ds{4g \over a}(7a - x)$ for $6a<x<7a$ 
and 
$\ds{g \over a}(7a - x)$ for $7a< x < 9a\,$.

Find the time taken for the particle to reach $B$.
\end{question}

%%%%%%%%%% Q11

\begin{question}
Particles $P$, of  mass $2$, and $Q$, of  mass $1$, 
move along a line. Their distances from a fixed point are $x_1$ and $x_2$, respectively
where $x_2>x_1\,$.
Each particle is subject  to a repulsive force from the other 
of magnitude $\displaystyle {2 \over z^3}$, where $z = x_2-x_1 \,$.

Initially, $x_1=0$, $x_2 = 1$, $Q$ is at rest and 
$P$ moves towards $Q$ with speed 1.  
Show that $z$ obeys the equation 
$\displaystyle {\mathrm{d}^2 z \over \mathrm{d}t^2} = {3 \over z^3}$. 

By first writing 
$\displaystyle {\mathrm{d}^2 z \over \mathrm{d}t^2} = v {\mathrm{d}v \over \mathrm{d}z} \,$, 
where $\displaystyle v={\mathrm{d}z \over \mathrm{d}t}\,$,
show that $z=\sqrt{4t^2-2t+1}\,$.

By considering the equation satisfied by $2x_1+x_2\,$,
find $x_1$ and $x_2$ in terms of $t \,$.
\end{question}
	

	
	\newpage
\section*{Section C: \ \ \ Probability and Statistics}


%%%%%%%%%% Q12
\begin{question}
A team of $m$ players, numbered from $1$ to $m$, 
puts on a set of a $m$ shirts, similarly numbered from $1$ to $m$. 
The players change in a hurry, so that the shirts are assigned to them randomly, 
one to each player.


Let $C_i$ be the random variable that takes the value $1$ if player $i$ is wearing shirt $i$, 
and 0 otherwise. Show that $\mathrm{E}\left(C_1\right)={1 \over m}$ 
and find 
$\var \left(C_1\right)$ and $\mathrm{Cov}\left(C_1 \, , \; C_2 \right) \,$.

Let $\, N = C_1 + C_2 + \cdots + C_m \,$ 
be the random variable whose value is the number of players who are wearing the correct shirt. 
Show that $\mathrm{E}\left(N\right)= \var \left(N\right) = 1 \,$.

Explain why a Normal approximation to $N$ is not likely to be appropriate for any $m$, 
but that a Poisson approximation might be reasonable.

In the case $m = 4$, find, by listing equally likely possibilities or otherwise, 
the probability that no player is wearing the correct shirt 
and verify that an appropriate Poisson approximation to $N$ 
gives this probability with a relative error  of about $2\%$. [Use $\e \approx 2\frac{72}{100} \,$.] 
\end{question}

%%%%%%%%%% Q13
\begin{question}
A men's endurance competition has an unlimited number of rounds. 
In each round, a competitor has, independently, a probability $p$ of making it through the round; 
otherwise, he fails the round. 
Once a competitor fails a round, he drops out of the competition; 
before he drops out, he takes part in every round. 
The grand prize is awarded to any competitor who makes it through a round 
which all the other remaining competitors fail; 
if all the remaining competitors fail at the same round the grand prize is not awarded.

If the competition begins with three competitors, find the probability that:
\begin{questionparts}
\item all three drop out in the same round;
\item two of them drop out in round $r$ (with $r \ge 2$) and the third in an earlier round;
\item the grand prize is awarded.
\end{questionparts}
\end{question}

%%%%%%%%%% Q14
\begin{question}
\textit{In this question, $\Phi(z)$ is the cumulative distribution 
function of a standard normal random variable.}

A random variable is known to have a 
Normal distribution with mean $\mu$ and standard deviation 
either $\sigma_0$ or $\sigma_1$, where $\sigma_0 < \sigma_1\,$. 
The mean, $\overline{X}$, of a random sample of $n$ values of $X$ 
is to be used to test the hypothesis 
$\mathrm{H}_0: \sigma = \sigma_0$ against the alternative $\mathrm{H}_1: \sigma = \sigma_1\,$.

Explain carefully why it is appropriate 
to use a two sided test of the form: 
accept $\mathrm{H}_0$ if \phantom{} $\mu - c < \overline{X} < \mu+c\,$, otherwise accept $\mathrm{H}_1$.

Given that the probability of accepting $\mathrm{H}_1$ 
when $\mathrm{H}_0$ is true is $\alpha$, 
determine $c$ in terms of $n$,  $\sigma_0$ and $z_{\alpha}$, where 
$z_\alpha $ is defined by $\ds\Phi(z_{\alpha}) = 1 - \tfrac{1}{2}\alpha$.

The probability of accepting $\mathrm{H}_0$ when $\mathrm{H}_1$ is true
is denoted by $\beta$. Show that $\beta$ is independent of $n$.

Given that $\Phi(1.960)\approx 0.975$ and that $\Phi(0.063) \approx 0.525\,$, 
determine, approximately, the minimum value of $\ds \frac{\sigma_1}{\sigma_0}$ 
if $\alpha$ and $\beta$ are both to be less than $0.05\,$.
\end{question}
	
\end{document}
