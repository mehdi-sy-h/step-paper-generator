\documentclass[a4, 11pt]{report}


\pagestyle{myheadings}
\markboth{}{Paper II, 2005
\ \ \ \ \ 
\today 
}               

\RequirePackage{amssymb}
\RequirePackage{amsmath}
\RequirePackage{graphicx}
\RequirePackage{color}
\RequirePackage[flushleft]{paralist}[2013/06/09]



\RequirePackage{geometry}
\geometry{%
  a4paper,
  lmargin=2cm,
  rmargin=2.5cm,
  tmargin=3.5cm,
  bmargin=2.5cm,
  footskip=12pt,
  headheight=24pt}


\newcommand{\comment}[1]{{\bf Comment} {\it #1}}
%\renewcommand{\comment}[1]{}

\newcommand{\bluecomment}[1]{{\color{blue}#1}}
%\renewcommand{\comment}[1]{}
\newcommand{\redcomment}[1]{{\color{red}#1}}



\usepackage{epsfig}
\usepackage{pstricks-add}
\usepackage{tgheros} %% changes sans-serif font to TeX Gyre Heros (tex-gyre)
\renewcommand{\familydefault}{\sfdefault} %% changes font to sans-serif
%\usepackage{sfmath}  %%%% this makes equation sans-serif
%\input RexFigs


\setlength{\parskip}{10pt}
\setlength{\parindent}{0pt}

\newlength{\qspace}
\setlength{\qspace}{20pt}


\newcounter{qnumber}
\setcounter{qnumber}{0}

\newenvironment{question}%
 {\vspace{\qspace}
  \begin{enumerate}[\bfseries 1\quad][10]%
    \setcounter{enumi}{\value{qnumber}}%
    \item%
 }
{
  \end{enumerate}
  \filbreak
  \stepcounter{qnumber}
 }


\newenvironment{questionparts}[1][1]%
 {
  \begin{enumerate}[\bfseries (i)]%
    \setcounter{enumii}{#1}
    \addtocounter{enumii}{-1}
    \setlength{\itemsep}{5mm}
    \setlength{\parskip}{8pt}
 }
 {
  \end{enumerate}
 }



\DeclareMathOperator{\cosec}{cosec}
\DeclareMathOperator{\Var}{Var}

\def\d{{\mathrm d}}
\def\e{{\mathrm e}}
\def\g{{\mathrm g}}
\def\h{{\mathrm h}}
\def\f{{\mathrm f}}
\def\p{{\mathrm p}}
\def\s{{\mathrm s}}
\def\t{{\mathrm t}}


\def\A{{\mathrm A}}
\def\B{{\mathrm B}}
\def\E{{\mathrm E}}
\def\F{{\mathrm F}}
\def\G{{\mathrm G}}
\def\H{{\mathrm H}}
\def\P{{\mathrm P}}


\def\bb{\mathbf b}
\def \bc{\mathbf c}
\def\bx {\mathbf x}
\def\bn {\mathbf n}

\newcommand{\low}{^{\vphantom{()}}}
%%%%% to lower suffices: $X\low_1$ etc


\newcommand{\subone}{ {\vphantom{\dot A}1}}
\newcommand{\subtwo}{ {\vphantom{\dot A}2}}




\def\le{\leqslant}
\def\ge{\geqslant}


\def\var{{\rm Var}\,}

\newcommand{\ds}{\displaystyle}
\newcommand{\ts}{\textstyle}
\def\half{{\textstyle \frac12}}
\def\l{\left(}
\def\r{\right)}



\begin{document}
\setcounter{page}{2}

 
\section*{Section A: \ \ \ Pure Mathematics}

%%%%%%%%%%Q1
\begin{question}
Find the three values of $x$ for which the derivative 
of $x^2 \e^{-x^2}$ is zero.

Given that $a$ and $b$ are
distinct positive numbers,
find a  polynomial $\P(x)$ such that the derivative of
$\P(x)\e^{-x^2}$ is zero for 
$x=0$, $x=\pm a$ and $x=\pm b\,$, but for no other values of $x$.
\end{question}

%%%%%%%%%%Q2
\begin{question}
For any positive integer $N$, the function $\f(N)$ is defined by
\[
\f(N) = N\Big(1-\frac1{p_1}\Big)\Big(1-\frac1{p_2}\Big)
\cdots\Big(1-\frac1{p_k}\Big)
\]
where $p_1$, $p_2$, $\dots$ , $p_k$ are the only prime numbers that are factors
 of $N$. 
\\
Thus $\f(80)=80(1-\frac12)(1-\frac15)\,$.


\begin{questionparts}
\item
\textbf{(a)} \ \  Evaluate $\f(12)$  and $\f(180)$. 

\textbf{(b)} \  
Show that $\f(N)$ is an integer for all $N$.
\item Prove, or disprove by means of a counterexample, 
each of the following:

\textbf{(a)} \ \   $\f(m) \f(n) = \f(mn)\,$;


\textbf{(b)} \ $\f(p) \f(q) = \f(pq)$ if $p$ and $q$ are distinct prime numbers;

\textbf{(c)}
  $\f(p) \f(q) = \f(pq)$ only if $p$ and $q$ are distinct prime numbers.
\item Find 
a positive integer $m$ and a prime number $p$ such that
$\f(p^m) = 146410\,$.
\end{questionparts}
\end{question}

%%%%%%%%% Q3
\begin{question}
Give a sketch, for  
$0 \le x \le \frac{1}{2}\pi$, of the curve
$$
y = (\sin x - x\cos x)\;,
$$
and show that $0\le y \le 1\,$.

Show that:
\begin{questionparts}
\item \ \ $\displaystyle
\int_0^{\frac{1}{2}\pi}\,y\;\d x = 2 -\frac \pi 2 \;;
$
\item \ \ $\displaystyle
\int_0^{\frac{1}{2}\pi}\,y^2\,\d x = \frac{\pi^3}{48}-\frac \pi 8 \;.
$
\end{questionparts}

Deduce that
$\pi^3 +18 \pi< 96\,$.
\end{question}

%%%%%% Q4 
\begin{question}
The positive numbers $a$, $b$ and $c$ satisfy
$bc=a^2+1$. Prove that
$$
\arctan\left(\frac1  {a+b}\right)+
\arctan\left(\frac1 {a+c}\right)=
\arctan\left(\frac1 a   \right).
$$

The positive numbers $p$, $q$, $r$, $s$, $t$, $u$ and $v$ satisfy
$$
st = (p+q)^2 + 1 \;, \ \ \ \ \ \ uv=(p+r)^2 + 1 \;, \ \ \ \ \ \ 
qr = p^2+1\;.
$$
Prove that 
$$
\arctan  \! \!\left(\!\frac1 {p+q+s}\!\right)  + 
\arctan  \! \!\left(\!\frac 1{p+q+t}\!\right)   + 
\arctan  \! \!\left(\!\frac 1 {p+r+u}\!\right)  + 
\arctan  \! \!\left(\!\frac1  {p+r+v}\!\right) 
=\arctan \! \!\left(    \! \frac1 p \! \right) .
$$ 

Hence show that
$$
\arctan\left(\frac1  {13}\right)
+\arctan\left(\frac1 {21}\right)
+\arctan\left(\frac1  {82}\right)
+\arctan\left(\frac1 {187}\right)
=\arctan\left(\frac1  {7}\right).
$$
[\,Note that $\arctan x$ is another notation for $ \tan^{-1}x \,.\,$] 
\end{question}

%%%%%%%%% Q5
\begin{question}
 The angle $A$ of triangle $ABC$ is a right angle and the sides
$BC$, $CA$ and $AB$ are  of lengths $a$, $b$ and $c$,
respectively.  
Each side of the triangle is  tangent to the circle $S_1$ which is
of radius $r$.
Show that $2r = b+c-a$.



Each vertex of the triangle 
lies on the circle~$S_2$.
The ratio of  the area  of the  region between~$S_1$ 
and the triangle to the area of $S_2$ is denoted by $R\,$. 
Show that 
$$
\pi R = -(\pi-1)q^2 + 2\pi q -(\pi+1) \;,
$$
where  $q=\dfrac{b+c}a\,$.
Deduce that 
$$
R\le \frac1 {\pi( \pi - 1)} \;.
$$
	\end{question}
	
%%%%%%%%% Q6
\begin{question}
\begin{questionparts}
\item  Write down the general term in the expansion
in  powers of $x$ 
of  $(1-x)^{-1}$,  $(1-x)^{-2}$ and   $(1-x)^{-3}$, where
 $\vert x \vert <1\,$.

Evaluate \
$\ds \sum_{n=1}^\infty n 2^{-n}\;$ \ 
and \ \
$\ds \sum_{n=1}^\infty n^22^{-n}\;$.

\item   Show that 
\ 
$\ds (1-x)^{-\frac12} = \sum_{n=0}^\infty \frac{(2n)!}{(n!)^2} \, 
\frac{x^n}{2^{2n}}\,$, \ for $\vert x \vert <1\,$.

Evaluate \
  $\ds \sum_{n=0}^\infty \frac{(2n)!} {(n!)^2 2^{2n}3^{n}}\; $ \ and \ \ 
$\ds \sum_{n=1}^\infty \frac{n(2n)!} {(n!)^2 2^{2n}3^{n}}\; $.


\end{questionparts}
\end{question}
	
%%%%%%%%% Q7
\begin{question}
The position vectors, relative to an origin $O$,
at time $t$  of the particles $P$ and $Q$ are
$$\cos t \; {\bf i} + \sin t\;{\bf j} + 0 \; {\bf k}
\hbox{ \ \ \ \ \ \ and \ \ \ \ \ \ }
 \cos (t+\tfrac14{\pi})\, \big[{\tfrac32}{\bf i} + 
{ \tfrac  {3\sqrt{3}}2} {\bf k}\big]
+
3\sin(t+\tfrac14{\pi}) \; {\bf j}\;,$$ 
respectively, where $0\le t \le 2\pi\,$.


\begin{questionparts}
\item
Give a geometrical description of the motion of $P$ and $Q$.

\item
Let $\theta$ be the angle $POQ$ at time $t$ that satisfies 
$0\le\theta\le\pi\,$. Show that
\[
\cos\theta = \tfrac{3\surd2}{8} -\tfrac14 \cos( 2t +\tfrac14 \pi)\;.
\]


\item  Show that 
the total time  for which 
$\theta \ge \frac14 \pi$ is $\tfrac32 \pi\,$.

\end{questionparts}
\end{question}
		
%%%%%%%%% Q8
\begin{question}	
For $x \ge 0$ the curve $C$ is defined by 
$$
{\frac{\d y} {\d x}} = \frac{x^3y^2}{(1 + x^2)^{5/2}}
$$
with $y = 1$ when $x=0\,$. Show that  
\[
\frac 1 y = \frac {2+3x^2}{3(1+x^2)^{3/2}} +\frac13
\]
and hence 
that for large positive $x$ 
$$
y \approx 3 - \frac 9 x\;.
$$ 
Draw a sketch of $C$.

On a separate diagram draw a sketch of the two curves 
defined for $x \ge 0$ by
$$
\frac {\d z} {\d x} = \frac{x^3z^3}{2(1 + x^2)^{5/2}}
$$
with $z = 1$ at $x=0$ on one curve, and
$z = -1$ at $x=0$ on the other. 
\end{question}	
		

		
	
\newpage
\section*{Section B: \ \ \ Mechanics}


	
%%%%%%%%%% Q9
\begin{question}
 Two particles, $A$ and $B$,  of masses $m$ and $2m$,
respectively, are placed on a line of greatest slope, $\ell$, of a 
rough inclined plane which makes
an angle of $30^{\circ}$ with the horizontal. The coefficient
of friction between $A$ and the plane is $\frac16\sqrt{3}$ 
and the coefficient of 
friction between $B$ and the plane is $\frac13 \sqrt{3}$. 
The particles  are at rest with
$B$ higher up $\ell$ than $A$ and are connected by a light inextensible string 
which is taut. A force $P$ is applied to $B$.

\begin{questionparts}
\item Show that the least magnitude of $P$ for which 
the two particles move upwards along $\ell$ is 
$\frac{11}8 \sqrt{3}\, mg$ and give, in this case,
 the direction in which $P$ acts.

\item Find the least magnitude of $P$ for which the particles
do not slip downwards along~$\ell$.
\end{questionparts}
	\end{question}
	
%%%%%%%%%% Q10 
\begin{question}	
 The points $A$ and $B$ are $180$ metres apart and lie 
 on  horizontal ground. 
A missile is launched  from $A$ 
at  speed of $100\,$m\,s$^{-1}$ and at an
acute angle of elevation to the 
line $AB$ of $\arcsin \frac35$. A time $T$ seconds later,
an anti-missile missile  is launched from $B$, 
at speed of $200\,$m\,s$^{-1}$ 
and at an acute 
angle of elevation to the line $BA$ of $\arcsin \frac45$.    
The motion of both missiles
takes place in the vertical plane containing $A$ and $B$, and the 
missiles collide.

Taking $g =10\,$m\,s$^{-2}$ and ignoring air resistance,
find $T$.

\noindent
[Note that $\arcsin \frac35$ is another notation for $\sin^{-1} \frac35\,$.]
\end{question}

%%%%%%%%%% Q11

\begin{question}
A plane is   inclined at an 
angle $\arctan \frac34$ to the horizontal and 
 a small, smooth, light pulley~$P$ 
is fixed to the top of the  plane.  A string, $APB$, passes over the pulley.
A particle  of mass~$m_1$ 
is attached to  the string at $A$ and rests on the inclined plane with $AP$ 
parallel to a line of greatest slope in the plane. 
A particle of mass $m_2$, where $m_2>m_1$,
 is attached to the string at $B$ 
and hangs freely with $BP$ 
vertical. The coefficient of 
friction between the particle at $A$ 
and the plane is $ \frac{1}{2}$.  
 
The system is released from rest with the string taut. 
Show that the acceleration of the 
particles is $\ds \frac{m_2-m_1}{m_2+m_1}g$. 
 
At a time $T$ after release, the string breaks.
Given that the particle at $A$
does not reach the pulley at any point in its motion,
find an expression  in terms of $T$ for the time 
after release at which the particle at $A$ 
reaches its maximum height. It is found that, regardless 
of when the string broke, this time is equal to the time 
taken by the particle at $A$  to descend 
from its point of maximum height to the point 
at which it was released. Find the ratio $m_1 : m_2$.  

\noindent
[Note that $\arctan \frac34$ is another notation for $\tan^{-1} \frac34\,$.]
\end{question}
	

	
	\newpage
\section*{Section C: \ \ \ Probability and Statistics}


%%%%%%%%%% Q12
\begin{question}
 The twins Anna and Bella share a computer and never sign their e-mails.
 When I e-mail them, only the twin
currently online responds.  The
 probability that it is Anna who is online is $p$ and she answers each
 question I ask her truthfully with probability $a$, independently of all her
 other answers, even if a question is repeated. The probability that it is
 Bella  who is online is~$q$, where $q=1-p$, and she answers each question
 truthfully with probability $b$, independently of all her other answers,
 even if a question is repeated.

\begin{questionparts}
\item
I send the twins the  e-mail: 
`Toss a fair coin and answer the following  question. 
Did the coin come down heads?'. I receive the answer `yes'.
Show that the probability that the coin
did come down heads is $\frac{1}{2}$ if and
only if  $2(ap+bq)=1$.

\item
I send the twins the e-mail: 
`Toss a fair coin and answer  the following question. 
Did the coin come down heads?'. I receive the answer `yes'.
I then send the e-mail: `Did the coin come down heads?' and I receive
the answer `no'. Show that the probability (taking into
account these answers) that the coin did come down heads is $\frac{1}{2}\,$.

\item
I send the twins the e-mail: `Toss a fair coin and answer the following
question. Did the coin come down heads?'. I receive the answer `yes'.
I then send the e-mail: `Did the coin come down heads?' and I receive
the answer `yes'. Show that, if $2(ap+bq)=1$,
the probability (taking into account these answers) that the coin did
come down heads is $\frac{1}{2}\,$.

\end{questionparts}
\end{question}

%%%%%%%%%% Q13
\begin{question}
The number of printing errors on any page of a large book of $N$ pages is 
modelled by a 
Poisson variate with parameter $\lambda$ and is statistically 
independent of the number of printing errors on any other page. The number
of pages in a random sample of $n$ pages (where $n$ is much smaller than $N$
 and $n\ge2$)
which contain fewer than two errors is denoted by $Y$.
 Show that  $\P(Y=k) = \binom n k p^kq^{n-k}$ where 
$p=(1+\lambda)e^{-\lambda}$ and $q=1-p\,$.

Show also that, if $\lambda$ is sufficiently small, 
\begin{questionparts}
\item $q\approx \frac12 \lambda^2\,$;
\item the largest value of 
$n$ for which $\P(Y=n)\ge 1-\lambda$ is approximately $2/\lambda\,$;
\item $
\P(Y>1 \;\vert\; Y>0) \approx 1-n(\lambda^2/2)^{n-1}\;.$

\end{questionparts}
\end{question}

%%%%%%%%%% Q14
\begin{question}
The probability density function $\f(x)$
of the random variable $X$ is given by
$$
\f(x) = k\left[{\phi}(x) + {\lambda}\g(x)\right],\,\,\,\,
$$
where 
${\phi}(x)$ is the probability density function  
of a normal variate with mean~0 and variance~1, 
$\lambda $ is a positive constant, and $\g(x)$ is a probability density function defined by
\[
\g(x)=
\begin{cases}
1/\lambda  & \mbox{for $0 \le x \le {\lambda}$}\,;\\
             0& \mbox{otherwise} .
\end{cases}
\]
Find $\mu$, the mean of $X$, in terms of $\lambda$, and prove that 
$\sigma$, the standard deviation of $X$, satisfies.
$$
\sigma^2 = \frac{\lambda^4 +4{\lambda}^3+12{\lambda}+12} 
         {12(1 + \lambda )^2}\;.
$$ 

In  the case $\lambda=2$:
\begin{questionparts}
\item draw a sketch of the curve $y=\f(x)$;
\item express the cumulative distribution function of $X$ in terms of $\Phi(x)$,
the cumulative distribution function corresponding to $\phi(x)$;
\item evaluate $\P(0<X<\mu+2\sigma)$, given that 
$\Phi (\frac 23 + \frac23 \surd7)=0.9921$. 


\end{questionparts}
\end{question}
	
\end{document}
