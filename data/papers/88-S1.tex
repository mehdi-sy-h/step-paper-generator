
\documentclass[a4, 11pt]{report}


\pagestyle{myheadings}
\markboth{}{Paper I, 1988
\ \ \ \ \ 
\today 
}               

\RequirePackage{amssymb}
\RequirePackage{amsmath}
\RequirePackage{graphicx}
\RequirePackage{color}
\RequirePackage[flushleft]{paralist}[2013/06/09]



\RequirePackage{geometry}
\geometry{%
  a4paper,
  lmargin=2cm,
  rmargin=2.5cm,
  tmargin=3.5cm,
  bmargin=2.5cm,
  footskip=12pt,
  headheight=24pt}


\newcommand{\comment}[1]{{\bf Comment} {\it #1}}
%\renewcommand{\comment}[1]{}

\newcommand{\bluecomment}[1]{{\color{blue}#1}}
%\renewcommand{\comment}[1]{}
\newcommand{\redcomment}[1]{{\color{red}#1}}



\usepackage{epsfig}
\usepackage{pstricks-add}
\usepackage{tgheros} %% changes sans-serif font to TeX Gyre Heros (tex-gyre)
\renewcommand{\familydefault}{\sfdefault} %% changes font to sans-serif
%\usepackage{sfmath}  %%%% this makes equation sans-serif
%\input RexFigs


\setlength{\parskip}{10pt}
\setlength{\parindent}{0pt}

\newlength{\qspace}
\setlength{\qspace}{20pt}


\newcounter{qnumber}
\setcounter{qnumber}{0}

\newenvironment{question}%
 {\vspace{\qspace}
  \begin{enumerate}[\bfseries 1\quad][10]%
    \setcounter{enumi}{\value{qnumber}}%
    \item%
 }
{
  \end{enumerate}
  \filbreak
  \stepcounter{qnumber}
 }


\newenvironment{questionparts}[1][1]%
 {
  \begin{enumerate}[\bfseries (i)]%
    \setcounter{enumii}{#1}
    \addtocounter{enumii}{-1}
    \setlength{\itemsep}{5mm}
    \setlength{\parskip}{8pt}
 }
 {
  \end{enumerate}
 }



\DeclareMathOperator{\cosec}{cosec}
\DeclareMathOperator{\Var}{Var}

\def\d{{\rm d}}
\def\e{{\rm e}}
\def\g{{\rm g}}
\def\h{{\rm h}}
\def\f{{\rm f}}
\def\p{{\rm p}}
\def\s{{\rm s}}
\def\t{{\rm t}}


\def\A{{\rm A}}
\def\B{{\rm B}}
\def\E{{\rm E}}
\def\F{{\rm F}}
\def\G{{\rm G}}
\def\H{{\rm H}}
\def\P{{\rm P}}


\def\bb{\mathbf b}
\def \bc{\mathbf c}
\def\bx {\mathbf x}
\def\bn {\mathbf n}

\newcommand{\low}{^{\vphantom{()}}}
%%%%% to lower suffices: $X\low_1$ etc


\newcommand{\subone}{ {\vphantom{\dot A}1}}
\newcommand{\subtwo}{ {\vphantom{\dot A}2}}




\def\le{\leqslant}
\def\ge{\geqslant}


\def\var{{\rm Var}\,}

\newcommand{\ds}{\displaystyle}
\newcommand{\ts}{\textstyle}




\begin{document}
\setcounter{page}{2}

 
\section*{Section A: \ \ \ Pure Mathematics}

%%%%%%%%%%Q1
\begin{question}
Sketch the graph of the function $\mathrm{h}$, where 
\[
\mathrm{h}(x)=\frac{\ln x}{x},\qquad(x>0).
\]
Hence, or otherwise, find all pairs of distinct positive integers
$m$ and $n$ which satisfy the equation 
\[
n^{m}=m^{n}.
\]

\end{question}

%%%%%%%%%%Q2
\begin{question}
The function $\mathrm{f}$ and $\mathrm{g}$ are related (for all
real $x$) by 
\[
\mathrm{g}(x)=\mathrm{f}(x)+\frac{1}{\mathrm{f}(x)}\,.
\]
Express $\mathrm{g}'(x)$ and $\mathrm{g}''(x)$ in terms of $\mathrm{f}(x)$
and its derivatives. 


If $\mathrm{f}(x)=4+\cos2x+2\sin x$, find the stationary points of
$\mbox{\ensuremath{\mathrm{g}}}$ for $0\leqslant x\leqslant2\pi,$
and determine which are maxima and which are minima.
\end{question}

%%%%%%%%% Q3
\begin{question}
Two points $P$ and $Q$ lie within, or on the boundary of, a square
of side 1cm, one corner of which is the point $O$. Show that the
length of at least one of the lines $OP,PQ$ and $QO$ must be less
than or equal to $(\sqrt{6}-\sqrt{2})$ cm.
\end{question}


%%%%%% Q4 

\begin{question}
Each of $m$ distinct points on the positive $y$-axis is joined by
a line segment to each of $n$ distinct points on the positive $x$-axis.
Except at the endpoints, no three of these segments meet in a single
point. Derive formulae for 

\begin{questionparts}
\item the number of such line segments; 
\item the number of points of intersections of the segments, ignoring intersections
at the endpoints of the segments. 
\end{questionparts}

If $m=n\geqslant3,$ and the two segments with the greatest number
of points of intersection, and the two segments with the least number
of points of intersection, are excluded, prove that the average number
of points of intersection per segment on the remaining segments is
\[
\frac{n^{3}-7n+2}{4(n+2)}\,.
\]
\end{question}


%%%%%%%%% Q5
\begin{question}
	Given that $b>a>0$, find, by using the binomial theorem, coefficients
$c_{m}$ ($m=0,1,2,\ldots$) such that
\[
\frac{1}{\left(1-ax\right)\left(1-bx\right)}=c_{0}+c_{1}x+c_{2}x^{2}+\ldots+c_{m}x^{m}+\cdots
\]
for $b\left|x\right|<1$. 


Show that 
\[
c_{m}^{2}=\frac{a^{2m+2}-2(ab)^{m+1}+b^{2m+2}}{(a-b)^{2}}\,.
\]



Hence, or otherwise, show that
\[
c_{0}^{2}+c_{1}^{2}x+c_{2}^{2}x^{2}+\cdots+c_{m}^{2}x^{m}+\cdots=\frac{1+abx}{\left(1-abx\right)\left(1-a^{2}x\right)\left(1-b^{2}x\right)}\,,
\]
for $x$ in a suitable interval which you should determine. 
\end{question}
	
	%%%%%%%%% Q6
	\begin{question}
The complex numbers $z_{1},z_{2},\ldots,z_{6}$ are represented by
six distinct points $P_{1},P_{2},\ldots,P_{6}$ in the Argand diagram.
Express the following statements in terms of complex numbers: 

\begin{questionparts}
\item $\overrightarrow{P_{1}P_{2}}=\overrightarrow{P_{5}P_{4}}$ and $\overrightarrow{P_{2}P_{3}}=\overrightarrow{P_{6}P_{5}}\,$; 
\item $\overrightarrow{P_{2}P_{4}}$ is perpendicular to $\overrightarrow{P_{3}P_{6}}\,$. 
\end{questionparts}

If \textbf{(i) }holds, show that $\overrightarrow{P_{3}P_{4}}=\overrightarrow{P_{1}P_{6}}\,$. 


Suppose that the statements \textbf{(i) }and \textbf{(ii) }both hold,
and that $z_{1}=0,$ $z_{2}=1,$ $z_{3}=z,$ $z_{5}=\mathrm{i}$ and
$z_{6}=w.$ Determine the conditions which $\mathrm{Re}(z)$ and $\mathrm{Re}(w)$
must satisfy in order that $P_{1}P_{2}P_{3}P_{4}P_{5}P_{6}$ should
form a convex hexagon. 


Find the distance between $P_{3}$ and $P_{6}$ when $\tan(\angle P_{3}P_{2}P_{6})=-2/3.$ 
	 \end{question}
	 
	 %%%%%%%%% Q7
\begin{question}
The function $\mathrm{f}$ is defined by 
\[
\mathrm{f}(x)=ax^{2}+bx+c.
\]
Show that 
\[
\mathrm{f}'(x)=\mathrm{f}(1)\left(x+\tfrac{1}{2}\right)+\mathrm{f}(-1)\left(x-\tfrac{1}{2}\right)-2\mathrm{f}(0)x.
\]



If $a,b$ and $c$ are real and such that $\left|\mathrm{f}(x)\right|\leqslant1$
for $\left|x\right|\leqslant1$, show that $\left|\mathrm{f}'(x)\right|\leqslant4$
for $\left|x\right|\leqslant1$. 


Find particular values of $a,b$ and $c$ such that, for the corresponding
function $\mathrm{f}$ of the above form $\left|\mathrm{f}(x)\right|\leqslant1$
for all $x$ with $\left|x\right|\leqslant1$ and $\left\mathrm{f}'(x)\right=4$
for some $x$ satisfying $\left|x\right|\leqslant1$.
	\end{question}
	
	%%%%%%%%% Q8
	\begin{question}
$ABCD$ is a skew (non-planar) quadrilateral, and its pairs of opposite
sides are equal, i.e. $AB=CD$ and $BC=AD$. Prove that the line joining
the midpoints of the diagonals $AC$ and $BD$ is perpendicular to
each diagonal.
		
		\end{question}
		
		
%%%%%%%%% Q9
\begin{question}
Find the following integrals:

\begin{questionparts}
\item $\ {\displaystyle \int_{1}^{\mathrm{e}}\frac{\ln x}{x^{2}}\,\mathrm{d}x}\,,$
\item $\ {\displaystyle \int\frac{\cos x}{\sin x\sqrt{1+\sin x}}\,\mathrm{d}x.}$
\end{questionparts}
\end{question}

\newpage
\section*{Section B: \ \ \ Mechanics}


	
%%%%%%%%%% Q10
\begin{question}
A sniper at the top of a tree of height $h$ is hit by a bullet fired
from the undergrowth covering the horizontal ground below. The position
and elevation of the gun which fired the shot are unknown, but it
is known that the bullet left the gun with speed $v$. Show that it
must have been fired from a point within a circle centred on the base
of the tree and of radius $(v/g)\sqrt{v^{2}-2gh}$. 


{[}Neglect air resistance.{]}

	\end{question}
	
%%%%%%%%%% Q11
\begin{question}	
Derive a formula for the position of the centre of mass of a uniform
circular arc of radius $r$ which subtends an angle $2\theta$ at
the centre. 


\noindent \begin{center}
\psset{xunit=0.8cm,yunit=0.8cm,algebraic=true,dotstyle=o,dotsize=3pt 0,linewidth=0.5pt,arrowsize=3pt 2,arrowinset=0.25} \begin{pspicture*}(-2.27,-5.26)(3.1,3.1) \psline(-2,2)(2,2) \psline(2,2)(2,-3) \psline(2,-3)(-2,-3) \psline(-2,-3)(-2,2) \parametricplot{3.141592653589793}{6.283185307179586}{1*2*cos(t)+0*2*sin(t)+0|0*2*cos(t)+1*2*sin(t)+-3} \psline{->}(0,-3)(-1.15,-4.63) \psline{->}(-1.15,-4.63)(0,-3) \psline{->}(0,-2.52)(-2,-2.52) \psline{->}(-2,-2.52)(0,-2.52) \psline{->}(0,-2.52)(2,-2.52) \psline{->}(2,-2.52)(0,-2.52) \psline{->}(2.42,1.96)(2.42,-3) \psline{->}(2.42,-3)(2.42,1.96) \psline{->}(-1.96,2.36)(1.95,2.37) \psline{->}(1.95,2.37)(-1.96,2.36) \rput[tl](-1.01,-1.94){$r$} \rput[tl](1.02,-1.94){$r$} \rput[tl](-0.45,-3.8){$r$} \rput[tl](2.59,-0.3){$2h$} \rput[tl](-0.23,3.06){$2r$} \end{pspicture*}
\par\end{center}


A plane framework consisting of a rectangle and a semicircle, as in
the above diagram, is constructed of uniform thin rods. It can stand
in equilibrium if it is placed in a vertical plane with any point
of the semicircle in contact with a horizontal floor. Express $h$
in terms of $r$. 
\end{question}

%%%%%%%%%% Q12

\begin{question}A skater of mass $M$ is skating inattentively on a smooth frozen
canal. She suddenly realises that she is heading perpendicularly towards
the straight canal bank at speed $V$. She is at a distance $d$ from
the bank and can choose one of two methods of trying to avoid it;
either she can apply a force of constant magnitude $F$, acting at
right-angles to her velocity, so that she travels in a circle; or
she can apply a force of magnitude $\frac{1}{2}F(V^{2}+v^{2})/V^{2}$
directly backwards, where $v$ is her instantaneous speed. Treating
the skater as a particle, find the set of values of $d$ for which
she can avoid hitting the bank. Comment \textbf{briefly} on the assumption
that the skater is a particle. 
\end{question}
	
%%%%%%%%%% Q13
\begin{question}
A piece of circus apparatus consists of a rigid uniform plank of mass
1000$\,$kg, suspended in a horizontal position by two equal light
vertical ropes attached to the ends. The ropes each have natural length
10$\,$m and modulus of elasticity 490$\,$000 N. Initially the plank
is hanging in equilibrium. Nellie, an elephant of mass 4000$\,$kg,
lands in the middle of the plank while travelling vertically downwards
at speed 5$\,$ms$^{-1}.$ While carrying Nellie, the plank comes
instantaneously to rest at a negligible height above the floor, and
at this instant Nellie steps nimbly and gently off the plank onto
the floor. Assuming that the plank remains horizontal, and the rope
remain vertical, throughout the motion, find to three significant
figures its initial height above the floor. 


During the motion after Nellie alights, do the ropes ever become slack?


{[}Take $g$ to be $9.8\mbox{\,\ ms}^{-1}.${]} 
\end{question}
	
	\newpage
\section*{Section C: \ \ \ Probability and Statistics}


%%%%%%%%%% Q14
\begin{question}
Let $X$ be a standard normal random variable. If $M$ is any real
number, the random variable $X_{M}$ is defined in terms of $X$ by
\[
X_{M}=\begin{cases}
X & \mbox{if }X<M,\\
M & \mbox{if }X\geqslant M.
\end{cases}
\]
Show that the expectation of $X_{M}$ is given by 
\[
\mathrm{E}(X_{M})=-\phi(M)+M(1-\Phi(M)),
\]
where $\phi$ is the probability density function, and $\Phi$ is
the cumulative distribution function of $X$. 


Fifty times a year, 1024 tourists disembark from a cruise liner at
the port of Slaka. From there they must travel to the capital either
by taxi or by bus. Officials of HOGPo are equally likely to direct
a tourist to the bus station or to the taxi rank. Each bus of the
bus coorperative holds 31 passengers, and the coorperative currently
runs 16 buses. The bus coorperative makes a profit of 1 vloska for
each passenger carried. It carries all the passengers it can, with
any excess being (eventually) transported by taxi. What is the larges
annual bribe the bus coorperative should consider paying to HOGPo
in order to be allowed to run an extra bus?
\end{question}

%%%%%%%%%% Q15
\begin{question}
In Fridge football, each team scores two points for a goal and one
point for a foul committed by the opposing team. In each game, for
each team, the probability that the team scores $n$ goals is $\left(3-\left|2-n\right|\right)/9$
for $0\leqslant n\leqslant4$ and zero otherwise, while the number
of fouls committed against it will with equal probability be one of
the numbers from $0$ to $9$ inclusive. The numbers of goals and
fouls of each team are mutually independent. What is the probability
that in some game a particular team gains more than half its points
from fouls?


In response to criticisms that the game is boring and violent, the
ruling body increases the number of penalty points awarded for a foul,
in the hope that this will cause large numbers of fouls to be less
probable. During the season following the rule change, 150 games are
played and on 12 occasions (out of 300) a team committed 9 fouls.
Is this good evidence of a change in the probability distribution
of the number of fouls? Justify your answer. 
\end{question}

%%%%%%%%%% Q16
\begin{question}
Wondergoo is applied to all new cars. It protects them completely
against rust for three years, but thereafter the probability density
of the time of onset of rust is proportional to $t^{2}/(1+t^{2})^{2}$
for a car of age $3+t$ years $(t\geqslant0)$. Find the probability
that a car becomes rusty before it is $3+t$ years old. 


Every car is tested for rust annually on the anniversary of its manufacture.
If a car is not rusty, it will certainly pass; if it is rusty, it
will pass with probability $\frac{1}{2}.$ Cars which do not pass
are immediately taken off the road and destroyed. What is the probability
that a randomly selected new car subsequently fails a test taken on
the fifth anniversary of its manufacture?


Find also the probability that a car which was destroyed immediately
after its fifth anniversary test was rusty when it passed its fourth
anniversary test.
	\end{question}
\end{document}
