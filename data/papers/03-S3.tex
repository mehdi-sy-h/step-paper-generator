\documentclass[a4, 11pt]{report}


\pagestyle{myheadings}
\markboth{}{Paper III, 2003
\ \ \ \ \ 
\today 
}               

\RequirePackage{amssymb}
\RequirePackage{amsmath}
\RequirePackage{graphicx}
\RequirePackage{color}
\RequirePackage[flushleft]{paralist}[2013/06/09]



\RequirePackage{geometry}
\geometry{%
  a4paper,
  lmargin=2cm,
  rmargin=2.5cm,
  tmargin=3.5cm,
  bmargin=2.5cm,
  footskip=12pt,
  headheight=24pt}


\newcommand{\comment}[1]{{\bf Comment} {\it #1}}
%\renewcommand{\comment}[1]{}

\newcommand{\bluecomment}[1]{{\color{blue}#1}}
%\renewcommand{\comment}[1]{}
\newcommand{\redcomment}[1]{{\color{red}#1}}



\usepackage{epsfig}
\usepackage{pstricks-add}
\usepackage{tgheros} %% changes sans-serif font to TeX Gyre Heros (tex-gyre)
\renewcommand{\familydefault}{\sfdefault} %% changes font to sans-serif
%\usepackage{sfmath}  %%%% this makes equation sans-serif
%\input RexFigs


\setlength{\parskip}{10pt}
\setlength{\parindent}{0pt}

\newlength{\qspace}
\setlength{\qspace}{20pt}


\newcounter{qnumber}
\setcounter{qnumber}{0}

\newenvironment{question}%
 {\vspace{\qspace}
  \begin{enumerate}[\bfseries 1\quad][10]%
    \setcounter{enumi}{\value{qnumber}}%
    \item%
 }
{
  \end{enumerate}
  \filbreak
  \stepcounter{qnumber}
 }


\newenvironment{questionparts}[1][1]%
 {
  \begin{enumerate}[\bfseries (i)]%
    \setcounter{enumii}{#1}
    \addtocounter{enumii}{-1}
    \setlength{\itemsep}{5mm}
    \setlength{\parskip}{8pt}
 }
 {
  \end{enumerate}
 }



\DeclareMathOperator{\cosec}{cosec}
\DeclareMathOperator{\Var}{Var}

\def\d{{\mathrm d}}
\def\e{{\mathrm e}}
\def\g{{\mathrm g}}
\def\h{{\mathrm h}}
\def\f{{\mathrm f}}
\def\p{{\mathrm p}}
\def\s{{\mathrm s}}
\def\t{{\mathrm t}}


\def\A{{\mathrm A}}
\def\B{{\mathrm B}}
\def\E{{\mathrm E}}
\def\F{{\mathrm F}}
\def\G{{\mathrm G}}
\def\H{{\mathrm H}}
\def\P{{\mathrm P}}


\def\bb{\mathbf b}
\def \bc{\mathbf c}
\def\bx {\mathbf x}
\def\bn {\mathbf n}

\newcommand{\low}{^{\vphantom{()}}}
%%%%% to lower suffices: $X\low_1$ etc


\newcommand{\subone}{ {\vphantom{\dot A}1}}
\newcommand{\subtwo}{ {\vphantom{\dot A}2}}




\def\le{\leqslant}
\def\ge{\geqslant}


\def\var{{\rm Var}\,}

\newcommand{\ds}{\displaystyle}
\newcommand{\ts}{\textstyle}
\def\half{{\textstyle \frac12}}
\def\l{\left(}
\def\r{\right)}



\begin{document}
\setcounter{page}{2}

 
\section*{Section A: \ \ \ Pure Mathematics}

%%%%%%%%%%Q1
\begin{question}
Given that $x+a>0$ and $x+b>0\,$, and that $b>a\,$, show that   
\[  
\frac{\mathrm{d} \ }{\mathrm{d} x}   
\arcsin \l \frac{x + a }{ \ x + b} \r =   
\frac{ \sqrt{\;b - a\;}} {\vphantom{\Big(} \l x + b \r \sqrt{ a + b + 2x\;} \ \ }  
\]  
and find $\displaystyle   
\frac{\mathrm{d} \ }{  \mathrm{d} x} \;  \mathrm{arcosh} \l \frac{x + b }{ \ x + a} \r \,$.  
  
Hence, or otherwise, integrate, for $x > -1\,$,  
\begin{questionparts}  
\item $\displaystyle \int
\frac{1}{ \vphantom{\dot (}\l x + 1 \r \sqrt{x + 3} \ } \; \mathrm{d} x\;$,  
\item $\displaystyle \int
\frac{1} {\vphantom{\dot (}\l x + 3 \r \sqrt{x + 1} \ } \; \mathrm{d} x\;$.  
\end{questionparts}  
  
\noindent[You may use the results \  
$\ds \frac{\d \ }{\d x} \arcsin x = \frac 1 {\sqrt{1-x^2\;}\;}$  
\ and \   
$\ds \frac{\d \ }{\d x} \; {\rm arcosh } \; x = \frac 1 {\sqrt{x^2-1}\;}\;$. \ ]  
\end{question}

%%%%%%%%%%Q2
\begin{question}
 Show that 
 $\ds   
 ^{2r} \! {\rm C}_r =\frac{1\times3\times\dots\times (2r-1)}{r!} \, \times 2^r
 \;,
 $
 for $r\ge1\,$.
 
 \begin{questionparts}  
 \item Give  the first four terms of the binomial series for 
 $\l 1 - p \r^{-\frac12}$.  
  
 By choosing a suitable value for $p$ in this series, or otherwise, show that 
 $$
 \displaystyle \sum_{r=0}^\infty \frac{  {\vphantom {\A}}^{2r} \! {\rm C}_r }{ 8^r} = \sqrt 2
 \;
 .$$  
  
 \item Show that 
 $$
 \displaystyle 
 \sum_{r=0}^\infty   
 \frac{\l 2r + 1 \r \; {\vphantom{A}}^{2r} \! {\rm C} _r }{ 5^r} =\big( \sqrt 5\big)^3
 \;.
 $$ 
 \end{questionparts}  
  
 [{\bf Note: } 
 $
 {\vphantom{A}}^n {\rm C}_r
 $ 
 is an alternative notation for  
 $\ds \
 \binom n r 
 \,
 $ for $r\ge1\,$, and $
 {\vphantom{A}}^0 {\rm C}_0 =1
 $ .]
\end{question}

%%%%%%%%% Q3
\begin{question}
If $m$ is a positive integer,    
show that $\l 1+x \r^m + \l 1-x \r^m \ne 0$ for any real $x\,$.   
   
The function $\f$ is defined by   
\[   
\f (x) = \frac{ (1+x )^m - ( 1-x )^m}{ (1+x )^m + (1-x )^m}   
\;.   
\]   
Find and simplify an expression for $\f'(x)$.   
   
In the case $m=5\,$,   
sketch the curves    
$y = \f (x)$ and  $\displaystyle y = \frac1 { \f (x )}\;$. 
\end{question}

%%%%%% Q4 
\begin{question}
A curve is defined parametrically by   
\[   
x=t^2 \;, \ \ \    
y=t (1 + t^2 )   
\;.   
\]   
The tangent at the  point with parameter $t$, where $t\ne0\,$, meets the    
curve again at the point with parameter $T$, where $T\ne t\,$. Show that   
\[   
T = \frac{1 - t^2 }{2t} \mbox { \ \ \ and \ \ \ } 3t^2\ne 1\;.   
\]   
Given a point $P_0\,$ on the curve, with parameter $t_0\,$,    
a sequence of points $P_0 \, , \; P_1 \, , \; P_2 \, , \ldots$    
on the curve is constructed such that the tangent at $P_i$ meets    
the curve again at $P_{i+1}$. If $t_0 = \tan \frac{ 7 } {18}\pi\,$,    
show that $P_3 = P_0$ but $P_1\ne P_0\,$.    
Find a second value of $t_0\,$, with $t_0>0\,$,   
for which $P_3 = P_0$ but  $P_1\ne P_0\,$.   
\end{question}

%%%%%%%%% Q5
\begin{question}
Find the coordinates of the turning point on the curve $y = x^2 - 2bx + c\,$.   
 Sketch the curve in the case that the equation $x^2 - 2bx + c=0$ has two    
distinct real roots. Use your sketch to determine necessary and sufficient   
conditions on $b$ and $c$ for the equation $x^2 - 2bx + c = 0$    
to have two distinct real roots. Determine   
necessary and sufficient   
conditions on $b$ and $c$ for this equation to have two distinct positive roots.   
   
Find the coordinates of the turning points on the curve    
$y = x^3 - 3b^2x + c$ (with $b>0$) and hence determine necessary and sufficient   
conditions on $b$ and $c$ for the equation $x^3 - 3b^2x + c = 0$    
to have three distinct real roots. Determine necessary and sufficient   
conditions on $a\,$, $b$ and $c$ for the equation    
$\l x - a \r^3 - 3b^2 \l x - a \r + c = 0$ to have three distinct positive roots.   
   
Show that the equation $2x^3 - 9x^2 + 7x - 1 = 0$ has three distinct positive roots.   
	\end{question}
	
%%%%%%%%% Q6
\begin{question}
Show that 
\[   
\ts
2\sin \frac12 \theta \, \cos r\theta  = \sin\big(r+\frac12\big)\theta -
\sin\big(r-\frac12\big)\theta
\;.   
\]   
Hence, or otherwise,    
find all solutions of the equation   
\[   
\cos a\theta + \cos (a + 1) \theta + \dots + \cos(b-2)\theta+\cos (b - 1 ) \theta = 0
\;,   
\]   
where $a$ and $b$ are positive integers with $a < b-1\,$.  
\end{question}
	
%%%%%%%%% Q7
\begin{question}
In the $x$--$y$ plane, the point $A$ has coordinates    
$(a\,,0)$ and the point $B$ has coordinates $(0\,,b)\,$,    
where $a$ and $b$ are positive.    
The point $P\,$, which is distinct from $A$ and $B$, has coordinates~$(s,t)\,$.
$X$ and $Y$ are the feet of the perpendiculars from $P$ to the $x$--axis and
$y$--axis respectively, and     
$N$ is the foot of the perpendicular from $P$ to the line $AB\,$.   
Show that the coordinates $(x\,,y)$ of $N$ are given by   
\[   
x= \frac {ab^2 -a(bt-as)}{a^2+b^2} \;, \ \ \    
y = \frac{a^2b +b(bt-as)}{a^2+b^2} \;.   
\]   
   
Show that, if    
$\ds   \
\left( \frac{t-b} s\right)\left( \frac t {s-a}\right) = -1\;$, then $N$ lies on    
the line $XY\,$.   
   
Give a geometrical interpretation of this result.   
\end{question}
		
%%%%%%%%% Q8
\begin{question}	
\begin{questionparts}
\item Show that the gradient at a point $\l x\,, \, y \r$ on the curve 
\[
\l y + 2x \r^3 \l y - 4x \r = c\;,
\]
where $c$ is a constant, is given by
\[
\frac{\d y}{\d x} = \frac{16 x -y}{2y-5x} \;.
\]

\item By considering the derivative with respect to $x$ of 
$\l y + ax \r^n \l y + bx \r\,$, or otherwise, 
find the general solution of the differential equation
\[
\frac{\mathrm{d}y}{ \mathrm{d}x} = \frac{10x - 4y}{  3x - y}\;.
\]

\end{questionparts}
\end{question}	
		

		
	
\newpage
\section*{Section B: \ \ \ Mechanics}


	
%%%%%%%%%% Q9
\begin{question}
A particle $P$ of mass $m$ is constrained to move on a vertical 
circle of smooth wire with centre~$O$ and of radius $a$. 
$L$ is the lowest point of the circle and $H$ the highest and 
$\angle LOP = \theta\,$. The particle is attached to $H$ by an 
elastic string of natural length $a$ and modulus of elasticity~$\alpha mg\,$, 
where $\alpha > 1\,$.  Show that,  if $\alpha>2\,$, there is an
equilibrium position with $0<\theta<\pi\,$.

   
Given that $\alpha =2+\sqrt 2\,$, and that 
$\displaystyle \theta = \tfrac{1}{2}\pi + \phi\,$, show that   
\[   
\ddot{\phi} \approx -\frac{g (\sqrt2+1)}{2a }\, \phi
\]   
when $\phi$ is small.   
   
For this value of $\alpha$, 
explain briefly what happens to the particle if it 
is given a small displacement when $ \theta = \frac{1}{2}\pi$.
	\end{question}
	
%%%%%%%%%% Q10 
\begin{question}	
A particle moves along the $x$-axis in such a way that 
its acceleration is  
$kx \dot{x}\,$ where  $k$ is a positive constant.
When $t = 0$,  $x = d$ (where $d>0$) and $\dot{x} =U\,$.   
\begin{questionparts}   
\item Find $x$ as a function of $t$ in the case $U = kd^2$ 
and show that $x$ tends to infinity as $t$~tends~to 
$\displaystyle \frac{\pi }{2 dk}\,$.   
\item If $U < 0$, find $x$ as a function of $t$ 
and show that it tends to a limit, which you should state in terms of $d$ and $U\,$, 
as $t$~tends to infinity.   
\end{questionparts}   
\end{question}

%%%%%%%%%% Q11

\begin{question}
Point $B$ is a distance $d$ due south of point $A$ on a horizontal plane. 
Particle $P$ is at rest at $B$ at $t=0$, when it begins 
to move with constant acceleration $a$ in a straight line
with  fixed bearing~$\beta\,$. 
Particle $Q$ is projected  from 
point $A$ at $t=0$ and moves in a straight line with constant
speed $v\,$. Show that if the direction of projection of $Q$ can be chosen so that
$Q$ strikes $P$, then   
\[   
v^2 \ge ad \l 1 - \cos \beta \r\;.   
\]   

Show further that if $v^2 >ad(1-\cos\beta)$ then the direction of projection of $Q$
can be chosen so that $Q$ strikes $P$ before $P$ has moved a distance $d\,$.
\end{question}
	

	
	\newpage
\section*{Section C: \ \ \ Probability and Statistics}


%%%%%%%%%% Q12
\begin{question}
Brief interruptions to my work occur on average every ten minutes 
and the number of interruptions in any given time period has a Poisson distribution. 
Given that an interruption has just occurred, find the probability 
that I will have less than $t$ minutes to work before the next interruption. 
If the random variable $T$ is the time I have to work before the next interruption, 
find the probability density function of $T\,$.   
      
I need an uninterrupted half hour to finish an important paper.  
Show that the expected number of interruptions before my first 
uninterrupted period of half an hour or more is $\e^3-1$. 
Find also the expected length of time between interruptions 
that are less than half an hour apart.
Hence write down 
the expected wait before my first uninterrupted period of half an hour or more.
\end{question}

%%%%%%%%%% Q13
\begin{question}
In a rabbit warren, underground chambers 
$A, B, C$ and $D$ are at the vertices of a square, 
and burrows join $A$ to $B$, \  $B$ to $C$, \ $C$ to $D$ and $D$ to $A$. 
Each of the chambers also has a tunnel to the surface. 
A rabbit finding itself in any chamber runs along one 
of the two burrows to a neighbouring chamber, or 
leaves the burrow through the tunnel to the surface. 
Each of these three possibilities is equally likely.

Let $p_A\,$, $p_B\,$, $p_C$ and $p_D$ be the probabilities 
of a rabbit leaving the burrow through the tunnel from chamber $A$, 
given that it is currently in chamber $A, B, C$ or $D$, respectively.
\begin{questionparts}
\item Explain why  $p_A = \frac13 + \frac13p_B + \frac13 p_D$.
\item Determine $p_A\,$.
\item Find the probability 
that a rabbit which starts in chamber $A$ does not visit chamber~$C$, 
given that it eventually leaves the burrow through the tunnel in chamber $A$.
\end{questionparts}
\end{question}

%%%%%%%%%% Q14
\begin{question}
Write down the probability generating function for 
the score on a standard, fair six-faced die whose faces are labelled 
$1, 2, 3, 4, 5, 6$. Hence show that the probability 
generating function for the sum of the scores on two standard, 
fair six-faced dice, rolled independently, can be written as
\[
\frac1{36} t^2 \l 1 + t \r^2 \l 1 - t + t^2 \r^2 \l 1 + t + t^2 \r^2 \;.
\]

Write down, in factorised form, the probability generating functions for 
the scores on two fair six-faced dice whose faces are labelled with the 
numbers $1, 2, 2, 3, 3, 4$ and $1, 3, 4, 5, 6, 8,$ 
and hence show that when these dice are rolled independently, 
the probability of any given sum of the scores is the same as for the two standard 
fair six-faced dice.

Standard, fair four-faced dice are tetrahedra whose faces are labelled $1, 2, 3, 4,$ 
the score being taken from the face which is not 
visible after throwing, and each score being equally likely. 
Find all the ways in which two fair four-faced dice can have 
their faces labelled with positive integers if the probability 
of any given sum of the scores is to be the same as for the two standard 
fair four-faced dice.
\end{question}
	
\end{document}
