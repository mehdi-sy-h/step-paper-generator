\documentclass[a4, 11pt]{report}


\pagestyle{myheadings}
\markboth{}{Paper III, 2002
\ \ \ \ \ 
\today 
}               

\RequirePackage{amssymb}
\RequirePackage{amsmath}
\RequirePackage{graphicx}
\RequirePackage{color}
\RequirePackage[flushleft]{paralist}[2013/06/09]



\RequirePackage{geometry}
\geometry{%
  a4paper,
  lmargin=2cm,
  rmargin=2.5cm,
  tmargin=3.5cm,
  bmargin=2.5cm,
  footskip=12pt,
  headheight=24pt}


\newcommand{\comment}[1]{{\bf Comment} {\it #1}}
%\renewcommand{\comment}[1]{}

\newcommand{\bluecomment}[1]{{\color{blue}#1}}
%\renewcommand{\comment}[1]{}
\newcommand{\redcomment}[1]{{\color{red}#1}}



\usepackage{epsfig}
\usepackage{pstricks-add}
\usepackage{tgheros} %% changes sans-serif font to TeX Gyre Heros (tex-gyre)
\renewcommand{\familydefault}{\sfdefault} %% changes font to sans-serif
%\usepackage{sfmath}  %%%% this makes equation sans-serif
%\input RexFigs


\setlength{\parskip}{10pt}
\setlength{\parindent}{0pt}

\newlength{\qspace}
\setlength{\qspace}{20pt}


\newcounter{qnumber}
\setcounter{qnumber}{0}

\newenvironment{question}%
 {\vspace{\qspace}
  \begin{enumerate}[\bfseries 1\quad][10]%
    \setcounter{enumi}{\value{qnumber}}%
    \item%
 }
{
  \end{enumerate}
  \filbreak
  \stepcounter{qnumber}
 }


\newenvironment{questionparts}[1][1]%
 {
  \begin{enumerate}[\bfseries (i)]%
    \setcounter{enumii}{#1}
    \addtocounter{enumii}{-1}
    \setlength{\itemsep}{5mm}
    \setlength{\parskip}{8pt}
 }
 {
  \end{enumerate}
 }



\DeclareMathOperator{\cosec}{cosec}
\DeclareMathOperator{\Var}{Var}

\def\d{{\mathrm d}}
\def\e{{\mathrm e}}
\def\g{{\mathrm g}}
\def\h{{\mathrm h}}
\def\f{{\mathrm f}}
\def\p{{\mathrm p}}
\def\s{{\mathrm s}}
\def\t{{\mathrm t}}


\def\A{{\mathrm A}}
\def\B{{\mathrm B}}
\def\E{{\mathrm E}}
\def\F{{\mathrm F}}
\def\G{{\mathrm G}}
\def\H{{\mathrm H}}
\def\P{{\mathrm P}}


\def\bb{\mathbf b}
\def \bc{\mathbf c}
\def\bx {\mathbf x}
\def\bn {\mathbf n}

\newcommand{\low}{^{\vphantom{()}}}
%%%%% to lower suffices: $X\low_1$ etc


\newcommand{\subone}{ {\vphantom{\dot A}1}}
\newcommand{\subtwo}{ {\vphantom{\dot A}2}}




\def\le{\leqslant}
\def\ge{\geqslant}


\def\var{{\rm Var}\,}

\newcommand{\ds}{\displaystyle}
\newcommand{\ts}{\textstyle}
\def\half{{\textstyle \frac12}}
\def\l{\left(}
\def\r{\right)}



\begin{document}
\setcounter{page}{2}

 
\section*{Section A: \ \ \ Pure Mathematics}

%%%%%%%%%%Q1
\begin{question}
Find the area of the region between the curve 
$\displaystyle y = {\ln x \over x}\,$ and the $x$-axis, for $1 \le x \le a$.
What happens to this area as $a$ tends to infinity?

Find the volume of the solid obtained when the region  between the curve 
$\displaystyle y = {\ln x \over x}\,$ and the $x$-axis, for $1 \le x\le a$,
 is rotated through $2 \pi$ radians about the $x$-axis.
What happens to this volume as $a$ tends to infinity?
\end{question}

%%%%%%%%%%Q2
\begin{question}
Prove that  
$\displaystyle \arctan a + \arctan b = \arctan \l {a + b \over 1-ab} \r\,$
when $0<a<1$ and $0<b<1\,$.

Prove by induction that, for $n \ge 1\,$,
\[
\sum_{r = 1}^n \arctan \l {1 \over r^2 + r + 1} \r = \arctan \l {n \over n+2} \r
\]
and hence find
\[
\sum_{r = 1}^\infty \arctan \l {1 \over r^2 + r + 1} \r\,.
\]

Hence prove that 
\[
\sum_{r = 1}^\infty \arctan \l {1 \over r^2 - r + 1} \r = {\pi \over 2}\,.
\]
\end{question}

%%%%%%%%% Q3
\begin{question}
Let 
\[\f(x) = a \sqrt{x} - \sqrt{x - b}\;,
\]
 where $x\ge b >0$ and $a>1\,$.
Sketch the graph of $\f(x)\,$.
Hence show that the equation $\f(x)  = c$, where $c>0$, 
has no solution when $c^2 < b \l a^2 - 1 \r\,$.
Find conditions on $c^2$ in terms of $a$ and $b$ 
for the equation to have exactly one or exactly two solutions.

Solve the equations 
\begin{questionparts}
	\item  $3 \sqrt{x} - \sqrt{x - 2} = 4\, ,$ 
\item   $3 \sqrt{x} - \sqrt{x - 3} = 5\;$.
\end{questionparts}

\end{question}

%%%%%% Q4 
\begin{question}
Show that if $x$ and $y$ are positive  and $x^3 + x^2 = y^3 - y^2$ then $x < y\,$.

Show further that if $0 < x \le y - 1$, then $x^3 + x^2 < y^3 - y^2$.

Prove that there does not exist a pair of {\sl positive} integers 
such that the difference of their cubes is
 equal to the sum of their squares.

Find all the pairs of integers such that the 
difference of their cubes  is equal to the sum of their squares.

\end{question}

%%%%%%%%% Q5
\begin{question}
Give a condition that must be satisfied by  $p$, $q$ and $r$ for it 
to be possible to write the quadratic polynomial 
$px^2 + qx + r$ in the form $p \l x + h \r^2$, for some $h$.

Obtain  an equation, which you need not simplify,
that  must be satisfied by $t$ if it is possible to write 
\[
\l x^2 + \textstyle{{1 \over 2}} bx + t \r^2 - \l x^4 + bx^3 + cx^2 +dx +e \r 
\]
in the form $k \l x + h \r^2$, for some $k$ and $h$. 

Hence, or otherwise, write $x^4 + 6x^3 + 9x^2 -2x -7$ as a product of two quadratic factors.
	\end{question}
	
%%%%%%%%% Q6
\begin{question}
Find all the solution curves of the differential equation
\[
y^4 \l {\mathrm{d}y \over \mathrm{d}x }\r^{\! \! 4} = \l y^2 - 1 \r^2
\]
that  pass through either of the points
\begin{questionparts}
\item $\l 0, \, \frac{1}{2}\sqrt3 \r$,
\item $\l 0, \, \frac{1}{2}\sqrt5 \r$.
\end{questionparts}
Show also that $y = 1$ and $y = -1$ are solutions of the differential equation.
Sketch all these solution curves on a single set of axes.

\end{question}
	
%%%%%%%%% Q7
\begin{question}
Given that  $\alpha$ and $\beta$ are acute angles,  show that
$\alpha + \beta = \tfrac{1}{2}\pi$ if and only if $\cos^2 \alpha + \cos^2 \beta = 1$.

In the $x$--$y$ plane, the point $A$ has coordinates $(0,s)$ and the point
$C$ has coordinates $(s,0)$, where $s>0$. The point $B$ lies in the 
first quadrant ($x>0$, $y>0$). The lengths of $AB$, $OB$ and $CB$
are respectively $a$, $b$ and $c$.

Show that
\[
(s^2 +b^2 - a^2)^2 + (s^2 +b^2 -c^2)^2 = 4s^2b^2
\]
and hence that
\[
(2s^2 -a^2-c^2)^2 + (2b^2 -a^2-c^2)^2 =4a^2c^2\;.
\]


Deduce that
$$
\l a - c \r^2 \le 2b^2 \le \l a + c \r^2\;. 
$$


%Show,
%by considering the case $a=1+\surd2\,$, $b=c=1\,$,
% that the condition $\l \ast \r\,$ 
%is not sufficient to ensure that $B$ lies in the first quadrant.
\end{question}
		
%%%%%%%%% Q8
\begin{question}	
Four complex numbers $u_1$, $u_2$, $u_3$ and $u_4$ 
have unit modulus, and arguments $\theta_1$, 
$\theta_2$, $\theta_3$ and $\theta_4$, 
respectively, with $-\pi < \theta_1 < \theta_2 < \theta_3 < \theta_4 < \pi$.

Show that
\[
\arg \l u_1 - u_2 \r  = \tfrac{1}{2} \l  \theta_1 + \theta_2 -\pi \r + 2n\pi
\]
where $n =  0 \hspace{4 pt} \mbox{or} \hspace{4 pt} 1\,$.
Deduce that
\[
\arg \l \l u_1 - u_2 \r \l u_4 - u_3 \r \r 
= \arg \l \l u_1 - u_4 \r \l u_3 - u_2 \r \r + 2n\pi 
\]
for some integer $n$.

Prove that
\[
| \l u_1 - u_2 \r \l u_4 - u_3 \r | + | \l u_1 - u_4 \r \l u_3 - u_2 \r |
= | \l u_1 - u_3 \r \l u_4 - u_2 \r |\;.
\]
\end{question}	
		

		
	
\newpage
\section*{Section B: \ \ \ Mechanics}


	
%%%%%%%%%% Q9
\begin{question}
A tall container made of light material of negligible thickness has
 the form of a  prism, with a square base  of area $a^2$. It contains 
a volume $ka^3$ of fluid of uniform density. The container is held so that it stands 
on a rough plane, which   is  inclined at angle 
$\theta$ to the horizontal,  with two   
of the edges  of the   base of the container  horizontal. 
In the case $k> \frac12 \tan\theta$, show that the centre of mass of the fluid
is at a distance $x$ from the lower side of the container and at a 
distance $y$ from the base of the container, where
\[
\frac x a = \frac12 - \frac {\tan\theta}{12k}\;,
\ \ \ \ \ \
\frac y a = \frac k 2 + \frac{\tan^2\theta}{24k}\;.
\]
Determine  the corresponding coordinates in the case $k< \frac12 \tan\theta$.


The container is now released.
Given that  $k < {1 \over 2}$, show that the container will topple if $\theta >45^\circ$.
	\end{question}
	
%%%%%%%%%% Q10 
\begin{question}	
A light hollow cylinder of radius $a$ can rotate freely 
about its axis of symmetry, 
which is fixed and horizontal. 
A particle of mass $m$ is fixed to the cylinder, 
and a second particle, also of mass $m$, moves 
on the rough inside surface of the cylinder. 
Initially, the cylinder is at rest, 
with the fixed particle on the same horizontal level as its axis
and the second particle at rest vertically below this axis. 
The system is then released. 
Show that, if $\theta$ is the angle through which the cylinder has rotated, then
\[
\ddot{\theta} = {g \over 2a} \l \cos \theta - \sin \theta \r \,,
\]
provided that the second particle does not slip.

Given that the coefficient of friction is 
$ (3 + \sqrt{3})/6$, show that  the second particle 
starts to slip when the cylinder has rotated through $60^\circ$.
\end{question}

%%%%%%%%%% Q11

\begin{question}
A particle  moves on a smooth triangular horizontal surface  $AOB$ with angle 
$AOB = 30^\circ$. 
The surface  is bounded by two vertical walls 
$OA$ and $OB$ and the coefficient of restitution 
between the particle and the walls is $e$, where $e < 1$. 
The particle, which is initially at point $P$ on the surface 
and moving with velocity $u_1$, 
strikes the wall $OA$ at $M_1$, with angle $PM_1A = \theta$, and rebounds, 
with velocity $v_1$, to strike the wall $OB$ at $N_1$, 
with angle $M_1N_1B = \theta$. 
Find $e$ and $\displaystyle {v_1 \over u_1}$ in terms of $\theta$.

The motion continues, 
with the particle striking side $OA$ at $M_2$, $M_3$,   $ \ldots $ and striking 
side $OB$ at $N_2$, $N_3$,  $\ldots $. 
Show that, if $\theta < 60^\circ\,$, 
 the particle reaches $O$ in a finite time.  
\end{question}
	

	
	\newpage
\section*{Section C: \ \ \ Probability and Statistics}


%%%%%%%%%% Q12
\begin{question}
In a game, a player tosses a biased coin 
repeatedly until two successive tails occur, when the game terminates. 
For each head which occurs the player wins $\pounds 1$. 
If $E$ is the expected number of tosses of the 
coin in the course of a game, and $p$ is the probability of a head, explain why
\[
E = p \l 1 + E \r + \l 1 - p \r p \l 2 + E \r + 2 \l 1 - p \r ^2\,,
\]
and hence determine $E$ in terms of $p$. 
Find also, in terms of $p$, the expected winnings in the course of a game.

A second game is played, 
with the same rules, except that the player continues to 
toss the coin until $r$ successive tails occur. 
Show that the expected number of tosses in the 
course of a game is given by the expression 
$\displaystyle {1 - q^r \over p q^r}\,$, where $q = 1 - p$.

\end{question}

%%%%%%%%%% Q13
\begin{question}
A continuous random variable is said to have an exponential distribution 
with parameter $\lambda$ if its density function is 
$\f(t) = \lambda \e ^{- \lambda t} \; \l 0 \le t < \infty \r\,$. 
If $X_1$ and $X_2$, which are independent random variables, 
have exponential distributions with parameters $\lambda_1$ and $\lambda_2$ respectively, 
find an expression for the probability that either $X_1$ or $X_2$ (or both) 
is less than $x$. Prove that if $X$ is the random variable 
whose value is the lesser of the values of $X_1$ and $X_2$, 
then $X$ also has an exponential distribution.

Route A and Route B buses run from my house to my college. 
The time between buses on each route has an 
exponential distribution and the mean time between buses is 
15 minutes for Route A  and 30 minutes for Route B. 
The timings of the buses on the two routes are independent. 
If I emerge from my house one day to see a Route A bus 
and a Route B bus just leaving the stop, 
show that the median wait for the next bus to my college will be approximately 7 minutes.
\end{question}

%%%%%%%%%% Q14
\begin{question}
Prove that, for any two discrete random variables $X$ and $Y$,
\[
\mathrm{Var} \left(X + Y \right) 
= \mathrm{Var}(X) + \mathrm{Var}(Y) +
2 \, \mathrm{Cov}(X,Y),
\]
where $\mathrm{Var}(X)$ 
is the variance of $X$ and $\mathrm{Cov}(X,Y)$  is 
the covariance of $X$ and $Y$.

When a Grandmaster plays a sequence of 
$m$ games of chess, she is, independently,
 equally likely to win, lose or draw each game. 
If the values of the random variables $W$, $L$ and $D$ are 
the numbers of her wins, losses and draws respectively, justify briefly
the following claims:
\begin{questionparts}
\item$W + L + D$ has variance $0\,$;
\item $W + L$ has a  binomial distribution.
\end{questionparts}
Find the value of 
%$\displaystyle \rho \left[ W , \, L \right] = 
$\displaystyle {\mathrm{Cov}(W,L) \over \sqrt{\mathrm{Var}(W) \mathrm{Var}(L)}}\;$.
\end{question}
	
\end{document}
