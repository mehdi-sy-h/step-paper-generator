
\documentclass[a4, 11pt]{report}


\pagestyle{myheadings}
\markboth{}{Paper I, 1991
\ \ \ \ \ 
\today 
}               

\RequirePackage{amssymb}
\RequirePackage{amsmath}
\RequirePackage{graphicx}
\RequirePackage{color}
\RequirePackage[flushleft]{paralist}[2013/06/09]



\RequirePackage{geometry}
\geometry{%
  a4paper,
  lmargin=2cm,
  rmargin=2.5cm,
  tmargin=3.5cm,
  bmargin=2.5cm,
  footskip=12pt,
  headheight=24pt}


\newcommand{\comment}[1]{{\bf Comment} {\it #1}}
%\renewcommand{\comment}[1]{}

\newcommand{\bluecomment}[1]{{\color{blue}#1}}
%\renewcommand{\comment}[1]{}
\newcommand{\redcomment}[1]{{\color{red}#1}}



\usepackage{epsfig}
\usepackage{pstricks-add}
\usepackage{tgheros} %% changes sans-serif font to TeX Gyre Heros (tex-gyre)
\renewcommand{\familydefault}{\sfdefault} %% changes font to sans-serif
%\usepackage{sfmath}  %%%% this makes equation sans-serif
%\input RexFigs


\setlength{\parskip}{10pt}
\setlength{\parindent}{0pt}

\newlength{\qspace}
\setlength{\qspace}{20pt}


\newcounter{qnumber}
\setcounter{qnumber}{0}

\newenvironment{question}%
 {\vspace{\qspace}
  \begin{enumerate}[\bfseries 1\quad][10]%
    \setcounter{enumi}{\value{qnumber}}%
    \item%
 }
{
  \end{enumerate}
  \filbreak
  \stepcounter{qnumber}
 }


\newenvironment{questionparts}[1][1]%
 {
  \begin{enumerate}[\bfseries (i)]%
    \setcounter{enumii}{#1}
    \addtocounter{enumii}{-1}
    \setlength{\itemsep}{5mm}
    \setlength{\parskip}{8pt}
 }
 {
  \end{enumerate}
 }



\DeclareMathOperator{\cosec}{cosec}
\DeclareMathOperator{\Var}{Var}

\def\d{{\rm d}}
\def\e{{\rm e}}
\def\g{{\rm g}}
\def\h{{\rm h}}
\def\f{{\rm f}}
\def\p{{\rm p}}
\def\s{{\rm s}}
\def\t{{\rm t}}


\def\A{{\rm A}}
\def\B{{\rm B}}
\def\E{{\rm E}}
\def\F{{\rm F}}
\def\G{{\rm G}}
\def\H{{\rm H}}
\def\P{{\rm P}}


\def\bb{\mathbf b}
\def \bc{\mathbf c}
\def\bx {\mathbf x}
\def\bn {\mathbf n}

\newcommand{\low}{^{\vphantom{()}}}
%%%%% to lower suffices: $X\low_1$ etc


\newcommand{\subone}{ {\vphantom{\dot A}1}}
\newcommand{\subtwo}{ {\vphantom{\dot A}2}}




\def\le{\leqslant}
\def\ge{\geqslant}


\def\var{{\rm Var}\,}

\newcommand{\ds}{\displaystyle}
\newcommand{\ts}{\textstyle}




\begin{document}
\setcounter{page}{2}

 
\section*{Section A: \ \ \ Pure Mathematics}

%%%%%%%%%%Q1
\begin{question}
If $\theta+\phi+\psi=\tfrac{1}{2}\pi,$ show that 
\[
\sin^{2}\theta+\sin^{2}\phi+\sin^{2}\psi+2\sin\theta\sin\phi\sin\psi=1.
\]
By taking $\theta=\phi=\tfrac{1}{5}\pi$ in this equation, or otherwise,
show that $\sin\tfrac{1}{10}\pi$ satisfies the equation 
\[
8x^{3}+8x^{2}-1=0.
\]
\end{question}

%%%%%%%%%%Q2
\begin{question}
Frosty the snowman is made from two uniform spherical snowballs, of
initial radii $2R$ and $3R.$ The smaller (which is his head) stands
on top of the larger. As each snowball melts, its volume decreases
at a rate which is directly proportional to its surface area, the
constant of proportionality being the same for both snowballs. During
melting each snowball remains spherical and uniform. When Frosty is
half his initial height, find the ratio of his volume to his initial
volume. 


If $V$ and $S$ denote his total volume and surface area respectively,
find the maximum value of $\dfrac{\mathrm{d}V}{\mathrm{d}S}$ up to
the moment when his head disappears.
\end{question}

%%%%%%%%% Q3
\begin{question}
A path is made up in the Argand diagram of a series of straight line
segments $P_{1}P_{2},$ $P_{2}P_{3},$ $P_{3}P_{4},\ldots$ such that
each segment is $d$ times as long as the previous one, $(d\neq1)$,
and the angle between one segment and the next is always $\theta$
(where the segments are directed from $P_{j}$ towards $P_{j+1}$,
and all angles are measured in the anticlockwise direction). If $P_{j}$
represents the complex number $z_{j},$ express 
\[
\frac{z_{n+1}-z_{n}}{z_{n}-z_{n-1}}
\]
as a complex number (for each $n\geqslant2$), briefly justifying
your answer. 


If $z_{1}=0$ and $z_{2}=1$, obtain an expression for $z_{n+1}$
when $n\geqslant2$. By considering its imaginary part, or otherwise,
show that if $\theta=\frac{1}{3}\pi$ and $d=2$, then the path crosses
the real axis infinitely often.  
\end{question}


%%%%%% Q4 

\begin{question}
$\ $\vspace{-1cm}



\noindent \begin{center}
\psset{xunit=1cm,yunit=1cm,algebraic=true,dotstyle=o,dotsize=3pt 0,linewidth=0.5pt,arrowsize=3pt 2,arrowinset=0.25} \begin{pspicture*}(0.3,-0.03)(8.14,7.64) \pspolygon[linewidth=0pt,linecolor=lightgray,hatchcolor=lightgray,fillstyle=hlines,hatchangle=45.0,hatchsep=0.17](3,5)(3,3)(5,3)(5,5) \psline(1,7)(1,1) \psline(1,1)(7,1) \psline(7,1)(7,7) \psline(7,7)(1,7) \rput[tl](1.15,6.88){$D$} \rput[tl](1.15,1.5){$A$} \rput[tl](6.6,1.5){$B$} \rput[tl](6.6,6.88){$C$} \rput[tl](2.91,1.5){$E$} \rput[tl](0.53,0.88){$(0,0)$} \rput[tl](6.71,0.88){$(12,0)$} \rput[tl](2.73,0.88){$(4,0)$} \rput[tl](0.55,7.5){$(0,12)$} \rput[tl](6.73,7.5){$(12,12)$} \rput[tl](3.1,3.3){$X$} \rput[tl](4.7,3.3){$Y$} \rput[tl](3.14,4.9){$T$} \rput[tl](4.7,4.87){$Z$} \rput[tl](2.73,2.92){$(4,4)$} \rput[tl](4.8,2.93){$(8,4)$} \rput[tl](4.8,5.5){$(8,8)$} \rput[tl](2.73,5.5){$(4,8)$} \psline(3,5)(3,3) \psline(3,3)(5,3) \psline(5,3)(5,5) \psline(5,5)(3,5) \end{pspicture*}
\par\end{center}


\vspace{-0.5cm}
The above diagram is a plan of a prison compound. The outer square
$ABCD$ represents the walls of the compound (whose height may be
neglected), while the inner square $XYZT$ is the Black Tower, a solid
stone structure. A guard patrols along segment $AE$ of the walls,
for a distance of up to 4 units from $A$. Determine the distance
from $A$ of points at which the area of the courtyard that he can
see is
\begin{itemize}
\setlength{\itemsep}{3mm}
\item[\bf (i)]  as small as possible, 
\item[\bf (ii)] as large as possible. 
\end{itemize}

[\textbf{Hint. }It is suggested that you express the area he \textit{cannot
}see in terms of $p$, his distance from $A$.]
\end{question}


%%%%%%%%% Q5
\begin{question}
A set of $n$ distinct vectors $\mathbf{a}_{1},\mathbf{a}_{2},\ldots,\mathbf{a}_{n},$
where $n\geqslant2$, is called \textit{regular }if it satisfies the
following two conditions: 

\begin{questionparts}
\item there are constants $\alpha$ and $\beta$, with $\alpha>0$, such
that for any $i$ and $j$, 
\[
\mathbf{a}_{i}\cdot\mathbf{a}_{j}=\begin{cases}
\alpha^{2} & \mbox{ when }i=j\\
\beta & \mbox{ when }i\neq j,
\end{cases}
\]

\item the centroid of $\mathbf{a}_{1},\mathbf{a}_{2},\ldots,\mathbf{a}_{n}$
is the origin $\mathbf{0}.$ {[}The centroid of vectors $\mathbf{b}_{1},\mathbf{b}_{2},\ldots,\mathbf{b}_{m}$
is the vector $\frac{1}{m}(\mathbf{b}_{1}+\mathbf{b}_{2}+\cdots+\mathbf{b}_{m}).${]} 
\end{questionparts}

Prove that \textbf{(i) }and \textbf{(ii) }imply that $(n-1)\beta=-\alpha^{2}.$


If $\mathbf{a}_{1}=\begin{pmatrix}1\\
0
\end{pmatrix},$ where $\mathbf{a}_{1},\mathbf{a}_{2},\ldots,\mathbf{a}_{n}$ is a
regular set of vectors in 2-dimensional space, show that either $n=2$
or $n=3$, and in each case find the other vectors in the set. 


Hence, or otherwise, find all regular sets of vectors in 3-dimensional
space for which $\mathbf{a}_{1}=\begin{pmatrix}1\\
0\\
0
\end{pmatrix}$ \phantom{ } and $\mathbf{a}_{2}$ lies in the $x$-$y$ plane. 
\end{question}
	
	%%%%%%%%% Q6
	\begin{question}
Criticise each step of the following arguments. You should correct
the arguments where necessary and possible, and say (with justification)
whether you think the conclusion are true even though the argument
is incorrect. 

\begin{questionparts}
\item The function $g$ defined by 
\[
\mathrm{g}(x)=\frac{2x^{3}+3}{x^{4}+4}
\]
satisfies $\mathrm{g}'(x)=0$ only for $x=0$ or $x=\pm1.$ Hence
the stationary values are given by $x=0$, $\mathrm{g}(x)=\frac{3}{4}$
and $x=\pm1,$ $\mathrm{g}(x)=1.$ Since $\frac{3}{4}<1,$ there is
a minimum at $x=0$ and maxima at $x=\pm1.$ Thus we must have $\frac{3}{4}\leqslant\mathrm{g}(x)\leqslant1$
for all $x$. 
\item ${\displaystyle \int(1-x)^{-3}\,\mathrm{d}x=-3(1-x)^{-4}}\quad$ and so
$\quad{\displaystyle \int_{-1}^{3}(1-x)^{-3}\,\mathrm{d}x=0.}$
\end{questionparts}
	 \end{question}
	 
	 %%%%%%%%% Q7
\begin{question}
According to the Institute of Economic Modelling Sciences, the Slakan
economy has alternate years of growth and decline, as in the following
model. The number $V$ of vloskan (the unit of currency) in the Slakan
Treasury is assumed to behave as a continuous variable, as follows.
In a year of growth it increases continuously at an annual rate $aV_{0}\left(1+(V/V_{0})\right)^{2}.$
During a year of decline, as long as there is still money in the Treasury,
the amount decreases continuously at an annual rate $bV_{0}\left(1+(V/V_{0})\right)^{2};$
but if $V$ becomes zero, it remains zero until the end of the year.
Here $a,b$ and $V_{0}$ are positive constants. A year of growth
has just begun and there are $k_{0}V_{0}$ vloskan in the Treasury,
where $0\leqslant k_{0}<a^{-1}-1$. Explain the significance of these
inequalities for the model to be remotely sensible. 


If $k_{0}$ is as above and at the end of one year there are $k_{1}V_{0}$
vloskan in the Treasury, where $k_{1}>0$, find the condition involving
$b$ which $k_{1}$ must satisfy so that there will be some vloskan
left after a further year. Under what condition (involving $a,b$
and $k_{0}$) does the model predict that unlimited growth will take
place in the third year (but not before)?

	\end{question}
	
	%%%%%%%%% Q8
	\begin{question}
\begin{questionparts}
\item{By a substitution of the form $y=k-x$ for suitable $k$, prove
that, for any function $\mathrm{f}$, 
\[
\int_{0}^{\pi}x\mathrm{f}(\sin x)\,\mathrm{d}x=\pi\int_{0}^{\frac{1}{2}\pi}\mathrm{f}(\sin x)\,\mathrm{d}x.
\]
Hence or otherwise evaluate 
\[
\int_{0}^{\pi}\frac{x}{2+\sin x}\,\mathrm{d}x.
\]
}
\item{Evaluate 
\[
\int_{0}^{1}\frac{(\sin^{-1}t)\cos\left[(\sin^{-1}t)^{2}\right]}{\sqrt{1-t^{2}}}\,\mathrm{d}t.
\]
{[}No credit will be given for numerical answers obtained by use of
a calculator.{]} } \end{questionparts}
		\end{question}
		
		
%%%%%%%%% Q9
\begin{question}
\begin{questionparts}
\item{Suppose that the real number $x$ satisfies the $n$ inequalities
\begin{alignat*}{2}
1<\  & x &  & <2\\
2<\  & x^{2} &  & <3\\
3<\  & x^{3} &  & <4\\
 & \vdots\\
n<\  & x^{n} &  & <n+1
\end{alignat*}
Prove without the use of a calculator that $n\leqslant4$. }


\item{If $n$ is an integer strictly greater than 1, by considering
how many terms there are in 
\[
\frac{1}{n+1}+\frac{1}{n+2}+\cdots+\frac{1}{n^{2}},
\]
or otherwise, show that 
\[
\frac{1}{n}+\frac{1}{n+1}+\cdots+\frac{1}{n^{2}}>1.
\]
Hence or otherwise find, with justification, an integer $N$ such
that ${\displaystyle {\displaystyle \sum_{n=1}^{N}\frac{1}{n}>10.}}$} 
\end{questionparts}
\end{question}

\newpage
\section*{Section B: \ \ \ Mechanics}


	
%%%%%%%%%% Q10
\begin{question}
 $\ $\vspace{-1cm}



\noindent \begin{center}
\psset{xunit=1.1cm,yunit=0.7cm,algebraic=true,dotstyle=o,dotsize=3pt 0,linewidth=0.5pt,arrowsize=3pt 2,arrowinset=0.25} \begin{pspicture*}(-0.32,-0.43)(11.1,8.55) \pspolygon[linewidth=0pt,linecolor=white,hatchcolor=black,fillstyle=hlines,hatchangle=45.0,hatchsep=0.11](0,0)(0,-0.2)(10,-0.2)(10,0) \psline(0,0)(10,0) \psline(0,8)(1,5.5) \psline(1,5.5)(2,4) \psline(3,3)(4,2.5) \psline(5,2.3)(6,2.5) \psline(2,4)(3,3) \psline[linestyle=dashed,dash=1pt 1pt](4,2.5)(5,2.3) \parametricplot[linestyle=dashed,dash=1pt 1pt]{5.080052976030177}{5.934024592002003}{1*5.25*cos(t)+0*5.25*sin(t)+4.11|0*5.25*cos(t)+1*5.25*sin(t)+7.4} \psline(9,5.5)(10,8) \psline(1,5.5)(1,0) \psline(2,4)(2,0) \psline(3,3)(3,0) \psline(4,2.5)(4,0) \psline(5,2.3)(5,0) \psline(6,2.51)(6,0) \psline(9,5.5)(9,0) \rput[tl](0.02,8.53){$A_0$} \rput[tl](1.03,6.06){$A_1$} \rput[tl](2.01,4.52){$A_2$} \rput[tl](2.98,3.56){$A_3$} \rput[tl](3.98,3.12){$A_n$} \rput[tl](4.8,2.9){$A_{n+1}$} \rput[tl](5.75,3.2){$A_{n+2}$} \rput[tl](8.1,6.1){$A_{2n+1}$} \rput[tl](10.13,8.19){$A_{2n+2}$} \end{pspicture*}
\par\end{center}


The above diagram represents a suspension bridge. A heavy uniform
horizontal roadway is attached by vertical struts to a light flexible
chain at points $A_{1}=(x_{1},y_{1}),$ $A_{2}=(x_{2},y_{2}),\ldots,$
$A_{2n+1}=(x_{2n+1},y_{2n+1}),$ where the coordinates are referred
to horizontal and vertically upward axes $Ox,Oy$. The chain is fixed
to external supports at points 
\[
A_{0}=(x_{0},y_{0})\quad\mbox{ and }\quad A_{2n+2}=(x_{2n+2},y_{2n+2})
\]
at the same height. The weight of the chain and struts may be neglected.
Each strut carries the same weight $w$. The horizontal spacing $h$
between $A_{i}$ and $A_{i+1}$ (for $0\leqslant i\leqslant2n+1$)
is constant. Write down equations satisfied by the tensions $T_{i}$
in the portion $A_{i-1}A_{i}$ of the chain for $1\leqslant i\leqslant n+1$.
Hence or otherwise show that 
\[
\frac{h}{y_{n}-y_{n+1}}=\frac{3h}{y_{n-1}-y_{n}}=\cdots=\frac{(2n+1)y}{y_{0}-y_{1}}.
\]
Verify that the points $A_{0},A_{1},\ldots,A_{2n+1},A_{2n+2}$ lie
on a parabola. 
	\end{question}
	
%%%%%%%%%% Q11
\begin{question}	
A piledriver consists of a weight of mass $M$ connected to a lighter
counterweight of mass $m$ by a light inextensible string passing
over a smooth light fixed pulley. By considerations of energy or otherwise,
show that if the weights are released from rest, and move vertically,
then as long as the string remains taut and no collisions occur, the
weights experience a constant acceleration of magnitude 
\[
g\left(\frac{M-m}{M+m}\right).
\]
Initially the weight is held vertically above the pile, and is released
from rest. During the subsequent motion both weights move vertically
and the only collisions are between the weight and the pile. Treating
the pile as fixed and the collisions as completely inelastic, show
that, if just before a collision the counterweight is moving with
speed $v$, then just before the next collision it will be moving
with speed $mv/\left(M+m\right)$. {[}You may assume that when the
string becomes taut, the momentum lost by one weight equals that gained
by the other.{]}


Further show that the times between successive collisions with the
pile form a geometric progression. Show that the total time before
the weight finally comes to rest is three times the time from the
start to the first impact. 
\end{question}

%%%%%%%%%% Q12

\begin{question}$\ $\vspace{-1.5cm}



\noindent \begin{center}
\psset{xunit=0.8cm,yunit=0.8cm,algebraic=true,dotstyle=o,dotsize=3pt 0,linewidth=0.5pt,arrowsize=3pt 2,arrowinset=0.25} \begin{pspicture*}(-2.4,-1.16)(12.46,5.7) \psline(0,0)(6,4) \psline(10,0)(5,5) \rput[tl](5.08,5.53){$D$} \rput[tl](5.31,4.3){$B$} \rput[tl](3.39,2.99){$2l$} \pscustom[fillcolor=black,fillstyle=solid,opacity=0]{\parametricplot{0.0}{0.5880026035475675}{1.23*cos(t)+0|1.23*sin(t)+0}\lineto(0,0)\closepath} \rput[tl](0.67,0.39){$\alpha$} \pscustom[fillcolor=black,fillstyle=solid,opacity=0]{\parametricplot{2.356194490192345}{3.141592653589793}{1.23*cos(t)+10|1.23*sin(t)+0}\lineto(10,0)\closepath} \rput[tl](9.09,0.56){$\beta$} \psline{->}(8,2)(8,1.3) \rput[tl](7.64,1.31){$Mg$} \rput[tl](9.17,1.38){$x$} \rput[tl](7.27,3.32){$x$} \rput[tl](-0.29,-0.18){$A$} \rput[tl](10.15,-0.2){$C$} \psline(-2,0)(12,0) \end{pspicture*}
\par\end{center}


The above diagram illustrates a makeshift stepladder, made from two
equal light planks $AB$ and $CD$, each of length $2l$. The plank
$AB$ is smoothly hinged to the ground at $A$ and makes an angle
of $\alpha$ with the horizontal. The other plank $CD$ has its bottom
end $C$ resting on the same horizontal ground and makes an angle
$\beta$ with the horizontal. It is pivoted smoothly to $B$ at a
point distance $2x$ from $C$. The coefficient of friction between
$CD$ and the ground is $\mu.$ A painter of mass $M$ stands on $CD$,
half between $C$ and $B$. Show that, for equilibrium to be possible,
\[
\mu\geqslant\frac{\cot\alpha\cot\beta}{2\cot\alpha+\cot\beta}.
\]
Suppose now that $B$ coincides with $D$. Show that, as $\alpha$
varies, the maximum distance from $A$ at which the painter will be
standing is 
\[
l\sqrt{\frac{1+81\mu^{2}}{1+9\mu^{2}}}.
\]
\end{question}
	
%%%%%%%%%% Q13
\begin{question}
 $\ $\vspace{-1.5cm}

\noindent \begin{center}
\psset{xunit=1.0cm,yunit=1.0cm,algebraic=true,dotstyle=o,dotsize=3pt 0,linewidth=0.5pt,arrowsize=3pt 2,arrowinset=0.25} \begin{pspicture*}(-0.08,-2.26)(6.28,6.22) \psline(0,0)(1,4) \psline(1,4)(5,4) \psline(5,4)(6,0) \psline(6,0)(0,0) \psline[linewidth=1.2pt](3,6)(3,4) \psline[linewidth=1.2pt](3,0)(3,-2) \parametricplot[linewidth=2pt]{0.0}{3.141592653589793}{1*2*cos(t)+0*2*sin(t)+3|0*2*cos(t)+1*2*sin(t)+0} \psdot[dotstyle=*](1,-0.8) \psline{->}(1,-0.6)(1,-0.1) \rput[tl](0.84,-1.04){$m$} \rput[tl](0.6,0.45){$A$} \rput[tl](2.86,2.4){$B$} \rput[tl](5.15,0.45){$C$} \psline[linewidth=1.2pt,linestyle=dashed,dash=1pt 2pt](3,4)(3,2.42) \psline[linewidth=1.2pt,linestyle=dashed,dash=1pt 2pt](3,0)(3,2) \rput[tl](1.2,-0.26){$V$} \end{pspicture*}
\par\end{center}


A heavy smooth lamina of mass $M$ is free to slide without rotation
along a straight line on a fixed smooth horizontal table. A smooth
groove $ABC$ is inscribed in the lamina, as indicated in the above
diagram. The tangents to the groove at $A$ and at $B$ are parallel
to the line. When the lamina is stationary, a particle of mass $m$
(where $m<M$) enters the groove at $A$. The particle is travelling,
with speed $V$, parallel to the line and in the plane of the lamina
and table. Calculate the speeds of the particle and of the lamina,
when the particle leaves the groove at $C$. 


Suppose now that the lamina is held fixed by a peg attached to the
line. Supposing that the groove $ABC$ is a semicircle of radius $r$,
obtain the value of the average force per unit time exerted on the
peg by the lamina between the instant that the particle enters the
groove and the instant that it leaves it.
\end{question}
	
	\newpage
\section*{Section C: \ \ \ Probability and Statistics}


%%%%%%%%%% Q14
\begin{question}
A set of $2N+1$ rods consists of one of each length $1,2,\ldots,2N,2N+1$,
where $N$ is an integer greater than 1. Three different rods are
selected from the set. Suppose their lengths are $a,b$ and $c$,
where $a>b>c$. Given that $a$ is even and fixed, show, by considering
the possible values of $b$, that the number of selections in which
a triangle can then be formed from the three rods is 
\[
1+3+5+\cdots+(a-3),
\]
where we allow only non-degenerate triangles (i.e. triangles with
non-zero area). 


Similarly obtain the number of selections in which a triangle may
be formed when $a$ takes some fixed odd value. Write down a formula
for the number of ways of forming a non-degenerate triangle and verify
it for $N=3$. 


Hence show that, if three rods are drawn at random without replacement,
then the probability that they can form a non-degenerate triangle
is 
\[
\frac{(N-1)(4N+1)}{2(4N^{2}-1)}.
\]
\end{question}

%%%%%%%%%% Q15
\begin{question}
A fair coin is thrown $n$ times. On each throw, 1 point is scored
for a head and 1 point is lost for a tail. Let $S_{n}$ be the points
total for the series of $n$ throws, i.e. $S_{n}=X_{1}+X_{2}+\cdots+X_{n},$
where 
\[
X_{j}=\begin{cases}
1 & \mbox{ if the \ensuremath{j}th throw is a head}\\
-1 & \mbox{ if the \ensuremath{j}th throw is a tail.}
\end{cases}
\]


\begin{itemize}
\setlength{\itemsep}{3mm}
\item[\bf (i)]  If $n=10\,000,$ find an approximate value for the probability that
$S_{n}>100.$
\item[\bf (ii)] Find an approximate value for the least $n$ for which $\mathrm{P}(S_{n}>0.01n)<0,01.$ 
\end{itemize}

Suppose that instead no points are scored for the first throw, but
that on each successive threw, 2 points are scored if both it and
the first throw are heads, two points are deducted if both are tails,
and no points are scored or lost if the throws differ. Let $Y_{k}$
be the score on the $k$th throw, where $2\leqslant k\leqslant n.$
Show that $Y_{k}=X_{1}+X_{k}.$


Calculate the mean and variance of each $Y_{k}$ and determine whether
it is true that 
\[
\mathrm{P}(Y_{2}+Y_{3}+\cdots+Y_{n}>0.01(n-1))\rightarrow0\quad\mbox{ as }n\rightarrow\infty.
\]

\end{question}

%%%%%%%%%% Q16
\begin{question}
At any instant the probability that it is safe to cross a busy road
is $0.1$. A toad is waiting to cross this road. Every minute she
looks at the road. If it is safe, she will cross; if it is not safe,
she will wait for a minute before attempting to cross again. Find
the probability that she eventually crosses the road without mishap. 


Later on, a frog is also trying to cross the same road. He also inspects
the traffic at one minute intervals and crosses if it is safe. Being
more impatient than the toad, he may also attempt to cross when it
is not safe. The probability that he will attempt to cross when it
is not safe is $n/3$ if $n\leqslant3,$ where $n$ minutes have elapsed
since he firrst inspected the road. If he attempts to cross when it
is not safe, he is run over with probability $0.8,$ but otherwise
he reaches the other side safely. Find the probability that he eventually
crosses the road without mishap. 


What is the probability that both reptiles safely cross the road with
the frog taking less time than the toad? If the frog has not arrived
at the other side 2 minutes after he began his attempt to cross, what
is the probability that the frog is run over (at some stage) in his
attempt to cross?


{[}Once moving, the reptiles spend a negligible time on their attempt
to cross the road.{]}
\end{question}
\end{document}
