\documentclass[a4, 11pt]{report}


\pagestyle{myheadings}
\markboth{}{Paper II, 2011
\ \ \ \ \ 
\today 
}               

\RequirePackage{amssymb}
\RequirePackage{amsmath}
\RequirePackage{graphicx}
\RequirePackage{color}
\RequirePackage[flushleft]{paralist}[2013/06/09]



\RequirePackage{geometry}
\geometry{%
  a4paper,
  lmargin=2cm,
  rmargin=2.5cm,
  tmargin=3.5cm,
  bmargin=2.5cm,
  footskip=12pt,
  headheight=24pt}


\newcommand{\comment}[1]{{\bf Comment} {\it #1}}
%\renewcommand{\comment}[1]{}

\newcommand{\bluecomment}[1]{{\color{blue}#1}}
%\renewcommand{\comment}[1]{}
\newcommand{\redcomment}[1]{{\color{red}#1}}



\usepackage{epsfig}
\usepackage{pstricks-add}

\usepackage{tgheros} %% changes sans-serif font to TeX Gyre Heros (tex-gyre)
\renewcommand{\familydefault}{\sfdefault} %% changes font to sans-serif
%\usepackage{sfmath}  %%%% this makes equation sans-serif

%\input RexFigs


\setlength{\parskip}{10pt}
\setlength{\parindent}{0pt}

\newlength{\qspace}
\setlength{\qspace}{20pt}


\newcounter{qnumber}
\setcounter{qnumber}{0}

\newenvironment{question}%
 {\vspace{\qspace}
  \begin{enumerate}[\bfseries 1\quad][10]%
    \setcounter{enumi}{\value{qnumber}}%
    \item%
 }
{
  \end{enumerate}
  \filbreak
  \stepcounter{qnumber}
 }


\newenvironment{questionparts}[1][1]%
 {
  \begin{enumerate}[\bfseries (i)]%
    \setcounter{enumii}{#1}
    \addtocounter{enumii}{-1}
    \setlength{\itemsep}{5mm}
    \setlength{\parskip}{8pt}
 }
 {
  \end{enumerate}
 }



\DeclareMathOperator{\cosec}{cosec}
\DeclareMathOperator{\Var}{Var}

\def\d{{\mathrm d}}
\def\e{{\mathrm e}}
\def\g{{\mathrm g}}
\def\h{{\mathrm h}}
\def\f{{\mathrm f}}
\def\p{{\mathrm p}}
\def\s{{\mathrm s}}
\def\t{{\mathrm t}}


\def\A{{\mathrm A}}
\def\B{{\mathrm B}}
\def\E{{\mathrm E}}
\def\F{{\mathrm F}}
\def\G{{\mathrm G}}
\def\H{{\mathrm H}}
\def\P{{\mathrm P}}


\def\bb{\mathbf b}
\def \bc{\mathbf c}
\def\bx {\mathbf x}
\def\bn {\mathbf n}

\newcommand{\low}{^{\vphantom{()}}}
%%%%% to lower suffices: $X\low_1$ etc


\newcommand{\subone}{ {\vphantom{\dot A}1}}
\newcommand{\subtwo}{ {\vphantom{\dot A}2}}




\def\le{\leqslant}
\def\ge{\geqslant}
\def\arcosh{{\rm arcosh}\,}


\def\var{{\rm Var}\,}

\newcommand{\ds}{\displaystyle}
\newcommand{\ts}{\textstyle}
\def\half{{\textstyle \frac12}}
\def\l{\left(}
\def\r{\right)}



\begin{document}
\setcounter{page}{2}

 
\section*{Section A: \ \ \ Pure Mathematics}

%%%%%%%%%%Q1
\begin{question}
\begin{questionparts}
\item
Sketch the curve  $y=\sqrt{1-x} + \sqrt{3+x}\;$.

Use your sketch to show that only one  real value of $x$  satisfies
\[
\sqrt{1-x} + \sqrt{3+x} = x+1\,,
\]
and give this value.

\item
Determine graphically the number of real values of $x$ that satisfy
\[
2\sqrt{1-x} = \sqrt{3+x} + \sqrt{3-x}\;.
\]
Solve this equation.         
\end{questionparts}
\end{question}

%%%%%%%%%%Q2
\begin{question}
Write down the cubes of the integers 1, 2, $\ldots$ , 10\,.

The positive integers $x$, $y$ and $z$, where $x<y$, satisfy
\[
x^3+y^3 = kz^3\,,
\tag{$*$}
\]
where $k$ is a given positive integer.

\begin{questionparts}
\item
In the case  $x+y =k$, show that 
\[
z^3 = k^2  -3kx+3x^2\,.
\]
Deduce that $(4z^3 - k^2)/3$ is a perfect square
and that $\frac14 {k^2} \le z^3 < k^2\,$.

Use these results to  find a solution of $(*)$ when $k=20$.

\item By considering the case $x+y = z^2$, 
find two solutions of $(*)$ when  
$k=19$. 

\end{questionparts}
\end{question}

%%%%%%%%% Q3
\begin{question}
In this question, you may assume without proof that 
 any function $\f$ for which  $\f'(x)\ge 0$ is {\em 
 increasing}; that is,
$\f(x_2)\ge \f(x_1)$ if $x_2\ge x_1\,$. 

\begin{questionparts}
\item 
\begin{itemize}
\item[{\bf (a)}]
 Let $\f(x) =\sin x -x\cos x$.
 Show that $\f(x)$ is 
increasing 
for $0\le x \le \frac12\pi\,$
and deduce that $\f(x)\ge 0\,$ for 
$0\le x \le \frac12\pi\,$.
\\
\item[{\bf (b)}] 
Given that $\dfrac{\d}{\d x} (\arcsin x)  \ge1$ for 
$0\le x< 1$,
show that 
\[
\hspace {1.1cm}
\arcsin x\ge  x
\hspace*{3.8cm}(0\le x <   1).
\]
\item[{\bf (c)}] Let $\g(x)= x\cosec x\,$ for $0<x<\frac12\pi$. Show that
$\g$ is increasing and deduce that
\[
\hspace{1cm}
({\arcsin x})\,  x^{-1} \ge    x\,{\cosec x}
\hspace{1.7cm} (0<x<1). 
\]
\end{itemize}
 \item
Given that $\dfrac{\d}{\d x}
 (\arctan x)\le 1$ for $x\ge 0$, show
by considering the function $x^{-1} \tan x$ 
that 
\[
\hspace{2cm}(\tan x)( \arctan x) \ge x^2
\hspace{2.3cm}
(0<x<\tfrac12\pi).
\]
\end{questionparts}
\end{question}

%%%%%% Q4 
\begin{question}
\begin{questionparts}
\item
 Find all the values of $\theta$, in the range $0^\circ <\theta<180^\circ$,
for which $\cos\theta=\sin 4\theta$. Hence show that
\[
\sin 18^\circ = \frac14\left( \sqrt 5 -1\right).
\]

\item Given that 
\[
4\sin^2 x + 1 = 4\sin^2 2x
\,,
\]
find
  all possible values of $\sin x\,$, giving your answers in the 
form $p+q\sqrt5$ where $p$ and $q$ are rational numbers.

\item
Hence find two values of  $\alpha$ with $0^\circ <\alpha<90^\circ$
for which
\[
\sin^23\alpha + \sin^25\alpha = \sin^2 6\alpha\,.
\]
\end{questionparts}
	\end{question}
	
%%%%%%%%% Q5
\begin{question}
The points  $A$ and $B$ have position vectors $\bf a $ and $\bf b$ 
with respect to an origin $O$, and $O$, $A$~and~$B$ are non-collinear.
The point $C$, with position vector $\bf c$,
 is the reflection of $B$ in the line through
$O$ and $A$. Show that $\bf c$ can be written in the 
form 
\[
\bf c = \lambda \bf a -\bf b
\]
where  $\displaystyle \lambda = \frac{2\,{\bf a .b}}{{\bf a.a}}$.

The point $D$, with position  vector $\bf d$, is the reflection of $C$ in 
the line through $O$ and $B$.
Show that  $\bf d$ can be written in the form
\[
\bf d = \mu\bf b - \lambda \bf a
\]
for some scalar $\mu$ to be determined. 

Given that $A$, $B$ and $D$ are collinear, find the relationship
between $\lambda$ and $\mu$. In the case $\lambda = -\frac12$, determine
the cosine of $\angle AOB$ and describe the relative positions
of $A$, $B$ and $D$.
\end{question}
	
%%%%%%%%% Q6
\begin{question}
For any given function $\f$, let
\[
I = \int [\f'(x)]^2 \,[\f(x)]^n \d x\,,
\tag{$*$}
\]
where $n$ is a positive integer.
Show that, if $\f(x)$ satisfies $\f''(x) =k \f(x)\f'(x)$ for some constant
$k$, then ($*$) can be integrated to obtain an expression 
for $I$ in terms of $\f(x)$, $\f'(x)$, $k$ and $n$.  

\begin{questionparts}
\item 
Verify your result in the case $\f(x) = \tan x\,$.
Hence find
\[
\displaystyle \int \frac{\sin^4x}{\cos^{8}x} \, \d x\;.
\]
\item Find
 \[
\displaystyle \int \sec^2x\, (\sec x + \tan x)^6\,\d x\;.
\]
\end{questionparts}
\end{question}
		
%%%%%%%%% Q7
\begin{question}
The two sequences $a_0$, $a_1$, $a_2$, $\ldots$
and   
 $b_0$, $b_1$, $b_2$, $\ldots$
have general terms
\[
a_n = \lambda^n +\mu^n
\text { \ \ \ and \ \ \ }
b_n = \lambda^n - \mu^n\,,
\]
respectively, where $\lambda = 1+\sqrt2$ and $\mu= 1-\sqrt2\,$.


\begin{questionparts}
\item Show that $\displaystyle \sum_{r=0}^nb_r = -\sqrt2 + \frac
1 {\sqrt2} \,a_{\low n+1}\,$,
and give a corresponding result for
 $\displaystyle \sum_{r=0}^na_r\,$.
\item Show that, if $n$ is odd,
 $$\sum_{m=0}^{2n}\left( \sum_{r=0}^m a_{\low r}\right) 
= \tfrac12 b_{n+1}^2\,,$$ 
 and give a corresponding result when $n$
is even.
\item Show that, if $n$ is even,
 $$\left(\sum_{r=0}^na_r\right)^{\!2}
 -\sum_{r=0}^n a_{\low 2r+1} =2\,,$$
and give a corresponding result when
$n$ is odd.
\end{questionparts}
\end{question}	

%%%%%%%%% Q8
\begin{question}
The end $A$ of an inextensible string $AB$ of length $\pi$
is attached to a point  on the circumference 
of a fixed circle of unit radius and 
centre $O$. Initially the string is straight
and tangent to the circle. The string is then wrapped round the circle
until the end $B$ comes into
contact with the circle. 
The string remains taut during the motion,
so that a section of the string is in contact with the circumference
and the remaining section is straight. 

Taking $O$ to be the origin of cartesian coordinates with $A$ at $(-1,0)$
and $B$ initially at $(-1, \pi)$, show that the
curve described by $B$  is given parametrically by
\[
x= \cos t + t\sin t\,, \ \ \ \ \ \
y=  \sin t - t\cos t\,,
\]
where $t$ is the angle shown in the diagram.

\begin{center}
\psset{xunit=0.8cm,yunit=0.8cm,algebraic=true,dimen=middle,dotstyle=o,dotsize=3pt 0,linewidth=0.3pt,arrowsize=3pt 2,arrowinset=0.25}
\begin{pspicture*}(-5.4,-1)(7,7)
\pspolygon(-1.22,3.03)(-0.87,3.17)(-1.01,3.52)(-1.36,3.38)
\parametricplot{-0.17}{3.3}{1*3.64*cos(t)+0*3.64*sin(t)+0|0*3.64*cos(t)+1*3.64*sin(t)+0}
\psline(-1.36,3.38)(6.23,6.37)
\psline[linestyle=dashed,dash=1pt 1pt](0,0)(-1.36,3.38)
\parametricplot{-0.0}{1.9540453733056695}{1.06*cos(t)+0|1.03*sin(t)+0}
\rput[tl](-0.45,-0.1){$O$}
\rput[tl](-4.12,0.46){$A$}
\rput[tl](6.11,6.8){$B$}
\rput[tl](0.25,0.6){$t$}
\psline{->}(-7.22,0)(5.78,0)
\psline{->}(0,-1.53)(0,6)
\rput[tl](-0.08,6.45){$y$}
\rput[tl](5.85,0.1){$x$}
\end{pspicture*}
\end{center}

Find the value, $t_0$, of $t$ for which $x$
takes its maximum value on the curve,
and sketch the curve.

Use   the area integral $\displaystyle \int y \frac{\d x}{\d t} \,
\d t\,$
to find the area between the curve and the $x$ axis 
for~\hbox{$\pi \ge t \ge t_0$}.

Find the area swept out by the string (that is, the area between the 
curve described by 
$B$ and the semicircle shown in the diagram).
\end{question}	
		

		
	
\newpage
\section*{Section B: \ \ \ Mechanics}


	
%%%%%%%%%% Q9
\begin{question}
Two particles, $A$ of mass $2m$ and $B$ of mass $m$, are moving towards
each  other in   a straight line on a smooth horizontal plane, with speeds
$2u$ and $u$ respectively. They collide directly. Given that the
coefficient of restitution between the particles is $e$, where 
$0<e\le1$, determine the speeds of the particles after the collision.

After the collision, $B$ collides directly with a smooth vertical wall,
rebounding and then colliding directly with $A$ for a second time.
The coefficient of restitution between $B$ and the wall is $f$, where
$0<f\le1$.
Show that the velocity of $B$ after  
its second collision with $A$ is
\[
\tfrac23 (1-e^2)u - \tfrac13(1-4e^2)fu
\]
towards the wall
and that 
$B$ moves towards (not away from) the wall for all values of $e$~and~$f$.
	\end{question}
	
%%%%%%%%%% Q10 
\begin{question}	
A particle is projected from a point on a 
horizontal plane, at speed
$u$ and at an angle~$\theta$ above the horizontal.
Let $H$ be 
the maximum height of the particle above the plane.
 Derive an expression
for $H$ in terms of $u$, $g$ and 
$\theta$.

A particle $P$ is projected from a point $O$ on a  smooth
horizontal plane,
at speed $u$ and at an angle~$\theta$ above the horizontal. At the 
same instant, a second particle $R$ is projected horizontally from $O$
in such a way that $R$ is vertically below $P$ in the ensuing motion. 
A light inextensible string of length $\frac12 H$ connects
$P$ and $R$. Show that the time that elapses
before the string becomes taut is 
\[
(\sqrt2 -1)\sqrt{H/g\,}\,.
\]

When the string becomes taut, $R$ leaves the plane,  the string
remaining taut. Given that $P$ and $R$ have equal masses, 
determine 
the total horizontal distance, $D$, travelled by $R$
from the moment its motion begins  to the moment it lands on the plane
again, giving your answer
in terms of $u$, $g$ and $\theta$.

Given that $D=H$, find the value of $\tan\theta$.
\end{question}

%%%%%%%%%% Q11

\begin{question}
Three non-collinear points $A$, $B$ and $C$ lie in
a horizontal ceiling. A particle $P$ of weight $W$
is suspended from this ceiling by means of three
light inextensible strings $AP$, $BP$ and $CP$,
as shown in the diagram. The point $O$ lies
vertically above $P$ in the ceiling.


\begin{center}
\psset{xunit=1.0cm,yunit=1.0cm,algebraic=true,dimen=middle,dotstyle=o,dotsize=3pt 0,linewidth=0.3pt,arrowsize=3pt 2,arrowinset=0.25}
\begin{pspicture*}(-1.15,-1)(10.53,4.67)
\psline(-1,2)(6,2)
\psline(-1,2)(3.5,4.5)
\psline(6,2)(10.5,4.5)
\psline(10.5,4.5)(3.5,4.5)
\psline[linestyle=dashed,dash=1pt 2.5pt](3.99,3.19)(4,-0.26)
\psline[linewidth=1.2pt](4,-0.26)(2.06,2.05)
\psline[linewidth=1.2pt](4,-0.26)(3.41,2)
\psline[linewidth=1.2pt](4,-0.26)(5.78,2)
\psline[linewidth=1pt,linestyle=dashed,dash=2pt 2.5pt](5.78,2)(7.02,3.66)
\psline[linewidth=1pt,linestyle=dashed,dash=2pt 2.5pt](3.41,2)(2.99,3.62)
\psline[linewidth=1pt,linestyle=dashed,dash=2pt 2.5pt](2.06,2.05)(1.56,2.64)
\rput[tl](3.87,3.6){$O$}
\rput[tl](7.05,4.05){$C$}
\rput[tl](2.81,4){$A$}
\rput[tl](1.28,3){$B$}
\rput[tl](3.85,-0.65){$P$}
\begin{scriptsize}
\psdots[dotsize=13pt 0,dotstyle=*](4,-0.26)
\end{scriptsize}
\end{pspicture*}
\end{center}

The angles $AOB$ and $AOC$ are $90^\circ+\theta$
and $90^\circ+\phi$, respectively, where $\theta$ and $\phi$
are acute angles such that $\tan\theta = \sqrt2$ and 
$\tan\phi =\frac14\sqrt2$. 

The strings $AP$, $BP$ and $CP$ make angles $30^\circ$, $90^\circ-\theta$
and $60^\circ$, respectively, with the vertical, and the tensions
in these strings have magnitudes $T$, $U$ and $V$ respectively.

\begin{questionparts}
\item
Show that the unit vector in the direction $PB$ can be written
in the form
\[
 -\frac13\, {\bf i} - \frac{\sqrt2\,}3\, {\bf j} + 
\frac{\sqrt2\, }{\sqrt3 \,} \,{\bf k}
\,,\]
where $\bf i\,$, $\, \bf j$ and $\bf k$ are the usual mutually perpendicular
unit vectors 
with $\bf j$ parallel to $OA$ and $\bf k$ vertically upwards.

\item
Find expressions in vector form for the forces acting on $P$.

\item
Show  that $U=\sqrt6 V$ and find $T$, $U$ and $V$ in terms of $W$.
\end{questionparts}
\end{question}
	

	
	\newpage
\section*{Section C: \ \ \ Probability and Statistics}


%%%%%%%%%% Q12
\begin{question}
Xavier and Younis are playing a match.
The match consists of a series of games and 
 each game consists of three points.


Xavier has
probability $p$ 
and Younis has probability $1-p$
of winning the first point of any game.
In the second and third points of each game,
the player who won the previous point
 has
probability $p$ 
and the player who lost the previous point
 has probability $1-p$
of winning the  point.
If a player wins two consecutive points in a single game,
the match ends and that player has won; otherwise the 
match continues with another game.


\begin{questionparts}
\item Let $w$ be the probability that Younis wins the match.
 Show that, for $p\ne0$, 
\[
w = \frac{1-p^2}{2-p}.
\]
Show that $w>\frac12$ if $p<\frac12$, and $w<\frac12$ if $p>\frac12$. Does $w$ increase whenever 
$p$ decreases? 

\item If Xavier wins the match, Younis gives him $\pounds1$; if 
Younis wins the match, Xavier gives him $\pounds k$.
      Find the value of $k$ for which the game is `fair' 
in the case when  $p =\frac23$.

\item What happens when  $p = 0$?
\end{questionparts}
\end{question}

%%%%%%%%%% Q13
\begin{question}
What property of a distribution is measured by its 
{\em skewness}?

\begin{questionparts}
\item 
One measure of  skewness, $\gamma$,  is given by 
\[
\displaystyle
\gamma=
\frac{ \E\big((X-\mu)^3\big)}{\sigma^3}\,,
\] 
where $\mu$ and $\sigma^2$ are the mean and variance of the random
variable  $X$.
Show that 
\[
\gamma = \frac{ \E(X^3) -3\mu \sigma^2 - \mu^3}{\sigma^3}\,.
\]

The continuous random variable $X$
has probability density function $\f$ where 
\[
\f(x) 
= \begin{cases}
2x & \text{for } 0\le x\le 1\,, \\[2mm]
0 & \text{otherwise}\,.
\end{cases}
\]
Show that for this distribution $\gamma= -\dfrac{2\sqrt2}{5}$.

\item The {\em decile skewness}, $D$,  of a distribution is defined by
\[D=
\frac { {\rm F}^{-1}(\frac9{10}) - 2{\rm F} ^{-1}(\frac12) + {\rm F}^{-1} (\frac1{10}) } 
{{\rm F}^{-1}(\frac9{10}) - {\rm F} ^{-1} (\frac1{10})}\,,
\]
where ${\rm F}^{-1}$ is the inverse of the 
cumulative distribution function.
Show that, for the above distribution,
$ 
D= 2 -\sqrt5\,.
$ 

 
 The {\em Pearson skewness},~$P$, of a distribution is defined by
\[
P = \frac{3(\mu-M)}{\sigma}
\,,\]
where $M$ is the median.
Find $P$ for the above distribution and show 
that
  $D>P>\gamma\,$.

\end{questionparts}
\end{question}

\end{document}
