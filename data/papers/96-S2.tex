\documentclass[a4, 11pt]{report}


\pagestyle{myheadings}
\markboth{}{Paper II, 1996
\ \ \ \ \ 
\today 
}               

\RequirePackage{amssymb}
\RequirePackage{amsmath}
\RequirePackage{graphicx}
\RequirePackage{color}
\RequirePackage[flushleft]{paralist}[2013/06/09]



\RequirePackage{geometry}
\geometry{%
  a4paper,
  lmargin=2cm,
  rmargin=2.5cm,
  tmargin=3.5cm,
  bmargin=2.5cm,
  footskip=12pt,
  headheight=24pt}


\newcommand{\comment}[1]{{\bf Comment} {\it #1}}
%\renewcommand{\comment}[1]{}

\newcommand{\bluecomment}[1]{{\color{blue}#1}}
%\renewcommand{\comment}[1]{}
\newcommand{\redcomment}[1]{{\color{red}#1}}



\usepackage{epsfig}
\usepackage{pstricks-add}
\usepackage{tgheros} %% changes sans-serif font to TeX Gyre Heros (tex-gyre)
\renewcommand{\familydefault}{\sfdefault} %% changes font to sans-serif
%\usepackage{sfmath}  %%%% this makes equation sans-serif
%\input RexFigs


\setlength{\parskip}{10pt}
\setlength{\parindent}{0pt}

\newlength{\qspace}
\setlength{\qspace}{20pt}


\newcounter{qnumber}
\setcounter{qnumber}{0}

\newenvironment{question}%
 {\vspace{\qspace}
  \begin{enumerate}[\bfseries 1\quad][10]%
    \setcounter{enumi}{\value{qnumber}}%
    \item%
 }
{
  \end{enumerate}
  \filbreak
  \stepcounter{qnumber}
 }


\newenvironment{questionparts}[1][1]%
 {
  \begin{enumerate}[\bfseries (i)]%
    \setcounter{enumii}{#1}
    \addtocounter{enumii}{-1}
    \setlength{\itemsep}{5mm}
    \setlength{\parskip}{8pt}
 }
 {
  \end{enumerate}
 }



\DeclareMathOperator{\cosec}{cosec}
\DeclareMathOperator{\Var}{Var}

\def\d{{\rm d}}
\def\e{{\rm e}}
\def\g{{\rm g}}
\def\h{{\rm h}}
\def\f{{\rm f}}
\def\p{{\rm p}}
\def\s{{\rm s}}
\def\t{{\rm t}}


\def\A{{\rm A}}
\def\B{{\rm B}}
\def\E{{\rm E}}
\def\F{{\rm F}}
\def\G{{\rm G}}
\def\H{{\rm H}}
\def\P{{\rm P}}


\def\bb{\mathbf b}
\def \bc{\mathbf c}
\def\bx {\mathbf x}
\def\bn {\mathbf n}

\newcommand{\low}{^{\vphantom{()}}}
%%%%% to lower suffices: $X\low_1$ etc


\newcommand{\subone}{ {\vphantom{\dot A}1}}
\newcommand{\subtwo}{ {\vphantom{\dot A}2}}




\def\le{\leqslant}
\def\ge{\geqslant}


\def\var{{\rm Var}\,}

\newcommand{\ds}{\displaystyle}
\newcommand{\ts}{\textstyle}




\begin{document}
\setcounter{page}{2}

 
\section*{Section A: \ \ \ Pure Mathematics}

%%%%%%%%%%Q1
\begin{question}
\begin{questionparts} 
\item Find the
coefficient of $x^{6}$ in
\[(1-2x+3x^{2}-4x^{3}+5x^{4})^{3}.\]
You should set out your working clearly.
 
\item By considering the binomial expansions of $(1+x)^{-2}$
and $(1+x)^{-6}$, or otherwise, find the
coefficient of $x^{6}$ in
\[(1-2x+3x^{2}-4x^{3}+5x^{4}-6x^{5}+7x^{6})^{3}.\]
\end{questionparts}
\end{question}

%%%%%%%%%%Q2
\begin{question}
Consider the system of equations
\begin{alignat*}{1}
2yz+zx-5xy & =2\\
yz-zx+2xy & =1\\
yz-2zx+6xy & =3
\end{alignat*}             
Show that 
\[xyz=\pm 6\] and find
the possible values of $x$, $y$ and $z$.
\end{question}

%%%%%%%%% Q3
\begin{question}
 The Fibonacci numbers $F_{n}$ are defined by the conditions
$F_{0}=0$, $F_{1}=1$ and
\[F_{n+1}=F_{n}+F_{n-1}\]
for all $n\geqslant 1$. Show that $F_{2}=1$,
$F_{3}=2$, $F_{4}=3$ and compute
$F_{5}$, $F_{6}$ and~$F_{7}$.

Compute $F_{n+1}F_{n-1}-F_{n}^{2}$ for a few values of $n$; guess
a general formula and prove it by induction, or otherwise.

By induction on $k$, or otherwise, show that
\[F_{n+k}=F_{k}F_{n+1}+F_{k-1}F_{n}\]
for all positive integers $n$ and $k$.
\end{question}

%%%%%% Q4 
\begin{question}
Show that $\cos 4u=8\cos^{4}u-8\cos^{2}u+1$.

If
\[
I=\int_{-1}^{1}
\frac{1}{\vphantom{{\big(}^2}\; 
\surd(1+x)+\surd(1-x)+2\;
}\;{\rm d}x  ,\]
show, by using the change of variable $x=\cos t$, that
\[
I= \int_0^\pi
\frac{\sin t}{4\cos^{2}\left(\frac{t}{4}-\frac{\pi}{8}\right)}\,{\rm
d}t.\]

By using the further change of variable $u=\frac{t}{4}-\frac{\pi}{8}$,
or otherwise, show that
\[I=4\surd{2}-\pi-2.\]

\noindent[You may assume that $\tan\frac{\pi}{8}=\surd{2}-1$.]
	\end{question}

%%%%%%%%% Q5
\begin{question}
If
$$                 
z^{4}+z^{3}+z^{2}+z+1=0\eqno(*) 
$$               
and $u=z+z^{-1}$,
find the possible values of $u$. Hence find the possible
values of $z$. [Do not try to simplify your answers.]

Show that, if $z$ satisfies $(*)$, then
\[z^{5}-1=0.\]
Hence write the solutions of $(*)$ in the form
$z=r(\cos\theta+i\sin\theta)$ for suitable real $r$ and $\theta$.
Deduce that
\[\sin\frac{2\pi}{5}=\frac{\surd(10+2\surd 5)}{4}
\ \ \hbox{and}\ \ \cos\frac{2\pi}{5}=\frac{-1+\surd 5}{4}.\]
	\end{question}
	
%%%%%%%%% Q6
\begin{question}
A {\sl proper factor} of a positive integer $N$
is an integer $M$, with $M\ne 1$ and $M\ne N$, which divides $N$ without remainder.
Show that $12$ has $4$ proper factors and $16$ has $3$.

Suppose that $N$ has the prime factorisation
\[N=p_{1}^{m_{1}}p_{2}^{m_{2}}\dots p_{r}^{m_{r}},\]
where  $p_{1}$, $p_{2}$, \dots, $p_{r}$ are distinct 
primes and
$m_{1}$, $m_{2}$, \dots, $m_{r}$ are positive integers.
How many proper factors does $N$ have and why?

Find:

\begin{questionparts} 
\item the smallest positive integer which has
precisely 12 proper factors;

\item the smallest positive integer which has
at least 12 proper factors.
\end{questionparts}
\end{question}
	
%%%%%%%%% Q7
\begin{question}
Consider a fixed square $ABCD$ and a variable
point $P$ in the plane of the square. We write the perpendicular
distance from $P$ to $AB$ as $p$, from $P$ to $BC$ as $q$,
from $P$ to $CD$ as $r$ and from $P$ to $DA$ as $s$.
(Remember that distance is never negative, so $p,q,r,s\geqslant 0$.)
If $pr=qs$, show that the only possible positions of $P$
lie on two straight lines and a circle and that every point
on these two lines and a circle is indeed a possible position
of $P$.
\end{question}
		
%%%%%%%%% Q8
\begin{question}	
Suppose that
\[{\rm f}''(x)+{\rm f}(-x)=x+3\cos 2x\]
and ${\rm f}(0)=1$, ${\rm f}'(0)=-1$.
If ${\rm g}(x)={\rm f}(x)+{\rm f}(-x)$,  find ${\rm g}(0)$
and show that ${\rm g}'(0)=0$.
Show that
\[{\rm g}''(x)+{\rm g}(x)=6\cos 2x,\]
and hence find ${\rm g}(x)$.

Similarly, if
${\rm h}(x)={\rm f}(x)-{\rm f}(-x)$, find ${\rm h}(x)$
and show that
\[{\rm f}(x)=2\cos x -\cos2x-x.\]
\end{question}	
		

		
	
\newpage
\section*{Section B: \ \ \ Mechanics}


	
%%%%%%%%%% Q9
\begin{question}
A child's toy consists of a  solid cone
of height $\lambda a$ and a 
solid hemisphere
of radius $a$,
made out of the same uniform material and fastened together
so that their plane faces coincide. (Thus the diameter
of the hemisphere is equal to that of the base of the cone.)
Show that if 
$\lambda<\sqrt{3}$ the toy will always move to an upright
position if placed with the surface of the hemisphere on a
horizontal table, but that if
$\lambda>\sqrt{3}$ the toy may overbalance.

Show, however, that if the toy is placed with the surface of the cone
touching the table it will remain there whatever the value of $\lambda$.

\noindent[The centre of gravity of a uniform 
solid cone of height $h$
is a height $h/4$ above its base. The centre of gravity of
a uniform solid hemisphere of radius $a$ is at distance $3a/8$
from the centre of its base.] 
	\end{question}
	
%%%%%%%%%% Q10
\begin{question}	
The plot of `Rhode Island Red and the Henhouse of Doom' calls
for the heroine to cling on to the circumference of a
fairground wheel of radius $a$ rotating with constant angular
velocity $\omega$ about its horizontal axis
and then let go. Let $\omega_{0}$ be the largest value
of $\omega$ for which it is not possible for her subsequent
path to carry her higher than the top
of the wheel. Find $\omega_{0}$ in terms of $a$ and $g$.

If $\omega>\omega_{0}$ show that the greatest height
above          the top of the wheel to which she
can rise is
\[\frac{a}{2}\left(\frac{\omega}{\omega_{0}}
-\frac{\omega_{0}}{\omega}\right)^{\!\!2}.\]
\end{question}

%%%%%%%%%% Q11

\begin{question}
A particle hangs in equilibrium from the ceiling
of a stationary lift, to which it is attached by an elastic
string of natural length $l$ extended to a length $l+a$.
The lift now descends with constant acceleration $f$
such that $0<f<g/2$. Show that the extension $y$ of the
string from its equilibrium length satisfies the 
differential equation 
$$
{{\rm  d}^2 y \over {\rm d} t^2}  +{g \over a}\, y = g-f.
$$
Hence show that 
the string never becomes slack and the
amplitude of the oscillation of the particle is $af/g$.

After a time $T$ the lift stops accelerating and moves with constant
velocity. Show that the string never becomes slack
and the amplitude of the oscillation is now
\[\frac{2af}{g}|\sin {\textstyle \frac{1}{2}}\omega T|,\]
where $\omega^{2}=g/a$.
\end{question}
	

	
	\newpage
\section*{Section C: \ \ \ Probability and Statistics}


%%%%%%%%%% Q12
\begin{question}
\begin{questionparts} 
\item Let $X_{1}$, $X_{2}$, \dots, $X_{n}$
be independent
random variables each of which is
uniformly distributed on $[0,1]$.
Let $Y$ be the
largest of $X_{1}$, $X_{2}$, \dots, $X_{n}$. By using the fact
that $Y<\lambda$ if and only if $X_{j}<\lambda$ for
$1\leqslant j\leqslant n$, find the probability density function of $Y$.
Show that the variance of $Y$
is 
\[\frac{n}{(n+2)(n+1)^{2}}.\]

\item 
The probability that a neon light switched on at time $0$ will have failed
by a time $t>0$ is $1-\mathrm{e}^{-t/\lambda}$ where $\lambda>0$. I switch
on $n$ independent neon lights at time zero. Show that the expected time
until the first failure is $\lambda/n$.
\end{questionparts}
\end{question}

%%%%%%%%%% Q13
\begin{question}
By considering the coefficients of $t^{n}$
in the equation
\[(1+t)^{n}(1+t)^{n}=(1+t)^{2n},\]
or otherwise,
show that
\[\binom{n}{0}\binom{n}{n}+\binom{n}{1}\binom{n}{n-1}+\cdots
+\binom{n}{r}\binom{n}{n-r}+\cdots+\binom{n}{n}\binom{n}{0}
=\binom{2n}{n}.\]

The large American city of Triposville is laid out in
a square grid with equally
spaced streets running east-west and avenues
running north-south. My friend is staying
at a hotel $n$ avenues west and $n$  streets north
of my hotel. Both hotels are at intersections.
We set out from our own hotels at the same
time. We walk at the same speed, taking 1 minute to go from
one intersection to the next. Every time I reach an intersection 
I go north with probability $1/2$ or west with probability
$1/2$. Every time my friend reaches an intersection 
she goes south with probability $1/2$ or east with probability
$1/2$. Our choices are independent of each other
and of our previous decisions.
Indicate by a sketch or by a brief description
the set of points where we could meet. Find the
probability that we meet.

Suppose that I oversleep and leave my hotel $2k$ minutes
later than my friend leaves hers, 
where $k$ is an integer and $0\leqslant 2k\leqslant n$. 
Find the probability
that we meet. Have you any comment?
If $n=1$ and I leave my hotel $1$ minute later than my friend
leaves hers, what is the probability that we meet and why?
\end{question}

%%%%%%%%%% Q14
\begin{question}
The random variable $X$ is
uniformly distributed on $[0,1]$. A new random variable
$Y$ is defined by the rule 
\[
Y=\begin{cases}
1/4 & \mbox{ if }X\leqslant1/4,\\
X & \mbox{ if }1/4\leqslant X\leqslant3/4\\
3/4 & \mbox{ if }X\geqslant3/4.
\end{cases}
\]
Find ${\mathrm E}(Y^{n})$ for all integers $n\geqslant 1$.

Show that ${\mathrm E}(Y)={\mathrm E}(X)$ and that
\[{\mathrm E}(X^{2})-{\mathrm E}(Y^{2})=\frac{1}{24}.\]
By using the fact that $4^{n}=(3+1)^{n}$, or otherwise,
show that ${\mathrm E}(X^{n})>{\mathrm E}(Y^{n})$ for $n\geqslant 2$.

Suppose that $Y_{1}$, $Y_{2}$, \dots are independent random variables
each having the same distribution as $Y$.
Find, to a good approximation, $K$ such that
\[{\rm P}(Y_{1}+Y_{2}+\cdots+Y_{240000}<K)=3/4.\]

\end{question}
	
\end{document}
