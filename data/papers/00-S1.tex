\documentclass[a4, 11pt]{report}


\pagestyle{myheadings}
\markboth{}{Paper I, 2000
\ \ \ \ \ 
\today 
}               

\RequirePackage{amssymb}
\RequirePackage{amsmath}
\RequirePackage{graphicx}
\RequirePackage{color}
\RequirePackage[flushleft]{paralist}[2013/06/09]



\RequirePackage{geometry}
\geometry{%
  a4paper,
  lmargin=2cm,
  rmargin=2.5cm,
  tmargin=3.5cm,
  bmargin=2.5cm,
  footskip=12pt,
  headheight=24pt}


\newcommand{\comment}[1]{{\bf Comment} {\it #1}}
%\renewcommand{\comment}[1]{}

\newcommand{\bluecomment}[1]{{\color{blue}#1}}
%\renewcommand{\comment}[1]{}
\newcommand{\redcomment}[1]{{\color{red}#1}}



\usepackage{epsfig}
\usepackage{pstricks-add}
\usepackage{tgheros} %% changes sans-serif font to TeX Gyre Heros (tex-gyre)
\renewcommand{\familydefault}{\sfdefault} %% changes font to sans-serif
%\usepackage{sfmath}  %%%% this makes equation sans-serif
%\input RexFigs


\setlength{\parskip}{10pt}
\setlength{\parindent}{0pt}

\newlength{\qspace}
\setlength{\qspace}{20pt}


\newcounter{qnumber}
\setcounter{qnumber}{0}

\newenvironment{question}%
 {\vspace{\qspace}
  \begin{enumerate}[\bfseries 1\quad][10]%
    \setcounter{enumi}{\value{qnumber}}%
    \item%
 }
{
  \end{enumerate}
  \filbreak
  \stepcounter{qnumber}
 }


\newenvironment{questionparts}[1][1]%
 {
  \begin{enumerate}[\bfseries (i)]%
    \setcounter{enumii}{#1}
    \addtocounter{enumii}{-1}
    \setlength{\itemsep}{5mm}
    \setlength{\parskip}{8pt}
 }
 {
  \end{enumerate}
 }



\DeclareMathOperator{\cosec}{cosec}
\DeclareMathOperator{\Var}{Var}

\def\d{{\mathrm d}}
\def\e{{\mathrm e}}
\def\g{{\mathrm g}}
\def\h{{\mathrm h}}
\def\f{{\mathrm f}}
\def\p{{\mathrm p}}
\def\s{{\mathrm s}}
\def\t{{\mathrm t}}


\def\A{{\mathrm A}}
\def\B{{\mathrm B}}
\def\E{{\mathrm E}}
\def\F{{\mathrm F}}
\def\G{{\mathrm G}}
\def\H{{\mathrm H}}
\def\P{{\mathrm P}}


\def\bb{\mathbf b}
\def \bc{\mathbf c}
\def\bx {\mathbf x}
\def\bn {\mathbf n}

\newcommand{\low}{^{\vphantom{()}}}
%%%%% to lower suffices: $X\low_1$ etc


\newcommand{\subone}{ {\vphantom{\dot A}1}}
\newcommand{\subtwo}{ {\vphantom{\dot A}2}}




\def\le{\leqslant}
\def\ge{\geqslant}


\def\var{{\rm Var}\,}

\newcommand{\ds}{\displaystyle}
\newcommand{\ts}{\textstyle}
\def\half{{\textstyle \frac12}}




\begin{document}
\setcounter{page}{2}

 
\section*{Section A: \ \ \ Pure Mathematics}

%%%%%%%%%%Q1
\begin{question}
To nine decimal places, $\log_{10}2=0.301029996$ and
 $\log_{10}3=0.477121255$.

\begin{questionparts}
\item  Calculate $\log_{10}5$ and  $\log_{10}6$ to three decimal
places. By taking logs, or otherwise,  show that 
\[
5\times 10^{47} < 3^{100} < 6\times 10^{47}.
\]
Hence write down the first digit of $3^{100}$.

\item  Find the first digit of each of the following numbers:
$2^{1000}$; \ $2^{10\,000}$; \ and 
$2^{100\, 000}$.
\end{questionparts}
\end{question}

%%%%%%%%%%Q2
\begin{question}
Show that the coefficient of $x^{-12}$ in the expansion of 
\[
\left(x^{4}-\frac{1}{x^{2}}\right)^{5} 
\left(x-\frac{1}{x}\right)^{6} 
\]
is $-15$, and calculate the coefficient of $x^2$.

Hence, or otherwise,
calculate the coefficients of $x^4$ and $x^{38}$ in the expansion of 
\[
(x^2-1)^{11}(x^4+x^2+1)^5.
\]
\end{question}

%%%%%%%%% Q3
\begin{question}
For any number $x$, the largest integer less than or equal to $x$ is
denoted by $[x]$. For example, $[3.7]=3$ and $[4]=4$.

Sketch the graph of $y=[x]$ for $0\le x<5$ and evaluate 
\[
\int_0^5 [x]\;\d x.
\]
  
Sketch the graph of 
$y=[\e^{x}]$ for $0\le x< \ln n$, 
where $n$ is an integer, and show that 
\[
\int_{0}^{\ln n}[\e^{x}]\,  \d x =n\ln n - \ln (n!).
\] 
\end{question}

%%%%%% Q4 
\begin{question}
\begin{questionparts}  

\item
Show that, for $0\le
 x\le 1$, the largest value of 
$ \
\displaystyle 
\frac{x^6}{(x^2+1)^4}
\ $   
is 
$
\ \displaystyle 
\frac1{16}
$.
\item
Find constants $A$, $B$, $C$ and $D$ such that, for all $x$, 
\[
\frac{1}{(x^2+1)^4}= 
\frac{\d \ }{\d x} 
\left(
\frac{Ax^5+Bx^3+Cx}{(x^2+1)^3}\right)
+\frac{Dx^6}{(x^2+1)^4}.
\] 

\item
Hence, or otherwise, prove that 
\[
\frac{11}{24}
\le
\int_{0}^{1}\frac{1}{(x^{2}+1)^{4}}\, \d x 
\le
\frac{11}{24} + \frac 1{16} \;  .
\] 

\end{questionparts}  
\end{question}

%%%%%%%%% Q5
\begin{question}
Arthur  and Bertha stand at a point $O$ on an inclined plane. 
The steepest line in the plane through $O$ makes an 
angle $\theta$ with the horizontal. Arthur walks uphill at a steady pace in
a straight line which makes an angle $\alpha$ with the steepest line.
Bertha walks uphill at the same speed in a straight line which makes an angle
$\beta$ with the steepest line (and is on the same side of the steepest line
as Arthur).
Show that, when Arthur  has walked a distance $d$, the distance between Arthur
and Bertha is $2d \vert\sin\half(\alpha-\beta)\vert$.
Show also that, if $\alpha\ne\beta$, 
the line joining Arthur  and Bertha makes an angle $\phi$ 
with the vertical, where
\[
\cos\phi = \sin\theta \sin \half(\alpha+\beta).
\]
	\end{question}
	
%%%%%%%%% Q6
\begin{question}
Show that 
\[
x^2-y^2 +x+3y-2 = (x-y+2)(x+y-1)
\]
and hence, or otherwise, indicate by means of a
 sketch the region of the $x$-$y$ plane for which
$$
x^2-y^2 +x+3y>2.
$$

Sketch also the region of the $x$-$y$ plane for which
$$
x^2-4y^2 +3x-2y<-2.
$$

Give the coordinates of 
a point for which both inequalities are satisfied or explain why
no such point exists.
\end{question}
\newpage
%%%%%%%%% Q7
\begin{question}
Let 
\[
{\f}(x)=a x-\frac{x^{3}}{1+x^{2}},
\]
where  $a$ is a constant. Show that, if 
$a\ge  9/8$, then
$\mathrm{f}' (x) \ge0$ for all $x$.
\end{question}
		
%%%%%%%%% Q8
\begin{question}	
Show that 
\[
\int_{-1}^1 \vert \, x\e^x \,\vert  \d x =- \int_{-1}^0  x\e^x  \d x +
 \int_0^1  x\e^x   \d x
\]
and hence evaluate the integral.

Evaluate the following integrals:
\begin{questionparts}

\item
$\displaystyle
\int_0^4 \vert\, x^3-2x^2-x+2 \,\vert \, \d x\,;
$

\item
$\displaystyle 
\int_{-\pi}^\pi 
\vert\, \sin x  +\cos x \,\vert \; \d x\,.
$

\end{questionparts}
\end{question}	
		

		
	
\newpage
\section*{Section B: \ \ \ Mechanics}


	
%%%%%%%%%% Q9
\begin{question}
A child is playing with a toy cannon on the floor of a long
railway carriage. 
The carriage is moving horizontally in a northerly direction with 
acceleration $a$. The child points the cannon southward at an angle $\theta$
to the horizontal and fires a toy shell which 
leaves the cannon at speed $V$.  Find, in terms of $a$ and $g$,
the value of $\tan 2\theta$  
for which the cannon has maximum range (in the carriage). 

If $a$ is small compared with $g$, show that the value of $\theta$ which 
gives the maximum range is approximately  
\[
\frac \pi 4 + \frac a {2g},
\]
and show that the maximum range is approximately 
$\displaystyle
\frac {V^2} g + \frac {V^2a}{g^2}.
$
	\end{question}
	
%%%%%%%%%% Q10
\begin{question}	
Three particles $P_1$, $P_2$ and $P_3$
of masses $m_{1}$, $m_{2}$ and  $m_{3}$ respectively 
lie at rest in a straight line on a smooth horizontal table. 
$P_1$ is projected with speed $v$ 
towards $P_2$ and brought to rest by the collision. 
After $P_2$  collides with $P_3$, 
the latter moves forward with speed $v$. The coefficients of restitution
in the first and second collisions are $e$ and $e'$, respectively.
Show that 
\[
e'=
\frac{m_{2}+m_{3}-m_{1}}{m_{1}}.
\] 
 
Show that 
$2m_1\ge m_2 +m_3\ge m_1$
for such collisions to be possible.

If $m_1$,  $m_3$ and $v$ are fixed, find,  in terms of $m_1$, $m_3$ and 
$v$,
the largest and smallest possible
values for the final  energy of the system.
\end{question}

%%%%%%%%%% Q11

\begin{question}
A rod $AB$ of length 0.81 m and mass 5 kg is
in equilibrium with the end $A$ on a rough floor and the end $B$ against
a very rough vertical wall. The rod is in a vertical plane perpendicular
to the wall and is inclined at $45^{\circ}$ to the horizontal. 
 The centre of gravity of the rod is at $G$, where $AG = 0.21$ m. 
The coefficient of friction between the
rod and the floor is 0.2, and the coefficient of friction
between the rod and the wall is 1.0. Show that the friction cannot 
be limiting at both $A$ and $B$.


A mass of 5 kg is attached to the rod at the point $P$ such that now
the friction is limiting  at both $A$ and $B$. Determine the length of 
$AP$. 
\end{question}
	

	
	\newpage
\section*{Section C: \ \ \ Probability and Statistics}


%%%%%%%%%% Q12
\begin{question}
I have $k$ different keys on my key ring. When I 
come home at night I try one key after another until I 
find the key that fits my front door. What is the probability 
that I find the correct key in exactly $n$ attempts in 
each of the following three cases?
 \begin{questionparts}
 \item At each attempt,
I choose a key that I have not tried 
before but otherwise each choice is equally likely. 
 
\item
[(ii)] At each attempt,
I choose a key from all my 
keys and each of the $k$ choices is equally likely. 
 
\item
At the first attempt,
I choose from all my 
keys and each of the $k$ choices is equally likely. Thereafter, 
I choose from the keys that I did not try the previous time 
but otherwise each choice is equally likely. 
 \end{questionparts}

\end{question}

%%%%%%%%%% Q13
\begin{question}
Every person  carries two genes which can each be either of 
 type $A$ or of type $B$. 
It is known that $81\%$ of the population are $AA$ (i.e. both genes are 
of type $A$), $18\%$ are $AB$ (i.e. there is one gene of type $A$ 
and one of type $B$) and $1\%$ are $BB$. A child inherits 
one gene from each of its parents. If one parent is $AA$, the child
 inherits a gene of type  $A$ from that parent;  
if the  parent is $BB$, the child
 inherits a gene of type  $B$ from that parent; 
 if the parent 
is $AB$, the inherited gene is equally likely to be $A$ or $B$. 

\begin{questionparts}
\item
Given that two $AB$ parents have four children, 
show that the probability 
that two of them are $AA$ and two of them are $BB$ is $3/128$. 
 
\item
My mother is $AB$ and I am $AA$. 
Find the probability that my father is $AB$.
\end{questionparts} 
\end{question}

%%%%%%%%%% Q14
\begin{question}
The random variable $X$ is uniformly distributed on the interval
$[-1,1]$. Find $\E(X^2)$ and $\var (X^2)$.

A second random variable $Y$, independent of $X$, is also uniformly
distributed on $[-1,1]$, and $Z=Y-X$. Find $\E(Z^2)$ and show that 
$\var (Z^2) = 7 \var (X^2)$.
\end{question}
	
\end{document}
