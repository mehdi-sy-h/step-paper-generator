
\documentclass[a4, 11pt]{report}


\pagestyle{myheadings}
\markboth{}{Paper I, 1990
\ \ \ \ \ 
\today 
}               

\RequirePackage{amssymb}
\RequirePackage{amsmath}
\RequirePackage{graphicx}
\RequirePackage{color}
\RequirePackage[flushleft]{paralist}[2013/06/09]



\RequirePackage{geometry}
\geometry{%
  a4paper,
  lmargin=2cm,
  rmargin=2.5cm,
  tmargin=3.5cm,
  bmargin=2.5cm,
  footskip=12pt,
  headheight=24pt}


\newcommand{\comment}[1]{{\bf Comment} {\it #1}}
%\renewcommand{\comment}[1]{}

\newcommand{\bluecomment}[1]{{\color{blue}#1}}
%\renewcommand{\comment}[1]{}
\newcommand{\redcomment}[1]{{\color{red}#1}}



\usepackage{epsfig}
\usepackage{pstricks-add}
\usepackage{pst-coil}
\usepackage{tgheros} %% changes sans-serif font to TeX Gyre Heros (tex-gyre)
\renewcommand{\familydefault}{\sfdefault} %% changes font to sans-serif
%\usepackage{sfmath}  %%%% this makes equation sans-serif
%\input RexFigs


\setlength{\parskip}{10pt}
\setlength{\parindent}{0pt}

\newlength{\qspace}
\setlength{\qspace}{20pt}


\newcounter{qnumber}
\setcounter{qnumber}{0}

\newenvironment{question}%
 {\vspace{\qspace}
  \begin{enumerate}[\bfseries 1\quad][10]%
    \setcounter{enumi}{\value{qnumber}}%
    \item%
 }
{
  \end{enumerate}
  \filbreak
  \stepcounter{qnumber}
 }


\newenvironment{questionparts}[1][1]%
 {
  \begin{enumerate}[\bfseries (i)]%
    \setcounter{enumii}{#1}
    \addtocounter{enumii}{-1}
    \setlength{\itemsep}{5mm}
    \setlength{\parskip}{8pt}
 }
 {
  \end{enumerate}
 }



\DeclareMathOperator{\cosec}{cosec}
\DeclareMathOperator{\Var}{Var}

\def\d{{\rm d}}
\def\e{{\rm e}}
\def\g{{\rm g}}
\def\h{{\rm h}}
\def\f{{\rm f}}
\def\p{{\rm p}}
\def\s{{\rm s}}
\def\t{{\rm t}}


\def\A{{\rm A}}
\def\B{{\rm B}}
\def\E{{\rm E}}
\def\F{{\rm F}}
\def\G{{\rm G}}
\def\H{{\rm H}}
\def\P{{\rm P}}


\def\bb{\mathbf b}
\def \bc{\mathbf c}
\def\bx {\mathbf x}
\def\bn {\mathbf n}

\newcommand{\low}{^{\vphantom{()}}}
%%%%% to lower suffices: $X\low_1$ etc


\newcommand{\subone}{ {\vphantom{\dot A}1}}
\newcommand{\subtwo}{ {\vphantom{\dot A}2}}




\def\le{\leqslant}
\def\ge{\geqslant}


\def\var{{\rm Var}\,}

\newcommand{\ds}{\displaystyle}
\newcommand{\ts}{\textstyle}




\begin{document}
\setcounter{page}{2}

 
\section*{Section A: \ \ \ Pure Mathematics}

%%%%%%%%%%Q1
\begin{question}
$\,$


\noindent \begin{center}
\psset{xunit=1.0cm,yunit=1.0cm,algebraic=true,dotstyle=o,dotsize=3pt 0,linewidth=0.5pt,arrowsize=3pt 2,arrowinset=0.25} \begin{pspicture*}(-3.02,0.12)(6.18,5.8) \parametricplot{0.0}{3.141592653589793}{1*4.17*cos(t)+0*4.17*sin(t)+1.63|0*4.17*cos(t)+1*4.17*sin(t)+0.88} \psline(-2.54,0.88)(5.8,0.88) \psline(1.82,0.88)(-1.07,4.05) \psline(-1.07,4.05)(3.33,4.69) \psline(3.33,4.69)(1.82,0.88) \psline(-1.07,4.05)(-2.54,0.88) \psline(3.33,4.69)(5.8,0.88) \rput[tl](1.58,1.5){$\alpha$} \rput[tl](1.18,1.34){$\theta $} \rput[tl](3.42,5.32){$Y$} \rput[tl](-1.24,4.72){$X$} \rput[tl](1.12,5.62){$B$} \rput[tl](-2.92,0.96){$A$} \rput[tl](1.7,0.8){$D$} \rput[tl](5.68,0.84){$C$} \end{pspicture*}
\par\end{center}


In the above diagram, $ABCD$ represents a semicircular window of
fixed radius $r$ and centre $D$, and $AXYC$ is a quadrilateral
blind. If $\angle XDY=\alpha$ is fixed and $\angle ADX=\theta$ is
variable, determine the value of $\theta$ which gives the blind \textbf{maximum}
area. 


If now $\alpha$ is allowed to vary but $r$ remains fixed, find the
maximum possible area of the blind.
\end{question}

%%%%%%%%%%Q2
\begin{question}
Let $\omega=\mathrm{e}^{2\pi\mathrm{i}/3}.$ Show that $1+\omega+\omega^{2}=0$
and calculate the modulus and argument of $1+\omega^{2}.$ 


Let $n$ be a positive integer. By evaluating $(1+\omega^{r})^{n}$
in two ways, taking $r=1,2$ and $3$, or otherwise, prove that 
\[
\binom{n}{0}+\binom{n}{3}+\binom{n}{6}+\cdots+\binom{n}{k}=\frac{1}{3}\left(2^{n}+2\cos\left(\frac{n\pi}{3}\right)\right),
\]
where $k$ is the largest multiple of $3$ less than or equal to $n$.
Without using a calculator, evaluate 
\[
\binom{25}{0}+\binom{25}{3}+\cdots+\binom{25}{24}
\]
and 
\[
\binom{24}{2}+\binom{24}{5}+\cdots+\binom{24}{23}\,.
\]
{[}$2^{25}=33554432.${]} 
\end{question}

%%%%%%%%% Q3
\begin{question}
Given a curve described by $y=\mathrm{f}(x)$, and such that $y\geqslant0$,
a \textit{push-off }of the curve is a new curve obtained as follows:
for each point $(x,\mathrm{f}(x))$ with position vector $\mathbf{r}$
on the original curve, there is a point with position vector $\mathbf{s}$
on the new curve such that $\mathbf{s-r}=\mathrm{p}(x)\mathbf{n},$
where $\mathrm{p}$ is a given function and $\mathbf{n}$ is the downward-pointing
unit normal to the original curve at $\mathbf{r}$. 

\begin{questionparts}
\item For the curve $y=x^{k},$ where $x>0$ and $k$ is a positive integer,
obtain the function $\mathrm{p}$ for which the push-off is the positive
$x$-axis, and find the value of $k$ such that, for all points on
the original curve, $\left|\mathbf{r}\right|=\left|\mathbf{r-s}\right|$. 
\item Suppose that the original curve is $y=x^{2}$ and $\mathrm{p}$ is such
that the gradient of the curves at the points with position vectors
$\mathbf{r}$ and $\mathbf{s}$ are equal (for every point on the
original curve). By writing $\mathrm{p}(x)=\mathrm{q}(x)\sqrt{1+4x^{2}},$
where $\mathrm{q}$ is to be determined, or otherwise, find the form
of $\mathrm{p}$.
\end{questionparts}
\end{question}


%%%%%% Q4 

\begin{question}
The sequence $a_{1},a_{2},\ldots,a_{n},\ldots$ forms an arithmetic
progression. Establish a formula, involving $n,$ $a_{1}$ and $a_{2}$
for the sum $a_{1}+a_{2}+\cdots+a_{n}$ of the first $n$ terms. 


A sequence $b_{1},b_{2},\ldots,b_{n},\ldots$ is called a \textit{double
arithmetic progression} if the sequence of differences
\[
b_{2}-b_{1},b_{3}-b_{2},\ldots,b_{n+1}-b_{n},\ldots
\]
is an arithmetic progression. Establish a formula, involving $n,b_{1},b_{2}$
and $b_{3}$, for the sum $b_{1}+b_{2}+b_{3}+\cdots+b_{n}$ of the
first $n$ terms of such a progression. 


A sequence $c_{1},c_{2},\ldots,c_{n},\ldots$ is called a \textit{factorial
progression} if $c_{n+1}-c_{n}=n!d$ for some non-zero $d$ and every
$n\geqslant1$. Suppose $1,b_{2},b_{3},\ldots$ is a double arithmetic
progression, and also that $b_{2},b_{4},b_{6}$ and $220$ are the first
four terms in a factorial progression. Find the sum $1+b_{2}+b_{3}+\cdots+b_{n}.$
\end{question}


%%%%%%%%% Q5
\begin{question}
\begin{questionparts} 
\item Evaluate 
\[
\int_{1}^{3}\frac{1}{6x^{2}+19x+15}\,\mathrm{d}x\,.
\]



\item Sketch the graph of the function $\mathrm{f}$, where $\mathrm{f}(x)=x^{1760}-x^{220}+q$,
and $q$ is a constant. Find the possible numbers of \textit{distinct
}roots of the equation $\mathrm{f}(x)=0$, and state the inequalities
satisfied by $q$.
 \end{questionparts}
\end{question}
	
	%%%%%%%%% Q6
	\begin{question}
Let $ABCD$ be a parallelogram. By using vectors, or otherwise, prove
that: 

\begin{itemize}
\setlength{\itemsep}{3mm}
\item[\bf (i)] $AB^{2}+BC^{2}+CD^{2}+DA^{2}=AC^{2}+BD^{2}$; 
\item[\bf (ii)] $\left|AC^{2}-BD^{2}\right|$ is 4 times the area of the rectangle
whose sides are \textit{any} side of the parallelogram and the projection
of an adjacent side on that side. 
\end{itemize}

State and prove a result like \textbf{(ii)} about $\left|AB^{2}-AD^{2}\right|$
and the diagonals.
	 \end{question}
	 
	 %%%%%%%%% Q7
\begin{question}
Let $y,u,v,P$ and $Q$ all be functions of $x$. Show that the substitution
$y=uv$ in the differential equation
\[
\frac{\mathrm{d}y}{\mathrm{d}x}+Py=Q
\]
leads to an equation for $\dfrac{\mathrm{d}v}{\mathrm{d}x}$ in terms
of $x,Q$ and $u$, provided that $u$ satisfies a suitable first
order differential equation. 


Hence or otherwise solve
\[
\frac{\mathrm{d}y}{\mathrm{d}x}-\frac{2y}{x+1}=\left(x+1\right)^{\frac{5}{2}},
\]
given that $y(1)=0$. For what set of values of $x$ is the solution
valid?
	\end{question}
	
	%%%%%%%%% Q8
	\begin{question}
Show that
\[
\cos\left(\frac{\alpha}{2}\right)\cos\left(\frac{\alpha}{4}\right)=\frac{\sin\alpha}{4\sin\left(\dfrac{\alpha}{4}\right)}\,,
\]
where $\alpha\neq k\pi$, $k$ is an integer. 


Prove that, for such $\alpha$, 
\[
\cos\left(\frac{\alpha}{2}\right)\cos\left(\frac{\alpha}{4}\right)\cdots\cos\left(\frac{\alpha}{2^{n}}\right)=\frac{\sin\alpha}{2^{n}\sin\left(\dfrac{\alpha}{2^{n}}\right)}\,,
\]
where $n$ is a positive integer. 


Deduce that
\[
\alpha=\frac{\sin\alpha}{\cos\left(\dfrac{\alpha}{2}\right)\cos\left(\dfrac{\alpha}{4}\right)\cos\left(\dfrac{\alpha}{8}\right)\cdots}\,,
\]
and hence that
\[
\frac{\pi}{2}=\frac{1}{\sqrt{\frac{1}{2}}\sqrt{\frac{1}{2}+\frac{1}{2}\sqrt{\frac{1}{2}}}\sqrt{\frac{1}{2}+\frac{1}{2}\sqrt{\frac{1}{2}+\frac{1}{2}\sqrt{\frac{1}{2}}}}\cdots}\,.
\]
		\end{question}
		
		
%%%%%%%%% Q9
\begin{question}
Let $A$ and $B$ be the points $(1,1)$ and $(b,1/b)$ respectively,
where $b>1$. The tangents at $A$ and $B$ to the curve $y=1/x$
intersect at $C$. Find the coordinates of $C$. 


Let $A',B'$ and $C'$ denote the projections of $A,B$ and $C$,
respectively, to the $x$-axis. Obtain an expression for the sum of
the areas of the quadrilaterals $ACC'A'$ and $CBB'C'$. Hence or
otherwise prove that, for $z>0$, 
\[
\frac{2z}{2+z}\leqslant\ln\left(1+z\right)\leqslant z.
\]
\end{question}

\newpage
\section*{Section B: \ \ \ Mechanics}


	
%%%%%%%%%% Q10
\begin{question}
In a certain race, runners run 5$\,$km in a straight line to a fixed
point and then turn and run back to the starting point. A steady wind
of 3$\,$ms$^{-1}$ is blowing from the start to the turning point.
At steady racing pace, a certain runner expends energy at a constant
rate of 300$\,$W. Two resistive forces act. One is of constant magnitude
50$\,$N. The other, arising from air resistance, is of magnitude
$2w\,\mathrm{N}$, where $w\,$ms$^{-1}$ is the runner's speed relative
to the air. Give a careful argument to derive formulae from which
the runner's steady speed in each half of the race may be found. Calculate,
to the nearest second, the time the runner will take for the whole
race. 


{[}Effects due to acceleration and deceleration at the start and turn
may be ignored.{]}


The runner may use alternative tactics, expending the same total energy
during the race as a whole, but applying different constant powers,
$x_{1}\,$W in the outward trip, and $x_{2}\,$W on the return trip.
Prove that, with the wind as above, if the outward and return speeds
are $v_{1}\,$ms$^{-1}$ and $v_{2}\,$ms$^{-1}$ respectively, then
$v_{1}+v_{2}$ is independent of the choices for $x_{1}$ and $x_{2}$.
Hence show that these alternative tactics allow the runner to run
the whole race approximately 15 seconds faster. 
	\end{question}
	
%%%%%%%%%% Q11
\begin{question}	
A shell of mass $m$ is fired at elevation $\pi/3$ and speed $v$.
Superman, of mass $2m$, catches the shell at the top of its flight,
by gliding up behind it in the same horizontal direction with speed
$3v$. As soon as Superman catches the shell, he instantaneously clasps
it in his cloak, and immediately pushes it vertically downwards, without
further changing its horizontal component of velocity, but giving
it a downward vertical component of velocity of magnitude $3v/2$.
Calculate the total time of flight of the shell in terms of $v$ and
$g$. Calculate also, to the nearest degree, the angle Superman's
flight trajectory initially makes with the horizontal after releasing
the shell, as he soars upwards like a bird. 


{[}Superman and the shell may be regarded as particles.{]}
\end{question}

%%%%%%%%%% Q12

\begin{question}$\,$


\noindent \begin{center}
\psset{xunit=1.0cm,yunit=1.0cm,algebraic=true,dotstyle=o,dotsize=3pt 0,linewidth=0.5pt,arrowsize=3pt 2,arrowinset=0.25} \begin{pspicture*}(-3.39,-0.76)(7.24,4.37) \psline(2,0)(6,3) \psline(-2,3)(2,0) \psline(-2,3)(-3,3) \psline(-2,3)(6,3) \psline(6,3)(7,3) \rput[tl](-2.15,3.55){$A$} \rput[tl](2.01,3.8){$B$} \rput[tl](6.08,3.55){$C$} \rput[tl](1.85,-0.15){$D$} \begin{scriptsize} \psdots[dotstyle=*](2,0) \end{scriptsize} \pscoil[coilheight=0.8,coilwidth=0.5](-2.35,3)(6.42,3) \end{pspicture*}
\par\end{center}


In the above diagram, $ABC$ represents a light spring of natural
length $2l$ and modulus of elasticity $\lambda,$ which is coiled
round a smooth fixed horizontal rod. $B$ is the midpoint of $AC.$
The two ends of a light inelastic string of length $2l$ are attached
to the spring at $A$ and $C$. A particle of mass $m$ is fixed to
the string at $D$, the midpoint of the string. The system can be
in equilibrium with the angle $CAD$ equal to $\pi/6.$ Show that
\[
mg=\lambda\left(\frac{2}{\sqrt{3}}-1\right).
\]
Write the length $AC$ as $2xl$, obtain an expression for the potential
energy of the system as a function of $x$. 


The particle is held at $B$, and the spring is restored to its natural
length $2l.$ The particle is then released and falls vertically.
Obtain an equation satisfied by $x$ when the particle next comes
to rest. Verify numerically that a possible solution for $x$ is approximately
$0.66.$ 
\end{question}
	
%%%%%%%%%% Q13
\begin{question}
A rough circular cylinder of mass $M$ and radius $a$ rests on a
rough horizontal plane. The curved surface of the cylinder is in contact
with a smooth rail, parallel to the axis of the cylinder, which touches
the cylinder at a height $a/2$ above the plane. Initially the cylinder
is held at rest. A particle of mass $m$ rests in equilibrium on the
cylinder, and the normal reaction of the cylinder on the particle
makes an angle of $\theta$ with the upward vertical. The particle
is on the same side of the centre of the cylinder as the rail. The
coefficient of friction between the cylinder and the particle and
between the cylinder and the plane are both $\mu$. Obtain the condition
on $\theta$ for the particle to rest in equilibrium. Show that, if
the cylinder is released, equilibrium of both particle and cylinder
is possible provided another inequality involving $\mu$ and $\theta$
(which should be found explicitly) is satisfied. Determine the largest
possible value of $\theta$ for equilibrium, if $m=7M$ and $\mu=0.75$. 
\end{question}
	
	\newpage
\section*{Section C: \ \ \ Probability and Statistics}


%%%%%%%%%% Q14
\begin{question}
A bag contains 5 white balls, 3 red balls and 2 black balls. In the
game of Blackball, a player draws a ball at random from the bag, looks
at it and replaces it. If he has drawn a white ball, he scores one
point, while for a red ball he scores two points, these scores being
added to his total score before he drew the ball. If he has drawn
a black ball, the game is over and his final score is zero. After
drawing a red or white ball, he can either decide to stop, when his
final score for the game is the total so far, or he may elect to draw
another ball. The starting score is zero. 


Juggins' strategy is to continue drawing until either he draws a black
ball (when of course he must stop, with final score zero), or until
he has drawn three (non-black) balls, when he elects to stop. Find
the probability that in any game he achieves a final score of zero
by employing this strategy. Find also his expected final score. 


Muggins has so far scored $N$ points, and is deciding whether to
draw another ball. Find the expected score if another ball is drawn,
and suggest a strategy to achieve the greatest possible average final
score in each game. 

\end{question}

%%%%%%%%%% Q15
\begin{question}
A coin has probability $p$ ($0<p<1$) of showing a head when tossed.
Give a careful argument to show that the $k$th head in a series of
consecutive tosses is achieved after \textit{exactly }$n$ tosses
with probability 
\[
\binom{n-1}{k-1}p^{k}(1-p)^{n-k}\qquad(n\geqslant k).
\]
Given that it took an even number of tosses to achieve exactly $k-1$
heads, find the probability that exactly $k$ heads are achieved after
an even number of tosses. 


If this coin is tossed until exactly 3 heads are obtained, what is
the probability that \textit{exactly }2 of the heads occur on even-numbered
tosses? 
\end{question}

%%%%%%%%%% Q16
\begin{question}
A bus is supposed to stop outside my house every hour on the hour.
From long observation I know that a bus will always arrive some time
between 10 minutes before and ten minutes after the hour. The probability
it arrives at a given instant increases linearly (from zero at 10
minutes before the hour) up to a maximum value at the hour, and then
decreases linearly at the same rate after the hour. Obtain the probability
density function of $T$, the time in minutes after the scheduled
time at which a bus arrives. 


If I get up when my alarm clock goes off, I arrive at the bus stop
at 7.55am. However, with probability 0.5, I doze for 3 minutes before
it rings again. In that case with probability 0.8 I get up then and
reach the bus stop at 7.58am, or, with probability 0.2, I sleep a
little longer, not reaching the stop until 8.02am. What is the probability
that I catch a bus by 8.10am? 


I buy a louder alarm clock which ensures that I reach the stop at
exactly the same time each morning. This clock keeps perfect time,
but may be set to an incorrect time. If it is correct, the alarm goes
off so that I should reach the stop at 7.55am. After 100 mornings
I find that I have had to wait for a bus until \textit{after }9am
(according to the new clock) on 5 occasions. Is this evidence that
the new clock is incorrectly set? 


{[}The time of arrival of different buses are independent of each
other.{]} 
\end{question}
\end{document}
