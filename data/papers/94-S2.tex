\documentclass[a4, 11pt]{report}


\pagestyle{myheadings}
\markboth{}{Paper II, 1994
\ \ \ \ \ 
\today 
}               

\RequirePackage{amssymb}
\RequirePackage{amsmath}
\RequirePackage{graphicx}
\RequirePackage{color}
\RequirePackage[flushleft]{paralist}[2013/06/09]



\RequirePackage{geometry}
\geometry{%
  a4paper,
  lmargin=2cm,
  rmargin=2.5cm,
  tmargin=3.5cm,
  bmargin=2.5cm,
  footskip=12pt,
  headheight=24pt}


\newcommand{\comment}[1]{{\bf Comment} {\it #1}}
%\renewcommand{\comment}[1]{}

\newcommand{\bluecomment}[1]{{\color{blue}#1}}
%\renewcommand{\comment}[1]{}
\newcommand{\redcomment}[1]{{\color{red}#1}}



\usepackage{epsfig}
\usepackage{pstricks-add}
\usepackage{tgheros} %% changes sans-serif font to TeX Gyre Heros (tex-gyre)
\renewcommand{\familydefault}{\sfdefault} %% changes font to sans-serif
%\usepackage{sfmath}  %%%% this makes equation sans-serif
%\input RexFigs


\setlength{\parskip}{10pt}
\setlength{\parindent}{0pt}

\newlength{\qspace}
\setlength{\qspace}{20pt}


\newcounter{qnumber}
\setcounter{qnumber}{0}

\newenvironment{question}%
 {\vspace{\qspace}
  \begin{enumerate}[\bfseries 1\quad][10]%
    \setcounter{enumi}{\value{qnumber}}%
    \item%
 }
{
  \end{enumerate}
  \filbreak
  \stepcounter{qnumber}
 }


\newenvironment{questionparts}[1][1]%
 {
  \begin{enumerate}[\bfseries (i)]%
    \setcounter{enumii}{#1}
    \addtocounter{enumii}{-1}
    \setlength{\itemsep}{5mm}
    \setlength{\parskip}{8pt}
 }
 {
  \end{enumerate}
 }



\DeclareMathOperator{\cosec}{cosec}
\DeclareMathOperator{\Var}{Var}

\def\d{{\rm d}}
\def\e{{\rm e}}
\def\g{{\rm g}}
\def\h{{\rm h}}
\def\f{{\rm f}}
\def\p{{\rm p}}
\def\s{{\rm s}}
\def\t{{\rm t}}


\def\A{{\rm A}}
\def\B{{\rm B}}
\def\E{{\rm E}}
\def\F{{\rm F}}
\def\G{{\rm G}}
\def\H{{\rm H}}
\def\P{{\rm P}}


\def\bb{\mathbf b}
\def \bc{\mathbf c}
\def\bx {\mathbf x}
\def\bn {\mathbf n}

\newcommand{\low}{^{\vphantom{()}}}
%%%%% to lower suffices: $X\low_1$ etc


\newcommand{\subone}{ {\vphantom{\dot A}1}}
\newcommand{\subtwo}{ {\vphantom{\dot A}2}}




\def\le{\leqslant}
\def\ge{\geqslant}


\def\var{{\rm Var}\,}

\newcommand{\ds}{\displaystyle}
\newcommand{\ts}{\textstyle}




\begin{document}
\setcounter{page}{2}

 
\section*{Section A: \ \ \ Pure Mathematics}

%%%%%%%%%%Q1
\begin{question}
In this question we consider only positive, non-zero integers written
out in the usual (decimal) way. We say, for example, that 207 ends
in 7 and that 5310 ends in 1 followed by 0. Show that, if $n$ does
not end in 5 or an even number, then there exists $m$ such that $n\times m$
ends in 1. 


Show that, given any $n$, we can find $m$ such that $n\times m$
ends either in 1 or in 1 followed by one or more zeros. 


Show that, given any $n$ which ends in 1 or in 1 followed by one
or more zeros, we can find $m$ such that $n\times m$ contains all
the digits $0,1,2,\ldots,9$. 
\end{question}

%%%%%%%%%%Q2
\begin{question}
If $\mathrm{Q}$ is a polynomial, $m$ is an integer, $m\geqslant1$
and $\mathrm{P}(x)=(x-a)^{m}\mathrm{Q}(x),$ show that \[
\mathrm{P}'(x)=(x-a)^{m-1}\mathrm{R}(x)
\]
where $\mathrm{R}$ is a polynomial. Explain why $\mathrm{P}^{(r)}(a)=0$
whenever $1\leqslant r\leqslant m-1$. ($\mathrm{P}^{(r)}$ is the
$r$th derivative of $\mathrm{P}.$)


If 
\[
\mathrm{P}_{n}(x)=\frac{\mathrm{d}^{n}}{\mathrm{d}x^{n}}(x^{2}-1)^{n}
\]
for $n\geqslant1$ show that $\mathrm{P}_{n}$ is a polynomial of
degree $n$. By repeated integration by parts, or otherwise, show
that, if $n-1\geqslant m\geqslant0,$ 
\[
\int_{-1}^{1}x^{m}\mathrm{P}_{n}(x)\,\mathrm{d}x=0
\]
and find the value of 
\[
\int_{-1}^{1}x^{n}\mathrm{P}_{n}(x)\,\mathrm{d}x.
\]
{[}\textbf{Hint. }\textit{You may use the formula 
\[
{\displaystyle \int_{0}^{\frac{\pi}{2}}\cos^{2n+1}t\,\mathrm{d}t=\frac{(2^{2n})(n!)^{2}}{(2n+1)!}}
\]
 without proof if you need it. However some ways of doing this question
do not use this formula.}{]}
\end{question}

%%%%%%%%% Q3
\begin{question}
The function $\mathrm{f}$ satisfies $\mathrm{f}(0)=1$ and 
\[
\mathrm{f}(x-y)=\mathrm{f}(x)\mathrm{f}(y)-\mathrm{f}(a-x)\mathrm{f}(a+y)
\]
for some fixed number $a$ and all $x$ and $y$. Without making any
further assumptions about the nature of the function show that $\mathrm{f}(a)=0$. 


Show that, for all $t$, 


\begin{itemize}
\setlength{\itemsep}{3mm}

\item[\bf (i)] $\mathrm{f}(t)=\mathrm{f}(-t)$, 
\item[\bf (ii)] $\mathrm{f}(2a)=-1$, 
\item[\bf (iii)] $\mathrm{f}(2a-t)=-\mathrm{f}(t)$, 
\item[\bf (iv)] $\mathrm{f}(4a+t)=\mathrm{f}(t)$. 
\end{itemize}

Give an example of a non-constant function satisfying the conditions
of the first paragraph with $a=\pi/2$. Give an example of an non-constant
function satisfying the conditions of the first paragraph with $a=-2$. 
\end{question}

%%%%%% Q4 
\begin{question}
By considering the area of the region defined in terms of Cartesian
coordinates $(x,y)$ by \[
\{(x,y):\ x^{2}+y^{2}=1,\ 0\leqslant y,\ 0\leqslant x\leqslant c\},
\]
show that 
\[
\int_{0}^{c}(1-x^{2})^{\frac{1}{2}}\,\mathrm{d}x=\tfrac{1}{2}[c(1-c^{2})^{\frac{1}{2}}+\sin^{-1}c],
\]
if $0<c\leqslant1.$


Show that the area of the region defined by 
\[
\left\{ (x,y):\ \frac{x^{2}}{a^{2}}+\frac{y^{2}}{b^{2}}=1,\ 0\leqslant y,\ 0\leqslant x\leqslant c\right\} ,
\]
is 
\[
\frac{ab}{2}\left[\frac{c}{a}\left(1-\frac{c^{2}}{a^{2}}\right)^{\frac{1}{2}}+\sin^{-1}\left(\frac{c}{a}\right)\right],
\]
if $0<c\leqslant a$ and $0<b.$ 


Suppose that $0<b\leqslant a.$ Show that the area of intersection
$E\cap F$ of the two regions defined by 
\[
E=\left\{ (x,y):\ \frac{x^{2}}{a^{2}}+\frac{y^{2}}{b^{2}}\leqslant1\right\} \qquad\mbox{ and }\qquad F=\left\{ (x,y):\ \frac{x^{2}}{b^{2}}+\frac{y^{2}}{a^{2}}\leqslant1\right\} 
\]
is 
\[
4ab\sin^{-1}\left(\frac{b}{\sqrt{a^{2}+b^{2}}}\right).
\]
	\end{question}

%%%%%%%%% Q5
\begin{question}
\begin{questionparts}
\item Show that the equation 
\[
(x-1)^{4}+(x+1)^{4}=c
\]
has exactly two real roots if $c>2,$ one root if $c=2$ and no roots
if $c<2$. 
\item How many real roots does the equation $\left(x-3\right)^{4}+\left(x-1\right)^{4}=c$
have?
\item How many real roots does the equation $\left|x-3\right|+\left|x-1\right|=c$
have?
\item How many real roots does the equation $\left(x-3\right)^{3}+\left(x-1\right)^{3}=c$
have?
\end{questionparts}

{[}The answers to parts \textbf{(ii)}, \textbf{(iii)} and \textbf{(iv)}
may depend on the value of $c$. You should give reasons for your
answers.{]}
	\end{question}
	
%%%%%%%%% Q6
\begin{question}
Prove by induction, or otherwise, that, if $0<\theta<\pi$, 
\[
\frac{1}{2}\tan\frac{\theta}{2}+\frac{1}{2^{2}}\tan\frac{\theta}{2^{2}}+\cdots+\frac{1}{2^{n}}\tan\frac{\theta}{2^{n}}=\frac{1}{2^{n}}\cot\frac{\theta}{2^{n}}-\cot\theta.
\]
Deduce that 
\[
\sum_{r=1}^{\infty}\frac{1}{2^{r}}\tan\frac{\theta}{2^{r}}=\frac{1}{\theta}-\cot\theta.
\]
\end{question}
	
%%%%%%%%% Q7
\begin{question}
Show that the equation 
\[
ax^{2}+ay^{2}+2gx+2fy+c=0
\]
where $a>0$ and $f^{2}+g^{2}>ac$ represents a circle in Cartesian
coordinates and find its centre. 


The smooth and level parade ground of the First Ruritanian Infantry
Division is ornamented by two tall vertical flagpoles of heights $h_{1}$
and $h_{2}$ a distance $d$ apart. As part of an initiative test
a soldier has to march in such a way that he keeps the angles of elevation
of the tops of the two flagpoles equal to one another. Show that if
the two flagpoles are of different heights he will march in a circle.
What happens if the two flagpoles have the same height?


To celebrate the King's birthday a third flagpole is added. Soldiers
are then assigned to each of the three different pairs of flagpoles
and are told to march in such a way that they always keep the tops
of their two assigned flagpoles at equal angles of elevation to one
another. Show that, if the three flagpoles have different heights
$h_{1},h_{2}$ and $h_{3}$ and the circles in which the soldiers
march have centres of $(x_{ij},y_{ij})$ (for the flagpoles of height
$h_{i}$ and $h_{j}$) relative to Cartesian coordinates fixed in
the parade ground, then the $x_{ij}$ satisfy 
\[
h_{3}^{2}\left(h_{1}^{2}-h_{2}^{2}\right)x_{12}+h_{1}^{2}\left(h_{2}^{2}-h_{3}^{2}\right)x_{23}+h_{2}^{2}\left(h_{3}^{2}-h_{1}^{2}\right)x_{31}=0,
\]
and the same equation connects the $y_{ij}$. Deduce that the three
centres lie in a straight line. 
\end{question}
		
%%%%%%%%% Q8
\begin{question}	
`24 Hour Spares' stocks a small, widely used and cheap component.
Every $T$ hours $X$ units arrive by lorry from the wholesaler, for
which the owner pays a total $\pounds (a+qX)$. It costs the owner $\pounds b$
per hour to store one unit. If she has the units in stock she expects
to sell $r$ units per hour at $\pounds(p+q)$ per unit. The other running
costs of her business remain at $\pounds c$ pounds an hour irrespective
of whether she has stock or not. (All of the quantities $T,X,a,b,r,q,p$
and $c$ are greater than 0.) Explain why she should take $X\leqslant rT$. 


Given that the process may be assumed continuous (the items are very
small and she sells many each hour), sketch $S(t)$ the amount of
stock remaining as a function of $t$ the time from the last delivery.
Compute the total profit over each period of $T$ hours. Show that,
if $T$ is fixed with $T\geqslant p/b$, the business can be made
profitable if 
\[
p^{2}>2\frac{(a+cT)b}{r}.
\]
\end{question}	
		

		
	
\newpage
\section*{Section B: \ \ \ Mechanics}


	
%%%%%%%%%% Q9
\begin{question}
A light rod of length $2a$ is hung from a point $O$ by two light
inextensible strings $OA$ and $OB$ each of length $b$ and each
fixed at $O$. A particle of mass $m$ is attached to the end $A$
and a particle of mass $2m$ is attached to the end $B.$ Show that,
in equilibrium, the angle $\theta$ that the rod makes the horizontal
satisfies the equation 
\[
\tan\theta=\frac{a}{3\sqrt{b^{2}-a^{2}}}.
\]
Express the tension in the string $AO$ in terms of $m,g,a$ and $b$.  
	\end{question}
	
%%%%%%%%%% Q10
\begin{question}	
A truck is towing a trailer of mass $m$ across level ground by means
of an elastic rope of natural length $l$ whose modulus of elasticity
is $\lambda.$ At first the rope is slack and the trailer stationary.
The truck then accelerates until the rope becomes taut and thereafter
the truck travels in a straight line at a constant speed $u$. Assuming
that the effect of friction on the trailer is negligible, show that
the trailer will collide with the truck at a time 
\[
\pi\left(\frac{lm}{\lambda}\right)^{\frac{1}{2}}+\frac{l}{u}
\]
after the rope first becomes taut.
\end{question}

%%%%%%%%%% Q11

\begin{question}
As part of a firework display a shell is fired vertically upwards
with velocity $v$ from a point on a level stretch of ground. When
it reaches the top of its trajectory an explosion it splits into two
equal fragments each travelling at speed $u$ but (since momentum
is conserved) in exactly opposite (not necessarily horizontal) directions.
Show, neglecting air resistance, that the greatest possible distance
between the points where the two fragments hit the ground is $2uv/g$
if $u\leqslant v$ and $(u^{2}+v^{2})/g$ if $v\leqslant u.$ 
\end{question}
	

	
	\newpage
\section*{Section C: \ \ \ Probability and Statistics}


%%%%%%%%%% Q12
\begin{question}
Calamity Jane sits down to play the game of craps with Buffalo Bill.
In this game she rolls two fair dice. If, on the first throw, the
sum of the dice is $2,3$ or $12$ she loses, while if it is $7$
or $11$ she wins. Otherwise Calamity continues to roll the dice until
either the first sum is repeated, in which case she wins, or the sum
is $7$, in which case she loses. Find the probability that she wins
on the first throw. 


Given that she throws more than once, show that the probability that
she wins on the $n$th throw is 
\[
\frac{1}{48}\left(\frac{3}{4}\right)^{n-2}+\frac{1}{27}\left(\frac{13}{18}\right)^{n-2}+\frac{25}{432}\left(\frac{25}{36}\right)^{n-2}.
\]
Given that she throws more than $m$ times, where $m>1,$ what is
the probability that she wins on the $n$th throw?
\end{question}

%%%%%%%%%% Q13
\begin{question}
The makers of Cruncho (`The Cereal Which Cares') are giving away a
series of cards depicting $n$ great mathematicians. Each packet of
Cruncho contains one picture chosen at random. Show that when I have
collected $r$ different cards the expected number of packets I must
open to find a new card is $n/(n-r)$ where $0\leqslant r\leqslant n-1.$


Show by means of a diagram, or otherwise, that 
\[
\frac{1}{r+1}\leqslant\int_{r}^{r+1}\frac{1}{x}\,\mathrm{d}x\leqslant\frac{1}{r}
\]
and deduce that 
\[
\sum_{r=2}^{n}\frac{1}{r}\leqslant\ln n\leqslant\sum_{r=1}^{n-1}\frac{1}{r}
\]
for all $n\geqslant2.$ 


My children will give me no peace until we have the complete set of
cards, but I am the only person in our household prepared to eat Cruncho
and my spouse will only buy the stuff if I eat it. If $n$ is large,
roughly how many packets must I expect to consume before we have the
set?
\end{question}

%%%%%%%%%% Q14
\begin{question}
	When Septimus Moneybags throws darts at a dart board they are certain
	to end on the board (a disc of radius $a$) but, it must be admitted,
	otherwise are uniformly randomly distributed over the board. 

	\begin{questionparts}
	\item Show that the distance $R$ that his shot lands from the centre of
	the board is a random variable with variance $a^{2}/18.$
	
	\item At a charity fête he can buy $m$ throws for $\pounds(12+m)$, but
	he must choose $m$ before he starts to throw. If at least one of
	his throws lands with $a/\sqrt{10}$ of the centre he wins back $\pounds 12$.
	In order to show that a good sport he is, he is determined to play
	but, being a careful man, he wishes to choose $m$ so as to minimise
	his expected loss. What values of $m$ should he choose? 
\end{questionparts}
\end{question}
	
\end{document}
