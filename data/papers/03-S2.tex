\documentclass[a4, 11pt]{report}


\pagestyle{myheadings}
\markboth{}{Paper II, 2003
\ \ \ \ \ 
\today 
}               

\RequirePackage{amssymb}
\RequirePackage{amsmath}
\RequirePackage{graphicx}
\RequirePackage{color}
\RequirePackage[flushleft]{paralist}[2013/06/09]



\RequirePackage{geometry}
\geometry{%
  a4paper,
  lmargin=2cm,
  rmargin=2.5cm,
  tmargin=3.5cm,
  bmargin=2.5cm,
  footskip=12pt,
  headheight=24pt}


\newcommand{\comment}[1]{{\bf Comment} {\it #1}}
%\renewcommand{\comment}[1]{}

\newcommand{\bluecomment}[1]{{\color{blue}#1}}
%\renewcommand{\comment}[1]{}
\newcommand{\redcomment}[1]{{\color{red}#1}}



\usepackage{epsfig}
\usepackage{pstricks-add}
\usepackage{tgheros} %% changes sans-serif font to TeX Gyre Heros (tex-gyre)
\renewcommand{\familydefault}{\sfdefault} %% changes font to sans-serif
%\usepackage{sfmath}  %%%% this makes equation sans-serif
%\input RexFigs


\setlength{\parskip}{10pt}
\setlength{\parindent}{0pt}

\newlength{\qspace}
\setlength{\qspace}{20pt}


\newcounter{qnumber}
\setcounter{qnumber}{0}

\newenvironment{question}%
 {\vspace{\qspace}
  \begin{enumerate}[\bfseries 1\quad][10]%
    \setcounter{enumi}{\value{qnumber}}%
    \item%
 }
{
  \end{enumerate}
  \filbreak
  \stepcounter{qnumber}
 }


\newenvironment{questionparts}[1][1]%
 {
  \begin{enumerate}[\bfseries (i)]%
    \setcounter{enumii}{#1}
    \addtocounter{enumii}{-1}
    \setlength{\itemsep}{5mm}
    \setlength{\parskip}{8pt}
 }
 {
  \end{enumerate}
 }



\DeclareMathOperator{\cosec}{cosec}
\DeclareMathOperator{\Var}{Var}

\def\d{{\mathrm d}}
\def\e{{\mathrm e}}
\def\g{{\mathrm g}}
\def\h{{\mathrm h}}
\def\f{{\mathrm f}}
\def\p{{\mathrm p}}
\def\s{{\mathrm s}}
\def\t{{\mathrm t}}


\def\A{{\mathrm A}}
\def\B{{\mathrm B}}
\def\E{{\mathrm E}}
\def\F{{\mathrm F}}
\def\G{{\mathrm G}}
\def\H{{\mathrm H}}
\def\P{{\mathrm P}}


\def\bb{\mathbf b}
\def \bc{\mathbf c}
\def\bx {\mathbf x}
\def\bn {\mathbf n}

\newcommand{\low}{^{\vphantom{()}}}
%%%%% to lower suffices: $X\low_1$ etc


\newcommand{\subone}{ {\vphantom{\dot A}1}}
\newcommand{\subtwo}{ {\vphantom{\dot A}2}}




\def\le{\leqslant}
\def\ge{\geqslant}


\def\var{{\rm Var}\,}

\newcommand{\ds}{\displaystyle}
\newcommand{\ts}{\textstyle}
\def\half{{\textstyle \frac12}}
\def\l{\left(}
\def\r{\right)}



\begin{document}
\setcounter{page}{2}

 
\section*{Section A: \ \ \ Pure Mathematics}

%%%%%%%%%%Q1
\begin{question}
Consider the equations
\begin{alignat*}{2}
ax-&y- \ z  && =3 \;,\\
2ax -&y -3z && = 7 \;,\\
3ax-&y-5z   && =b \;,
\end{alignat*}
where $a$ and $b$ are given constants.
\begin{questionparts}
\item In the case $a=0\,$, show that the equations have a solution if and only if
$b=11\,$.
\item In the case $a\ne0$ and $b=11\,$ show that the equations have
a solution with $z=\lambda$ for any given number $\lambda\,$.
\item  In the case $a=2$ and $b=11\,$ find the solution
for which $x^2+y^2+z^2$ is least.
\item Find a value for $a$ for which there is a solution such that
$x>10^6$ and $y^2+z^2<1\,$.
\end{questionparts}
\end{question}

%%%%%%%%%%Q2
\begin{question}
 Write down a value of $\theta\,$ in the interval $\frac{1}{4}\pi< \theta <\frac{1}{2}\pi$ that satisfies
the  equation
\[
             4\cos\theta+ 2\sqrt3\, \sin\theta = 5 \;.
\]
Hence, or otherwise,  show that
\[
         \pi=3\arccos(5/\sqrt28) + 3\arctan(\sqrt3/2)\;.
\]

Show that
\[
         \pi=4\arcsin(7\sqrt2/10) - 4\arctan(3/4)\;.
\]
\end{question}

%%%%%%%%% Q3
\begin{question}
Prove that the cube root of any irrational number is an
irrational number.

Let $\ds u_n = {5\vphantom{\dot A}}^{1/{(3^n)}}\,$.
Given that $\sqrt[3]5$ is an irrational number, prove by
induction that
$u_n$ is an irrational number 
for every positive integer $n$.

Hence, or otherwise, give an example of  an infinite sequence
of irrational numbers which converges to a given  integer $m\,$.



\noindent
[An irrational number is a number that  cannot be expressed as the ratio 
of two integers.]
\end{question}

%%%%%% Q4 
\begin{question}
The line $y=d\,$, where $d>0\,$,
intersects the circle $x^2+y^2=R^2$ at $G$ and $H$. Show
that the area of the minor segment $GH$ is equal to
\begin{equation}
R^2\arccos \left({d \over R}\right) -d\sqrt{R^2 - d^2}\;.
\tag
{$*$}
\end{equation}


In the following cases, the given line intersects the
given circle. Determine how, in each case, the expression $(*)$ should be modified
to give the area of the minor
segment.


  
\begin{questionparts}
\item
Line: $y=c\,$; \ \ \ circle: $(x-a)^2+(y-b)^2=R^2\,$. 

\item
Line: $y=mx+c\, $;  \ \ \  circle: $x^2+y^2=R^2\,$. 


\item
Line: $y=mx+c\,$; \ \ \ circle: $(x-a)^2+(y-b)^2=R^2\,$.
\end{questionparts}
\end{question}

%%%%%%%%% Q5
\begin{question}
The position vectors of 
the points $A\,$, $B\,$ and $P$ with respect to an origin $O$ 
are $a{\bf i}\,$,  $b{\bf j}\,$ and  $l{\bf i}+m{\bf j}+n{\bf k}\,$, respectively,
where $a$, $b$, and $n$ are all non-zero. The points $E$, $F$, $G$ and~$H$ 
are the midpoints of $OA$, $BP$, $OB$ and  $AP$, respectively.  
Show that the lines
$EF$ and $GH$  intersect.


Let $D$ be the point with position vector $d{\bf k}$, where $d$ is non-zero, 
and let $S$ be the point of intersection of $EF$ and $GH.$
The point $T$ is such that the mid-point of $DT$ is $S$. 
Find the position vector of $T$ and hence find $d$ in terms of $n$
if $T$ lies  in the plane $OAB$. 
	\end{question}
	
%%%%%%%%% Q6
\begin{question}
 The function $\f$  is defined by 
$$
\f(x)=  \vert x-1 \vert\;,
$$
where the domain is ${\bf R}\,$, the set of all real numbers. 
The function $\g_n =\f^n$, with domain ${\bf R}\,$, 
so for example $\g_3(x) = \f(\f(\f(x)))\,$.
In separate diagrams, sketch graphs of $\g_1\,$, $\g_2\,$, $\g_3\,$ and~$\g_4\,$.



The function $\h$ is defined by
\[
\h(x) = \left\vert \sin {{{\pi}x} \over 2} \right\vert\;,
\]
where  the domain is ${\bf R}\,$. Show that if $n$ is even,
\[
\int_0^n\,\big( \h(x)-\g_n(x)\big)\,\d x = \frac{2n}{\pi} -\frac{n}2\;.
\]
\end{question}
	
%%%%%%%%% Q7
\begin{question}
Show that, 
 if $n>0\,$, then
$$
\int_{e^{1/n}}^\infty\,{{\ln x} \over {x^{n+1}}}\,\d x
=  {2 \over  {n^2\e}}\;.  
$$
You may assume that $\ds \frac{\ln x} x \to 0\;$ as $x\to\infty\,$.


Explain why, if $1<a<b\,$, then
$$
\int_b^\infty\,{{\ln x} \over {x^{n+1}}}\,\d x  
<
\int_a^\infty\,{{\ln x} \over {x^{n+1}}}\,\d x\;.
$$


Deduce that
$$
\sum_{n=1}^{N}{1 \over n^2} <
{\e \over 2}\int_{\e^{1/N}}^{\infty}
\left({1-x^{-N}} \over {x^2-x}\right) \ln x\,\d x\;,
$$
where  $N\,$ is any integer greater than $1$.
\end{question}
		
%%%%%%%%% Q8
\begin{question}	
It is given that $y$ satisfies
 $$
{{\d y} \over { \d t}} + 
k\left({{t^2-3t+2} \over {t+1}}\right)y = 0\;,
$$
where $k$ is a constant,  and $y=A $ when $t=0\,$, where $A$ is a positive constant.
Find $y$ in terms of $t\,$, $k$ and $A\,$.


Show that $y$ has two stationary values whose 
ratio is 
$(3/2)^{6k}
\e^{-5{k}/2}.$


Describe the behaviour of $y$ as $t \to +\infty$ for the case
where $k> 0$ and for the case where $k<0\,.$

In separate diagrams, sketch the graph of $y$  for $t>0$
for each of these cases. 
\end{question}	
		

		
	
\newpage
\section*{Section B: \ \ \ Mechanics}


	
%%%%%%%%%% Q9
\begin{question}
$AB$ is a uniform rod of weight $W\,$.
The point $C$  on $AB$ is such that $AC>CB\,$. The 
rod is in contact with a rough horizontal floor  at $A\,$ 
and with a cylinder at $C\,$. The cylinder   
is fixed to the floor with its axis horizontal.
The rod makes an angle ${\alpha}$ with 
the horizontal and lies in a vertical plane perpendicular to
the axis of the cylinder.
The coefficient  of friction between the rod
and the floor  is $\tan \lambda_1$ and the coefficient of friction
between the rod and the 
cylinder is $\tan \lambda_2\,$.  

Show that if friction is limiting 
both at $A$ and at $C$, and ${\alpha} \ne {\lambda}_2 - {\lambda}_1\,$, 
then the frictional force acting on the rod 
at $A$ has magnitude 
$$
\frac{ W\sin {\lambda}_1  \, \sin({\alpha}-{\lambda}_2)}
{\sin ({\alpha}+{\lambda}_1-{\lambda}_2)}
\;.$$
%and that
%$$
%p=\frac{\cos{\alpha} \, \sin({\alpha}+{\lambda}_1-{\lambda}_2)}
%{2\cos{\lambda}_1 \, \sin {\lambda}_2}\;.
%$$
	\end{question}
	
%%%%%%%%%% Q10 
\begin{question}	
A bead $B$ of mass $m$ can slide along a rough horizontal wire.
A light inextensible string of length $2\ell$ has one end attached 
to a fixed point $A$
of the wire and the other to $B\,$. 
A particle $P$ of mass $3m$ is attached to the mid-point of the string
and $B$ is held at a distance
$\ell$ from~$A\,$. The bead is released from rest.


Let $a_1$ and $a_2$ be the magnitudes of the horizontal and vertical components of the
initial acceleration of $P\,$. Show by considering the motion of $P$ relative to $A\,$, or 
otherwise, that $a_1= \sqrt 3 a_2\,$. Show also that the magnitude
of the initial acceleration of $B$ is 
$2a_1\,$.

Given that the frictional force opposing the motion
of $B$ is equal to $({\sqrt{3}}/6)R$, where $R$ is the normal reaction 
between $B$ and the wire, show that the magnitude of the initial acceleration of 
$P$ is~$g/18\,$.
\end{question}

%%%%%%%%%% Q11

\begin{question}
A particle $P_1$ is projected with speed $V$ at an angle of elevation
${\alpha}\,\,\,( > 45^{\circ})\,,\,\,\,$ 
from a point in a horizontal plane. 
Find $T_1$, the flight time of $P_1$, in terms of
${\alpha}, V \hbox{    and    } g\,$.
Show that the time after projection at 
which the direction of motion of $P_1$ first 
makes an angle of 
$45^{\circ}$ with the horizontal is $\frac12 (1-\cot \alpha)T_1\,$.


A particle $P_2$  is projected 
under the same conditions.
When the direction of the motion of $P_2$ 
first makes an angle of  $45^{\circ}$ with the horizontal, the speed of
$P_2$ is instantaneously doubled. If $T_2$ is the total flight time of
$P_2$,  show that 
$$
\frac{2T_2}{T_1}
=  1+\cot{\alpha}
+\sqrt{1+3\cot^2{\alpha}} \;.
$$ 
\end{question}
	

	
	\newpage
\section*{Section C: \ \ \ Probability and Statistics}


%%%%%%%%%% Q12
\begin{question}
The life of a certain species of elementary particles
can be described as follows. Each particle has a life time 
of $T$ seconds, after which it disintegrates into $X$ particles
of the same species, where $X$ is a random variable with 
binomial distribution $\mathrm{B}(2,p)\,$.
A population of these particles starts with the creation
of a single such particle at $t=0\,$. Let $X_n$ be 
the number of particles in existence in the time interval 
$nT < t < (n+1)T\,$, where $n=1\,$, $2\,$, $\ldots$.  

Show that $\P(X_1=2 \mbox { and } X_2=2) = 6p^4q^2\;$, where $q=1-p\,$.
Find the possible values of $p$ if it is known that 
$\P(X_1=2 \vert X_2=2) =9/25\,$.

Explain briefly why  $\E(X_n) =2p\E(X_{n-1})$ and hence determine $\E(X_n)$
in terms of $p$.
Show that for one of the values of $p$ found above  
$\lim_{n \to \infty}\E(X_n) = 0$ 
and that for the other
$\lim_{n \to \infty}\E(X_n) = + \infty\,$. 
\end{question}

%%%%%%%%%% Q13
\begin{question}
The random variable $X$ takes the values $k=1$, $2$, $3$, $\dotsc$,
and has probability distribution 
$$
\P(X=k)= A{{{\lambda}^k\e^{-{\lambda}}} \over {k!}}\,,
$$
where $\lambda $ is a positive constant.
Show that $A = (1-\e^{-\lambda})^{-1}\,$. Find
the mean ${\mu}$ in terms of  ${\lambda}$  and show that
$$
\var(X) = {\mu}(1-{\mu}+{\lambda})\;.
$$
Deduce that ${\lambda} < {\mu} < 1+{\lambda}\,$.

Use a normal approximation to find
the value of $P(X={\lambda})$
in the case where ${\lambda}=100\,$, giving your answer to 2 decimal places.
\end{question}

%%%%%%%%%% Q14
\begin{question}
The probability of throwing a 6 with a biased die
is $p\,$. It is known that  $p$
is equal to one or other of the numbers $A$ and $B$
where  $0< A <B<1 \,$. 
Accordingly the following  statistical 
test of the hypothesis $H_0: \,p=B$ against the alternative
hypothesis $H_1: \,p=A$ is performed.

The die is thrown  repeatedly until a 6 is obtained. Then if
$X$ is the total number of throws,  $H_0$ is accepted if $X \le M\,$,
where $M$ is a given positive integer;  otherwise $H_1$ is accepted.
Let ${\alpha}$ be the probability that $H_1$ is accepted
if $H_0$ is true, and let ${\beta}$ be  the probability that $H_0$
is accepted if $H_1$ is true.  

Show that ${\beta} = 1- {\alpha}^K,$ where $K$ is independent of $M$ and 
is to be determined in terms of 
$A$ and $B\,$.  Sketch the graph of ${\beta}$ 
against ${\alpha}\,$.
\end{question}
	
\end{document}
