
\documentclass[a4, 11pt]{report}


\pagestyle{myheadings}
\markboth{}{Paper II, 1988
\ \ \ \ \ 
\today 
}               

\RequirePackage{amssymb}
\RequirePackage{amsmath}
\RequirePackage{graphicx}
\RequirePackage{color}
\RequirePackage[flushleft]{paralist}[2013/06/09]



\RequirePackage{geometry}
\geometry{%
  a4paper,
  lmargin=2cm,
  rmargin=2.5cm,
  tmargin=3.5cm,
  bmargin=2.5cm,
  footskip=12pt,
  headheight=24pt}


\newcommand{\comment}[1]{{\bf Comment} {\it #1}}
%\renewcommand{\comment}[1]{}

\newcommand{\bluecomment}[1]{{\color{blue}#1}}
%\renewcommand{\comment}[1]{}
\newcommand{\redcomment}[1]{{\color{red}#1}}



\usepackage{epsfig}
\usepackage{tgheros} %% changes sans-serif font to TeX Gyre Heros (tex-gyre)
\renewcommand{\familydefault}{\sfdefault} %% changes font to sans-serif
%\usepackage{sfmath}  %%%% this makes equation sans-serif
%\input RexFigs


\setlength{\parskip}{10pt}
\setlength{\parindent}{0pt}

\newlength{\qspace}
\setlength{\qspace}{20pt}


\newcounter{qnumber}
\setcounter{qnumber}{0}

\newenvironment{question}%
 {\vspace{\qspace}
  \begin{enumerate}[\bfseries 1\quad][10]%
    \setcounter{enumi}{\value{qnumber}}%
    \item%
 }
{
  \end{enumerate}
  \filbreak
  \stepcounter{qnumber}
 }


\newenvironment{questionparts}[1][1]%
 {
  \begin{enumerate}[\bfseries (i)]%
    \setcounter{enumii}{#1}
    \addtocounter{enumii}{-1}
    \setlength{\itemsep}{5mm}
    \setlength{\parskip}{8pt}
 }
 {
  \end{enumerate}
 }



\DeclareMathOperator{\cosec}{cosec}
\DeclareMathOperator{\Var}{Var}

\def\d{{\rm d}}
\def\e{{\rm e}}
\def\g{{\rm g}}
\def\h{{\rm h}}
\def\f{{\rm f}}
\def\p{{\rm p}}
\def\s{{\rm s}}
\def\t{{\rm t}}


\def\A{{\rm A}}
\def\B{{\rm B}}
\def\E{{\rm E}}
\def\F{{\rm F}}
\def\G{{\rm G}}
\def\H{{\rm H}}
\def\P{{\rm P}}


\def\bb{\mathbf b}
\def \bc{\mathbf c}
\def\bx {\mathbf x}
\def\bn {\mathbf n}

\newcommand{\low}{^{\vphantom{()}}}
%%%%% to lower suffices: $X\low_1$ etc


\newcommand{\subone}{ {\vphantom{\dot A}1}}
\newcommand{\subtwo}{ {\vphantom{\dot A}2}}




\def\le{\leqslant}
\def\ge{\geqslant}


\def\var{{\rm Var}\,}

\newcommand{\ds}{\displaystyle}
\newcommand{\ts}{\textstyle}




\begin{document}
\setcounter{page}{2}

 
\section*{Section A: \ \ \ Pure Mathematics}

%%%%%%%%%%Q1
\begin{question}
The function $\mathrm{f}$ is defined, for $x\neq1$ and $x\neq2$
by
\[
\mathrm{f}(x)=\frac{1}{\left(x-1\right)\left(x-2\right)}
\]
Show that for $\left|x\right|<1$
\[
\mathrm{f}(x)=\sum_{n=0}^{\infty}x^{n}-\frac{1}{2}\sum_{n=0}^{\infty}\left(\frac{x}{2}\right)^{n}
\]
and that for $1<\left|x\right|<2$ 
\[
\mathrm{f}(x)=-\sum_{n=1}^{\infty}x^{-n}-\frac{1}{2}\sum_{n=0}^{\infty}\left(\frac{x}{2}\right)^{n}
\]
Find an expression for $\mbox{f}(x)$ which is valid for $\left|x\right|>2$.
\end{question}

%%%%%%%%%%Q2
\begin{question}
The numbers $x,y$ and $z$ are non-zero, and satisfy
\[
2a-3y=\frac{\left(z-x\right)^{2}}{y}\quad\mbox{ and }\quad2a-3z=\frac{\left(x-y\right)^{2}}{z},
\]
for some number $a$. If $y\neq z$, prove that 
\[
x+y+z=a,
\]
and that 
\[
2a-3x=\frac{\left(y-z\right)^{2}}{x}.
\]
Determine whether this last equation holds \textit{only} if $y\neq z$. 
\end{question}

%%%%%%%%% Q3
\begin{question}
The quadratic equation $x^{2}+bx+c=0$, where $b$ and $c$ are real,
has the properly that if $k$ is a (possibly complex) root, then $k^{-1}$
is a root. Determine carefully the restriction that this property
places on $b$ and $c$. If, in addition to this property, the equation
has the further property that if $k$ is a root, then $1-k$ is a
root, find $b$ and $c$. 


Show that 
\[
x^{3}-\tfrac{3}{2}x^{2}-\tfrac{3}{2}x+1=0
\]
is the only cubic equation of the form $x^{3}+px^{2}+qx+r=0$, where
$p,q$ and $r$ are real, which has both the above properties.
\end{question}

%%%%%% Q4 

\begin{question}
The complex number $w$ is such that $w^{2}-2x$ is real. 

\begin{itemize}
\setlength{\itemsep}{3mm}
\item[\bf (i)] Sketch the locus of $w$ in the Argand diagram. 
\item[\bf (ii)] If $w^{2}=x+\mathrm{i}y,$ describe fully and sketch the locus of
points $(x,y)$ in the $x$-$y$ plane. 
\end{itemize}

The complex number $t$ is such that $t^{2}-2t$ is imaginary. If
$t^{2}=p+\mathrm{i}q$, sketch the locus of points $(p,q)$ in the
$p$-$q$ plane. 


\end{question}


%%%%%%%%% Q5
\begin{question}
	By considering the imaginary part of the equation $z^{7}=1,$ or otherwise,
	find all the roots of the equation 
	\[
	t^{6}-21t^{4}+35t^{2}-7=0.
	\]
	You should justify each step carefully. 


	Hence, or otherwise, prove that 
	\[
	\tan\frac{2\pi}{7}\tan\frac{4\pi}{7}\tan\frac{6\pi}{7}=\sqrt{7}.
	\]
	Find the corresponding result for 
	\[
	\tan\frac{2\pi}{n}\tan\frac{4\pi}{n}\cdots\tan\frac{(n-1)\pi}{n}
	\]
	in the two cases $n=9$ and $n=11.$
	\end{question}
	
%%%%%%%%% Q6
\begin{question}
Show that the following functions are positive when $x$ is positive: 

\begin{itemize}
 \setlength{\itemsep}{3mm}
\item[\bf (i)] $x-\tanh x$
\item[\bf (ii)] $x\sinh x-2\cosh x+2$
\item[\bf (iii)] $2x\cosh2x-3\sinh2x+4x$. 
\end{itemize}

The function $\mathrm{f}$ is defined for $x>0$ by 
\[
\mathrm{f}(x)=\frac{x(\cosh x)^{\frac{1}{3}}}{\sinh x}.
\]
Show that $\mathrm{f}(x)$ has no turning points when $x>0,$ and
sketch $\mathrm{f}(x)$ for $x>0.$ 

\end{question}
	 
	 %%%%%%%%% Q7
\begin{question}
The integral $I$ is defined by 
\[
I=\int_{1}^{2}\frac{(2-2x+x^{2})^{k}}{x^{k+1}}\,\mathrm{d}x
\]
where $k$ is a constant. Show that
\[
I=\int_{0}^{1}\frac{(1+x^{2})^{k}}{(1+x)^{k+1}}\,\mathrm{d}x=\int_{0}^{\frac{1}{4}\pi}\frac{\mathrm{d}\theta}{\left[\sqrt{2}\cos\theta\cos\left(\frac{1}{4}\pi-\theta\right)\right]^{k+1}}=2\int_{0}^{\frac{1}{8}\pi}\frac{\mathrm{d}\theta}{\left[\sqrt{2}\cos\theta\cos\left(\frac{1}{4}\pi-\theta\right)\right]^{k+1}}.
\]
Hence show that 
\[
I=2\int_{0}^{\sqrt{2}-1}\frac{(1+x^{2})^{k}}{(1+x)^{k+1}}\,\mathrm{d}x
\]
Deduce that
\[
\int_{1}^{\sqrt{2}}\left(\frac{2-2x^{2}+x^{4}}{x^{2}}\right)^{k}\frac{1}{x}\,\mathrm{d}x=\int_{1}^{\sqrt{2}}\left(\frac{2-2x+x^{2}}{x}\right)^{k}\frac{1}{x}\,\mathrm{d}x
\]
\end{question}
	
%%%%%%%%% Q8
\begin{question}
In a crude model of population dynamics of a community of aardvarks
and buffaloes, it is assumed that, if the numbers of aardvarks and
buffaloes in any year are $A$ and $B$ respectively, then the numbers
in the following year at $\frac{1}{4}A+\frac{3}{4}B$ and $\frac{3}{2}B-\frac{1}{2}A$
respectively. It does not matter if the model predicts fractions of
animals, but a non-positive number of buffaloes means that the species
has become extinct, and the model ceases to apply. Using matrices
or otherwise, show that the ratio of the number of aardvarks to the
number of buffaloes can remain the same each year, provided it takes
one of two possible values. 


Let these two possible values be $x$ and $y$, and let the numbers
of aardvarks and buffaloes in a given year be $a$ and $b$ respectively.
By writing the vector $(a,b)$ as a linear combination of the vectors
$(x,1)$ and $(y,1),$ or otherwise, show how the numbers of aardvarks
and buffaloes in subsequent years may be found. On a sketch of the
$a$-$b$ plane, mark the regions which correspond to the following
situations 
\begin{itemize}
 \setlength{\itemsep}{3mm}
\item[\bf (i)] an equilibrium population is reached as time $t\rightarrow\infty$; 
\item[\bf (ii)] buffaloes become extinct after a finite time; 
\item[\bf (iii)] buffaloes approach extinction as $t\rightarrow\infty.$ 
\end{itemize}
\end{question}
		
		
%%%%%%%%% Q9
\begin{question}
Give a careful argument to show that, if $G_{1}$ and $G_{2}$ are
subgroups of a finite group $G$ such that every element of $G$ is
either in $G_{1}$ or in $G_{2},$ then either $G_{1}=G$ or $G_{2}=G$. 


Give an example of a group $H$ which has three subgroups $H_{1},H_{2}$
and $H_{3}$ such that every element of $H$ is either in $H_{1},H_{2}$
or $H_{3}$ and $H_{1}\neq H,H_{2}\neq H,H_{3}\neq H$. 
\end{question}
		
	
%%%%%%%%%% 10
\begin{question}
The surface $S$ in 3-dimensional space is described by the equation
\[
\mathbf{a}\cdot\mathbf{r}+ar=a^{2},
\]
where $\mathbf{r}$ is the position vector with respect to the origin
$O$, $\mathbf{a}(\neq\mathbf{0})$ is the position vector of a fixed
point, $r=\left|\mathbf{r}\right|$ and $a=\left|\mathbf{a}\right|.$
Show, with the aid of a diagram, that $S$ is the locus of points
which are equidistant from the origin $O$ and the plane $\mathbf{r}\cdot\mathbf{a}=a^{2}.$


The point $P$, with position vector $\mathbf{p},$ lies in $S$,
and the line joining $P$ to $O$ meets $S$ again at $Q$. Find the
position vector of $Q$. 


The line through $O$ orthogonal to $\mathbf{p}$ and $\mathbf{a}$
meets $S$ at $T$ and $T'$. Show that the position vectors of $T$
and $T'$ are 
\[
\pm\frac{1}{\sqrt{2ap-a^{2}}}\mathbf{a}\times\mathbf{p},
\]
where $p=\left|\mathbf{p}\right|.$ 


Show that the area of the triangle $PQT$ is 
\[
\frac{ap^{2}}{2p-a}.
\]
			\end{question}
			
		
		
		
	
\newpage
\section*{Section B: \ \ \ Mechanics}


	
%%%%%%%%%% Q11
\begin{question}
A heavy particle lies on a smooth horizontal table, and is attached
to one end of a light inextensible string of length $L$. The other
end of the string is attached to a point $P$ on the circumference
of the base of a vertical post which is fixed into the table. The
base of the post is a circle of radius $a$ with its centre at a point
$O$ on the table. Initially, at time $t=0$, the string is taut and
perpendicular to the line $OP.$ The particle is then struck in such
a way that the string starts winding round the post and remains taut.
At a later time $t$, a length $a\theta(t)\ (<L)$ of the string is
in contact with the post. Using cartesian axes with origin $O$, find
the position and velocity vectors of the particle at time $t$ in
terms of $a,L,\theta$ and $\dot{\theta},$ and hence show that the
speed of the particle is $(L-a\theta)\dot{\theta}.$ 


If the initial speed of the particle is $v$, show that the particle
hits the post at a time $L^{2}/(2av).$ 

	\end{question}
	
%%%%%%%%%% Q12
\begin{question}	
One end of a thin uniform inextensible, but perfectly flexible, string
of length $l$ and uniform mass per unit length is held at a point
on a smooth table a distance $d(<l)$ away from a small vertical hole
in the surface of the table. The string passes through the hole so
that a length $l-d$ of the string hangs vertically. The string is
released from rest. Assuming that the height of the table is greater
than $l$, find the time taken for the end of the string to reach
the top of the hole. 

\end{question}

%%%%%%%%%% Q13

\begin{question}
A librarian wishes to pick up a row of identical books from a shelf,
by pressing her hands on the outer covers of the two outermost books
and lifting the whole row together. The covers of the books are all
in parallel vertical planes, and the weight of each book is $W$.
With each arm, the librarian can exert a maximum force of $P$ in
the vertical direction, and, independently, a maximum force of $Q$
in the horizontal direction. The coefficient of friction between each
pair of books and also between each hand and a book is $\mu.$ Derive
an expression for the maximum number of books that can be picked up
without slipping, using this method. 


{[}You may assume that the books are thin enough for the rotational
effect of the couple on each book to be ignored.{]}
\end{question}
	
%%%%%%%%%% Q14
\begin{question}
Two particles of mass $M$ and $m$ $(M>m)$ are attached to the ends
of a light rod of length $2l.$ The rod is fixed at its midpoint to
a point $O$ on a horizontal axle so that the rod can swing freely
about $O$ in a vertical plane normal to the axle. The axle rotates
about a \textit{vertical }axis through $O$ at a constant angular
speed $\omega$ such that the rod makes a constant angle $\alpha$
$(0<\alpha<\frac{1}{2}\pi)$ with the vertical. Show that 
\[
\omega^{2}=\left(\frac{M-m}{M+m}\right)\frac{g}{l\cos\alpha}.
\]



Show also that the force of reaction of the rod on the axle is inclined
at an angle 
\[
\tan^{-1}\left[\left(\frac{M-m}{M+m}\right)^{2}\tan\alpha\right]
\]
with the downward vertical. 
\end{question}
	
	\newpage
\section*{Section C: \ \ \ Probability and Statistics}


%%%%%%%%%% Q15
\begin{question}
An examination consists of several papers, which are marked independently.
The mark given for each paper can be an integer from $0$ to $m$
inclusive, and the total mark for the examination is the sum of the
marks on the individual papers. In order to make the examination completely
fair, the examiners decide to allocate the mark for each paper at
random, so that the probability that any given candidate will be allocated
$k$ marks $(0\leqslant k\leqslant m)$ for a given paper is $(m+1)^{-1}$.
If there are just two papers, show that the probability that a given
candidate will receive a total of $n$ marks is 
\[
\frac{2m-n+1}{\left(m+1\right)^{2}}
\]
for $m<n\leqslant2m$, and find the corresponding result for $0\leqslant n\leqslant m$. 


If the examination consists of three papers, show that the probability
that a given candidate will receive a total of $n$ marks is 
\[
\frac{6mn-4m^{2}-2n^{2}+3m+2}{2\left(m+1\right)^{2}}
\]
in the case $m<n\leqslant2m$. Find the corresponding result for $0\leqslant n\leqslant m$,
and deduce the result for $2m<n\leqslant3m$. 
\end{question}

%%%%%%%%%% Q16
\begin{question}
Find the probability that the quadratic equation 
\[
X^{2}+2BX+1=0
\]
has real roots when $B$ is normally distributed with zero mean and
unit variance. 


Given that the two roots $X_{1}$ and $X_{2}$ are real, find: 
\begin{itemize}
\setlength{\itemsep}{3mm}
\item[\bf (i)] the probability that both $X_{1}$ and $X_{2}$ are greater than $\frac{1}{5}$;
\item[\bf (ii)] the expected value of $\left|X_{1}+X_{2}\right|$; 
\end{itemize}

giving your answers to three significant figures.
\end{question}
\end{document}
