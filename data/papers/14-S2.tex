
\documentclass[a4, 11pt]{report}


\pagestyle{myheadings}
\markboth{}{Paper II, 2014
\ \ \ \ \ 
\today 
}               
\RequirePackage{amssymb}
\RequirePackage{amsmath}
\RequirePackage{graphicx}
\RequirePackage{color}
\RequirePackage[flushleft]{paralist}[2013/06/09]
\RequirePackage{asymptote}


\RequirePackage{geometry}
\geometry{%
  a4paper,
  lmargin=2cm,
  rmargin=2.5cm,
  tmargin=3.5cm,
  bmargin=2.5cm,
  footskip=12pt,
  headheight=24pt}
 

\newcommand{\comment}[1]{{\bf Comment} {\it #1}}
%\renewcommand{\comment}[1]{}

\newcommand{\bluecomment}[1]{{\color{blue}#1}}
%\renewcommand{\comment}[1]{}
\newcommand{\redcomment}[1]{{\color{red}#1}}



\usepackage{epsfig}
\usepackage{pstricks-add}
\usepackage{tgheros} %% changes sans-serif font to TeX Gyre Heros (tex-gyre)
\renewcommand{\familydefault}{\sfdefault} %% changes font to sans-serif
%\usepackage{sfmath}  %%%% this makes equation sans-serif
%\input RexFigs


\setlength{\parskip}{10pt}
\setlength{\parindent}{0pt}

\newlength{\qspace}
\setlength{\qspace}{20pt}


\newcounter{qnumber}
\setcounter{qnumber}{0}

\newenvironment{question}%
 {\vspace{\qspace}
  \begin{enumerate}[\bfseries 1\quad][10]%
    \setcounter{enumi}{\value{qnumber}}%
    \item%
 }
{
  \end{enumerate}
  \filbreak
  \stepcounter{qnumber}
 }


\newenvironment{questionparts}[1][1]%
 {
  \begin{enumerate}[\bfseries (i)]%
    \setcounter{enumii}{#1}
    \addtocounter{enumii}{-1}
    \setlength{\itemsep}{5mm}
    \setlength{\parskip}{8pt}
 }
 {
  \end{enumerate}
 }



\DeclareMathOperator{\cosec}{cosec}
\DeclareMathOperator{\Var}{Var}

\def\d{{\mathrm d}}
\def\e{{\mathrm e}}
\def\g{{\mathrm g}}
\def\h{{\mathrm h}}
\def\f{{\mathrm f}}
\def\p{{\mathrm p}}
\def\s{{\mathrm s}}
\def\t{{\mathrm t}}
\def\i{{\mathrm i}}

\def\A{{\mathrm A}}
\def\B{{\mathrm B}}
\def\E{{\mathrm E}}
\def\F{{\mathrm F}}
\def\G{{\mathrm G}}
\def\H{{\mathrm H}}
\def\P{{\mathrm P}}


\def\bb{\mathbf b}
\def \bc{\mathbf c}
\def\bx {\mathbf x}
\def\bn {\mathbf n}

\newcommand{\low}{^{\vphantom{()}}}
%%%%% to lower suffices: $X\low_1$ etc


\newcommand{\subone}{ {\vphantom{\dot A}1}}
\newcommand{\subtwo}{ {\vphantom{\dot A}2}}

\begin{asydef}
  import markers;
  import geometry;
  import graph;
  usepackage("amsmath");
\end{asydef}


\def\le{\leqslant}
\def\ge{\geqslant}
\def\arcosh{{\rm arcosh}\,}


\def\var{{\rm Var}\,}

\newcommand{\ds}{\displaystyle}
\newcommand{\ts}{\textstyle}
\def\half{{\textstyle \frac12}}
\def\l{\left(}
\def\r{\right)}
\renewcommand{\.}[1]{\ensuremath{\mathrm{#1}}}
\newcommand{\+}[1]{\ensuremath{\mathbf{#1}}}
\newcommand{\ud}{\mathop{}\!\mathrm{d}}



\begin{document}
\setcounter{page}{2}

 
\section*{Section A: \ \ \ Pure Mathematics}

%%%%%%%%%%Q1
\begin{question}
  In the triangle $ABC$, the base $AB$ is of length 1 unit and the
  angles at~$A$ and~$B$ are $\alpha$ and~$\beta$ respectively, where
  $0<\alpha\le\beta$.  The points $P$ and~$Q$ lie on the sides $AC$ and
  $BC$ respectively, with $AP=PQ=QB=x$.  The line $PQ$ makes an angle
  of~$\theta$ with the line through~$P$ parallel to~$AB$.
  \begin{questionparts}
  \item Show that $x\cos\theta = 1- x\cos\alpha - x\cos\beta$, and 
 obtain an expression for $x\sin\theta$  in
    terms of $x$, $\alpha$ and~$\beta$.  Hence show that
    \begin{equation}
      \label{eq:2*}
      \bigl(1+2\cos(\alpha+\beta)\bigr)x^2 - 2(\cos\alpha +
      \cos\beta)x + 1 = 0\,. \tag{$*$}
    \end{equation}
Show that $(*)$ is also satisfied if $P$ and $Q$ lie  on  
$AC$ produced and $BC$ produced, respectively.  [By definition,
$P$ lies on $AC$ produced if $P$ lies on the line through $A$ and~$C$
and the points are in the order $A$, $C$, $P$\,.]

  \item
State the condition on $\alpha$ and $\beta$ for 
   $(*)$
 to be  linear in $x$.
If this condition does not hold (but the condition
$0<\alpha \le \beta$ still holds), show that 
$(*)$ has
distinct real roots.

  \item Find the possible values of~$x$ in the two  cases (a)
 $\alpha = \beta =    45^\circ$ and (b) 
$\alpha = 30^\circ$, $\beta = 90^\circ$, and 
illustrate each case with a sketch. 
\end{questionparts}
\end{question}

%%%%%%%%%%Q2
\begin{question}
  This question concerns the  inequality
  \begin{equation}
    \label{eq6:*}
    \int_0^\pi  \bigl( \.f(x) \bigr)^2 \ud x \le \int_0^\pi \bigl(
    \.f'(x)\bigr)^2 \ud x\,.\tag{$*$}
  \end{equation}

  \begin{questionparts}
  \item Show that $(*)$ is satisfied in the case
    $\.f(x)=\sin nx$, where $n$~is a positive integer.

Show by means of counterexamples that  $(*)$ is not
necessarily satisfied if either $\.f(0) \ne 0$ or $\.f(\pi)\ne0$.
 

\item  You may now assume that
 $(*)$ is satisfied for any 
(differentiable) function~$\.f$ for which $\.f(0)=\.f(\pi)=0$.

 By setting $\.f(x) = ax^2 + bx +c$,
 where $a$, $b$ and $c$ are suitably chosen, show that 
$\pi^2\le 10$.

 By setting $\.f(x) = p \sin \frac12 x + q\cos \frac12 x +r$,
    where $p$, $q$ and $r$ are suitably chosen, obtain another inequality
    for $\pi$.

Which of these inequalities leads to a better estimate for $\pi^2\,$?
  \end{questionparts}
\end{question}

%%%%%%%%% Q3
\begin{question}
\begin{questionparts}
\item
Show, geometrically or otherwise, that the shortest distance between the 
origin and the line
$y= mx+c$, where $c\ge0$,  is $c(m^2+1)^{-\frac12}$. 

\item
  The curve $C$ lies in the $x$-$y$ plane. Let the line $L$ be tangent
  to~$C$ at a point~$P$ on~$C$, and let $a$~be the shortest distance
  between the origin and $L$.  The curve~$C$ has
  the property that the distance~$a$ is the same for all points~$P$
  on~$C$.

 Let $P$ be the point on $C$ with coordinates $(x,y(x))$. Given that the 
tangent to $C$ at $P$ is not vertical, show that
    \begin{equation}
      \label{eq:8*}
  (y-xy')^2 =  a^2\big (1+(y')^2 \big) 
\,.
  \tag{$*$}
    \end{equation}

 By first differentiating $(*)$ with respect to $x$,
show
that either $y= mx \pm a(1+m^2)^{\frac12}$ for some $m$
or $x^2+y^2 =a^2$. 

\item  Now suppose that $C$ (as defined above) is a continuous curve
for $-\infty < x < \infty$,  consisting of the arc of a circle and two
straight lines. Sketch an example of 
such a curve which has a non-vertical tangent at each point.

  \end{questionparts}
\end{question}

%%%%%% Q4 
\begin{question}
  \begin{questionparts}
  \item 
By using the substitution $u=1/x$, 
show that for
    $b>0$
    \[
    \int_{1/b}^b \frac{x \ln x}{(a^2+x^2)(a^2x^2+1)} \ud x =0 \,.
    \]

  \item 
By using the substitution $u=1/x$, 
show that for $b>0$,
    \[
    \int_{1/b}^b \frac{\arctan x}{x} \ud x = \frac{\pi \ln b} 2\,.
    \]

  \item 
By using the result 
$ \displaystyle \int_0^\infty \frac 1 {a^2+x^2} \ud x = \frac {\pi}{2 a} $
\ (where $a>0$),
and a substitution of the 
\\[5pt]
form $u=k/x$, for suitable $k$,
show that
    \[
    \int_0^\infty  \frac 1 {(a^2+x^2)^2} \ud x =  \frac {\pi}{4a^3 }
\, \ \ \ \ \ \ (a>0).
    \]
  \end{questionparts}
\end{question}

%%%%%%%%% Q5
\begin{question}
  Given that $y=xu$, where $u$ is a function of $x$, write down an
  expression for $\dfrac {\.dy}{\.dx}$.

  \begin{questionparts}
  \item Use the substitution $y=xu$ to solve
    \[
    \frac {\.dy}{\.dx} = \frac {2y+x}{y-2x}
    \]
    given that the solution curve passes through the point $(1,1)$.

    Give your answer in the form of a quadratic in $x$ and~$y$.

  \item Using the substitutions $x=X+a$ and $y=Y+b$ for appropriate values
    of $a$ and~$b$, or otherwise, solve
    \[
    \frac {\.dy}{\.dx} = \frac {x-2y-4} {2x+y-3}\,,
    \]
    given that the solution curve passes through the point $(1,1)$.

  \end{questionparts}

\end{question}
	
%%%%%%%%% Q6
\begin{question}
  By simplifying $\sin(r+\frac12)x - \sin(r-\frac12)x$ or
  otherwise show that, for $\sin\frac12 x \ne0$,
  \[
  \cos x + \cos 2x +\cdots + \cos nx = \frac{\sin(n+\frac12)x -
    \sin\frac12 x}{2\sin\frac12x}\,.
  \]

  The functions $\.S_n$, for $n=1$, $2$, \dots, are defined
  by
  \[
  \.S_n(x) = \sum_{r=1}^n \frac 1 r \sin rx \qquad (0\le
  x \le \pi).
  \]

  \begin{questionparts}
  \item Find the stationary points of $\.S_2(x)$ for $0\le x\le\pi$,
    and sketch this function.

  \item Show that if $\.S_n(x)$ has a stationary point at $x=x_0$,
    where $0< x_0 < \pi$, then
    \[
    \sin nx_0 = (1-\cos nx_0) \tan\tfrac12 x_0
    \]
    and hence that $\.S_n(x_0) \ge \.S_{n-1}(x_0)$.  Deduce that if
    $\.S_{n-1}(x)>0$ for all $x$ in the interval $0<x<\pi$, 
    then $\.S_{n}(x)>0$ for all $x$ in this interval.

  \item Prove that $\.S_n(x)\ge0$ for $n\ge1$ and 
 $0\le x\le\pi$.
  \end{questionparts}
\end{question}
	
%%%%%%%%% Q7
\begin{question}
  \begin{questionparts}
  \item The function $\.f$ is defined by $\.f(x)= |x-a| + |x-b| $,
    where $a<b$.  Sketch the graph of~$\.f(x)$, giving the gradient in
    each of the regions $x<a$, $a<x<b$ and $x>b$.  Sketch on the same
    diagram the graph of $\.g(x)$, where $\.g(x)= |2x-a-b|$.

    What shape is the quadrilateral with vertices $(a,0)$, $(b,0)$,
$(b,\.f(b))$ and 
    $(a, \.f(a))$? 

  \item Show graphically that the equation
    \[
    |x-a| + |x-b| = |x-c|\,,
    \]
    where $a<b$, has $0$, $1$ or $2$ solutions, stating the
    relationship of $c$ to $a$ and~$b$ in each case.

  \item For the equation
    \[
    |x-a| + |x-b| = |x-c|+|x-d|\,,
    \]
    where $a<b$, $c<d$ and $d-c <b-a$, determine the number of
    solutions in the various cases that arise, stating the
    relationship between $a$,~$b$, $c$ and~$d$ in each case.
  \end{questionparts}
\end{question}
		
%%%%%%%%% Q8
\begin{question}
For positive integers $n$, $a$ and $b$, the integer $c_r$
  ($0\le r\le n$) is defined to be the coefficient of~$x^r$ in the
  expansion in powers of $x$ of $(a+bx)^n$. Write down an expression
  for $c_r$ in terms of $r$, $n$, $a$ and~$b$.

  For given $n$, $a$ and $b$, let $m$~denote a value of~$r$ for which
  $c_r$~is greatest (that is, $c_m \ge c_r$ for $0\le r\le n$).

  Show that
  \[
  \frac{b(n+1)}{a+b} - 1 \le m \le \frac {b(n+1)}{a+b} \,.
  \]
  Deduce that $m$ is either a unique integer or one of 
  two consecutive integers.

Let  $\.G(n,a,b)$ denote the unique value
  of~$m$ (if there is one) or
 the larger of the two possible values of~$m$.

  \begin{questionparts}
  \item Evaluate $\.G(9,1,3)$ and $\.G(9,2,3)$.
  \item For any positive integer $k$,
   find $\.G(2k,a,a)$ and $\.G(2k-1,a,a)$ in terms of~$k$.
  \item For fixed $n$ and $b$, determine a value of~$a$ for which
    $\.G(n,a,b)$ is greatest.
  \item For fixed~$n$, find the greatest possible value of
    $\.G(n,1,b)$. For which  values of~$b$ is  this
    greatest value achieved?
  \end{questionparts}
\end{question}	
		

		
	
\newpage
\section*{Section B: \ \ \ Mechanics}


	
%%%%%%%%%% Q9
\begin{question}
  A uniform rectangular lamina $ABCD$ rests in equilibrium in a
  vertical plane with the \mbox{corner~$A$} in contact with a rough vertical
  wall. The plane of the lamina is perpendicular to the wall.
It is supported by a light inextensible string attached to the
  side $AB$ at a distance~$d$ from $A$. The other end of the string is
  attached to a point on the wall above $A$ 
where it makes an acute angle  $\theta$
  with the downwards vertical.  The side $AB$ makes an acute angle
  $\phi$ with the upwards vertical at $A$. The sides $BC$ and $AB$
  have lengths $2a$ and $2b$ respectively. The coefficient of friction
  between the lamina and the wall is~$\mu$.

  \begin{questionparts}
  \item Show that, when the lamina is in limiting equilibrium with the
    frictional force acting upwards,
    \begin{equation}
      \label{eq:9*}
      d\sin(\theta +\phi) = (\cos\theta +\mu \sin\theta)(a\cos\phi
      +b\sin\phi)\,.  \tag{$*$}
    \end{equation}

  \item How should $(*)$~be modified if the lamina is in
    limiting equilibrium with the frictional force acting downwards?

  \item Find a condition on $d$, in terms of $a$, $b$, $\tan\theta$
    and $\tan\phi$, which is necessary and sufficient for the
    frictional force to act upwards. Show that this condition cannot be
satisfied if $b(2\tan\theta+ \tan \phi)<a$.
  \end{questionparts}
	\end{question}
	
%%%%%%%%%% Q10 
\begin{question}	
  A particle is projected from a point $O$ on horizontal ground 
with initial speed $u$ and at an angle of
  $\theta$ above the ground.  The motion takes place in the
  $x$-$y$ plane, where the $x$-axis is horizontal, the $y$-axis is 
vertical and 
 the origin is  $O$.
    Obtain the Cartesian equation of the particle's trajectory in
    terms of $u$, $g$ and~$\lambda$, where $\lambda=\tan\theta$.
    
Now consider the trajectories for different values of $\theta$
    with $u$~fixed.  Show that for a given value of~$x$, the
    coordinate~$y$ can take all values up to a maximum value,~$Y$,
    which you should determine as a function of $x$, $u$ and~$g$.

    Sketch a graph of $Y$ against $x$ and indicate on your graph 
    the set of points that can be reached by a particle projected
    from $O$ with speed $u$.

    Hence find the furthest distance from $O$ that can be achieved 
     by such a projectile.
\end{question}

%%%%%%%%%% Q11

\begin{question}
  A small smooth ring $R$ of mass $m$ is free to slide on a fixed smooth
  horizontal rail. A light inextensible string of length~$L$ is
  attached to one end,~$O$, of the rail.  The string passes through
  the ring, and a particle~$P$ of mass~$km$ (where $k>0$)
is attached to its other
  end; this part of the string hangs at an acute 
  angle $\alpha$ to the vertical and 
  it is given that $\alpha$ is constant in the motion.

  Let $x$ be the distance between $O$ and the ring.  Taking the
  $y$-axis to be vertically upwards, write down the Cartesian
  coordinates of~$P$ relative to~$O$ in terms of $x$, $L$
  and~$\alpha$.
 

\begin{questionparts}
\item
By considering the vertical component of the equation of motion of $P$,
show that
\[
km\ddot x \cos\alpha = T \cos\alpha - kmg\,,
 \]
where $T$ is the tension in the string. Obtain two similar equations
relating to the horizontal components of the equations of motion of 
$P$ and $R$.


 \item Show that
$\dfrac {\sin\alpha}{(1-\sin\alpha)^2_{\vphantom|}} = k$, and
    deduce, by means of a sketch or otherwise, that motion with $\alpha$ 
constant 
 is   possible for all values of~$k$.
\item Show that $\ddot x = -g\tan\alpha\,$.
  \end{questionparts}
\end{question}
	

	
	\newpage
\section*{Section C: \ \ \ Probability and Statistics}


%%%%%%%%%% Q12
\begin{question}
  The lifetime of a fly (measured in hours) is given by the continuous
  random variable~$T$ with probability density function $\.f(t)$ and
  cumulative distribution function $\.F(t)$.  The \emph{hazard
    function}, $\.h(t)$, is defined, for $\.F(t)<1$, by
  \[
  \.h(t) = \frac{\.f(t)}{1-\.F(t)}\,.
  \]

  \begin{questionparts}
\item Given that the fly lives to at least  time $t$, show that the 
probability of its dying within the following $\delta t$ is 
approximately $\.h (t) \, \delta t$ for small values of $\delta t$. 


  \item Find the hazard function in the case $\.F(t) = t/a$ 
   for $0<    t <   a$. 
   Sketch $\.f(t)$ and $\.h(t)$ in this case.

  \item The random variable $T$ is distributed on the interval $t>  
    a$, where $a>0$, and its hazard function is $t^{-1}$.  Determine
    the probability density function for $T$.

  \item Show that $\.h(t)$ is constant for $t>b$          
    and zero otherwise if and only if $\.f(t) =k\.e^{-k(t-b)}$ for
    $t>b$,            where $k$~is a positive constant.

  \item The random variable $T$ is distributed on the interval $t>  0$
    and its hazard function is given by
    \[
    \.h(t) =
    \left(\frac{\lambda}{\theta^\lambda}\right)t^{\lambda-1}\,,
    \]
    where $\lambda$ and $\theta$ are positive constants.  Find the
    probability density function for $T$.
  \end{questionparts}
\end{question}

%%%%%%%%%% Q13
\begin{question}
  A random number generator prints out a sequence of integers $I_1$,
  $I_2$, $I_3$, \dots. Each integer is independently equally
  likely to be any one of $1$, $2$, \dots, $n$, where $n$ is
  fixed.  The random variable $X$ takes the value $r$, where $I_r$ is
  the first integer which is a repeat of some earlier integer.

  Write down an expression for $\.P(X=4)$.

  \begin{questionparts}
  \item Find an expression for $\.P(X=r)$, where $2\le r\le n+1$.  Hence show
    that, for any positive integer $n$,
    \[
    \frac 1n + \left(1-\frac1n\right) \frac 2 n +
    \left(1-\frac1n\right)\left(1-\frac2n\right) \frac3 n + \cdots  \ =  \ 1
    \,.
    \]
  \item Write down an expression for $\.E(X)$.  (You do not need to
    simplify it.)
  \item Write down an expression for $\.P(X\ge k)$.

  \item Show that, for any discrete random variable $Y$ taking the
    values $1$, $2$, \dots, $N$,
    \[
    \.E(Y) = \sum_{k=1}^N \.P(Y\ge k)\,.
    \]
    Hence show that, for any positive integer $n$,
    \[
    \left(1-\frac{1^2}n\right) +
    \left(1-\frac1n\right)\left(1-\frac{2^2}n\right) +
    \left(1-\frac1n\right)\left(1-\frac{2}n\right)\left(1-\frac{3^2}n\right)
    + \cdots \ = \ 0.
    \]
  \end{questionparts}
\end{question}
\end{document}
