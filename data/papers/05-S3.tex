\documentclass[a4, 11pt]{report}


\pagestyle{myheadings}
\markboth{}{Paper III, 2005
\ \ \ \ \ 
\today 
}               

\RequirePackage{amssymb}
\RequirePackage{amsmath}
\RequirePackage{graphicx}
\RequirePackage{color}
\RequirePackage[flushleft]{paralist}[2013/06/09]



\RequirePackage{geometry}
\geometry{%
  a4paper,
  lmargin=2cm,
  rmargin=2.5cm,
  tmargin=3.5cm,
  bmargin=2.5cm,
  footskip=12pt,
  headheight=24pt}


\newcommand{\comment}[1]{{\bf Comment} {\it #1}}
%\renewcommand{\comment}[1]{}

\newcommand{\bluecomment}[1]{{\color{blue}#1}}
%\renewcommand{\comment}[1]{}
\newcommand{\redcomment}[1]{{\color{red}#1}}



\usepackage{epsfig}
\usepackage{pstricks-add}
\usepackage{tgheros} %% changes sans-serif font to TeX Gyre Heros (tex-gyre)
\renewcommand{\familydefault}{\sfdefault} %% changes font to sans-serif
%\usepackage{sfmath}  %%%% this makes equation sans-serif
%\input RexFigs


\setlength{\parskip}{10pt}
\setlength{\parindent}{0pt}

\newlength{\qspace}
\setlength{\qspace}{20pt}


\newcounter{qnumber}
\setcounter{qnumber}{0}

\newenvironment{question}%
 {\vspace{\qspace}
  \begin{enumerate}[\bfseries 1\quad][10]%
    \setcounter{enumi}{\value{qnumber}}%
    \item%
 }
{
  \end{enumerate}
  \filbreak
  \stepcounter{qnumber}
 }


\newenvironment{questionparts}[1][1]%
 {
  \begin{enumerate}[\bfseries (i)]%
    \setcounter{enumii}{#1}
    \addtocounter{enumii}{-1}
    \setlength{\itemsep}{5mm}
    \setlength{\parskip}{8pt}
 }
 {
  \end{enumerate}
 }



\DeclareMathOperator{\cosec}{cosec}
\DeclareMathOperator{\Var}{Var}

\def\d{{\mathrm d}}
\def\e{{\mathrm e}}
\def\g{{\mathrm g}}
\def\h{{\mathrm h}}
\def\f{{\mathrm f}}
\def\p{{\mathrm p}}
\def\s{{\mathrm s}}
\def\t{{\mathrm t}}


\def\A{{\mathrm A}}
\def\B{{\mathrm B}}
\def\E{{\mathrm E}}
\def\F{{\mathrm F}}
\def\G{{\mathrm G}}
\def\H{{\mathrm H}}
\def\P{{\mathrm P}}


\def\bb{\mathbf b}
\def \bc{\mathbf c}
\def\bx {\mathbf x}
\def\bn {\mathbf n}

\newcommand{\low}{^{\vphantom{()}}}
%%%%% to lower suffices: $X\low_1$ etc


\newcommand{\subone}{ {\vphantom{\dot A}1}}
\newcommand{\subtwo}{ {\vphantom{\dot A}2}}




\def\le{\leqslant}
\def\ge{\geqslant}


\def\var{{\rm Var}\,}

\newcommand{\ds}{\displaystyle}
\newcommand{\ts}{\textstyle}
\def\half{{\textstyle \frac12}}
\def\l{\left(}
\def\r{\right)}



\begin{document}
\setcounter{page}{2}

 
\section*{Section A: \ \ \ Pure Mathematics}

%%%%%%%%%%Q1
\begin{question}
Show that $\sin A = \cos B$ if and only if $\ds A = (4n+1)\frac{\pi}{2}
\pm B$ for some integer $n$.

Show also that 
$\big\vert\sin x \pm \cos x \big\vert \le \sqrt{2}$ for all values of $x$ and
deduce  that there are no solutions to the equation $\sin
\left( \sin x \right) = \cos \left( \cos x \right)$.

Sketch, on the same axes, the graphs of $y= \sin \left( \sin x \right)$ and
$y = \cos \left( \cos x \right)$. Sketch, not on the previous axes,
the graph of  $y= \sin \left(2 \sin x \right)$.
\end{question}

%%%%%%%%%%Q2
\begin{question}
Find the general solution of the differential equation  
$\ds \frac{\mathrm{d}y}{\mathrm{d}x} = -\frac{xy}{x^2+a^2}\;$,  
where $a\ne0\,$, and show that it can be written in the form
$\ds y^2(x^2+a^2)= c^2\,$,  
where $c$ is an arbitrary constant.
Sketch this curve. 
 
Find an expression for  
$\ds \frac{\mathrm{d}}{\mathrm{d}x} (x^2+y^2)$  
and  show that
\[
 \frac{\mathrm{d^2}}{\mathrm{d}x^2} (x^2+y^2) = 
2\left(1 -\frac {c^2}{(x^2+a^2)^2} \right) + \frac{8c^2x^2}{(x^2+a^2)^3}\;.
\]

\begin{questionparts}
\item
Show that, if $0<c<a^2$,  
the points on the curve whose distance from the origin  
is least are $\ds \l 0\,,\;\pm \frac{c}{a}\r$. 
 \item
If $c>a^2$, determine the points  
on the curve whose distance from the origin is least. 
 \end{questionparts}
\end{question}

%%%%%%%%% Q3
\begin{question}
Let  
$\f(x)=x^2+px+q$ and $\g(x)=x^2+rx+s\,$.  
Find an expression for $\f ( \g (x))$  
and hence find a necessary and sufficient condition  
on $a$, $b$ and $c$  for it to be possible to  
write the quartic expression $x^4+ax^3+bx^2+cx+d$  
in the form $\f ( \g (x))$, for some choice of values  
of $p$, $q$, $r$ and~$s$.  
 
Show further that this condition holds  
if and only if it is possible to write  
the quartic expression  
$x^4+ax^3+bx^2+cx+d$ in the form $(x^2+vx+w)^2-k$,  
for some choice of values of $v$, $w$ and $k$. 
 
Find the roots of the quartic equation $x^4-4x^3+10x^2-12x+4=0\,$. 
\end{question}

%%%%%% Q4 
\begin{question}
The sequence $u_n$ ($n= 1, 2, \ldots$) satisfies the recurrence relation 
\[ 
u_{n+2}= \frac{u_{n+1}}{u_n}(ku_n-u_{n+1})  
\] 
where  $k$ is a constant. 
 
If $u_1=a$ and $u_2=b\,$,  
where $a$ and  $b$ are non-zero and $b \ne ka\,$, prove by induction that 
\[ 
u_{2n}=\Big(\frac b a \Big) u_{2n-1} 
\] 
\[ 
u_{2n+1}= c u_{2n} 
\] 
for $n \ge 1$,  
where $c$ is a constant to be found in terms of $k$, $a$ and $b$. 
Hence express $u_{2n}$ and $u_{2n-1}$ in terms of $a$, $b$, $c$ and $n$. 
 
Find conditions on $a$, $b$ and $k$ in the three cases: 
\begin{questionparts}
\item the sequence $u_n$ is geometric;
\item $u_n$ has period 2; 
\item the sequence $u_n$  has period 4. 
\end{questionparts} 
\end{question}

%%%%%%%%% Q5
\begin{question}
Let $P$ be the point on 
the curve
$y=ax^2+bx+c$ (where $a$ is non-zero) at which the gradient  
is $m$. 
Show   that the equation of the tangent at $P$ is 
\[ 
y-mx=c-\frac{(m-b)^2}{4a}\;. 
\] 
 
Show that the curves  
$y=a_1 x^2+b_1 x+c_1$ 
and $y=a_2 x^2+b_2 x+c_2$  
(where $a_1$ and $a_2$ 
are non-zero) have a common tangent  
with gradient $m$ if and only if 
\[ 
(a_2 -a_1 )m^2 +
2(a_1 b_2-a_2 b_1)m +
4a_1 a_2(c_2-c_1)+
a_2 b_1^2-a_1 b_2 ^2=0\;. 
\] 

Show that, in the case  $a_1 \ne a_2 \,$, 
the two curves have exactly one common  
tangent if and only if they touch each other.
In the case $a_1 =a_2\,$, 
find  a necessary and sufficient condition  
for the two curves to have exactly one common tangent.
	\end{question}
	
%%%%%%%%% Q6
\begin{question}
In this question, you may use without proof the results
\[
4 \cosh^3 y - 3 \cosh y = \cosh (3y)
\ \ \ \ \text{and} \ \ \ \ 
\mathrm{arcosh} \, y = \ln ( y+\sqrt{y^2-1}).
\] 
 
\noindent[
{\bf Note: } $\mathrm{arcosh}y$ is another notation for 
$\cosh^{-1}y\,$]

Show that the equation $x^3 - 3a^2x = 2a^3 \cosh T$  
is satisfied by $ 2a \cosh \l \frac13 T \r$  
and hence that, if $c^2\ge b^3>0$,  
one of the roots of the equation $x^3-3bx=2c$  
is $\ds u+\frac{b}{u}$, where 
$u = (c+\sqrt{c^2-b^3})^{\frac13}\;$. 
 
Show that the other two roots of  
the equation $x^3-3bx=2c$ are the roots of the  
quadratic equation  
\[\ds x^2 + \Big( u+\frac{b}{u}\Big) x + u^2+\frac{b^2}{u^2}-b=0\, ,\]  
and find these roots in terms of $u$, $b$  and $\omega$,  
where $\omega = \frac{1}{2}(-1 + \mathrm{i}\sqrt{3})$. 
 
Solve completely the equation $x^3-6x=6\,$. 
\end{question}
	
%%%%%%%%% Q7
\begin{question}
Show  that if  
$\ds \int\frac1{u \, \f(u)}\; \d u  = \F(u) + c\;$,  
then $\ds \int\frac{m}{x \, \f(x^m)} \;\d x = \F(x^m) + c\;$, 
where $m\ne0$.

 
Find:  
\begin{questionparts} 
\item $\ds \int\frac1{x^n-x} \, \d x\,$; 
\item $\ds \int\frac1 {\sqrt{x^n+x^2}}\, \d x\,$. 
\end{questionparts} 
\end{question}
		
%%%%%%%%% Q8
\begin{question}	
In this question, $a$ and $c$ are distinct non-zero complex numbers.  
The complex conjugate of any complex number $z$ is denoted by  
$z^*$. 
 
Show that  
\[ 
|a - c|^2 = aa^* + cc^* -ac^* - ca^* 
\] 
and hence prove that the triangle  
$OAC$ in the Argand diagram,  
whose vertices are represented by  
$0$, $a$ and $c$ respectively, is right angled at $A$  
if and only if $2aa^* = ac^*+ca^*\,$. 
 
Points $P$ and $P'$ in the Argand diagram  
are represented by the complex numbers $ab$ and  
$\ds \frac{a}{b^*}\,$, where $b$ is a non-zero complex number.  
A circle in the Argand diagram has centre $C$ and passes through the point $A$,
 and is such that $OA$ is a tangent to the circle.
Show that  the point $P$ lies on the circle  
if and only if the point $P'$ lies on the circle.  

 
Conversely, show that if the points represented  
by the complex numbers $ab$ and $\ds \frac{a}{b^*}$,  
for some non-zero complex number $b$ with $bb^* \ne 1\,$,  
both lie on a circle centre $C$ in the Argand diagram  
which passes through $A$, then $OA$ is a tangent to the circle.  
\end{question}	
		

		
	
\newpage
\section*{Section B: \ \ \ Mechanics}


	
%%%%%%%%%% Q9
\begin{question}
Two particles, A and B, 
move without friction along a horizontal line which is 
perpendicular to a vertical wall. 
The coefficient of restitution between the two particles is $e$ 
and the coefficient of restitution between particle B 
and the wall is also $e$, where $0<e<1$. 
The mass of particle~A is $4em$ (with $m>0$), 
and the mass of particle B is $(1-e)^2m$\,. 
 
Initially, A is moving towards the wall with speed $(1-e)v$ 
(where $v>0$) and B is moving away from the wall and towards A 
with speed $2ev$. The two particles collide at a 
distance $d$ from the wall. Find the speeds of A and B after the collision. 
 
When B strikes the wall, it rebounds along the same line. 
Show that a second collision will take place, 
at a distance $de$ from the wall. 
 
Deduce that further collisions will take place. Find 
the distance from the wall at which the $n$th collision takes place,
and show that the times between successive collisions are equal. 
	\end{question}
	
%%%%%%%%%% Q10 
\begin{question}	
Two thin discs, each of radius $r$ and mass $m$, are held on a rough
horizontal surface with their centres a distance $6r$ apart. A thin light
 elastic band, of natural length $2\pi r$ and modulus $\dfrac{\pi mg}{12}$, 
is wrapped once round
the discs, its straight sections being parallel. The contact between the
elastic band and the discs is smooth. The coefficient of static 
friction between
each disc and the horizontal surface is $\mu$, and each disc
experiences a force due to friction equal to $\mu mg$ when it is
sliding.

The discs are released simultaneously.  If the discs collide, 
they rebound and  a half of their total
kinetic energy is lost in the collision. 

\begin{questionparts}
\item Show that the discs start sliding, but
come to rest before colliding, if and only if \mbox{$\frac23 <\mu <1$}.

\item Show that, if the discs collide at least once, 
their total kinetic energy
just before the first collision is $\frac43 mgr(2-3\mu)$.
\item
Show that if $\frac 4 9 > \mu^2 >\frac{5}{27}$ the discs come to
rest exactly once after the first collision.
\end{questionparts}
\end{question}

%%%%%%%%%% Q11

\begin{question}
A horizontal spindle rotates freely in a fixed bearing. 
Three light rods are each attached by one end to the spindle
so that they rotate in a vertical plane.
A particle of mass $m$ is fixed to the other   end 
of each of the three rods.
The rods 
 have lengths $a$, $b$ and $c$, with  $a > b > c\,$ 
and the angle between any pair of rods is
$\frac23 \pi$.
The angle between the rod of length $a$ and the vertical  is
$\theta$, 
as shown in the diagram.

\vspace*{-0.1in}
\begin{center}
\psset{xunit=0.45cm,yunit=0.45cm,algebraic=true,dotstyle=o,dotsize=3pt 0,linewidth=0.5pt,arrowsize=3pt 2,arrowinset=0.25}
\begin{pspicture*}(-6.49,-3.44)(6.06,9.38)
\psline[linestyle=dashed,dash=1pt 1pt](0,9.15)(0,-3.05)
\psline(0,2.51)(2.41,8.3)
\psline(0,2.51)(3.89,-0.76)
\psline(0,2.51)(-4.39,0.87)
\parametricplot{-2.7855695569416454}{1.176155335856138}{1*1.77*cos(t)+0*1.77*sin(t)+0|0*1.77*cos(t)+1*1.77*sin(t)+2.51}
\parametricplot{1.1701030633139027}{1.5707963267948966}{1*2.47*cos(t)+0*2.47*sin(t)+0|0*2.47*cos(t)+1*2.47*sin(t)+2.54}
\rput[tl](0.08,4.53){$\theta$}
\rput[tl](0.59,3.24){$\frac{2}{3}\pi$}
\rput[tl](-0.46,2.08){$\frac{2}{3}\pi$}
\rput[tl](1.56,6.08){$a$}
\rput[tl](2.57,1.3){$b$}
\rput[tl](-2.8,2.31){$c$}
\begin{scriptsize}
\psdots[dotsize=6pt 0,dotstyle=*](2.41,8.3)
\psdots[dotsize=6pt 0,dotstyle=*](3.89,-0.76)
\psdots[dotsize=6pt 0,dotstyle=*](-4.39,0.87)
\end{scriptsize}
\end{pspicture*}
\end{center}


 
Find an expression for the energy of the system and 
show that, if the system is in equilibrium, then 
\[ 
\tan \theta = -\frac{(b-c) \sqrt{3}}{2a-b-c}\;. 
\] 
Deduce that there are exactly two equilibrium positions 
and determine
which of the two equilibrium positions is stable. 
 
Show that, for the system to make complete revolutions, 
it must pass through its position of stable equilibrium 
with an angular velocity of at least 
\[ 
\sqrt{\frac{4gR}{a^2+b^2+c^2}} \, , \]
where $2R^2 =  (a-b)^2+(b-c)^2 +(c-a)^2 \;$. 
\end{question}
	

	
	\newpage
\section*{Section C: \ \ \ Probability and Statistics}


%%%%%%%%%% Q12
\begin{question}
 Five independent timers time a runner as she runs four laps of a track.
Four of the timers measure the individual lap times, the results of the
measurements being the random variables $T_1$ to
$T_4$, each of which has  variance $\sigma^2$ and 
expectation equal to the true
time for the lap. The fifth timer measures the total time 
for the race, the result of the measurement being the random variable
$T$ which has  variance $\sigma^2$ and
expectation equal to the true race time (which 
is equal to the sum of the four true lap times).
                                
                                                
Find a random variable $X$ of the form $aT+b(T_1+T_2+T_3+T_4)$,
where $a$ and $b$ are constants independent
of the true lap times,
 with
the two properties:

(1) \ whatever the true lap times, the expectation of $X$ is equal to
the true race time;

(2) \ the variance of $X$ is as small as possible.


Find also a random variable $Y$ of the form 
$cT+d(T_1+T_2+T_3+T_4)$, 
where $c$ and $d$ are constants independent
of the true lap times,
 with
the property that, whatever the true lap times, the expectation of $Y^2$ is
equal to $\sigma^2$.

In one particular race, $T$ takes the value 220 seconds and
$(T_1 + T_2 + T_3 + T_4)$ takes the value $220.5$ seconds. Use the random 
variables $X$ and $Y$ to estimate an interval
in which the true race time lies.
\end{question}

%%%%%%%%%% Q13
\begin{question}
A pack of cards consists of  $n+1$ cards, 
which are printed with the integers from 
$0$ to $n$.   A~game consists of drawing cards  repeatedly at 
random from the pack until the 
card printed with 0 is drawn, at which point  the game ends. 
After each draw, the player 
receives $\pounds 1$ if the card  drawn shows any of the 
integers from $1$ to $w$ inclusive but receives nothing 
if  the card  drawn shows any of the 
integers from $w+1$ to $n$ inclusive.

\begin{questionparts}
\item[\bf (i)]  In one version of the game, each card drawn is replaced immediately
and randomly in the pack.
Explain clearly why the probability that the player 
wins a total of exactly $\pounds 3$ 
is equal to the probability of the following event 
occurring:
out of the first four cards drawn which show 
 numbers in the range $0$ to $w$, 
the numbers on the first three are non-zero and the  
number on the fourth is zero.
Hence show that the probability that the player 
wins a total  of exactly  $\pounds 3$ is equal to $\ds \frac{w^3}{(w+1)^4}$. 
 
Write down the probability that the player wins a total of exactly 
$\pounds r$ and hence find the  expected total win.  
 
\item[\bf (ii)] In another version of the game,  
each card drawn is removed from the pack.
Show that the  expected total win in this version is 
half of the expected total win in the other version. 

\end{questionparts}
\end{question}

%%%%%%%%%% Q14
\begin{question}
In this question, 
you may use the result 
\[
\ds \int_0^\infty \frac{t^m}{(t+k)^{n+2}} \; \mathrm{d}t
=\frac{m!\, (n-m)!}{(n+1)! \, k^{n-m+1}}\;,
\]
where $m$ and $n$ are positive integers with $n\ge m\,$,  and where $k>0\,$. 
 
The random variable $V$ has density function 
\[ 
\f(x) 
= \frac{C \, k^{a+1} \, x^a}{(x+k)^{2a+2}} \quad \quad (0 \le x < \infty) \;,
\] 
where $a$ is a positive integer. 
Show that  $\ds C = \frac{(2a+1)!}{a! \, a!}\;$. 
 
Show, by means of a suitable substitution, that 
\[ 
\int_0^v \frac{x^a}{(x+k)^{2a+2}} \; \mathrm{d}x 
= \int_{\frac{k^2}{v}}^\infty \frac{u^a}{(u+k)^{2a+2}} \; \mathrm{d}u  
\] 
and deduce that the median value of $V$ is  $k$. 
Find the expected value of $V$. 
 
The random variable $V$ represents the speed
of a randomly chosen gas molecule. 
The time taken for such a particle to travel a fixed distance $s$ is 
given by the random variable $\ds T=\frac{s}{V}$. 
 
Show that 
\begin{equation}
\mathrm{P}(T<t) = \ds \int_{\frac{s}{t}}^\infty \frac{C \, k^{a+1} \, 
x^a}{(x+k)^{2a+2}}\; \mathrm{d}x 
\tag{$ *$}
\end{equation}
and hence find the density function of $T$. 
You may find it helpful to make the substitution $\ds u = \frac{s}{x}$ 
in the integral $(*)$. 
 
Hence show that the product of the median time 
and the median speed is equal to the distance~$s$, 
but that the product of the 
expected time and the expected speed is greater than~$s$. 

\end{question}
	
\end{document}
