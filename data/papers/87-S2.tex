
\documentclass[a4, 11pt]{report}


\pagestyle{myheadings}
\markboth{}{Paper II, 1987
\ \ \ \ \ 
\today 
}               

\RequirePackage{amssymb}
\RequirePackage{amsmath}
\RequirePackage{graphicx}
\RequirePackage{color}
\RequirePackage[flushleft]{paralist}[2013/06/09]



\RequirePackage{geometry}
\geometry{%
  a4paper,
  lmargin=2cm,
  rmargin=2.5cm,
  tmargin=3.5cm,
  bmargin=2.5cm,
  footskip=12pt,
  headheight=24pt}


\newcommand{\comment}[1]{{\bf Comment} {\it #1}}
%\renewcommand{\comment}[1]{}

\newcommand{\bluecomment}[1]{{\color{blue}#1}}
%\renewcommand{\comment}[1]{}
\newcommand{\redcomment}[1]{{\color{red}#1}}



\usepackage{epsfig}
\usepackage{tgheros} %% changes sans-serif font to TeX Gyre Heros (tex-gyre)
\renewcommand{\familydefault}{\sfdefault} %% changes font to sans-serif
%\usepackage{sfmath}  %%%% this makes equation sans-serif
%\input RexFigs


\setlength{\parskip}{10pt}
\setlength{\parindent}{0pt}

\newlength{\qspace}
\setlength{\qspace}{20pt}


\newcounter{qnumber}
\setcounter{qnumber}{0}

\newenvironment{question}%
 {\vspace{\qspace}
  \begin{enumerate}[\bfseries 1\quad][10]%
    \setcounter{enumi}{\value{qnumber}}%
    \item%
 }
{
  \end{enumerate}
  \filbreak
  \stepcounter{qnumber}
 }


\newenvironment{questionparts}[1][1]%
 {
  \begin{enumerate}[\bfseries (i)]%
    \setcounter{enumii}{#1}
    \addtocounter{enumii}{-1}
    \setlength{\itemsep}{5mm}
    \setlength{\parskip}{8pt}
 }
 {
  \end{enumerate}
 }



\DeclareMathOperator{\cosec}{cosec}
\DeclareMathOperator{\Var}{Var}

\def\d{{\rm d}}
\def\e{{\rm e}}
\def\g{{\rm g}}
\def\h{{\rm h}}
\def\f{{\rm f}}
\def\p{{\rm p}}
\def\s{{\rm s}}
\def\t{{\rm t}}


\def\A{{\rm A}}
\def\B{{\rm B}}
\def\E{{\rm E}}
\def\F{{\rm F}}
\def\G{{\rm G}}
\def\H{{\rm H}}
\def\P{{\rm P}}


\def\bb{\mathbf b}
\def \bc{\mathbf c}
\def\bx {\mathbf x}
\def\bn {\mathbf n}

\newcommand{\low}{^{\vphantom{()}}}
%%%%% to lower suffices: $X\low_1$ etc


\newcommand{\subone}{ {\vphantom{\dot A}1}}
\newcommand{\subtwo}{ {\vphantom{\dot A}2}}




\def\le{\leqslant}
\def\ge{\geqslant}


\def\var{{\rm Var}\,}

\newcommand{\ds}{\displaystyle}
\newcommand{\ts}{\textstyle}




\begin{document}
\setcounter{page}{2}

 
\section*{Section A: \ \ \ Pure Mathematics}

%%%%%%%%%%Q1
\begin{question}
Prove that: 

\begin{questionparts}
\item if $a+2b+3c=7x$, then 
\[
a^{2}+b^{2}+c^{2}=\left(x-a\right)^{2}+\left(2x-b\right)^{2}+\left(3x-c\right)^{2};
\]

\item if $2a+3b+3c=11x$, then 
\[
a^{2}+b^{2}+c^{2}=\left(2x-a\right)^{2}+\left(3x-b\right)^{2}+\left(3x-c\right)^{2}.
\]

\end{questionparts}
Give a general result of which \textbf{(i) }and \textbf{(ii) }are
special cases.
\end{question}

%%%%%%%%%%Q2
\begin{question}
Show that if at least one of the four angles $A\pm B\pm C$ is a multiple
of $\pi$, then 
\begin{alignat*}{1}
\sin^{4}A+\sin^{4}B+\sin^{4}C & -2\sin^{2}B\sin^{2}C-2\sin^{2}C\sin^{2}A\\
 & -2\sin^{2}A\sin^{2}B+4\sin^{2}A\sin^{2}B\sin^{2}C=0.
\end{alignat*}
\end{question}

%%%%%%%%% Q3
\begin{question}
	Let $a$ and $b$ be positive integers such that $b<2a-1$. For any
	given positive integer $n$, the integers $N$ and $M$ are defined
	by 
	\[
	[a+\sqrt{a^{2}-b}]^{n}=N-r,
	\]
	\[
	[a-\sqrt{a^{2}-b}]^{n}=M+s,
	\]
	where $0\leqslant r<1$ and $0\leqslant s<1$. Prove that 

\begin{itemize}
	\item[\bf (i)] $M=0$, 
	\vspace{2mm}
	\item[\bf (ii)] $r=s$, 
	\vspace{2mm}
	\item[\bf (iii)] $r^{2}-Nr+b^{n}=0.$
\end{itemize}

	Show that for large $n$, $\left(8+3\sqrt{7}\right)^{n}$ differs
	from an integer by about $2^{-4n}$.
\end{question}

%%%%%% Q4 

\begin{question}
Explain the geometrical relationship between the points in the Argand
diagram represented by the complex numbers $z$ and $z\mathrm{e}^{\mathrm{i}\theta}.$ 


Write down necessary and sufficient conditions that the distinct complex
numbers $\alpha,\beta$ and $\gamma$ represent the vertices of an
equilateral triangle taken in anticlockwise order. 


Show that $\alpha,\beta$ and $\gamma$ represent the vertices of
an equilateral triangle (taken in any order) if and only if 
\[
\alpha^{2}+\beta^{2}+\gamma^{2}-\beta\gamma-\gamma\alpha-\alpha\beta=0.
\]
Find necessary and sufficient conditions on the complex coefficients
$a,b$ and $c$ for the roots of the equation 
\[
z^{3}+az^{2}+bz+c=0
\]
to lie at the vertices of an equilateral triangle in the Argand digram. 


\end{question}


%%%%%%%%% Q5
\begin{question}
	If $y=\mathrm{f}(x)$, then the inverse of $\mathrm{f}$ (when it
	exists) can be obtained from\textit{ Lagrange's identity}. This identity,
	which you may use without proof, is
	\[
	\mathrm{f}^{-1}(y)=y+\sum_{n=1}^{\infty}\frac{1}{n!}\frac{\mathrm{d}^{n-1}}{\mathrm{d}y^{n-1}}\left[y-\mathrm{f}\left(y\right)\right]^{n},
	\]
	provided the series converges. 

\begin{questionparts}
	\item Verify Lagrange's identity when $\mathrm{f}(x)=\alpha x$, $(0<\alpha<2)$. 
	\item Show that one root of the equation 
	\[
	\tfrac{1}{2}=x-\tfrac{1}{4}x^{3}
	\]
	 is
	\[
	x=\sum_{n=0}^{\infty}\frac{\left(3n\right)!}{n!\left(2n+1\right)!2^{4n+1}}
	\]

	\item Find a solution for $x$, as a series in $\lambda,$ of the equation
	\[
	x=\mathrm{e}^{\lambda x}.
	\]

\end{questionparts}

	{[}You may assume that the series in part \textbf{(ii) }converges,
	and that the series in part \textbf{(iii) }converges for suitable
	$\lambda$.{]}
	\end{question}
	
	%%%%%%%%% Q6
	\begin{question}
	Let
	\[
	I=\int_{-\frac{1}{2}\pi}^{\frac{1}{2}\pi}\frac{\cos^{2}\theta}{1-\sin\theta\sin2\alpha}\,\mathrm{d}\theta\, ,
	\]
	where $0<\alpha<\frac{1}{4}\pi$. Show that
	\[
	I=\int_{-\frac{1}{2}\pi}^{\frac{1}{2}\pi}\frac{\cos^{2}\theta}{1+\sin\theta\sin2\alpha}\,\mathrm{d}\theta\, ,
	\]
	and hence that


	\[
	I=\frac{\pi}{\sin^{2}2\alpha}-\cot^{2}2\alpha\int_{-\frac{1}{2}\pi}^{\frac{1}{2}\pi}\frac{\sec^{2}\theta}{1+\cos^{2}2\alpha\tan^{2}\theta}\,\mathrm{d}\theta.
	\]



	Show that $I=\frac{1}{2}\pi\sec^{2}\alpha$, and state the value of
	$I$ if $\frac{1}{4}\pi<\alpha<\frac{1}{2}\pi$.
	
	 \end{question}
	 
	 %%%%%%%%% Q7
\begin{question}
A definite integral can be evaluated approximately by means of the Trapezium rule:


	\[
	\int_{x_{0}}^{x_{N}}\mathrm{f}(x)\,\mathrm{d}x\approx\tfrac{1}{2}h\left\{ \mathrm{f}\left(x_{0}\right)+2\mathrm{f}\left(x_{1}\right)+\ldots+2\mathrm{f}\left(x_{N-1}\right)+\mathrm{f}\left(x_{N}\right)\right\} ,
	\]
	where the interval length $h$ is given by $Nh=x_{N}-x_{0}$, and
	$x_{r}=x_{0}+rh$. Justify briefly this approximation. 


	Use the Trapezium rule with intervals of unit length to evaluate approximately
	the integral 
	\[
	\int_{1}^{n}\ln x\,\mathrm{d}x,
	\]
	where $n(>2)$ is an integer. Deduce that $n!\approx\mathrm{g}(n)$,
	where 
	\[
	\mathrm{g}(n)=n^{n+\frac{1}{2}}\mathrm{e}^{1-n},
	\]
	and show by means of a sketch, or otherwise, that 
	\[
	n!<\mathrm{g}(n).
	\]
	By using the Trapezium rule on the above integral with intervals of
	width $k^{-1}$, where $k$ is a positive integer, show that 
	\[
	\left(kn\right)!\approx k!n^{kn+\frac{1}{2}}\left(\frac{\mathrm{e}}{k}\right)^{k\left(1-n\right)}.
	\]
	Determine whether this approximation or $\mathrm{g}(kn)$ is closer
	to $\left(kn\right)!$.
	
	
	\end{question}
	
	%%%%%%%%% Q8
	\begin{question}
		Let $\mathbf{r}$ be the position vector of a point in three-dimensional
		space. Describe fully the locus of the point whose position vector
		is $\mathbf{r}$ in each of the following four cases: 

		\begin{questionparts}
		\item $\left(\mathbf{a-b}\right) \cdot \mathbf{r}=\frac{1}{2}(\left|\mathbf{a}\right|^{2}-\left|\mathbf{b}\right|^{2});$
		\item $\left(\mathbf{a-r}\right)\cdot\left(\mathbf{b-r}\right)=0;$
		\item $\left|\mathbf{r-a}\right|^{2}=\frac{1}{2}\left|\mathbf{a-b}\right|^{2};$ 
		\item $\left|\mathbf{r-b}\right|^{2}=\frac{1}{2}\left|\mathbf{a-b}\right|^{2}.$ 
		\end{questionparts}

		Prove algebraically that the equations \textbf{(i) }and \textbf{(ii)
		}together are equivalent to \textbf{(iii) }and \textbf{(iv) }together.
		Explain carefully the geometrical meaning of this equivalence.
		
		\end{question}
		
		
%%%%%%%%% Q9
		\begin{question}
		For any square matrix $\mathbf{A}$ such that $\mathbf{I-A}$ is non-singular
		(where $\mathbf{I}$ is the unit matrix), the matrix $\mathbf{B}$
		is defined by 
		\[
		\mathbf{B}=(\mathbf{I+A})(\mathbf{I-A})^{-1}.
		\]
		Prove that $\mathbf{B}^{\mathrm{T}}\mathbf{B}=\mathbf{I}$ if and
		only if $\mathbf{A+A}^{\mathrm{T}}=\mathbf{O}$ (where $\mathbf{O}$
		is the zero matrix), explaining clearly each step of your proof.


		{[}\textit{You may quote standard results about matrices without proof.}{]}
		\end{question}
		
	
%%%%%%%%%% 10
\begin{question}
	The set $S$ consists of $N(>2)$ elements $a_{1},a_{2},\ldots,a_{N}.$
	$S$ is acted upon by a binary operation $\circ,$ defined by 
	\[
	a_{j}\circ a_{k}=a_{m},
	\]
	where $m$ is equal to the greater of $j$ and $k$. 


	Determine, giving reasons, which of the four group axioms hold for
	$S$ under $\circ,$ and which do not. 


	Determine also, giving reasons, which of the group axioms hold for
	$S$ under $*$, where $*$ is defined by 
	\[
	a_{j}*a_{k}=a_{n},
	\]
	where $n=\left|j-k\right|+1$.
			\end{question}
			
		
		
		
	
\newpage
\section*{Section B: \ \ \ Mechanics}


	
%%%%%%%%%% Q11
\begin{question}
	A rough ring of radius $a$ is fixed so that it lies in a plane inclined
	at an angle $\alpha$ to the horizontal. A uniform heavy rod of length
	$b(>a)$ has one end smoothly pivoted at the centre of the ring, so
	that the rod is free to move in any direction. It rests on the circumference
	of the ring, making an angle $\theta$ with the radius to the highest
	point on the circumference. Find the relation between $\alpha,\theta$
	and the coefficient of friction, $\mu,$ which must hold when the
	rod is in limiting equilibrium. 

	\end{question}
	
%%%%%%%%%% Q12
\begin{question}	
A long, inextensible string passes through a small fixed ring. One
end of the string is attached to a particle of mass $m,$ which hangs
freely. The other end is attached to a bead also of mass $m$ which
is threaded on a smooth rigid wire fixed in the same vertical plane
as the ring. The curve of the wire is such that the system can be
in static equilibrium for all positions of the bead. The shortest
distance between the wire and the ring is $d(>0).$ Using plane polar
coordinates centred on the ring, find the equation of the curve. 


The bead is set in motion. Assuming that the string remains taut,
show that the speed of the bead when it is a distance $r$ from the
ring is 
\[
\left(\frac{r}{2r-d}\right)^{\frac{1}{2}}v,
\]
where $v$ is the speed of the bead when $r=d.$ 
\end{question}

%%%%%%%%%% Q13

\begin{question}
Ice snooker is played on a rectangular horizontal table, of length
$L$ and width $B$, on which a small disc (the \textit{puck}) slides
without friction. The table is bounded by smooth vertical walls (the
\textit{cushions}) and the coefficient of restitution between the
puck and any cushion is $e$. If the puck is hit so that it bounces
off two adjacent cushions, show that its final path (after two bounces)
is parallel to its original path. 


The puck rests against the cushion at a point which divides the side
of length $L$ in the ratio $z:1$. Show that it is possible, whatever
$z$, to hit the puck so that it bounces off the three other cushions
in succession clockwise and returns to the spot at which it started. 


By considering these paths as $z$ varies, explain briefly why there
are two different ways in which, starting at any point away from the
cushions, it is possible to perform a shot in which the puck bounces
off all four cushions in succession clockwise and returns to its starting
point.
\end{question}
	
%%%%%%%%%% Q14
\begin{question}
A thin uniform elastic band of mass $m,$ length $l$ and modulus
of elasticity $\lambda$ is pushed on to a smooth circular cone of
vertex angle $2\alpha,$ in such a way that all elements of the band
are the same distance from the vertex. It is then released from rest.
Let $x(t)$ be the length of the band at time $t$ after release,
and let $t_{0}$ be the time at which the band becomes slack. 


Assuming that a small element of the band which subtends an angle
$\delta\theta$ at the axis of the cone experiences a force, due to
the tension $T$ in the band, of magnitude $T\delta\theta$ directed
towards the axis, and ignoring the effects of gravity, show that 
\[
\frac{\mathrm{d}^{2}x}{\mathrm{d}t^{2}}+\frac{4\pi^{2}\lambda}{ml}(x-l)\sin^{2}\alpha=0,\qquad(0<t<t_{0}).
\]
Find the value of $t_{0}.$ 
\end{question}
	
	\newpage
\section*{Section C: \ \ \ Probability and Statistics}


%%%%%%%%%% Q15
\begin{question}
A train of length $l_{1}$ and a lorry of length $l_{2}$ are heading
for a level crossing at speeds $u_{1}$ and $u_{2}$ respectively.
Initially the front of the train and the front of the lorry are at
distances $d_{1}$ and $d_{2}$ from the crossing. Find conditions
on $u_{1}$ and $u_{2}$ under which a collision will occur. On a
diagram with $u_{1}$ and $u_{2}$ measured along the $x$ and $y$
axes respectively, shade in the region which represents collision. 


Hence show that if $u_{1}$ and $u_{2}$ are two independent random
variables, both uniformly distributed on $(0,V)$, then the probability
of a collision in the case when initially the back of the train is
nearer to the crossing than the front of the lorry is 
\[
\frac{l_{1}l_{2}+l_{2}d_{1}+l_{1}d_{2}}{2d_{2}\left(l_{2}+d_{2}\right)}.
\]
Find the probability of a collision in each of the other two possible
cases. 
\end{question}

%%%%%%%%%% Q16
\begin{question}
My two friends, who shall remain nameless, but whom I shall refer
to as $P$ and $Q$, both told me this afternoon that there is a body
in my fridge. I'm not sure what to make of this, because $P$ tells
the truth with a probability of only $p$, while $Q$ (independently)
tells the truth with probability $q$. I haven't looked in the fridge
for some time, so if you had asked me this morning, I would have said
that there was just as likely to be a body in it as not. Clearly,
in view of what $P$ and $Q$ told me, I must revise this estimate.
Explain carefully why my new estimate of the probability of there
being a body in the fridge should be 
\[
\frac{pq}{1-p-q+2pq}.
\]
I have now been to look in the fridge, and there is indeed a body
in it; perhaps more than one. It seems to me that only my enemy $A$,
or my enemy $B$, or (with a bit of luck) both $A$ and $B$ could
be in my fridge, and this morning I would have judged these three
possibilities to be equally likely. But tonight I asked $P$ and $Q$
separately whether or not $A$ was in the fridge, and they each said
that he was. What should be my new estimate of the probability that
both $A$ and $B$ are in my fridge?


Of course, I tell the truth always.
\end{question}
\end{document}
