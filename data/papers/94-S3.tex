\documentclass[a4, 11pt]{report}


\pagestyle{myheadings}
\markboth{}{Paper III, 1994
\ \ \ \ \ 
\today 
}               

\RequirePackage{amssymb}
\RequirePackage{amsmath}
\RequirePackage{graphicx}
\RequirePackage{color}
\RequirePackage[flushleft]{paralist}[2013/06/09]



\RequirePackage{geometry}
\geometry{%
  a4paper,
  lmargin=2cm,
  rmargin=2.5cm,
  tmargin=3.5cm,
  bmargin=2.5cm,
  footskip=12pt,
  headheight=24pt}


\newcommand{\comment}[1]{{\bf Comment} {\it #1}}
%\renewcommand{\comment}[1]{}

\newcommand{\bluecomment}[1]{{\color{blue}#1}}
%\renewcommand{\comment}[1]{}
\newcommand{\redcomment}[1]{{\color{red}#1}}



\usepackage{epsfig}
\usepackage{pstricks-add}
\usepackage{tgheros} %% changes sans-serif font to TeX Gyre Heros (tex-gyre)
\renewcommand{\familydefault}{\sfdefault} %% changes font to sans-serif
%\usepackage{sfmath}  %%%% this makes equation sans-serif
%\input RexFigs


\setlength{\parskip}{10pt}
\setlength{\parindent}{0pt}

\newlength{\qspace}
\setlength{\qspace}{20pt}


\newcounter{qnumber}
\setcounter{qnumber}{0}

\newenvironment{question}%
 {\vspace{\qspace}
  \begin{enumerate}[\bfseries 1\quad][10]%
    \setcounter{enumi}{\value{qnumber}}%
    \item%
 }
{
  \end{enumerate}
  \filbreak
  \stepcounter{qnumber}
 }


\newenvironment{questionparts}[1][1]%
 {
  \begin{enumerate}[\bfseries (i)]%
    \setcounter{enumii}{#1}
    \addtocounter{enumii}{-1}
    \setlength{\itemsep}{5mm}
    \setlength{\parskip}{8pt}
 }
 {
  \end{enumerate}
 }



\DeclareMathOperator{\cosec}{cosec}
\DeclareMathOperator{\Var}{Var}

\def\d{{\rm d}}
\def\e{{\rm e}}
\def\g{{\rm g}}
\def\h{{\rm h}}
\def\f{{\rm f}}
\def\p{{\rm p}}
\def\s{{\rm s}}
\def\t{{\rm t}}


\def\A{{\rm A}}
\def\B{{\rm B}}
\def\E{{\rm E}}
\def\F{{\rm F}}
\def\G{{\rm G}}
\def\H{{\rm H}}
\def\P{{\rm P}}


\def\bb{\mathbf b}
\def \bc{\mathbf c}
\def\bx {\mathbf x}
\def\bn {\mathbf n}

\newcommand{\low}{^{\vphantom{()}}}
%%%%% to lower suffices: $X\low_1$ etc


\newcommand{\subone}{ {\vphantom{\dot A}1}}
\newcommand{\subtwo}{ {\vphantom{\dot A}2}}




\def\le{\leqslant}
\def\ge{\geqslant}


\def\var{{\rm Var}\,}

\newcommand{\ds}{\displaystyle}
\newcommand{\ts}{\textstyle}




\begin{document}
\setcounter{page}{2}

 
\section*{Section A: \ \ \ Pure Mathematics}

%%%%%%%%%%Q1
\begin{question}
Calculate 
\[
\int_{0}^{x}\mathrm{sech}\, t\,\mathrm{d}t.
\]
Find the reduction formula involving $I_{n}$ and $I_{n-2}$, where
\[
I_{n}=\int_{0}^{x}\mathrm{sech}^{n}t\,\mathrm{d}t
\]
and, hence or otherwise, find $I_{5}$ and $I_{6}.$ 
\end{question}

%%%%%%%%%%Q2
\begin{question}
\begin{questionparts}
 \item By setting $y=x+x^{-1},$ find the solutions
of 
\[
x^{4}+10x^{3}+26x^{2}+10x+1=0.
\]

\item Solve 
\[
x^{4}+x^{3}-10x^{2}-4x+16=0.
\]		
\end{questionparts}
\end{question}

%%%%%%%%% Q3
\begin{question}
Describe geometrically the possible intersections of a plane with
a sphere. 


Let $P_{1}$ and $P_{2}$ be the planes with equations 
\begin{alignat*}{1}
3x-y-1 & =0,\\
x-y+1 & =0,
\end{alignat*}
respectively, and let $S_{1}$ and $S_{2}$ be the spheres with equations
\begin{alignat*}{1}
x^{2}+y^{2}+z^{2} & =7,\\
x^{2}+y^{2}+z^{2}-6y-4z+10 & =0,
\end{alignat*}
respectively. Let $C_{1}$ be the intersection of $P_{1}$ and $S_{1},$
let $C_{2}$ be the intersection of $P_{2}$ and $S_{2}$ and let
$L$ be the intersection of $P_{1}$ and $P_{2}.$ Find the points
where $L$ meets each of $S_{1}$ and $S_{2}.$ Determine, giving
your reasons, whether the circles $C_{1}$ and $C_{2}$ are linked. 
\end{question}

%%%%%% Q4 
\begin{question}
Find the two solutions of the differential equation 
\[
\left(\frac{\mathrm{d}y}{\mathrm{d}x}\right)^{2}=4y
\]
which pass through the point $(a,b^{2}),$ where $b\neq0.$


Find two distinct points $(a_{1},1)$ and $(a_{2},1)$ such that one
of the solutions through each of them also passes through the origin.
Show that the graphs of these two solutions coincide and sketch their
common graph, together with the other solutions through $(a_{1},1)$
and $(a_{2},1)$. 


Now sketch sufficient members of the family of solutions (for varying
$a$ and $b$) to indicate the general behaviour. Use your sketch
to identify a common tangent, and comment briefly on its relevance
to the differential equation. 
	\end{question}

%%%%%%%%% Q5
\begin{question}
The function $\mathrm{f}$ is given by $\mathrm{f}(x)=\sin^{-1}x$
for $-1<x<1.$ Prove that 
\[
(1-x^{2})\mathrm{f}''(x)-x\mathrm{f}'(x)=0.
\]
Prove also that 
\[
(1-x^{2})\mathrm{f}^{(n+2)}(x)-(2n+1)x\mathrm{f}^{(n+1)}(x)-n^{2}\mathrm{f}^{(n)}(x)=0,
\]
for all $n>0$, where $\mathrm{f}^{(n)}$ denotes the $n$th derivative
of $\mathrm{f}$. Hence express $\mathrm{f}(x)$ as a Maclaurin series. 


The function $\mathrm{g}$ is given by 
\[
\mathrm{g}(x)=\ln\sqrt{\frac{1+x}{1-x}},
\]
for $-1<x<1.$ Write down a power series expression for $\mathrm{g}(x),$
and show that the coefficient of $x^{2n+1}$ is greater than that
in the expansion of $\mathrm{f},$ for each $n>0$. 
	\end{question}
	
%%%%%%%%% Q6
\begin{question}
The four points $A,B,C,D$ in the Argand diagram (complex plane) correspond
to the complex numbers $a,b,c,d$ respectively. The point $P_{1}$
is mapped to $P_{2}$ by rotating about $A$ through $\pi/2$ radians.
Then $P_{2}$ is mapped to $P_{3}$ by rotating about $B$ through
$\pi/2$ radians, $P_{3}$ is mapped to $P_{4}$ by rotating about
$C$ through $\pi/2$ radians and $P_{4}$ is mapped to $P_{5}$ by
rotating about $D$ through $\pi/2$ radians, each rotation being
in the positive sense. If $z_{i}$ is the complex number corresponding
to $P_{i},$ find $z_{5}$ in terms of $a,b,c,d$ and $z_{1}.$ 


Show that $P_{5}$ will coincide with $P_{1},$ irrespective of the
choice of the latter if, and only if \[a-c=\mathrm{i}(b-d)\] and interpret
this condition geometrically. 


The points $A,B$ and $C$ are now chosen to be distinct points on
the unit circle and the angle of rotation is changed to $\theta,$
where $\theta\neq0,$ on each occasion. Find the necessary and sufficient
condition on $\theta$ and the points $A,B$ and $C$ for $P_{4}$
always to coincide with $P_{1}.$ 
\end{question}
	
%%%%%%%%% Q7
\begin{question}
Let $S_{3}$ be the group of permutations of three objects and $Z_{6}$
be the group of integers under addition modulo 6. List all the elements
of each group, stating the order of each element. State, with reasons,
whether $S_{3}$ is isomorphic with $Z_{6}.$


Let $C_{6}$ be the group of 6th roots of unity. That is, $C_{6}=\{1,\alpha,\alpha^{2},\alpha^{3},\alpha^{4},\alpha^{5}\}$
where $\alpha=\mathrm{e}^{\mathrm{i}\pi/3}$ and the group operation
is complex multiplication. Prove that $C_{6}$ is isomorphic with
$Z_{6}.$ Is there any (multiplicative or additive) subgroup of the
complex numbers which is isomorphic with $S_{3}$? Give a reason for
your answer. 
\end{question}
		
%%%%%%%%% Q8
\begin{question}	
Let $a,b,c,d,p,q,r$ and $s$ be real numbers. By considering the
determinant of the matrix product 
\[
\begin{pmatrix}z_{1} & z_{2}\\
-z_{2}^{*} & z_{1}^{*}
\end{pmatrix}\begin{pmatrix}z_{3} & z_{4}\\
-z_{4}^{*} & z_{3}^{*}
\end{pmatrix},
\]
where $z_{1},z_{2},z_{3}$ and $z_{4}$ are suitably chosen complex
numbers, find expressions $L_{1},L_{2},L_{3}$ and $L_{4},$ each
of which is linear in $a,b,c$ and $d$ and also linear in $p,q,r$
and $s,$ such that 
\[
(a^{2}+b^{2}+c^{2}+d^{2})(p^{2}+q^{2}+r^{2}+s^{2})=L_{1}^{2}+L_{2}^{2}+L_{3}^{2}+L_{4}^{2}.
\]
\end{question}	
		

		
	
\newpage
\section*{Section B: \ \ \ Mechanics}


	
%%%%%%%%%% Q9
\begin{question}
A smooth, axially symmetric bowl has its vertical cross-sections determined
by $s=2\sqrt{ky},$ where $s$ is the arc-length measured from its
lowest point $V$, and $y$ is the height above $V$. A particle is
released from rest at a point on the surface at a height $h$ above
$V$. Explain why 
\[
\left(\frac{\mathrm{d}s}{\mathrm{d}t}\right)^{2}+2gy
\]
is constant. 


Show that the time for the particle to reach $V$ is 
\[
\pi\sqrt{\frac{k}{2g}}.
\]
Two elastic particles of mass $m$ and $\alpha m,$ where $\alpha<1,$
are released simultaneously from opposite sides of the bowl at heights
$\alpha^{2}h$ and $h$ respectively. If the coefficient of restitution
between the particles is $\alpha,$ describe the subsequent motion. 
	\end{question}
	
%%%%%%%%%% Q10
\begin{question}	
The island of Gammaland is totally flat and subject to a constant
wind of $w$ kh$^{-1},$ blowing from the West. Its southernmost shore
stretches almost indefinitely, due east and west, from the coastal
city of Alphabet. A novice pilot is making her first solo flight from
Alphaport to the town of Betaville which lies north-east of Alphaport.
Her instructor has given her the correct heading to reach Betaville,
flying at the plane's recommended airspeed of $v$ kh$^{-1},$ where
$v>w.$ 


On reaching Betaport the pilot returns with the opposite heading to
that of the outward flight and, so featureless is Gammaland, that
she only realises her error as she crosses the coast with Alphaport
nowhere in sight. Assuming that she then turns West along the coast,
and that her outward flight took $t$ hours, show that her return
flight takes 
\[
\left(\frac{v+w}{v-w}\right)t\ \text{hours.}
\]
If Betaville is $d$ kilometres from Alphaport, show that, with the
correct heading, the return flight should have taken 
\[
t+\frac{\sqrt{2}wd}{v^{2}-w^{2}}\ \text{hours.}
\]
\end{question}

%%%%%%%%%% Q11

\begin{question}
A step-ladder has two sections $AB$ and $AC,$ each of length $4a,$
smoothly hinged at $A$ and connected by a light elastic rope $DE,$
of natural length $a/4$ and modulus $W$, where $D$ is on $AB,$
$E$ is on $AC$ and $AD=AE=a.$ The section $AB,$ which contains
the steps, is uniform and of weight $W$ and the weight of $AC$ is
negligible. 


The step-ladder rests on a smooth horizontal floor and a man of weight
$4W$ carefully ascends it to stand on a rung distant $\beta a$ from
the end of the ladder resting on the floor. Find the height above
the floor of the rung on which the man is standing when $\beta$ is
the maximum value at which equilibrium is possible. 
\end{question}
	

	
	\newpage
\section*{Section C: \ \ \ Probability and Statistics}


%%%%%%%%%% Q12
\begin{question}
In certain forms of Tennis two players $A$ and $B$ serve alternate
games. Player $A$ has probability $p\low_{A}$ of winning a game in which
she serves and $p\low_{B}$ of winning a game in which player $B$ serves.
Player $B$ has probability $q\low_{B}=1-p\low_{B}$ of winning a game in
which she serves and probability $q\low_{A}=1-p\low_{A}$ of winning a game
in which player $A$ serves. In Shortened Tennis the first player
to lead by 2 games wins the match. Find the probability $P_{\text{short}}$
that $A$ wins a Shortened Tennis match in which she serves first
and show that it is the same as if $B$ serves first. 


In Standard Tennis the first player to lead by 2 or more games after
4 or more games have been played wins the match. Show that the probability
that the match is decided in 4 games is 
\[
p^{2}_Ap_{B}^{2}+q_{A}^{2}q_{B}^{2}+2(p\low_{A}p\low_{B}+q\low_{A}q\low_{B})(p\low_{A}q\low_{B}+q\low_{A}p\low_{B}).
\]
If $p\low_{A}=p\low_{B}=p$ and $q\low_{A}=q\low_{B}=q,$ find the probability $P_{\text{stan}}$
that $A$ wins a Standard Tennis match in which she serves first.
Show that 
\[
P_{\text{stan}}-P_{\text{short}}=\frac{p^{2}q^{2}(p-q)}{p^{2}+q^{2}}.
\]
\end{question}

%%%%%%%%%% Q13
\begin{question}
During his performance a trapeze artist is supported by two identical
ropes, either of which can bear his weight. Each rope is such that
the time, in hours of performance, before it fails is exponentially
distributed, independently of the other, with probability density
function $\lambda\exp(-\lambda t)$ for $t\geqslant0$ (and 0 for
$t<0$), for some $\lambda>0.$ A particular rope has already been
in use for $t_{0}$ hours of performance. Find the distribution for
the length of time the artist can continue to use it before it fails.
Interpret and comment upon your result. 


Before going on tour the artist insists that the management purchase
two new ropes of the above type. Show that the probability density
function of the time until both ropes fail is 
\[
\mathrm{f}(t)=\begin{cases}
2\lambda\mathrm{e}^{-\lambda t}(1-\mathrm{e}^{-\lambda t}) & \text{ if }t\geqslant0,\\
0 & \text{ otherwise.}
\end{cases}
\]
If each performance lasts for $h$ hours, find the probability that
both ropes fail during the $n$th performance. Show that the probability
that both ropes fail during the same performance is $\tanh(\lambda h/2)$. 
\end{question}

%%%%%%%%%% Q14
\begin{question}
Three points, $P,Q$ and $R$, are independently randomly chosen on
the perimeter of a circle. Prove that the probability that at least
one of the angles of the triangle $PQR$ will exceed $k\pi$ is $3(1-k)^{2}$
if $\frac{1}{2}\leqslant k\leqslant1.$ Find the probability if $\frac{1}{3}\leqslant k\leqslant\frac{1}{2}.$
\end{question}
	
\end{document}
