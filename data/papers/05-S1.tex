\documentclass[a4, 11pt]{report}


\pagestyle{myheadings}
\markboth{}{Paper I, 2005
\ \ \ \ \ 
\today 
}               

\RequirePackage{amssymb}
\RequirePackage{amsmath}
\RequirePackage{graphicx}
\RequirePackage{color}
\RequirePackage[flushleft]{paralist}[2013/06/09]



\RequirePackage{geometry}
\geometry{%
  a4paper,
  lmargin=2cm,
  rmargin=2.5cm,
  tmargin=3.5cm,
  bmargin=2.5cm,
  footskip=12pt,
  headheight=24pt}


\newcommand{\comment}[1]{{\bf Comment} {\it #1}}
%\renewcommand{\comment}[1]{}

\newcommand{\bluecomment}[1]{{\color{blue}#1}}
%\renewcommand{\comment}[1]{}
\newcommand{\redcomment}[1]{{\color{red}#1}}



\usepackage{epsfig}
\usepackage{pstricks-add}
\usepackage{tgheros} %% changes sans-serif font to TeX Gyre Heros (tex-gyre)
\renewcommand{\familydefault}{\sfdefault} %% changes font to sans-serif
%\usepackage{sfmath}  %%%% this makes equation sans-serif
%\input RexFigs


\setlength{\parskip}{10pt}
\setlength{\parindent}{0pt}

\newlength{\qspace}
\setlength{\qspace}{20pt}


\newcounter{qnumber}
\setcounter{qnumber}{0}

\newenvironment{question}%
 {\vspace{\qspace}
  \begin{enumerate}[\bfseries 1\quad][10]%
    \setcounter{enumi}{\value{qnumber}}%
    \item%
 }
{
  \end{enumerate}
  \filbreak
  \stepcounter{qnumber}
 }


\newenvironment{questionparts}[1][1]%
 {
  \begin{enumerate}[\bfseries (i)]%
    \setcounter{enumii}{#1}
    \addtocounter{enumii}{-1}
    \setlength{\itemsep}{5mm}
    \setlength{\parskip}{8pt}
 }
 {
  \end{enumerate}
 }



\DeclareMathOperator{\cosec}{cosec}
\DeclareMathOperator{\Var}{Var}

\def\d{{\mathrm d}}
\def\e{{\mathrm e}}
\def\g{{\mathrm g}}
\def\h{{\mathrm h}}
\def\f{{\mathrm f}}
\def\p{{\mathrm p}}
\def\s{{\mathrm s}}
\def\t{{\mathrm t}}


\def\A{{\mathrm A}}
\def\B{{\mathrm B}}
\def\E{{\mathrm E}}
\def\F{{\mathrm F}}
\def\G{{\mathrm G}}
\def\H{{\mathrm H}}
\def\P{{\mathrm P}}


\def\bb{\mathbf b}
\def \bc{\mathbf c}
\def\bx {\mathbf x}
\def\bn {\mathbf n}

\newcommand{\low}{^{\vphantom{()}}}
%%%%% to lower suffices: $X\low_1$ etc


\newcommand{\subone}{ {\vphantom{\dot A}1}}
\newcommand{\subtwo}{ {\vphantom{\dot A}2}}




\def\le{\leqslant}
\def\ge{\geqslant}


\def\var{{\rm Var}\,}

\newcommand{\ds}{\displaystyle}
\newcommand{\ts}{\textstyle}
\def\half{{\textstyle \frac12}}
\def\l{\left(}
\def\r{\right)}



\begin{document}
\setcounter{page}{2}

 
\section*{Section A: \ \ \ Pure Mathematics}

%%%%%%%%%%Q1
\begin{question}
$47231$ is a five-digit number whose digits sum to $4+7+2+3+1 = 17\,$.
\begin{itemize}
 \setlength{\itemsep}{3mm}
\item[\bf (i)] Show that there are $15$ five-digit numbers whose digits sum to $43$. 
You should explain  your reasoning clearly.
\item[\bf (ii)] How many five-digit numbers are there whose digits sum to $39$?
\end{itemize}
\end{question}

%%%%%%%%%%Q2
\begin{question}
The point $P$ has coordinates $\l p^2  ,  2p \r$ 
and the point $Q$ has coordinates $\l q^2  ,  2q \r$, 
where $p$ and~$q$ are non-zero and $p \neq q$. 
The curve $C$ is given by  $y^2 = 4x\,$.
The point $R$ is the intersection of the tangent to $C$  at $P$ 
and the tangent 
to $C$ at $Q$. 
Show that $R$ has coordinates $\l pq ,  p+q \r$.


The point $S$ is the intersection of the normal to $C$ at $P$ 
and the normal to $C$ at $Q$. 
If $p$ and $q$ are such  that $\l 1  ,  0 \r$ 
lies on the line $PQ$, 
show that $S$ has coordinates $\l p^2 + q^2 + 1  , \, p+q \r$, 
and that the quadrilateral $PSQR$ is a rectangle.
\end{question}

%%%%%%%%% Q3
\begin{question}
In this question $a$ and $b$ are distinct, 
non-zero real numbers, and $c$ is a real number.
\begin{questionparts}
\item Show that, if $a$ and $b$ are 
either both positive or both negative, then the equation 
\[
\displaystyle \frac {x }{ x-a} + \frac{x }{ x-b} = 1
\]
has two distinct real solutions.
\item Show that, if $c\ne1$, the equation 
\[\displaystyle \frac x { x-a} + \frac{x}{ x-b} = 1 + c\] 
has exactly one real solution if 
$
\displaystyle c^2 = - \frac {4ab}{\l a - b \r ^2}\;. \;\;
$
Show that
this condition can be written $\displaystyle
c^2= 1 - \l \frac {a+b}{a-b} \r ^2 $
and deduce that it  can only hold if $0 < c^2 \le 1\,$.
\end{questionparts}
\end{question}

%%%%%% Q4 
\begin{question}
\begin{questionparts}
\item Given that $\displaystyle \cos \theta = \frac{3}{ 5}$ 
and that $\displaystyle \frac{3\pi }{ 2} \le \theta \le 2\pi$, 
show that $\displaystyle \sin 2 \theta =  -
\frac{24 }{ 25}\,$, and evaluate $ \cos 3 \theta\,$.
\item Prove the identity 
$\displaystyle 
\tan 3\theta 
\equiv \frac {3 \tan \theta - \tan^3 \theta}{1 - 3 \tan^2 \theta}\;$. 

Hence evaluate $\tan \theta$, 
given that $\displaystyle \tan 3\theta = \frac{11}{ 2}$ 
and that $\displaystyle \frac{\pi}{ 4} \le \theta \le \frac{\pi}{ 2}\;$.
\end{questionparts}
\end{question}

%%%%%%%%% Q5
\begin{question}
\begin{questionparts}
\item Evaluate the integral
\[
\int_0^1  \l x + 1 \r ^{k-1} \; \mathrm{d}x
\]
in the cases  $k\ne0$ and $k = 0\,$.

Deduce that  $\displaystyle \frac{2^k - 1}{k} \approx \ln 2$ when $k \approx 0\,$.


\item Evaluate the integral
\[
\int_0^1  x  \l x + 1 \r ^m \; \mathrm{d}x \;
\]
in the different cases that arise according to the value of $m$.

\end{questionparts}
	\end{question}
	
%%%%%%%%% Q6
\begin{question}
\begin{questionparts}
\item The point $A$ has coordinates $\l 5 \, , 16 \r$ and the point 
$B$ has coordinates $\l -4 \, , 4 \r$. 
The variable point $P$ has coordinates $\l x \, ,  y \r\,$ 
and moves on a path such that $AP=2BP$. 
Show that the Cartesian equation of the path of $P$ is
\[
\displaystyle \l x+7 \r^2 + y^2 =100 \;.
\]

\item The point $C$ has coordinates $\l a \, ,  0 \r$ 
and the point $D$ has coordinates $\l b \, ,  0 \r$, where $a\ne b$.
 The variable point $Q$ moves on a path such that
\[
QC = k \times QD\;,
\]
where $k>1\,$.
Given that the path of $Q$ is the same as the path of $P$,
show that 
\[
\frac{a+7}{b+7}=\frac{a^2+51}{b^2+51}\;.
\]
Show further that $(a+7)(b+7)=100\,$.

\end{questionparts}
\end{question}
	
%%%%%%%%% Q7
\begin{question}
The notation $\displaystyle \prod^n_{r=1} \f (r)$ 
denotes the product $\f (1) \times \f (2) \times \f(3) \times \cdots \times
\f(n)$. 
%For example, $\displaystyle \prod_{r=1}^4 r = 24$.

%Simplify $\displaystyle \prod^n_{r=1} \frac{\g (r) }{ \g (r-1) }$. 
%You may assume that $\g (r) \neq 0$ for any integer $0 \le r \le n $.

Simplify the following products as far as possible:

\begin{questionparts}
\item $\displaystyle \prod^n_{r=1} \l \frac{r+ 1 }{ r } \r\,$;
\item $\displaystyle \prod^n_{r=2} \l \frac{r^2 -1}{r^2 } \r\,$;
\item $\displaystyle \prod^n_{r=1} \l {\cos \frac{2\pi }{ n} 
+ \sin \frac{2\pi}{ n} \cot \frac{\l 2r-1 \r \pi  }{ n} }\r\,$,
 where $n$ is even.

\end{questionparts}
\end{question}
		
%%%%%%%%% Q8
\begin{question}	
Show that, if $y^2 = x^k \f(x)$, 
then $\displaystyle 2xy \frac{\mathrm{d}y }{ \mathrm{d}x} = ky^2 + x^{k+1} 
\frac{\mathrm{d}\f }{ \mathrm{d}x}$\,. 
\begin{questionparts}
\item By setting $k=1$ in this result, find the solution of the differential equation
\[
\displaystyle 2xy \frac{\mathrm{d}y }{ \mathrm{d}x} = y^2 + x^2 - 1
\]
for which  $y=2$ when $x=1$. Describe geometrically this solution.
\item Find the solution of the differential equation 
\[
2x^2y\displaystyle \frac{\mathrm{d}y}{\mathrm{d}x} = 2 \ln(x) - xy^2
\]
for which  $y=1$ when $x=1\,$.
\end{questionparts}
\end{question}	
		

		
	
\newpage
\section*{Section B: \ \ \ Mechanics}


	
%%%%%%%%%% Q9
\begin{question}
A non-uniform rod $AB$ has weight $W$ and  length $3l$. 
When the rod  is suspended horizontally in equilibrium by  vertical 
strings attached to the ends $A$ and $B$, 
the tension in the string 
attached to $A$ is $T$.



When  instead the rod is held
in equilibrium in a horizontal position by means of  a smooth pivot at a distance $l$ from $A$
and a vertical 
string attached to  $B$, the tension in the string  is~$T$. 
Show that $5T = 2W$.

When instead the end $B$ of the rod
rests on rough horizontal ground and the rod is held in equilibrium
at an angle $\theta$ to the horizontal
by means of    
a string that is perpendicular to the rod and  attached to $A$, 
the tension in the string is $\frac12 T$. 
Calculate $\theta$ and find the smallest value of 
the 
coefficient of friction between the rod and the ground that will prevent slipping.
	\end{question}
	
%%%%%%%%%% Q10 
\begin{question}	
Three collinear, non-touching particles $A$, $B$ and $C$ have masses $a$, $b$ and $c$,
respectively, and  are 
at rest on a smooth horizontal surface. 
The particle $A$ is given an initial velocity $u$ towards~$B$. 
These particles collide, giving $B$ a velocity $v$ towards $C$. 
These two particles then collide, giving $C$ a velocity $w$. 

The coefficient of 
restitution is $e$ in both collisions. 
Determine an expression for $v$, and show that 
\[
\displaystyle w = \frac {abu \l 1+e \r^2}{\l a + b \r \l b+c \r}\;.
\]

Determine the final velocities of each of the three particles in the cases:
\begin{questionparts}
\item $\displaystyle \frac ab  = \frac bc = e\,$;
\item $\displaystyle \frac ba  = \frac cb  = e\,$.

\end{questionparts}
\end{question}

%%%%%%%%%% Q11

\begin{question}
A particle moves so that ${\bf r}$, 
its displacement from a fixed origin at time $t$, 
is given by 
\[{\bf r} = \l \sin{2t} \r {\bf i} + \l 2\cos t \r \bf{j}\,,\] 
where $0 \le t < 2\pi$.
\begin{questionparts}
\item  Show that the particle passes through the origin exactly twice.

\item  Determine 
the times when the velocity of the particle is perpendicular to its displacement. 

\item  Show that, when the particle is not at the origin,
its velocity is never parallel to its displacement.

\item  Determine the maximum distance of the particle from the origin, 
and sketch the path of the particle.
\end{questionparts}
\end{question}
	

	
	\newpage
\section*{Section C: \ \ \ Probability and Statistics}


%%%%%%%%%% Q12
\begin{question}
\begin{questionparts}
\item The probability that a hobbit smokes a pipe is 0.7 
and the probability that a hobbit wears a hat is 0.4\,. 
The probability that a hobbit smokes a pipe but does not wear a hat is $p$. 
Determine the range of values of $p$ consistent with this information.

\item  The probability that a wizard wears a hat is 0.7\,;
the probability that a wizard wears a cloak is 0.8\,; 
and the probability that a wizard wears a ring is 0.4\,. 
The probability that a wizard does not wear a hat, 
does not wear a cloak and does not wear a ring is 0.05\,. 
The probability that a wizard wears a hat, a cloak and also a ring is 0.1\,. 
Determine the probability that a wizard wears exactly two of a hat, a cloak, and a ring. 


The probability that a wizard wears a hat 
but not a ring, {\bf given} that he wears a cloak, is $q$. 
Determine the range of values of $q$ consistent with this information.
\end{questionparts}
\end{question}

%%%%%%%%%% Q13
\begin{question}
The random variable $X$ has mean $\mu$ and standard deviation $\sigma$.
The distribution of $X$
is symmetrical about $\mu$ and satisfies:
\[
\P \l X \le \mu + \sigma \r = a 
\mbox{ \ \ \ \ \ and  \ \ \ \ \ } 
\P \l X \le \mu + \tfrac{1}{ 2}\sigma \r = b\,,\]
 where $a$ and $b$ are fixed numbers. Do not assume that $X$ is Normally distributed.

\begin{questionparts}
\item 
Determine expressions (in terms of $a$ and $b$) 
for 
\[
\P \l \mu-\tfrac12 \sigma \le X \le \mu + \sigma \r
\mbox{ \ \ \ \ \ and \ \ \ \ \ \ } 
\P \l X \le \mu +\tfrac12 \sigma \; \big\vert \; X \ge \mu - \tfrac12 \sigma \r.\]

\item  My local supermarket sells cartons of skimmed 
milk and cartons of full-fat milk: 60\% of the cartons 
it sells  contain skimmed milk, and the rest contain full-fat milk. 


The volume of skimmed milk in a carton  is  modelled by 
$X$ ml, with $\mu = 500$ and $\sigma =10\,$.
The volume of full-fat milk in a  carton is  modelled by  $X$ ml,
 with $\mu = 495$ and $\sigma = 10\,$.

(a) 
Today, I bought one carton of milk, chosen at random, from this supermarket. 
When I get home, I find that it contains less than 505 ml. 
Determine an expression (in terms of $a$ and $b$) 
for the probability that this carton of milk contains more than 500 ml.


(b) Over the years, I have bought a very large number of cartons of milk, 
all chosen at random, 
from this supermarket. 70\% of the cartons 
I have bought have contained at most 505 ml of milk. 
Of all the cartons that have contained at least 495 ml of milk, 
one third of them have contained full-fat milk. 
Use this information to estimate the values of $a$ and $b$. 

\end{questionparts}
\end{question}

%%%%%%%%%% Q14
\begin{question}
The random variable $X$ can take the value \mbox{$X=-1$}, and also any
value in the range \mbox{$0\le X <\infty\,$}. The distribution of $X$ is given by
\[
\P(X=-1) =m  \,, \ \ \ \ \ \ \ \P(0\le X\le x) = k(1-\e^{-x})\,,
\]
for any non-negative number $x$, where $k$ and $m$ are  constants, and 
$m <\frac12\,$.

\begin{questionparts}
\item Find $k$ in terms of $m$. 
\item Show that $\E(X)= 1-2m\,$. 
\item Find, in terms of $m$, $\var (X)$ and 
  the median value of~$X$.
\item Given that 
\[
\int_0^\infty y^2 \e^{-y^2} \d y = \tfrac14 \sqrt{ \pi}\;,\]
find  $\E\big(\vert X \vert^{\frac12}\big)\,$ in terms of $m$.
\end{questionparts}
\end{question}
	
\end{document}
