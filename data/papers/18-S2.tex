\documentclass[a4, 11pt]{report}


%\pagestyle{myheadings}
%\markboth{}{Paper II, 2008  second vetter 
%\today 
%}               


\usepackage{pstricks-add}
\usepackage{epsfig}

\RequirePackage{amssymb}
\RequirePackage{amsmath}
\RequirePackage{graphicx}
\RequirePackage{color}
\RequirePackage{xcolor}


\RequirePackage[flushleft]{paralist}[2013/06/09]



\RequirePackage{geometry}
\geometry{%
  a4paper,
  lmargin=2cm,
  rmargin=2.5cm,
  tmargin=3.5cm,
  bmargin=2.5cm,
  footskip=12pt,
  headheight=24pt}


\newcommand{\comment}[1]{{\bf Comment} {\it #1}}
%\renewcommand{\comment}[1]{}

\newcommand{\bct}[1]{{\color{blue}#1}}
%\renewcommand{\comment}[1]{}
\newcommand{\rct}[1]{{\color{red}#1}}

\setlength{\parskip}{8 pt}
\setlength{\parindent}{0pt}

\newlength{\qspace}
\setlength{\qspace}{15pt}

\newcounter{qnumber}
\setcounter{qnumber}{0}

\newenvironment{question}%
 {\vspace{\qspace}
  \begin{enumerate}[\bfseries 1\quad][10]%
    \setcounter{enumi}{\value{qnumber}}%
    \item%
 }
{
  \end{enumerate}
  \filbreak
  \stepcounter{qnumber}
 }

\newenvironment{questionparts}[1][1]%
 {
  \begin{enumerate}[\bfseries (i)]%
    \setcounter{enumii}{#1}
    \addtocounter{enumii}{-1}
    \setlength{\itemsep}{5mm}
    \setlength{\parskip}{5pt}
 }
 {
  \end{enumerate}
 }


\DeclareMathOperator{\cosec}{cosec}
\DeclareMathOperator{\Var}{Var}

\def\d{{\rm d}}
\def\e{{\rm e}}
\def\g{{\rm g}}
\def\h{{\rm h}}
\def\f{{\rm f}}
\def\p{{\rm p}}
\def\q{{\rm q}}
\def\s{{\rm s}}
\def\t{{\rm t}}


\def\A{{\rm A}}
\def\B{{\rm B}}
\def\E{{\rm E}}
\def\F{{\rm F}}
\def\G{{\rm G}}
\def\H{{\rm H}}
\def\P{{\rm P}}


\def\bb{\mathbf b}
\def \bc{\mathbf c}
\def\bx {\mathbf x}
\def\bn {\mathbf n}

\makeatletter
\newcommand{\raisemath}[1]{\mathpalette{\raisem@th{#1}}}
\newcommand{\raisem@th}[3]{\raisebox{#1}{$#2#3$}}
\makeatother
%%%To raise suffices: e.g.  $\Pi_{\raisemath{2pt}{-}}$.


\def\le{\leqslant}
\def\ge{\geqslant}


\def\var{{\rm Var}\,}

\newcommand{\ds}{\displaystyle}
\newcommand{\ts}{\textstyle}


\begin{document}

\setcounter{page}{2}


 
\section*{Section A: \ \ \ Pure Mathematics}


%%%%%%%%%%%%% Q1
\begin{question}
Show that, if $k$ is a root of   
the quartic equation
\[
x^4 + ax^3 + bx^2 + ax + 1 = 0\,,
\tag{$*$}
\]
then $k^{-1}$ is a root.

You are now given that $a$ and $b$ in $(*)$ are both
real and are such that the roots are all real.




\begin{questionparts}
\item Write down all
the values of $a$ and $b$ for which $(*)$ has only one
distinct root.

 \item
Given that $(*)$ has exactly three distinct roots, show that 
either $b=2a-2$ or \mbox{$b=-2a-2\,$}. 
 
\item 
 Solve $(*)$ in the  case 
$b= 2  a -2\,$,
giving your solutions in terms of $a$.

\end{questionparts}

Given that $a$ and $b$ are both real and that the roots of $(*)$
are all real, 
find
necessary and sufficient conditions, in terms of $a$ and $b$,
for $(*)$ to have 
exactly three distinct real roots.

\end{question}


%%%%%%%%%%%%%Q2
\begin{question}
A  function $\f(x)$ is said to be {\em concave} for
  $a<   x <   b$
if
\[
 \ t\,\f(x_1) +(1-t)\,\f(x_2)
\le 
\f\big(tx_1+ (1-t)x_2\big) 
\,
,\]  
 for $a< x_1 < b\,$, \ 
 $a<   x_2<   b$ and $0\le t \le 1\,$.

Illustrate this definition by means of 
a sketch,
 showing the chord joining the points 
$\big(x_1, \f(x_1)\big) $
and
$\big(x_2, \f(x_2)\big) $, in the case $x_1<x_2$ and $\f(x_1)< \f(x_2)\,$.

Explain why a function 
$\f(x)$ 
satisfying  $\f''(x)<0$ for 
$a<   x <   b$  is concave for $a<   x <   b\,$.

\begin{questionparts}
\item
 By choosing $t$, $x_1$ and $x_2$ suitably,
show that, if $\f(x)$ is 
concave for $a<   x <   b\,$, then
\[
\f\Big(\frac{u+ v+w}3\Big) \ge \frac{ \f(u) +\f(v) +\f(w)}3
\,
,\]  
for  $a<   u <   b\,$,  $a<   v <   b\,$ and  $a<   w <   b\,$.
  



\item
Show that, 
if $A$, $B$ and $C$ are the  angles of a triangle, then

\[
\sin A +\sin B + \sin C \le \frac{3\sqrt3}2
\,.
\]
\item
By considering $\ln (\sin x)$, show that,
if $A$, $B$ and $C$ are the  angles of a triangle, then
\[
\sin A \times \sin B \times \sin C \le \frac {3 \sqrt 3}  8 \,.
\]


\end{questionparts}
\end{question}

%%%%%%%%%%%%%% Q3
\begin{question}
\begin{questionparts}
\item
Let 
\[
\f(x) 
 = \frac 1 {1+\tan x}
\]
for $0\le x < \frac12\pi\,$.

Show that $\f'(x)= -\dfrac{1}{1+\sin 2x}$ and hence find the 
range of $\f'(x)$. 

Sketch the curve $y=\f(x)$.

\item The function $\g(x)$ is continuous for $-1\le x \le 1\,$. 

Show that the curve $y=\g(x)$ has  rotational
 symmetry of order 2
about
the point $(a,b)$ on the curve if and only if
\[
\g(x) + \g(2a-x) = 2b\,.
\]

Given that the curve $y=\g(x)$ 
passes through the origin and
has rotational
 symmetry of order 2 
about the origin, 
write
down the value of 
\[\displaystyle \int_{-1}^1 \g(x)\,\d x\,.
\]
\item
Show that the curve $y=\dfrac{1}{1+\tan^kx}\,$,
where $k$ is a positive constant and $0<x<\frac12\pi\,$,
\\[3mm]
 has rotational 
symmetry of 
order~2 about a certain point (which you should specify) and evaluate
\[
\int_{\frac16\pi}^{\frac13\pi} \frac 1 {1+\tan^kx} \, \d x 
\,.
\]
\end{questionparts}

\end{question}
%%%%%%%%%%%%%% Q4
\begin{question}
In this question, 
you may use the following identity without proof:
\[
\cos A + \cos B = 2\cos\tfrac12(A+B) \, \cos \tfrac12(A-B)
\;.
\]

\begin{questionparts}

\item
Given that $0\le x \le   2\pi$, find all the values of $x$ that satisfy
the equation
\[
\cos x + 3\cos 2x + 3\cos 3 x + \cos 4x= 0
\,.
\]

\item
Given that
$0\le x \le \pi$ and $0\le y \le \pi$ and that
\[
\cos (x+y) + \cos (x-y) -\cos2x = 1
\,,
\]
show that  either $x=y$ or $x$ takes one specific value which you should find. 
\item
Given that
$0\le x \le \pi$ and $0\le y \le \pi\,$, 
find the values of $x$ and $y$ 
that satisfy the equation
\[
\cos x + \cos y -\cos (x+y) = \tfrac32
\,.
\]
\end{questionparts}
\end{question}



%%%%%%%%%%%%%% Q5
\begin{question}

In this question, you should ignore issues of convergence.

\begin{questionparts}
\item Write down the binomial expansion, for $\vert x \vert<1\,$, of 
$\;\dfrac{1}{1+x}\,$ and deduce that
%. By considering
%$  
%\displaystyle \int \frac 1 {1+x} \, \d x 
%\,,
%$  
%show that 
\[
\displaystyle
\ln (1+x) = -\sum_{n=1}^\infty \frac {(-x)^n}n
\,
\]
 for $\vert x \vert <1 \,$.
\item Write down the series expansion in powers of $x$ of 
$\displaystyle \e^{-ax}\,$.
Use this expansion to  show that 
\[
\int_0^\infty \frac {\left(1- \e^{-ax}\right)\e^{-x}}x
\,\d x = \ln(1+a)
\ \ \ \ \ \ \ (\vert a \vert <1)\,.
\]
\item
Deduce the value of  
\[
\int_0^1 \frac{x^p - x^q}{\ln x} \, \d x
\ \ \ \ \ \ (\vert p\vert <1, \ \vert q\vert <1)
\,.
\]
\end{questionparts}


\end{question}


%%%%%%%%%%%%%% Q6
\begin{question}
\begin{questionparts}
\item
Find all pairs of positive integers $(n,p)$, where $p$ is
a prime number,
that satisfy
\[
n!+ 5 =p
\,.
\]

\item
In this part of the 
question you may use the following two theorems:
\begin{enumerate}
\item[1.] 
For $n\ge 7$, 
$1! \times 3! \times \cdots \times (2n-1)! > (4n)!\,$.
\item [2.] For every positive integer
$n$, there is a prime number between $2n$ and $4n$.
\end{enumerate}

Find all pairs of positive integers $(n,m)$ that satisfy 
\[
1! \times 3! \times \cdots \times (2n-1)! = m! \,.
\]


\end{questionparts}
\end{question}

%%%%%%%%%%%%%% Q7
\begin{question}
The points $O$, $A$ and $B$ are the vertices 
of an acute-angled triangle. The points $M$ and $N$
lie on the sides $OA$ and $OB$ respectively, and the lines
$AN$ and $BM$  
intersect at $Q$. The position vector of $A$ with respect to 
$O$ is $\bf a$,  and the position vectors of the 
 other points are labelled similarly.

Given that $\vert MQ \vert = \mu \vert QB\vert $, and that
$\vert NQ \vert = \nu \vert QA\vert $, where $\mu$ and $\nu$ are positive and
 $\mu \nu <1$, show that
\[
{\bf m} = 
\frac {(1+\mu)\nu}{1+\nu} \,
{\bf a}
\,.
\]

The point $L$ lies
on
the
 side $OB$, and $\vert OL \vert = \lambda \vert OB \vert \,$.
Given that $ML$ is parallel to  $AN$,
express~$\lambda$ in terms of $\mu$ and $\nu$.

What is the geometrical significance of the condition $\mu\nu<1\,$?
\end{question}


%%%%%%%%%%%%%% Q8
\begin{question}
\begin{questionparts}
\item Use the substitution $v= \sqrt y$ 
to solve the differential equation
\[
\frac{\d y}{\d t} = \alpha y^{\frac12} - \beta y
\ \ \ \ \ \ \ \ \ \ (y\ge0, \ \ t\ge0)
\,,   
\]
where $\alpha$ and $\beta$ are positive constants.
Find the non-constant solution
$y_1(x)$
that satisfies $y_1(0)=0\,$. 

\item
Solve the differential equation
\[
\frac{\d y}{\d t} = \alpha y^{\frac23}  - \beta y
\ \ \ \ \ \ \ \ \ \ (y\ge0, \ \ t\ge0)
\,,
\]
where $\alpha$ and $\beta$ are positive constants.
Find the non-constant solution 
$y_2(x)$
 that satisfies
$y_2(0)=0\,$.

\item In the case $\alpha=\beta$, sketch 
$y_1(x)$ and $y_2(x)$ 
 on the same 
axes, indicating clearly which is 
$y_1(x)$ and which is $y_2(x)$.
You should explain how   you determined the positions of the
curves relative to each other.

\end{questionparts}
 
\end{question}

\newpage
\section*{Section B: \ \ \ Mechanics}


%%%%%%%%%%%%%% Q9
\begin{question} Two small beads, $A$ and $B$, 
of the same mass, are threaded onto a 
vertical wire on which they slide without friction, 
and which is fixed to the ground at $P$. 
They are released simultaneously from rest, $A$
 from a height of $8h$ above $P$ and $B$ from
 a height of $17h$ above~$P$. 

When  $A$ reaches the ground
for the first time, it is
moving with speed $ V$. It then 
rebounds with coefficient of restitution~$\frac{1}{2}$ 
and subsequently collides with  $B$ at height $H$ above~$P$.

Show that $H= \frac{15}8h$ and find, in terms of $g$ and $h$, the speeds $u_A$ and 
$u_B$ of the two beads just before the collision.
 
When $A$ reaches the ground for the second time, 
it is again moving  with speed $ V$.
Determine the coefficient of restitution between the two beads. 


\end{question}





%%%%%%%%%%%%%% Q10
\begin{question}
A uniform elastic string lies on a smooth horizontal table.
One end of the string
 is attached to a fixed peg, 
and the other
end is pulled at constant speed $u$. At time
$t=0$, the string is 
taut and its length is $a$. Obtain an expression for 
the speed, at time $t$,
of  the point on the string
which is a distance
$x$ from the peg at time~$t$.

An ant walks along the string starting at $t=0$ at the peg. 
The ant walks at constant speed~$v$ along the string (so that
its speed relative to the peg is the sum of $v$ and the speed of the 
point on the string beneath the ant).
At time $t$, the ant is a distance $x$ from the 
peg. 
Write down
 a first order differential equation
for $x$, and 
verify
that
\[
\frac{\d  }{\d t} \left( \frac x {a+ut}\right) = \frac v {a+ut} \,.
\]

Show that  the time $T$ taken for the ant to 
reach the end of the string is given by 
\[uT = a(\e^k-1)\,,\]
  where $k=u/v$.

On reaching the end of the string, the ant turns round and walks back to the 
peg. Find in terms of $T$ and $k$
 the time taken for the journey
back.


\end{question}

%%%%%%%%%%%%%% Q11
\begin{question}
The axles of the wheels of a motorbike of mass $m$
are a distance $b$ apart.  Its centre of 
mass is a horizontal distance of $d$ from the front axle, where
$d<b$, and a vertical
distance $h$ above the road, which is horizontal and straight.
The engine is connected to the rear wheel.
The coefficient of friction between the 
ground and the rear wheel is $\mu$, where $\mu<b/h$, 
and the front wheel is smooth. 

You may assume 
 that the sum of the moments of the forces acting on the motorbike
about the centre of mass is zero. By taking moments about the
centre of mass
show that, as the acceleration of the motorbike increases from zero,
the rear wheel will slip before the front wheel loses contact with the road
if
\[
\mu < \frac {b-d}h\,.
\tag{$*$}
\]
If the inequality $(*)$ holds and the rear wheel does not slip, 
show that the maximum acceleration is
\[
\frac{ \mu dg}{b-\mu h}
\,.
\]

If the inequality $(*)$ does not hold, find
the maximum acceleration 
given that
 the front wheel remains in contact 
with the road.





\end{question}

\newpage
\section*{Section C: \ \ \ Probability and Statistics}



%%%%%%%%%%%%%% Q12

\begin{question}
In  a game,  I toss a coin repeatedly. The probability, $p$, that
the coin shows Heads on any given toss is given by
\[
p= \frac N{N+1}
\,,
\]
where  $N$ is a positive integer.
The outcomes of any two tosses
are independent.  

The game has two versions.
 In each version, I can choose to 
stop playing after any number of tosses, in which case 
I win \pounds$H$, where $H$ is the number of Heads I have tossed. However,
the game may end before that, in which case I win nothing.
 
\begin{questionparts}
\item In version 1, the game ends when
the coin first shows Tails 
(if I haven't stopped playing before that).

I decide from the start  to  toss the coin until  
a total of $h$ Heads have been shown, unless the game ends before then. 
Find, in terms of $h$ and $p$, 
an expression for my expected winnings and show that
I can  maximise my 
expected
winnings by choosing $h=N$.  

\item In version 2, the game ends when the coin shows Tails on 
two {\em consecutive} tosses
(if I haven't stopped playing before that).

I decide from the start  to  toss the coin until  
a total of $h$ Heads have been shown, unless the game ends before then. 
Show that my expected winnings  
are 
\[
\frac{ hN^h (N+2)^h}{(N+1)^{2h}}
\,.
\]
In the case 
$N=2\,$,
use the approximation $\log_3 2 \approx 0.63$ to  show that 
the maximum value of my expected winnings is approximately \pounds 3.



\end{questionparts}
\end{question}




%%%%%%%%%%%%%% Q13

\begin{question}
Four children, $A$, $B$, $C$ and $D$, 
are playing a version of the game `pass the parcel'. 
They stand in a circle, so that $ABCDA$ 
is the clockwise order.
Each time a whistle is blown, the child holding the parcel is 
supposed to pass the parcel immediately exactly one place clockwise. In 
fact each child, independently of any other past event, 
passes the parcel clockwise with probability~$\frac{1}{4}$, 
passes it anticlockwise with probability 
$\frac{1}{4}$ and fails to pass it at all with~probability $\frac{1}{2}$. 
At the start of the game, child~$A$ is holding the parcel.

The probability that child $A$ is holding the parcel 
just after the whistle has been blown for the $n$th  time 
 is $A_n$, and   
$B_n$, $C_n$ and $D_n$ are defined similarly. 

\begin{questionparts}
\item 
Find $A_1$, $B_1$, $C_1$ and $D_1$. Find also 
 $A_2$, $B_2$, $C_2$ and $D_2$. 

\item 
%Write down 
%expressions for $A_{n+1}$, $B_{n+1}$, $C_{n+1}$ and $D_{n+1}$ in
%terms of $A_n$, $B_n$, $C_n$ and~$D_n$.

By first considering $B_{n+1}+D_{n+1}$, or  otherwise,  
find $B_n$ and $D_n$.

Find also expressions for $A_n$ and $C_n$ in terms of $n$.


 
\end{questionparts}
\end{question}






\end{document}


